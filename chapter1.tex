The “quest for the historical Jesus” is the modern scholarly effort to reconstruct Jesus within history rather than theology.
It began in the Enlightenment, flourished in the nineteenth century with many “Lives of Jesus,” was cut back by early twentieth-century skepticism, and has been renewed since the mid-century in successive waves.
Since then, countless works have proposed competing hypotheses about who Jesus was and what his message meant.
Before turning to our own analysis, it is useful to outline the most influential of these portraits, each of which has commanded serious attention in modern scholarship.

\section{Apocalyptic prophet.}\label{sec:apocalyptic}

In this portrait Jesus stands within the apocalyptic currents of late Second Temple Judaism.
His proclamation of the “kingdom of God” (βασιλεία τοῦ θεοῦ) is heard not as a timeless ethic but as a time–sensitive announcement that God is about to act: “the kingdom has drawn near” (ἤγγικεν; Mark 1:15).
The rhetoric is urgent—watchfulness, division, harvest, reckoning—and the frame is the same one found in Daniel, 1 Enoch, 4 Ezra, and the Dead Sea Scrolls: God judges the wicked, vindicates the righteous, restores Israel, and reorders the world.

Advocates point to a cluster of sayings that read naturally in this horizon.
Jesus speaks of the “Son of Man” (ὁ υἱὸς τοῦ ἀνθρώπου) who will be revealed, of cosmic portents, and of nearness indexed to “this generation” (ἡ γενεὰ αὕτη).
The “Little Apocalypse” (Mark 13 and synoptic parallels) sets tribulation, desecration, and the coming of the Son of Man “on the clouds” (cf. Dan 7:13) in a single arc.
Even Jesus’ reply to John the Baptist—“the blind see, the lame walk, the poor have good news preached to them”—echoes an eschatological sign list (cf. Isa 35; 61; 4Q521), tying healings to end–time renewal rather than mere compassion.
On this reading, prophetic sign–acts—the symbolic action in the Temple, the choosing of Twelve, open meals—function as enacted parables of Israel’s imminent restoration.

Some scholars frame this same material less in terms of timetable and more in terms of visionary experience.
Here Jesus’ authority rests not only on announcing the end but on intimacy with God in prayer, revelatory speech, and visionary moments such as baptism and transfiguration.
This “mystic” emphasis does not displace apocalyptic but stresses its experiential core: Jesus spoke urgently of God’s reign because he believed he had already seen and tasted it.

The model’s strength is historical fit.
It explains the urgency of Jesus’ summons, why his movement could be read as destabilizing by both priestly elites and Rome, why kingship language sounded political, and why crucifixion—the Roman penalty for seditious claimants—became the end of his public course.
It also explains why the earliest preaching carries eschatological pressure forward: Paul’s letters, for example, register both the nearness of the end and the need to order communities under that expectation.

The tensions are real.
Timelines are hard: sayings about “some standing here” who will see the kingdom, or about events arriving before a generation passes, sit uneasily with delay.
Redaction is a problem: Mark 13 looks shaped by post-70 trauma, raising questions about how much is Jesus and how much is crisis interpretation.
Not all remembered speech is crisis–driven; parables of mercy, commands to forgive, and counsel against anxiety can read like stable wisdom, not countdown rhetoric.
Yet for many historians these very difficulties strengthen rather than weaken the case.
Unfulfilled prophecy is precisely what one expects in real apocalyptic movements: bold timetables are proclaimed, fulfillment is delayed, and followers recalibrate.
The New Testament bears this stamp everywhere—Paul’s communities wrestle with delay, the Gospels frame “this generation” in shifting ways—showing not the absence of apocalyptic expectation but the lived process of its deferral.

For all its strengths, however, the apocalyptic prophet portrait leaves decisive paradoxes.
If Jesus was simply one among many Jewish apocalyptic preachers, his message proved strikingly unconvincing in its own homeland: within a generation the majority of Jews rejected it, and the movement held only a marginal foothold in Judea.
Yet the same message proved astonishingly persuasive among Greeks, spreading across cities and cultures at a pace no other apocalyptic sect achieved.
Equally striking, while the memory of Jesus preserved by his followers does contain apocalyptic urgency, it never appears as the core of his message.
Both the Gospels and the early epistles emphasize love, forgiveness, and community life far more strongly than eschatological timetables.
The text often regarded as the oldest in the New Testament, Galatians, centers on justification by faith, life in the Spirit, unity, and love, with apocalyptic expectation serving only as a backdrop.
Similarly, the core teachings that are most widely confirmed for historical Jesus, the Lords Prayer, the Sermon on the Mount, and the parables of mercy also treat end times as a backdrop rather than the core message.
And most telling, crucifixion was never the fate of an apocalyptic preacher for prophesying.
It was Rome’s penalty for crimes against the state—sedition, insurrection, or royal pretension.
If Jesus was only announcing the end of the world, he would not have been executed as the “King of the Jews.”
The apocalyptic model explains urgency, conflict, and expectation, but it cannot explain either the failure among Jews, the enduring appeal among Greeks, or why Rome treated him as a political threat rather than a harmless visionary.
That unresolved tension marks its greatest limitation.
The apocalyptic Jesus may not be the whole story, but he remains the unavoidable horizon against which all other accounts must measure themselves.

\section{Revolutionary or zealot.}\label{sec:revolutionary}

Another long-standing portrait identifies Jesus as a political insurgent against Rome.
It takes seriously the charge on the cross—“King of the Jews”—which is most naturally read as political, and it points out that crucifixion was Rome’s penalty for sedition.
On this model Jesus was not primarily a sage or mystic but a revolutionary claimant whose movement threatened the stability of Judea.

The fullest statement of this thesis came from S. G. F. Brandon in *Jesus and the Zealots* (1967) and *The Fall of Jerusalem and the Christian Church* (1951).
Brandon argued that Jesus stood in continuity with the nationalist zealot tradition, that the Temple action was a revolutionary sign-act, and that his disciples were not merely hearers but comrades in resistance.
Unlike later proposals that equate Jesus with specific rebel leaders, Brandon’s version did not depend on collapsing him into Judas or the Egyptian, but rather saw him as a distinct figure shaped by the same revolutionary currents.
This view is compelling because it makes sense of the political overtones of kingship language, the symbolic challenge to the Temple, and the Roman decision to crucify him.

Variants of the revolutionary thesis identify Jesus directly with known rebel leaders.
One proposal equates him with “the Egyptian” mentioned by Josephus, a prophet who led thousands to the Mount of Olives in the 50s CE and promised to bring down Jerusalem’s walls.
This parallel is attractive because it explains why Rome reacted so harshly and why later Christians might remember him as one who foretold the Temple’s fall.
The difficulty is chronological, since Josephus places the Egyptian two decades too late.

Another variant links Jesus with Judas the Galilean, who led a revolt against Rome at the time of Quirinius’ census in 6 CE and is remembered as the founder of the zealot movement.
Here the appeal is geographical—Judas was a Galilean like Jesus—and thematic, since both were proclaimed leaders challenging Roman rule.
The weakness is again chronology: Judas’ revolt predates Jesus’ ministry, and there is no clear evidence that they were the same person.

Even with these problems, the revolutionary model remains compelling because it foregrounds the political danger Jesus posed.
It explains why Rome crucified him, why the charge was “King of the Jews,” and why later memory of him could not be disentangled from royal and national expectations.
Still, all current variants of the theory suffer from chronological difficulties.
Shifting Jesus’ birth date, ministry, or crucifixion by a year or two is possible.
But moving them by decades, as required to identify him with the Egyptian or Judas the Galilean, is far beyond historical plausibility.
And while the revolutionary portrait is compelling, it is also clear that—like the apocalyptic model—it cannot be the whole story.

\section{Mythical figure.}\label{sec:mythical}

The mythicist position ranges from total denial of Jesus’ existence to the claim that, even if he lived, nothing of him remains recoverable beneath the myths.
The case rests on two pillars: the lateness and anonymity of the sources, and the density of mythic and literary parallels in the ancient world.

Jesus himself wrote nothing.
The Gospels known to us today are only first implied to exist when Irenaeus, writing in the late second century, describes Marcion’s activity around 140 CE.
While they are frequently dated to the late first century, the evidence for mid second century dates is strong as well.
First century and early second century writings show no clear references to the Gospels or epistles we know today, and a direct mention would be expected if they already circulated.
Even by the end of the first century most followers of Jesus were likely long dead, or at least too old to compose literary works.
The texts are anonymous, and were attributed to Matthew, Mark, Luke, and John only much later by the church fathers, often on the basis of speculation.
Irenaeus, who elsewhere makes bold and confident assertions, is notably more reserved when assigning Gospel authorship.
He explains his reasoning, giving the impression of an educated reconstruction rather than a well–preserved tradition handed down from the apostles.

The Gospels do not present independent eyewitness voices but rather a layered literary tradition: Mark written first, Matthew and Luke revising it, and John reshaping it theologically.
Most scholars agree we are looking at second or third edition of gospels, and our texts are copies of copies of copies with much of the original content likely lost.
The narratives are saturated with scriptural borrowing.
The passion reads like a deliberate pastiche of Psalm 22, Isaiah 53, and Daniel 7.
Matthew overtly echoes Exodus, Hosea, and Micah, and builds the whole story to highlight how Jesus is the modern Moses.

But the parallels extend well beyond scripture.
Jesus’ healings, exorcisms, and life story have hundreds of uncanny parallels in the life of Apollonius of Tyana.
His wine miracle at Cana, the Eucharist, and resurrection recalls Dionysian cult.
The language of “rebirth” and “new creation” parallels rites of Mithras, Isis, and Eleusis.
His passion and death recall motifs of dying-and-rising gods such as Osiris, Attis, and Dionysus.
Stories of demoniac of Gerasenes and the Legion, bear uncanny resemblance to the story of Odysseus and Polyphemus.
The Lord’s Prayer bears uncanny resemblance to the solar hymns to Aten.
The dove descending at baptism is exactly how Zeus appears when coming down to earth.

Even the trial and death fit known templates.
The nobly suffering teacher brought before the state echoes the trial of Socrates.
Philosophers condemned for corrupting youth or challenging gods form a clear cultural backdrop.
The crucifixion becomes, in this reading, a Mediterranean variation on the righteous sage unjustly executed by human power yet vindicated by the divine.

Paul’s letters, when read this way, describe no Galilean rabbi.
They proclaim a heavenly Christ revealed through visions and scripture.
Paul never quotes a parable, names Nazareth, or narrates a miracle.
His Jesus is cosmic, not biographical.

The earliest external witnesses are late and second-hand.
Josephus is dismissed as interpolated.
Tacitus is seen as reporting Christian belief, not independent archival knowledge.
Suetonius’ “Chrestus” is treated as unrelated.

On this reconstruction, Christianity begins not with memory of a man but with myth, allegory, and ritual.
The parallels to mystery cults, Dionysian faith, Apollonius, and philosophical trials are so dense that the story of Jesus reads less like unique biography and more like a collage of known cultural motifs.
Whether or not a man once lived at the root, the Jesus of the texts is viewed as myth from the ground up, a theological construct rather than a historical teacher.

While many critics claim the parallels are forced and not genuine, whether these parallels are real bears little consequence for likelihood of this model being historically accurate.
That was simply how stories were told in the context of Jesus.
It does not mean that the core of the story is false.
Other critics point our the extreme precision of geographic and unrelated historical details in the texts.
Even if we grant the correctness of all of these, these too have very little bearing on the core question of whether Jesus was a mythical figure or not.
Stan Lee wrote about New York and events happening there in excruciating detail but that does not in any way make the stories of Spiderman more historically accurate.
The weaknesses of the model lie elsewhere.

The most fundamental issue of this theory comes from the minor details that are significant to the story but clearly not understood by the authors of the texts.
Unlike background realism that can be freely invented, these fragments look like authentic pieces of memory copied without comprehension.
For example John’s Gospel reports that Joseph of Arimathea and Nicodemus bring a mixture of myrrh and aloes to anoint Jesus’ body.
A late mythical John would not have produced medical agents while Jesus was buried in a hurry.
Or Matthew talking about the father who will reward you in secret.
A late mythical Matthew would not have quoted overheard Egyptian theology in a Jewish prophetic context.
Or later texts naming the magi as Caspar, Melchior, and Balthazar.
A late mythical birth narrative would not have been aware of these courtly titles while calling them magi.
We will explain the background behind these and many other examples in later chapters.

On top of that, it is very hard to explain the sheer number of authors who seem to selectively know the geography and language of the events they are alleged to be a part of.
We find too much corroboration in sources that at the same time show sharp disagreement on elements that likely did not happen.
Each version of the story presents different miracles and different words of Jesus on the cross, but the description of the trial is highly consistent.
We are quite certain that there were already major disagreements about Jesus’ identity and message within just years of his death.
The early authors did not hesitate to produce different interpretations, but they were very reserved about inventing the framework of events.
It is hard to imagine a myth where some central elements were so widely divergent while many seemingly minor details were so widely agreed upon.
This pattern of corroboration and disagreement is very difficult to reconcile with pure invention.

What emerges is not fabrication but history refracted through mythic language—an account rooted in lived experience, though expressed in the idiom of the age.
And once again, while myth is likely responsible for many elements of the story, it is unlikely a very large part of the story.

\section{Healer, exorcist, and social prophet.}\label{sec:healer}

One of the most persistent portraits of Jesus is that of a healer and exorcist whose acts of power drew crowds and gave his mission immediate authority.
The Gospels present this dimension not as ornament but as central: exorcisms of demons, cures of blindness and lameness, restorations of lepers, and actions that reintegrated the ritually impure.
Such traditions are multiple, early, and widespread, and they align with Jewish memories of holy men like Honi the Circle Drawer and Hanina ben Dosa, who were recalled for prayer, rainmaking, or healing.
They also resonate with Mediterranean traditions of wonder–workers and divine men whose authority rested on visible power.
Not surprisingly, an enormous number of books have been written that develop this portrait in different ways, sometimes emphasizing the Jewish holy man, sometimes focusing on social reintegration, sometimes on political economy, sometimes on religious experience, and sometimes on broader Mediterranean parallels.
While these approaches vary in emphasis, together they create a composite picture: Jesus’ deeds of power were integral to his reputation and message.

Within this framework, healings can be read as enacted parables of God’s reign.
An exorcism signals the defeat of hostile powers.
A cure enacts the restoration of Israel.
A shared meal erases boundaries of purity and status.
Social readings underline that illness meant exclusion and healing meant reintegration into family, village, and covenant life.
Political readings stress that purity codes and debt systems served as instruments of elite control, so that Jesus’ healings and meals enacted an alternative social order.
Experiential readings emphasize his sense of sonship and prayer as the source of power.
Comparative readings place him among Mediterranean healers and sages, but note that his idiom is Israel’s scripture and his horizon is covenant renewal.
Liturgical readings show how stories of healing became rituals of memory, shaping how communities taught, preached, and worshiped.

The strengths of this composite portrait are breadth and explanatory force.
It makes sense of why crowds gathered, why discipleship formed, and why opponents reacted strongly.
It shows how “kingdom” talk took flesh in visible action, not just in speech.
It explains why memory of Jesus endured: people remembered help received.

The weaknesses are equally important.
The problem is not that healings are incredible but that they are insufficient as a total explanation.
Many figures in antiquity were remembered as healers, and countless priests across centuries have performed exorcisms.
That category alone cannot explain why this Galilean became the center of a movement that reshaped history.

Methodological limits are obvious: historians cannot verify miracles as they can coins or inscriptions, and narrative shaping blurs what occurred.
The traditions overlap with nearly every other portrait: apocalyptic prophets, wisdom teachers, revolutionaries, and restorationist prophets all absorb healings as evidence for their own frames.
Early letters like Paul’s, our first Christian texts, mention visions and spiritual gifts but do not narrate Jesus as healer in detail, suggesting that miracle stories grew in prominence after his death.
The stories themselves vary—some cures are instant, others gradual; some tied to faith, others not—pointing to community shaping as well as memory.
And most decisive, healing alone does not explain crucifixion.
Rome did not execute men for curing the sick.
The danger must have lain in his proclamation of God’s reign, his symbolic challenge to the Temple, his dynastic claim, or his perceived political threat.

The result is a portrait that secures one indispensable strand of memory but cannot stand alone.
Jesus likely did perform healings and exorcisms, and these were remembered as signs of divine power and social renewal.
But they must be framed within a larger horizon—apocalyptic urgency, dynastic claim, challenge to authority—if they are to explain why Jesus was crucified as “King of the Jews” and why his followers proclaimed him as Lord.

\section{Sage and teacher.}

This composite portrait gathers the strands that see Jesus first as a teacher.
He speaks in parables and aphorisms that invert conventional honor and redraw moral horizons.
He debates halakhah with rigor while pressing mercy and the “weightier matters of the law.”
He performs symbolic acts that point to covenant renewal rather than private piety.
In this frame he is a wisdom teacher within Israel’s sapiential tradition, a near-Pharisaic legal interlocutor, and—at points—a village sage whose style can look Cynic-like in its sharp speech, voluntary poverty, and open table.

The strengths are substantial.
Parables are the most stable core of remembered speech and fit Jewish wisdom forms.
Brief aphorisms travel well across communities and explain why sayings collections arose.
Legal debates over Sabbath, purity, divorce, and tithes situate Jesus inside Jewish law rather than outside it.
Symbolic actions—especially the Temple action and the choosing of the Twelve—cohere with hopes for Israel’s renewal.
Comparative work shows that his social performance—itinerancy, counter-status meals, pointed critique—has real Mediterranean analogues without erasing his scriptural idiom.

This portrait also clarifies why Jesus drew disciples.
Teachers gather learners.
Memorable speech, enacted in public practices, generates communities that preserve and repeat it.
It explains durability: wisdom survives delays and disappointments better than timelines do.

Its weaknesses are different in kind.
“Teacher” is too elastic to explain the end of the story.
Rome did not crucify sages for parables.
To make sense of the charge “King of the Jews,” the teacher profile must be integrated with claims that sounded political—kingdom, Temple, tribe, and rule.
The Cynic analogy can overreach if it downplays Jesus’ thick engagement with Torah and Israel’s story.
The law-debater emphasis can flatten if it turns him into a conventional Pharisee with a softer tone.
The restoration theme can drift into vagueness unless one specifies what changes in Temple, land, and kingship were actually envisioned.

There is also a problem of selectivity.
Sayings can be arranged to yield almost any portrait if context is pruned.
Parables about mercy sit alongside warnings of judgment and royal imagery.
Wisdom and eschatology are intertwined in the sources, not neatly separable.

The most coherent reading treats teaching as the medium rather than the message.
Parables, aphorisms, and legal disputes are how Jesus prosecuted a larger claim about God’s rule and Israel’s renewal.
On that view the sage-teacher profile is indispensable for understanding how his words were remembered, transmitted, and practiced.
Standing alone, it cannot explain why he was executed by Rome or why his followers proclaimed him Lord.
As with the other portraits, it names a real center of gravity that still requires integration with apocalyptic pressure, symbolic kingship, and public challenge to authority.

\section{Dynastic king or royal claimant.}\label{sec:dynastic}

A further portrait interprets Jesus as a dynastic figure, a royal claimant whose household preserved succession after his death.
Modern works that develop this theme include James D.\ Tabor, \emph{The Jesus Dynasty}; Robert Eisenman, \emph{James the Brother of Jesus}; Hyam Maccoby, \emph{Revolution in Judea}; Barbara Thiering, \emph{Jesus the Man}; Simcha Jacobovici and Charles Pellegrino, \emph{The Jesus Family Tomb}; and Michael Baigent, Richard Leigh, and Henry Lincoln, \emph{Holy Blood, Holy Grail}.

Several lines of argument appear across this body of work.
One emphasizes the crucifixion charge ``King of the Jews'' and reads it as the plain accusation of kingship.
On this view the triumphal entry is a royal procession, the Temple action a constitutional gesture of sovereignty, and the choosing of the Twelve a redistribution of tribal administration under restored rule.
The genealogies in Matthew and Luke are treated as dynastic records designed to assert legal descent and political right.

A second emphasis falls on family succession as the mechanism of continuity.
Jesus’ brothers and kin, remembered as the \emph{desposyni}, appear as the continuation of his line.
James the Just is described as immediate successor in Jerusalem, presiding over halakhic decisions and community leadership.
The dynasty extends further through relatives such as Simeon son of Clopas, yielding a remembered sequence of royal claimants.

Another current stresses a dual pattern of messianic office.
In this model John the Baptist functions as priestly messiah and Jesus as royal messiah.
The baptism is interpreted as an investiture that fuses priesthood and kingship and establishes a joint framework of legitimacy.

Another line of argument links Jesus directly to the Hasmonean–Herodian dynasty.
In this reconstruction he is remembered not only as a Davidic claimant but as a grandson or descendant of Herod the Great, through Antipater or another branch of the family.
Such a pedigree would explain the consistent association of his birth with royal danger, the efforts of Herod’s court to eliminate him, and the political charge inscribed on the cross.
It would also account for the Gospel genealogies as dynastic propaganda, crafted to frame his lineage within Israel’s idiom while positioning him in rivalry to other heirs of the throne.
This version of the dynastic portrait interprets his crucifixion less as the fate of a marginal prophet and more as the neutralization of a dangerous claimant to power within the Herodian house itself.

Archaeological arguments add another dimension.
The Talpiot tomb in Jerusalem is read as a family burial with inscribed ossuaries whose names correspond to Gospel figures.
Statistical treatment of the onomastic cluster is used to claim that such a combination is unlikely to occur by chance.
Within the dynastic portrait the site functions as material confirmation of an urban household with royal pretensions.

Some lines extend the portrait into later history through a bloodline hypothesis.
In this reading the ``Holy Grail'' is understood as royal blood (\emph{sang real}).
Descent is traced through exile and memory, with survival of the line narrated in medieval and early modern traditions.

Several weaknesses are often raised against the dynastic portrait.
The genealogies of Matthew and Luke diverge in detail and appear inconsistent with later virginal-conception traditions.
Royal titulature is claimed to be rare in surviving references, leading some to doubt that Jesus was ever remembered as a king.
The Talpiot tomb has been debated because the names it contains are both common and yet clustered in a way that recalls the Gospel family.
These weaknesses deserve careful attention, since in-depth treatment can have a major impact on the overall assessment of dynastic models.
We will examine each of these in later chapters, where we test whether they are truly fatal or whether they can be reconciled within a coherent dynastic reconstruction.

The much more difficult problem is the social one.
If Jesus’ claim was simply that of a Jewish dynasty, it is hard to explain why his death was followed not by enduring Jewish allegiance but by an overwhelmingly Gentile movement.
The earliest evidence shows synagogues divided, but within a generation Jewish adherence is marginal while Greek cities across the empire host flourishing assemblies.
If the crucifixion titulus “King of the Jews” captured his claim, why did the Jews themselves not preserve his kingship?
Why instead did the gentiles, who had no stake in a Davidic line, become his primary adherents?
This imbalance between Jewish rejection and Gentile embrace remains the hardest tension to reconcile within a narrowly Judean dynastic model.

\section{Supernatural figure.}\label{sec:supernatural}

Most modern scholarship works within a naturalistic frame, but in this project we do not accept that God, Allah, or even non-human intelligences should be treated as impossible explanations in advance.
The ``supernatural figure'' portrait takes seriously the claim that Jesus was more than human, or that he truly spoke with the agency of a divine or non-human being.
This includes the orthodox confession of God the almighty creator incarnate.
It also encompasses rival ancient visions in which he descends as a heavenly revealer, an emissary of a higher God, prophet of Allah, or a being from beyond the created order, as well as modern reframings that cast him as otherworldly in technological terms.
Across these variants the core thesis is the same: Jesus acted with non-human agency, and people encountered him as a living power.

The positive case is straightforward.
Earliest worship treats him as κύριος and Son, not merely teacher.
Communities invoke his name in prayer, hymn, and meal as if divine presence is operative.
Miracle and exorcism traditions are central, not decorative.
Rival movements that denied his flesh or split him from Israel’s God still presupposed a supernatural Jesus.
Persistence and global spread of devotion are unusual if he was only a village sage.

The decisive pressure point is continuity.
If a being with non-human agency entered history, we should expect ongoing public interaction with the world.
We should see signs that are clear, repeatable, and available to outsiders, not only to insiders.
An incarnate God, a heavenly emissary, or a supra-human agent would not vanish into a brief Galilean interval and then recede into silence.
We would expect durable footprints in the shared record of nature and society.
This expectation itself reflects modern scientific standards, but it highlights the gap between ancient testimony and the kinds of continuous evidence demanded today.

Science sharpens that expectation.
Modern inquiry looks for effects that can be specified in advance, observed under controlled conditions, and reproduced.
Hospitals, demographic data, and large cohorts supply continuous measurement of recovery, mortality, and anomalous events.
Astronomy, seismology, and climate series track the heavens and the earth with high precision.
Against this background there is no stable, public pattern of events that requires a standing supernatural term.
Reports of wonders are sporadic, local, and non-replicable.
They are mediated by testimony rather than by instrumented observation.
They do not accumulate into a signal that forces revision of physics, biology, or the causal structure we currently use.

Ancient records show a similar selectivity.
While reports of miracles and sightings are common, imperial annals, temple archives, and civic inscriptions rarely corroborate the kinds of spectacular signs one would expect if interventions were widespread.
Where wonders appear, they follow the contours of community memory and polemic rather than the long-series data that usually mark persistent causes.

This does not refute every claim.
Absence of replicable signal is not proof of impossibility.
Testimony to healings, deliverance, sudden transformations, and visions is widespread across cultures and centuries.
Some recoveries remain medically unexplained.
Some experiences remake lives and communities in ways ordinary causation struggles to capture.
Those data exist, but they do not yet form a continuous empirical trace that satisfies scientific criteria for public verification.

The result is a balanced judgment.
The supernatural portrait has real historical weight because devotion to Jesus as more-than-human is early, intense, and persistent.
Its critical weakness is the continuity problem: a world allegedly under continued supernatural agency does not display a commensurate, public, reproducible pattern of effects.
In this book we acknowledge testimony to miracles and do not attempt to falsify such experiences.
What can be said here is that our reconstruction shows no contradiction with the supernatural portrait, and we leave it to the reader to judge whether the evidence we present is more consistent with or without it.
That said, we do find multiple strands of evidence suggesting that many Christian traditions are more credible than often assumed, and very limited evidence that the generally accepted traditions might be false.

\medskip

Every portrait of Jesus surveyed so far is supported by serious arguments.
They would not have lasted in scholarship if they did not illuminate real features of the evidence.

But historical judgment cannot be built on the number of arguments alone.
Probability is not a matter of tallying strengths and weaknesses like votes.
One clear contradiction with secure data—chronology, geography, language, political context—is enough to break a theory.
By contrast, a cluster of minor tensions may weaken a case but does not destroy it.
The question is not how many problems a model has, but whether any of its problems are fatal.

On this strict ground, each of the standard models proves partial.
They explain urgency, or healings, or wisdom teaching, or political danger, but always at the cost of leaving some other fixed fact in conflict.
The result is that none of the models we discussed so far can be correct.
The way forward is not to choose among them, but to recover what remains coherent once contradictions are stripped away and the surviving elements are integrated into a single, consistent account.
It is to that background and reconstruction we now turn.
