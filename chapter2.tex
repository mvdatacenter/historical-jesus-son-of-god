\section{Chapter 2 - He Truly was the Son of God}\label{sec:chapter-2---he-truly-was-the-son-of-god}

Next we re-investigate Jesus's royal claim within the context of the greek empire.
We want to show that all the titles given to Jesus were titles given to the rulers of the greek empire.
That the events surrounding Jesus's life were consistent with the events surrounding the rulers of the greek empire and were deeply rooted in greek imperial cult.

\paragraph{0.
Jesus is the Christ.}\label{par:jesus-is-the-christ.}

In the gospels and all the earliest materials we can see by far the most common title given to Jesus is Christ.
It is not the Messiah, but the Christ.
In the very few cases the word Messiah is used, the gospels clearly state that Messiah is the Hebrew translation of the word Christ.
Christ is the title used by Josephus, Christ is used by Pliny the Younger, Christ is used by Tacitus, Christ is used by Suetonius, Christ is used by Paul.
Christ is the only term used by Jesus own highly educated, greek speaking brother James.
It is not the other way around, Jesus was not actually called the Messiah, which later morphed into Christ.
The gospels just use the Messiah to explain to the Hebrews what the Christ was, not the other way around.
We should look at the usage of the word Christ in the context of the greek empire and not purely in the context of the Hebrew apocalyptic literature.

\paragraph{1.
Jesus is the Son of God.}\label{par:jesus-is-the-son-of-god.}

Alexander was the son of God and so were the Seleucids, Ptolemys.
The title was used for the rulers of the greek empire.

\paragraph{2.
Jesus is the Son Of Father.}\label{par:jesus-is-the-son-of-father.}

The well known ruler of Egypt, Cleopatra, was given the title ``father loving goddess''.
Seleucus IV Philopator was also given the title ``father loving god''.
This or similar title was used by countless other rulers in the greek world.
In this context the multiple references to being the only-begotten son of the father signify the rightful heir, and the only rightful claim to the throne.

\paragraph{3.
Jesus is the Logos, the Word of God.}\label{par:jesus-is-the-logos-the-word-of-god.}

The Stoics, particularly Zeno of Citium (c.~334--262 BCE) and later Chrysippus, played a significant role in further developing the idea of Logos.
In Stoic thought, Logos became the rational principle that permeates and organizes the cosmos.
It was understood as a divine rationality that was present in the universe, giving it structure and coherence.
Very notably, Philo of Alexandria who was closely related to the Herodian and Hasmonean dynasties and contemporary to Jesus, described the Logos as a mediator between God and the world.
Thus, if Jesus and some of the other members of the court were indeed visiting Alexandria in Jesus's youth, they must have had close exposure to the idea of logos.
Based on the age difference and the family relations, Jesus would have likely treated Philo as his distinguished uncle, and so would John the Evangelist.

Notably in the context of the greek empire, the Logos was a divine principle that was both God's expression and as such the ruler would have been considered to speak the Logos of the God.
This is consistent with the idea of multiple quotes as nobody can get to the father except through me.

Notably Philo described the Logos as the firstborn son of God, and so the rightful heir to the throne and the intermediary between God and the world.

Even though countless philosophers claimed rulers should follow the Logos and called the Logos the divine, the title of the Logos itself does not appear to have been given to any ruler prior to John's gospel.
And as the idea of Logos in John appears so closely to the evolution of the Logos idea from Philo, it increases the likelihood that indeed John was in a close contact with Philo.

\paragraph{3.
Jesus is the God manifested in the flesh.}\label{par:jesus-is-the-god-manifested-in-the-flesh.}

Epiphanes is a title given to multiple rulers, such as Antiochus IV and Ptolemy V .
The title signify the ruler is the god manifested in the flesh.

\paragraph{4.
Jesus is the Savior of the World.}\label{par:jesus-is-the-savior-of-the-world.}

Essentially every other ruler of the greek empire was given Soter as a title, which means savior.
The first Ptolemy was given the title ``Soter'', and so was the last king of the greek kingdom Strato II Soter.
Antiochus III was even given the title ``The God the Savior''.

\paragraph{5.
Jesus is the Good Shepherd.}\label{par:jesus-is-the-good-shepherd.}

Plato taught that a true philosopher king is a good shepherd.
Likely it stuck as a title for Jesus.

\paragraph{6.
Jesus is the Light of the World.}\label{par:jesus-is-the-light-of-the-world.}

Here the title relates to the cave from Plato's Republic.
The journey from darkness to light symbolizes the philosopher's ascent from ignorance to knowledge, particularly knowledge of the Good.

\paragraph{7.
Jesus is the King of Kings and Lord of Lords.}\label{par:jesus-is-the-king-of-kings-and-lord-of-lords.}

Similarly, King of Kings and Lord of Lords was a title used for the rulers of empires, not only greeks, but also of India and Persia.

\paragraph{8.
Jesus is the Prince of Peace.}\label{par:jesus-is-the-prince-of-peace.}

Ptolemy V Epiphanes famous of the Rosetta stone, was called the bringer of peace.

\paragraph{9.
Jesus is the Word.}\label{par:jesus-is-the-word.}

The Stoics, particularly Zeno of Citium (c.~334--262 BCE) and later Chrysippus, played a significant role in further developing the idea of Logos.
In Stoic thought, Logos became the rational principle that permeates and organizes the cosmos.
It was understood as a divine rationality that was present in the universe, giving it structure and coherence.
The Stoics taught that humans should live in accordance with the Logos, which represented divine reason and the natural law of the universe.
For the Stoics, the Logos was both immanent and cosmic, meaning it was part of everything and governed all things.

For Philo, the Logos was a mediator between God and the world.
He described the Logos as a divine principle that was both God's expression and the agent of creation.
Philo's Logos was similar to the Stoic idea, but with a more direct connection to the Jewish monotheistic tradition.

\paragraph{10.
Jesus has 12 apostles.}\label{par:jesus-has-12-apostles.}

Apostles were a common term for the emissaries of the greek empire.
Ptolemaic and Seleucid Kings often sent out ἀπόστολοι as official emissaries or envoys on diplomatic missions.
The emissaries represented the kings' political will and spread decrees across the empire.

\paragraph{11.
Jesus was crucified for being the King of The Jews}\label{par:jesus-was-crucified-for-being-the-king-of-the-jews}

This is the actual question that was asked to Jesus by the Roman governor Pontius Pilate.
If Jesus really had the title of the last rightful heir to the greek empire, then he would have indeed posed a threat to the Roman empire.
Generally apocalyptic preachers would not have been given this punishment, but a rightful heir to the greek empire would have been.
Typically, the punishment was reserved for those who posed a threat to the Roman empire.

\paragraph{12.
The writing on the cross was ``The King of the Jews''}\label{par:the-writing-on-the-cross-was-the-king-of-the-jews}

In the context of crucifixion being a punishment for those who posed a threat to the Roman empire, it would make sense that Romans would post a note like this not to mock Jesus, but to warn others against rebelling against the Roman empire.

\paragraph{13.
Jesus was not left on the cross to be eaten by scavengers.}\label{par:jesus-was-not-left-on-the-cross-to-be-eaten-by-scavengers.}

Typically, the bodies of the crucified were left on the cross to be eaten by scavengers, but Jesus was taken down from the cross and buried in a tomb.
This is consistent with the Roman being harsh but ultimately not trying to overstep their bounds.
In a case of a slave committing a crime, Jesus would be left on the cross, but as they could expect more rebellion if their cruelty had no bounds, it is conceivable that they would have allowed Jesus to be buried.

\paragraph{13.
Parables of Jesus.}\label{par:parables-of-jesus.}

There are many parables of Jesus that seem to hint at him being a good ambassador of a future kingdom.
The parable of the talents, the parable of the sower, the parable of the lost sheep, the parable of the prodigal son, the parable of the good samaritan, the parable of the wedding feast, the parable of the ten virgins, the parable of the wise and foolish builders, the parable of the rich fool, the parable of the barren fig tree, the parable of the great banquet, the parable of the mustard seed, the parable of the yeast, the parable of the hidden treasure, the parable of the pearl, the parable of the net, all seem to hint at a future kingdom that Jesus would rule over.

\subsubsection{14.
Kingdom of God was a common term for the greek empire and the greek kingdoms.}\label{subsubsec:kingdom-of-god-was-a-common-term-for-the-greek-empire-and-the-greek-kingdoms.}

The Hellenistic rulers weren't just kings---they were divine monarchs ruling by the will of the gods.
Βασιλεία τοῦ θεοῦ is a very common pre christian greek term to describe greek kingdoms, notably used in the writings of Philo of Alexandria who likely influenced Jesus and early Christian writers.
This culture was very deeply rooted in the Greek, Persian and Egyptian royal cultures for centuries by that time.
And so restoring a kingdom of God on earth much more likely refers to restoring the greek empire than to some abstract idea of a spiritual kingdom.
This is something fresh in the mind of Philo and Jesus himself.

\subsubsection{15.
Jesus birth was represented as a new star in the sky.}\label{subsubsec:jesus-birth-was-represented-as-a-new-star-in-the-sky.}

This is a common trope in the greek world, notably used for the birth of Alexander the Great.
The Ptolemies of Egypt (Greek rulers after Alexander) often linked their divine status to stars and celestial phenomena.
Ptolemy III (246--222 BC) was honored with a new star appearing, supposedly confirming his divine favor.
The tradition actually was transferred to the Roman empire, where the birth of Augustus was also represented as a new star in the sky, and so was the death and deification of Julius Caesar.

The story of Ptolemy III (246--222 BC) and a celestial sign is linked to the Canopus Decree (238 BC), an inscription issued by Egyptian priests during his reign.
This decree honors Ptolemy III and his wife, Berenice II, and includes references to astronomical phenomena associated with his rule.

The Canopus Decree (238 BC) This decree was issued by Egyptian priests to honor Ptolemy III for his military campaigns and religious patronage.
It mentions a new star appearing in the sky, likely in connection with his divine status.
The decree also orders the addition of a leap day to the Egyptian calendar, demonstrating Ptolemy III's association with astronomical knowledge.
Callimachus (Greek Poet, 3rd Century BC) In his lost work, Aetia, Callimachus possibly referenced Berenice's Lock, a constellation myth that was linked to a celestial omen for Ptolemy III .
The myth suggests that Berenice II dedicated a lock of her hair for her husband's victory, which disappeared and was later seen as a new star in the sky (Coma Berenices).
Manetho (Egyptian Historian, 3rd Century BC) Though most of Manetho's works are lost, later writers reference his accounts of omens, stars, and divine portents during the reign of Ptolemy III .

Antiochus III (the Great, r.
222--187 BC) was said to have had a new star appear before his greatest campaigns.
Seleucid coins often depicted Zeus with a star, symbolizing divine rule.

Mithridates VI of Pontus (120--63 BC) Celestial Sign: A comet appeared at his birth, interpreted as a sign of his future greatness.
Source: Justin's Epitome of Pompeius Trogus mentions his divine status.

\subsubsection{16.
Jesus was buried in a tomb.}\label{subsubsec:jesus-was-buried-in-a-tomb.}

No other crucified person was buried in a tomb.
They were left on the cross to be eaten by scavengers.
A regular revolutionary would have been left on the cross, but someone with a royal lineage could be given an extraordinary exception.
As the Romans likely could have conceived of Jesus being more divine by his royal lineage, they may have already be afraid of Gods wrath at the time of crucifixion.

\subsubsection{17.
Jesus was crucified on Wednesday in 31 AD.}\label{subsubsec:jesus-was-crucified-on-wednesday-in-31-ad.}

It is commonly believed that Jesus was crucified on Friday, but all the Christian sources actually very directly say that Jesus was crucified on Wednesday.
This is not a new idea, but for some unknown reason it is ignored by nearly all scholars.
The Friday crucifixion is a later liturgical development and not present in the bible or earliest Christian sources.
All gospels agree that Jesus was crucified on the day of preparation for the Passover.
The common misunderstanding is that the day of preparation is the day before the Sabbath, which is Saturday, but in this context the passover is also a Sabbath.
So for the texts to be consistent, the day of preparation must be Thursday.
If we assume Friday we have trials before Annas, Caiaphas, Sanhedrin, Herod, and Pilate, mocking, beatings, travel, flogging, crucifixion, death, burial before sunset all supposedly occurred on the same morning, which is highly implausible.
There is obviously the extremely common statement that Jesus was in the tomb for three days and three nights which only works if we assume Jesus was crucified on Thursday.

To summarize: Wednesday crucifixion in 34 AD (April 21 Julian) fits all the evidence: John 19:31 explicitly distinguishes the Sabbath following Jesus' death as a ``high day'' (i.e., not a regular Saturday Sabbath).
Leviticus 23:7 establishes that Nisan 15 is a mandatory Sabbath regardless of weekday---this is the Feast of Unleavened Bread.
Matthew 28:1 does indeed say ``after the Sabbaths'' (plural: σαββάτων), and this is not a scribal error.
Mark 16:1 -- ``When the Sabbath was past, Mary Magdalene, and Mary the mother of James, and Salome, bought spices\ldots'' → This refers to buying spices after the first (Thursday) Sabbath, so it must be Friday.
``Three days and three nights'' (Matthew 12:40) cannot be forced into a Friday-to-Sunday window without manipulating Jewish idiom or chronology.
Luke 23:56 -- ``They returned and prepared spices and ointments; and rested on the Sabbath according to the commandment.'' → This refers to preparing spices on Friday, and then resting on Saturday.

Day Event Here's the count if Jesus died Wednesday afternoon: Buried before sunset Wednesday Night 1: Wednesday night Day 1: Thursday (High Sabbath) Night 2: Thursday night Day 2: Friday (spice preparation day) Night 3: Friday night, third night in the tomb Day 3: Saturday (Weekly Sabbath), third day in the tomb Night 4: Jesus resurrected at the sundown of Saturday, that is the beginning of Sunday in Jewish calendar Day 4: Sunday (first day of the week) the tomb was found empty If resurrection occurred just after sunset on Saturday, it is exactly three days and three nights.
Leading to Thursday, Friday, and Saturday being the three days and three nights, and the evening of Saturday when the resurrection allegedly occurred is the beginning of Sunday.
So this is fully consistent with the numerous references to the resurrection on Sunday.

Why the tomb was found empty Sunday: Because the women waited until the Sabbath ended (Saturday evening), then came at dawn Sunday (Matthew 28:1, Luke 24:1).
That's when the resurrection was discovered --- not when it happened.

So: Three full days and nights: yes.
Resurrection not seen, only tomb found empty Sunday: yes.
Fits Jewish counting and Gospel timeline: yes.

Then the crucifixion must have happened on Wednesday, April 21, 31 AD .

Note that for example, the possibly oldest preserved Christian text, the Didache, explicitly states that fast on Wednesday and Friday.
Friday is likely linked to Sabbath, but Wednesday is a new addition and most logically it would be the day of crucifixion.

Notably the earliest Christian writers outside the gospels and before Tertullian (c.~160--220 AD) that talk about the resurrection are Justin Martyr (c.~100--165 AD) and Barnabas (c.~70--135 AD), both of whom place the resurrection on at the start of Sunday, but do not list the day of crucifixion.

\subsubsection{17.
Jesus survived crucifixion}\label{subsubsec:jesus-survived-crucifixion}

In this context it is not even inconceivable that the Romans would have allowed Jesus to be picked up from the cross before death.
It is highly speculative, but it has a really strong explanation power to it.
Perhaps something as trivial as lightnings and thunders could have already made the Roman soldiers and the crowds to superstitiously believe he truly was the son of God and got scared.
Joseph of Arimathea and Nicodemus did receive an agreement from Pilate to pull him from the cross early, and Jesus could have simply survived the trauma and barely alive.
And on the third day he got so much better that he was able to walk around and talk to the apostles and show his wounds.
Then Jesus died fifty days later due to infection and all started to believe he was resurrected and ascended to heaven.

Notably all burial care was done at the time of death.
It was not Sabbath yet, and they had allegedly plenty of time to do bury Jesus on Friday.
Yet why would they still tend to Jesus body on Sunday morning?
This could have been a medical care, not just continuation of unfinished burial process.
The eventual death of Jesus from infection, especially given the severe wounds he suffered, adds a realistic angle to the story.
After his brief recovery, it would be plausible for his body to succumb to the damage sustained during the crucifixion.
This would also explain why the apostles continued to believe in his resurrection, even after his eventual death.
They might have interpreted his survival and brief recovery as divine intervention and seen his later death as part of a larger divine plan.
Perhaps all the doubting really did happen as the apostles were certain that Jesus was dead as they were not eyewitnesses to the event itself.

Further support to the story is In Mark 15:44, Pilate is described as being surprised by the news of Jesus' death, as he expected Jesus to have been on the cross longer.
Mark states: ``Pilate was surprised to hear that he was already dead.
Summoning the centurion, he asked him if Jesus had already died.
When he learned from the centurion that it was so, he gave the body to Joseph.'' This detail suggests that Jesus' death was unexpectedly quick.
Crucifixion was a prolonged form of execution designed to last for hours, if not days, as the condemned person typically died from a combination of blood loss, exposure, and suffocation.
For Pilate to be surprised, it could imply that Jesus' death occurred more quickly than usual, which is significant because:

Jesus cries out loudly before dying (Mark 15:37, Luke 23:46): Crucifixion victims typically die slowly, often suffocating, with fading strength.
A loud cry right before death is unusual and may imply he still had significant strength---suggesting he was not yet at death's door.
A burst of strength like this would point more towards a theatrical performance to convince the others the death was real.

Roman soldiers were typically experienced in carrying out executions, and the death on the cross was intended to be slow and torturous.
The standard time for death was several hours, and for someone to die within less than six hours, as Jesus did, would have been unusual.
Pilate's surprise may indicate that Jesus' death was significantly quicker than expected.
It may be that Jesus wasn't fully dead at the time he was taken down from the cross.
It seems likely small omens in the sky mixed with fear in the crowd and even among the soldiers made Pilate more receptive to Joseph of Arimathea's request without carefully checking Jesus fully passed away.
Note that Joseph of Arimathea was a member of the Sanhedrin, and he was likely a person of influence making it even more likely Pilate would have been more receptive to his request.

When the Roman soldiers pierced Jesus' side with a spear, blood and water flowed out, which is often interpreted as a sign of death.
``Instead, one of the soldiers pierced Jesus' side with a spear, bringing a sudden flow of blood and water.'' (John 19:34) Trauma, scourging, and prolonged stress could have caused a buildup of fluid around the lungs (pleural effusion) or around the heart (pericardial effusion).
If pierced, these fluids could flow out as a mix of clear fluid (``water'') and blood.
If Jesus were completely dead, the blood would have likely clotted in his body, and the wound wouldn't produce such a ``sudden flow.'' The fact that both blood and water ``flowed out'' immediately suggests the body still had some circulatory activity, meaning the heart might not have fully stopped yet.
Pleural or pericardial effusion does NOT mean the person is already dead---it can happen before death in cases of extreme shock or injury.
It would be likely more than enough to convince the centurion that Jesus was dead and pass the news to Pilate.
Joseph of Arimathea taking Jesus to his own garden, according to the gospel of Peter, to bury him there, is also highly suspect.
If Jesus were truly dead, it would be strange for Joseph, not his family, to take initiative.
But if alive, placing him in a personal tomb under control of a sympathizer makes sense.
The excuse that Jesus had to be buried quickly before the Sabbath in a temporary grave may actually be a plot to hide the fact that Jesus was not dead yet.
Aloes and myrrh for treatment, not burial (John 19:39): 75 pounds of myrrh and aloes were brought by Nicodemus.
That's far more than needed for burial alone and both have known medicinal properties---especially for healing wounds.
The quantity hints at treatment, not embalming.
It is also worth noting that using armed guards to protect the tomb is not a common practice.
Armed guards when the person is still alive would make a lot more sense.
For example some of the opponents of Jesus suspected a foul play and wanted to make sure Jesus was really dead and not taken down from the cross alive.

Many scholars point out that Arimathea is a place that does not exist, and so it is likely a made up name.
However, Ar-Ram, also known as Ramathaim, today better known as Ramallah or Ram Allah, is an ancient town a few kilometers north of Jerusalem that is more than likely to be that place.
We actually already know Ramallah from the Old Testament, where it is mentioned as the birthplace of Samuel.
Notably early versions of Septuagint translated the birthplace as ``Αρμαθαιμ'' in 1 Samuel 1:1, while hebrew text used Ramathaim.
It is a natural translation of the name into Greek, and the Septuagint should be considered as a strong evidence that Arimathea is indeed Ramallah.
It is also notable enough and close enough to Jerusalem that it would have been very logical origin place for a prominent member of the Sanhedrin.
It is also notable enough to have been mentioned in the gospels as a place elevating the status of Joseph of Arimathea and familiar enough to the people of the story to not be needing any further explanation.
The doubting found in all the gospels is also highly natural.
Simply Jesus being tortured and left for near-certain death but then eventually surviving would have still been treated as a miracle.
Most likely the apostles really did not believe much in miracles and were not expecting Jesus to survive.

Finally, many scholars point out that the victim of crucifixion were always left on the cross to be eaten by scavengers in every known record.
However, even Philo of Alexandria, featured in many discussions in this book for unrelated reasons, described a case of numerous Jew insurrectionists in Alexandria in 38AD were actually taken down from the cross in exchange for a bribe.
It is not completely clear from the text, but the more plausible reading is that some of the victims may have been taken down from the cross before they died.
In this light it should not be considered as implausible that a member of the Sanhedrin would have been able to bargain with the centurion to convince Pilate Jesus already died.

Josephus, Jewish War 4.5.2 (333): He says he recognized three of his acquaintances being crucified, asked for their removal, and one of them survived.

Indications do not fully end at the burial, Thomas' request to touch wounds (John 20:27) only makes sense if the wounds are fresh and still healing---rather than glorified.

Summarizing, although partially speculative, there is a lot of speculation that matches unexpectedly well on close inspection.
Addition of so much of this detail We need to consider the possibilities that Jesus really died and then numerous propaganda stories were created with strictly controlled narrative.
The non-resurrection theory actually does suffer very substantially from the problem of consistent narrative.
The ungrounded claims would not be corroborated by everyone in the same way.
There is bound to be more serious discrepancies in the story and more variants of the story.
To witness the fact, there is famously very serious discrepancy between all the gospels as to how the resurrection was discovered.
This actually strongly corroborates the idea that a lot of earlier highly consistent narratives were actually independently attested in the gospels, while the discovery of the missing body must have been a deliberate attempt to cover up the actual story.

The other alternative is that there really was empty tomb and misunderstanding.
For example, Joseph and Nicodemus really did use a temporary grave and then moved Jesus to a different grave without telling anyone.
Then the women came to the grave and found it empty and spread the news to Peter and John and so the story started to spread.
Then the empty grave was undeniable but everyone doubted the resurrection as they had no certainty that Jesus was resurrected or simply his body was secretly moved.

So here we need to consider the plausibility that Jesus survived because of luck or a conspiracy of Joseph of Arimathea and Nicodemus, and contrast it with the alternative of the event being completely fabricated.

\subsection{Was God the Father the God of Moses or the God of Plato?}\label{subsec:was-god-the-father-the-god-of-moses-or-the-god-of-plato}

In modern Christianity it is completely beyond doubt that God the Father is YHWH, the God of Moses, the God of Abraham, Isaac and Jacob.

Not many are aware that this is actually a surprisingly recent development in Christianity.
Many believe that the distinction of God the Father from the God of Moses was the heresy of Marcion and one Marcion has been excommunicated, the Church has always believed that God the Father is the God of Moses.
However, this is not the case.
Marcion's theory went further than that, he believed that the God of Moses was an evil god, and that the God of Jesus was a good god.
This theory was far more radical than the idea that God the Father is not the God of Moses, and that was what was condemned by other thinkers of the early Church.

In the Gospel of John, Jesus is described as the Logos, the Word of God with is very unequivocal reference to the Greek creator deity and not the God of Moses.
A lot more of the philosophy and theology of Logos has been described by Philo who seems to have had very substantial influence on the New Testament texts.
Saint Paul refers to the God the Father by ``For we are indeed his offspring'' which was one of the most common phrases related to Zeus.
Just a moment later he uses the common greek thought ``The God who made the world and everything in it\ldots{} does not live in temples\ldots{} we should not think the divine being is like gold or silver\ldots{} In him we live and move and have our being.''

Perhaps the most common perspective was expressed by Clement of Alexandria ``God is one and the same, the universal Father, being known under many names.''

An epitaph from the 5th century AD, written by deeply Christian family, reads: ``Weak at birth with defence of crosses walled, Guiltless of any dark stain of sin, Little Theodosius, who with pure mind parents Chose to dip in the sacred baptismal font Cruel death seized.
May the ruler of Olympus' height Give rest to these members with the noble sign Of the cross marked, proclaiming an heir of Christ.''

Then we obviously have the images of God

\subsubsection{Pater Noster}\label{subsubsec:pater-noster}

In catechisms and commentaries the Pater Noster is presented as a quintessentially Jewish prayer.
Two Gospel forms survive---Matthew 6:9--13 (Sermon on the Mount) and Luke 11:2--4 (disciple request).
Its usual gloss runs like this: • ``Our Father in heaven'' echoes synagogue formulas (e.g., the later Kaddish: ``Exalted and hallowed be His great Name'').
• ``Hallowed be thy Name'' = sanctifying YHWH's Name (already holy in Israel).
• ``Thy kingdom come / Thy will be done on earth as in heaven'' = Israel's hope from Daniel and the Prophets.
• ``Give us today our daily bread'' = manna typology or Psalmic providence.
• ``Forgive us our debts\ldots{} as we forgive'' = Jubilee/Leviticus ethic.
• ``Lead us not into temptation, but deliver us from evil'' = moral temptations; God's protection from sin.
On this reading the prayer is ``Second-Temple Jewish'' through and through.

\begin{enumerate}
\def\labelenumi{\arabic{enumi})}
\setcounter{enumi}{1}
\item
  Why that neat picture collapses
\end{enumerate}

Read closely, the prayer pointedly avoids Israel's covenant markers.
There is no Sinai, no Torah, no Zion, no Abraham, no Sabbath, no sacrifices---nothing that nails it to Moses.
Instead, the vocabulary is cosmic, royal, solar: • A universal Father in the heavens, not the covenant God ``who brought you out of Egypt.'' • A sanctified Name without the Tetragrammaton or Temple.
• A Kingdom that descends from heaven to earth (cosmic axis), not the restoration of David on Mount Zion.
• A petition for daily bread does not echo manna (which was not daily) or psalmic providence (which is not daily).
• A trial (Greek peirasmos) and rescue from the Evil One that ring like an eschatological ordeal with a devouring adversary, not like a generic plea about private temptations.

The standard reading ``works'' only by importing background the text doesn't supply, while ignoring the imagery it does supply.

\begin{enumerate}
\def\labelenumi{\arabic{enumi})}
\setcounter{enumi}{2}
\item
  The Egyptian solar--royal reading (what fits cleanly, line by line)
\end{enumerate}

\begin{enumerate}
\def\labelenumi{(\alph{enumi})}
\item
  ``Our Father in the heavens'' Egyptian hymns to Aten and Amun-Ra address the high god as father of all; the Pharaoh is the son of the Sun.
The address is cosmic, not ethnic.
It's the right register for the prayer.
\item
  ``Hallowed be thy Name'' (ἁγιασθήτω τὸ ὄνομά σου) Egyptian piety centers on the Name (rèn).
Amun literally means ``the Hidden (One)''; his hidden Name is praised and protected.
Refrains like ``Your name is Amun---Amun, Amun'' are liturgical.
This is precisely a sanctified Name without pronouncing it---a far tighter fit than Moses' tetragrammaton practices as usually described for lay prayer.
\item
  ``Thy kingdom come.
Thy will be done on earth as in heaven.'' Every dawn the Sun restores Maʿat (order) in the heavens and, through the king, on earth.
Solar kingship is a heaven-to-earth pipeline of will/order.
That is exactly the structure of this petition.
\item
  ``Give us today our daily bread'' (τὸν ἄρτον\ldots{} τὸν ἐπιούσιον) The Great Hymn to Aten praises the god who ``daily makes bread for humankind.'' Egyptian offering formulas (``bread and beer, daily'') are standard temple language.
This line is almost a quotation in sense.
\item
  ``Forgive us our debts as we forgive our debtors.'' Egypt frames justice as weight at judgment---the heart weighed against the feather of Maʿat.
Being ``set right'' (absolved of moral weight/debt) is the difference between survival and obliteration.
The ethical turn (``as we forgive'') binds the worshiper to enact Maʿat socially.
This is far closer to Egyptian moral weight/debt than to later juridical hair-splitting.
\item
  ``Do not bring us into the time of trial, but deliver us from the Evil One.'' (μὴ εἰσενέγκῃς\ldots{} εἰς πειρασμόν\ldots{} ῥῦσαι ἀπὸ τοῦ πονηροῦ) Peirasmos = trial/ordeal, not chiefly ``temptation.'' Read apocalyptically, it is the great ordeal; read visually, it is the judgment scene.
And the Evil One is not an abstraction: in Egyptian iconography the failed soul is devoured by Ammit (crocodile-lion-hippo) as the scales tip.
``Deliver us from the devourer'' is exactly how the scene works.
\item
  ``Amen.'' Before we pretend ``amen'' is safely, uniquely Hebrew, note the liturgical practice: Egyptian hymns and responses end with acclamations of Amun; congregational call-and-response reinforces the Name.
The sound and function of the close match the prayer's cadence (whatever the etymology textbooks insist).
In worship, ``All say Amun/Amen!'' is the logic.
\end{enumerate}

Bottom line: every clause fits solar-royal liturgy without strain.
The ``Jewish only'' reading must paper over the prayer's cosmic grammar; the Egyptian reading doesn't.

\begin{enumerate}
\def\labelenumi{\arabic{enumi})}
\setcounter{enumi}{3}
\item
  The historical pipeline (why this imagery would still be alive)
\end{enumerate}

This isn't Bronze-Age dust accidentally stuck to a 1st-century text.
It's continuous culture: • Egypt in Canaan (c.~1500--1150 BC).
For centuries the southern Levant was an Egyptian province.
Jerusalem appears in the Amarna archive (\textasciitilde1350 BC), with its ruler Abdi-Heba writing Pharaoh as ``my Sun.'' Egyptian garrisons and cult stood at Beth-Shean, Jaffa, Deir el-Balah, etc.
• Davidic psalmic core (c.~1150--970 BC).
The linguistically older psalms are drenched in sun, light, heaven, earth, kingship, divine rule (Ps 19; 29; 68; 84; 104).
These read like Hebrew adaptations of solar hymns, not Torah homilies.
• Aten's revolution and Amun-Ra's supremacy.
Akhenaten's Aten-monotheism collapses, but Amun-Ra returns stronger; the solar-monotheist pressure never disappears.
• Ptolemaic Egypt (3rd--1st c.~BC).
The dynasty crafts Serapis/Isis cult and keeps solar divine kingship explicit.
Cleopatra's death (30 BC) is within grandparent memory of Jesus' generation.
• Jesus' milieu.
A ``king of the Jews'' claim sits inside Roman Syria-Palestine, saturated with Helios/Sol imagery.
Early Christian art happily paints Christ as Helios; the holy day is Sunday.
The solar-royal idiom is not alien---it is the water everyone swims in.

Seen through this pipeline, the Pater Noster doesn't borrow a few Egyptian phrases; it belongs to the solar-royal register that ran from Aten → Amun-Ra → Ptolemaic kingship straight into the first century.

\begin{enumerate}
\def\labelenumi{\arabic{enumi})}
\setcounter{enumi}{4}
\item
  ``But isn't it still Jewish?'' --- the honest reconciliation
\end{enumerate}

Yes, the prayer can be prayed in a Jewish key (and was).
Luke embeds it in a lesson on dependence; Matthew frames it within piety and forgiveness.
The wording genuinely overlaps later synagogue language (``hallowed be His Name'').
But that overlap proves adaptability, not origin.
Crucially, the prayer: • avoids covenant particulars, • speaks in universal solar-cosmic grammar, and • lands perfectly inside Egyptian/Ptolemaic royal theology.

The better model is fusion: Israel's high-God devotion absorbed, translated, and reused the dominant solar-royal grammar everyone understood.
If Jesus stands---as our thesis argues---as a royal claimant in the wake of a just-collapsed Ptolemaic world, then this prayer reads not as a synagogue formula but as a dynastic solar-royal prayer: heavenly Father (Sun), kingdom descending, daily sustenance, righteous scales, rescue from the devourer---Amen.

\begin{enumerate}
\def\labelenumi{\arabic{enumi})}
\setcounter{enumi}{5}
\item
  Clause-by-clause gloss (for the reader who wants the map) • Father in heaven → Solar source (Aten/Amun-Ra) and royal sonship.
• Hallowed Name → the hidden Name exalted (Amun's rèn).
• Kingdom come / will be done → Maʿat restored from heaven to earth via the king.
• Daily bread → the sun-god who daily provides bread.
• Forgive debts → lighten the moral weight at the scales; enact Maʿat with others.
• Do not bring into the trial → spare us the ordeal/judgment.
• Deliver from the Evil One → rescue from the devourer who consumes the failed.
• Amen → the communal seal, functionally identical to Amun-acclamation.
\end{enumerate}

Why this matters for David and Jesus • David (1150--970 BC) sits close enough to the Amarna horizon for Egyptian solar kingship to be living memory; the oldest psalms sound like it because they grew in it.
• Jesus (early 1st c.~AD) stands within living memory of Ptolemaic solar monarchy.
If he is (as our book argues) a royal figure inside that political theology, the Pater Noster is exactly the kind of solar-royal prayer a claimant would teach: it translates Egypt's oldest grammar of kingship into a form his followers can pray anywhere.

That is the reading that explains everything the text actually says---without importing Sinai---and explains why the prayer crossed languages and empires so effortlessly.

\subsection{Jesus Christ believed in true immortal soul}\label{subsec:jesus-christ-believed-in-true-immortal-soul}

Sometimes overlooked, the idea of soul in greek and jewish beliefs is very different.
``Do not fear those who kill the body but cannot kill the soul'' ψυχή (psyche) ≠ body (Matt 10:28) ``What does it profit a man to gain the world and forfeit his soul?'' (Matt 16:26) ``Today you will be with me in Paradise'' (Luke 23:43)

The belief in post mortem conscious existence as a continuation of soul is not a belief of the Old Testament, but it is a common belief in the Greek world.
The Jews of that time believed in a resurrection of the body, but not in the immortality of the soul.

\subsection{Holy Mary, Mother of God, Perpetual Virgin, Queen of Heaven}\label{subsec:holy-mary-mother-of-god-perpetual-virgin-queen-of-heaven}

\paragraph{1.
Mary was the rightful heiress to the Hasmonean dynasty and so the royal titles we know very well today also correspond to the titles of the Greek rulers.}\label{par:mary-was-the-rightful-heiress-to-the-hasmonean-dynasty-and-so-the-royal-titles-we-know-very-well-today-also-correspond-to-the-titles-of-the-greek-rulers.}

There were countless greek female rulers with highly distinguished, divine titles, such as Cleopatra, who was also a common name in the Hasmonean dynasty.

\paragraph{2.
Holy Mary mother of God.}\label{par:holy-mary-mother-of-god.}

As Jesus was the Son of God, as a rightful heir, Mary as the mother of future ruler would have been considered the Mother of God.

``Mother of the Lord'' (Μήτηρ τοῦ Κυρίου) -- Found in Luke 1:43, where Elizabeth calls Mary the ``Mother of my Lord.'' This implies royal status since ``Lord'' (Kyrios) could mean a divine or kingly ruler.
Elizabeth greets Mary: ``And why is this granted to me, that the mother of my Lord should come to me?''

\paragraph{3.
Mary was a perpetual virgin.}\label{par:mary-was-a-perpetual-virgin.}

A very common trope in the greek world was that the royal women were pure often called perpetual virgins.
This is something that fits really well with the idea of Mary as royal but seems at odds with the idea of Mary as a mother to Jesus.

\paragraph{4.
Mary was the Queen of Heaven.}\label{par:mary-was-the-queen-of-heaven.}

The earliest known hymns and prayers to Mary refer to her as Βασίλισσα τῶν Οὐρανῶν -- explicit royal status.

``The Woman Clothed with the Sun'' -- Revelation 12:1 ``A great sign appeared in heaven: a woman clothed with the sun, with the moon under her feet, and a crown of twelve stars on her head.'' While this passage refers to Israel, early Christian writings (Hippolytus, 3rd century AD) link it to Mary as a royal mother figure.

\paragraph{5.
Mary was Blessed among all women}\label{par:mary-was-blessed-among-all-women}

Highlighting Mary's royal lineage

\paragraph{6.
Mary was the New Eve}\label{par:mary-was-the-new-eve}

``New Eve'' -- Early Church Fathers (Justin Martyr, Irenaeus) described Mary as the new Eve, implying a role in a divine dynasty.
Protoevangelium of James (c.~2nd century AD) -- While emphasizing her perpetual virginity, it also hints at a priestly and royal lineage, calling her ``set apart for the Lord.''
