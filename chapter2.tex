Most scholarly works on the historical Jesus begin with a brief overview of the historical background of the time.
And at the very start we already arrive at a major bias in the historiography of Jesus.
The overview is usually focused only on Jewish history and the Roman occupation of Judea.
While these are extremely important, for over 300 years preceding the birth of Jesus Christ the entire Eastern Mediterranean was shaped by the successors of Alexander the Great.
Hellenistic culture was prolific and deeply involved in every aspect of life in the Greek states.
Even though every single Christian text for decades after the birth of Jesus was written in Greek, and the Judaism we know today only began to be practiced in a Hellenistic state, this wider background is mostly ignored.

Among the successors of Alexander, those who ruled over Galilee and Judea the longest were the Ptolemaic dynasty, who governed from Alexandria in Egypt, and the Seleucid dynasty, who ruled from Antioch and Damascus in Syria.
It is worth noting that Galilee was directly adjacent to Syria and Phoenicia, while Judea bordered Egypt.
At the same time Galilee and Judea were separated by Samaria, which was not considered a friendly neighbor to either.

Most scholars do acknowledge Greek, Egyptian, and Syrian influence on the story of Jesus, but they often compare it to the ancient mythologies of these nations rather than to their actual religious and philosophical beliefs in the first century.
By then, both Greeks and Egyptians had been steeped in centuries of monotheistic thought.
In the world of Jesus, the Theos, the Demiurge, or the Creator God was already the most common way to refer to the supreme deity in Greek philosophy.
Meanwhile, Alexandrians worshiped Serapis, a syncretic deity combining Osiris and Apis from Egyptian religion with Zeus and Hades from Greek tradition.

On 31 August 326 BCE, Alexander the Great, King of Macedon, stood on the banks of the Hydaspes River in India and wept because there were no more worlds to conquer.
In 323 BCE, Alexander died in Babylon and left his empire to the strongest of his men.
His realm was divided, with the largest share and the imperial title going to Seleucus Nicator.
Under Greek rule an era of enlightenment and prosperity spread across the nations of the East.
In a short span of time countless colonies were founded and given Greek law, currency, and culture.
Among the cities of the world, Ephesus, Antioch, Thessalonica, Laodicea, Philippi, Corinth, Athens, Tarsus, and Alexandria rose as the greatest seats of learning.

In 146 BCE, the Roman general Lucius Mummius destroyed Corinth, and Polybius lamented, ``The day will come when men will ask where once stood mighty Corinth.''
In the year 85 BCE, to the shock of the world, the Roman general Lucius Cornelius Sulla fought and destroyed the combined 350,000–strong army of the Greek world.
Athens, once the teacher of the world, lay in ruins.
In 31 BCE, at Actium, fortune truly turned away from the Greeks and embraced the Romans as Gaius Julius Caesar Octavianus defeated the combined forces of the Greek world and Mark Antony.
In the East the remaining parts of the Greek empire were attacked by the Parthians and Scythians.
The last Greek king of Bactria, Strato II Soter, fell to King Rajuvula around 10 CE.

With that the fall of the entire Greek world was complete — well, not exactly\ldots{}
When the general Sulla sued for peace he did not fully incorporate Judea, a rebellious land, and permitted the Greek dynasties of the Hasmoneans and Herodians to continue to rule as client kingdoms of Galilee, Samaria, Judea, and the Decapolis.
And so the imperial court officials of the Greeks, the Head of the Imperial Guard, the Keeper of Imperial Light, and the Imperial Treasurer, came to Galilee from the East to seek the last rightful heir to the empire.

In this light let's first revisit the identity and background of Jesus Christ.
Where was he born and when?
Who were his parents?
What was the world like where he grew up?

Let us begin the exploration of the historical Jesus by revisiting the identity and background of Jesus Christ.

There is a long-standing assumption that historical Jesus and his apostles and companions were illiterate Jewish peasants from Judea and Galilee which were a backwater of the Roman empire.
This assumption underlays nearly all of modern scholarship and is treated as gospel while there is barely any evidence to support it, while there is overwhelming evidence to the contrary.
Breaking this assumption can categorically change the way we assign the probabilities to various theories about the life and death of Jesus Christ and the rise of Christianity.

\paragraph{1.
Timeline}\label{par:historical-background}

Let's revisit the the current mainstream scholarly consensus on the timeline of the life of Jesus Christ.

Jesus was born to Mary and Joseph in Nazareth in Galilee, around 4 BC.
The alleged birth in Bethlehem is considered to be a later invention to fulfill the prophecy of city of David.
Jesus did not flee to Egypt as a child, as this is also considered to be a later invention to fulfill the prophecy of Hosea.
The birth narrative is considered to be a later invention, and the gospels are deeply inconsistent and were interpolated much later to fit the narrative.
Jesus was a carpenter by trade and was illiterate living in a backwater village of the Roman empire.
His apostles were also illiterate peasants.

When looking more closely at the historical evidence, we find that the traditional biblical timeline is actually far more sensible than the current mainstream scholarly consensus.
When looking at most scholarly arguments we see a lot of overinterpreted events as prophetic fulfillment's and allegories where literal reading in the right historical context would make far more sense.

\paragraph{2. Was Jesus an illiterate peasant?}\label{par:nazareth-was-not-a-backwater-village.}

Nazareth was in the same area as Sepphoris (today Tzippori), the capital of Galilee.
The Church of the Annunciation, the Church of St. Joseph, and the Basilica of Jesus the Adolescent — our best indications of where Jesus lived as a child — are only about 4 km from the center of Sepphoris.
Sepphoris was a major city, and a settlement that close should be considered part of its immediate district rather than a remote village.
Herod the Great rebuilt Sepphoris as a royal city, and in many respects it was more important than Jerusalem at the time.
It had a Greek theater, a Roman-style forum, colonnaded streets, and elite villas decorated with mosaics such as the famous Dionysus and Nile Festival scenes.

Archaeology underscores how deeply Hellenized the region was.
While countless Greek inscriptions and mosaics have been uncovered in the Sepphoris area, almost no Hebrew inscriptions or clear signs of Torah observance or Second Temple Jewish practices have been found.
By contrast, Jerusalem and Samaria show abundant synagogues, mikva’ot, and Hebrew inscriptions from the same period.
Although Galilee was considered an Israelite region, it does not appear to have adopted Second Temple Judaism in the same way as Judea.

It has often been pointed out that Jesus was a humble “carpenter,” but this rests on a mistranslation of the Greek word τέκτων (*tekton*).
The term does not mean only “carpenter” but more broadly “builder” or even “stonemason.”
Jesus’ frequent use of construction and masonry metaphors in his teaching supports this reading.
Being a builder was a common occupation for members of royal households, and Herod the Great himself was praised as a master builder.
Thus τέκτων (*tekton*) strengthens, rather than weakens, the case for Jesus’ royal background.

This connection between Jesus’ sayings and the broader philosophical traditions cannot be reconciled with the assumption that he and his apostles were illiterate peasants.
Multiple Gospel pericopes presuppose literary allusions: Jesus’ dialogues echo Cynic-Stoic sayings, and his parables employ established rhetorical tropes.
It is difficult to imagine that a man unable to read Greek could have produced such forms, or that illiterate followers could have preserved them with such precision.
Either they were divinely inspired, or they were highly educated — able to rival the most learned philosophers of the time.

If Jesus was so well-educated, why did he not write anything himself?
The answer is simple: he had others record his teachings for him.
This would be unusual for a prophet or a commoner, but it is exactly what we expect from a person of high status — a nobleman or a royal figure.

One of the strongest arguments for the late dating of the Gospels is the assumption that Jesus and his apostles were illiterate peasants.
If that assumption cannot stand, then the case for late dating must be reconsidered.
We will revisit this in more depth, but once we grant that the Gospels were written close to the time of Jesus — by highly educated people who were either eyewitnesses or had access to eyewitnesses — the likelihood that they preserve genuine historical facts increases substantially.

\paragraph{3.
Messiah or Christ?.}\label{par:messiah-or-christ}

Mixing up the terms *Christos* and *Messiah* is deeply rooted in the historiography of Jesus.

The common narrative is to dismiss the idea of “Jesus, son of Mary and Joseph, Christ” as ridiculous, and then to say that *Christos* was simply the Greek translation of the Jewish term *Messiah*, and not a title in its own right.

Nearly every historical treatment of the topic forgets that *Christos* is a Greek term meaning “anointed one,” often used for royal or chosen figures, while *Messiah* is a Jewish term referring to a prophesied apocalyptic figure.
The New Testament overwhelmingly uses *Christos*, whereas *Messiah* appears only a few times, usually in dialogues with Jewish figures.
This suggests that the term *Messiah* was employed only when convincing Jewish audiences that Jesus the *Christos* was also their expected *Messiah*.

In Greek usage, *Christos* applied to athletes, rulers, and initiates — categories tied to public honor and kingship, not to hidden Jewish prophecy.
If *Christos* were simply a translation of *Messiah*, we would expect to find the Greek term used in Jewish texts outside Christianity.
Yet apart from a rare usage in Aeschylus (*Prometheus Bound*), its earliest sustained association is with Jesus.

And so we arrive at another strong argument: Jesus *Christos* may well have been understood as a royal figure.

\paragraph{4.
Royal lineage through his mother Mary}\label{par:royal-lineage-through-his-mother-mary}

The lineage of Jesus is described in the Gospels of Matthew and Luke, with Luke traditionally understood to trace the line through his mother Mary.
From this alone it follows that both Mary and Joseph were presented as of royal descent.
Mary’s early title Θεοτόκος (“God-bearer”) makes sense only if she was perceived as more than a peasant mother: she was viewed as a dynastic figure whose womb transmitted legitimate kingship.

The Gospels also show Mary in Jerusalem with notable frequency.
That pattern ties her not merely to Galilee but to the Judahite world centered on Jerusalem and Bethlehem.
Bethlehem itself functioned as a satellite of Jerusalem, the city of David, so tracing Jesus’ lineage through Mary places him within that dynastic line.

If so, Mary giving birth to Jesus in Bethlehem, the city of David, makes perfect sense — it is plausible and even expected.
It is not something that needed to be invented merely to fit prophecy.
As will be discussed later, Matthew had no shortage of Old Testament texts he could frame as prophecy; in practice he could find a proof-text for almost any verse in his Gospel.
Inventing an elaborate narrative just to match Bethlehem would have been a strange choice.
The same applies to the massacre of the innocents and the flight to Egypt, often claimed to be inventions to make Jesus a new Moses.
Yet Matthew again had no shortage of prophetic material, so the flight to Egypt could well preserve a real historical event.

The tradition that Joseph was not the father but only the guardian of Jesus may indicate that the real father died before Jesus was born; otherwise, more straightforward cover stories would have been available.
Herod the Great is known as a paranoid ruler who murdered even members of his own family.
Fleeing to Egypt — the nearest refuge outside Herod’s reach — would have been a sensible action for a threatened dynastic family.
If Jesus spent part of his childhood in Alexandria until Herod’s death, it would also help explain how he became so well-educated, with access to the libraries and intellectual circles of that city from a young age.

\paragraph{5.
Jesus's mother was Mary Christ, the last rightful heiress to the Hasmonean dynasty.}\label{par:jesuss-mother-was-mary-christ-the-last-rightful-heiress-to-the-hasmonean-dynasty.}

Although not the mainstream view, several traditions and sources suggest that Mary was connected to the Hasmonean dynasty.
We see this as a highly plausible theory that explains many otherwise puzzling details in the story of Jesus Christ.

The lineage of Mary as preserved in Luke includes names associated with the Hasmoneans.
The Protoevangelium of James — regarded as reliable by many early church fathers — gives Mary’s father as Joachim.
This agrees with Luke’s genealogy, which names her father as Heli, a shortened form of Eliakim, the same name as Joachim.
The text presents Mary in a biographical, almost royal fashion, describing her perpetual virginity and the miraculous conception of Jesus in terms that parallel traditions about Greek princesses.
Celsus, the second-century Jewish philosopher and sharp critic of Christianity, confirms knowledge of these traditions.

Other dynastic details also point in this direction.
Mary’s brother Simon, remembered as a high priest in the Temple, was executed by Herod the Great in 23 BC.
Her birth name, Miriam, was especially common in the Hasmonean dynasty.
Her birthplace was Sepphoris — a city conquered in 104 BC by Alexander Jannaeus of the Hasmonean line, who even made it his capital.
The rare name Jannaeus also appears in Luke’s genealogy of Mary.

Taken together, these details make sense if Mary was not merely a villager, but the rightful dynastic heiress.
Under this interpretation she may herself have been a “Christ” — the anointed one, perhaps the last to carry the legitimate rule of the Hasmonean house — and the one through whom Jesus inherited his royal claim.

And if both Mary and Joseph were dynastic figures, then perhaps Jesus Christ was indeed “Jesus Christ, son of Joseph and Mary Christ.”

\paragraph{6.
Jesus fleeing to Egypt can be a historical fact as the Hasmonean family had very close ties to Egypt.}\label{par:jesus-fleeing-to-egypt-can-be-a-historical-fact-as-the-hasmonean-family-had-very-close-ties-to-egypt.}

If we consider Alexander Jannaeus as Jesus’s great-great-great-grandfather, we can see further reasons why Jesus would flee to Egypt.
The son of Jannaeus, Aristobulus II, had a daughter Alexandra who married Philippion, a member of the Ptolemaic dynasty.
There were also other family ties between the Hasmoneans and the Ptolemies.
It is therefore very plausible that Mary had relatives in Alexandria who could shelter her family from the wrath of Herod the Great.
Jesus’ flight to Egypt thus fits the pattern of dynastic exile, not of a rustic folk tale.

\paragraph{7.
Jesus's father was killed by Herod the Great}\label{par:jesuss-father-was-killed-by-herod-the-great}

Celsus, a highly educated philosopher, used the name Panthera as one of his main arguments against the divinity of Jesus.
For such a claim to carry weight, he must have drawn on a solid source; otherwise it would have had no impact in debate.
Later rabbinic texts — Shabbat 104b and Tosefta Hullin 2:22 — also contain indirect references to Jesus and Mary (Miriam), implying accusations of adultery.
Celsus’ information was likely drawn from oral court history or family traditions, hostile to Christians but accurate in preserving a dynastic scandal.
That his testimony survived within Origen’s rebuttal shows that even Christian defenders could not simply dismiss it.

If Jesus’ father had a claim to the Herodian throne, while Mary carried Hasmonean blood, their union would have been a direct threat to Herod the Great.
Josephus records that Herod executed several of his own relatives, including his Hasmonean wife Mariamne I and her two sons, as well as his firstborn son Antipater.
The Gospel claim that Herod slaughtered all infants under two is implausible, but the killing of dynastic heirs is exactly what Josephus confirms.

It is significant that Herod’s firstborn son was named Antipater — a name closely related to the form Panthera.
The Greek Ἀντίπατρος (Antipatros) became Antipater in Latinized form.
This name was difficult to render naturally in Hebrew, which lacks the “nt” consonant cluster, and Hebrew commonly shortened or reshaped such names.
Alexander (Alexandros), for example, was often shortened to Sandros or Sendros.
Antiochus was reduced to Yochus or Yuki in Jewish tradition, and Antipas became Pas or Pasi in later Talmudic contractions.
By similar shifts, πατρος could morph into Pantera as the name passed from Greek into Hebrew and back into Greek, with single letters dropped or altered.

Thus, what appears as “Panthera” in Celsus may ultimately preserve the memory of Herod’s Antipater — precisely the kind of dynastic link that would explain both the polemics of Christianity’s critics and Herod’s lethal paranoia.

\paragraph{8.
The Magi scene preserves real Eastern court protocol, not folklore.}\label{par:magi-court-protocol}

The visit of the Magi is often dismissed as myth, even by readers otherwise sympathetic to the Gospels.
Yet when read carefully, Matthew’s account preserves the grammar of Near Eastern statecraft rather than the texture of folklore.

Matthew calls them \textit{μάγοι ἀπὸ ἀνατολῶν} (Magi from the East) and has them say, \textit{εἴδομεν γὰρ αὐτοῦ τὸν ἀστέρα ἐν τῇ ἀνατολῇ}—“we saw his star at its rising” (Matt 2:1–2).
This is technical language: Magi as astronomer–priests, “at its rising” as the heliacal rising of a natal star, and an embassy that performs \textit{προσκυνῆσαι} (royal obeisance) and opens \textit{θησαυρούς} (treasure chests) to present \textit{δῶρα} (state gifts).
The register is diplomatic, not fairy-tale.
Herod’s alarm and his consultation of scribes fits the reality that a foreign court publicly recognizing a rival Davidic heir in his territory was a political act.

The later names Caspar, Melchior, and Balthazar, though not found in Matthew, crystallize real court categories.
Caspar (Gaspar) comes from Hebrew \textit{gizbār}, a loanword from Old Persian via Imperial Aramaic, meaning “treasurer” (cf. Ezra 1:8).
Melchior reflects a title associated with the Zoroastrian concept of the royal light (*khvarenah*), the divine radiance that legitimized kingship—hence “king of light” or “keeper of light.”
Balthazar comes from the Akkadian royal name \textit{Bēl-šar-uṣur}, preserved in the biblical Belshazzar, meaning “Bel, protect the king.”
Taken together, these figures map to the royal treasurer, the priest and keeper of the royal light, and the commander of the royal guard.

The gifts in Matthew align with these roles exactly.
The treasurer brought gold, the symbol of wealth and kingship; the priest of light offered frankincense, the symbol of priesthood and divine worship; and the commander of the guard carried myrrh, the emblem of death and enthronement.
The fit between office and offering is precise: the treasurer brings gold, the priest of light incense, the guard myrrh.
This is court logic, not children’s-pageant symbolism.

The Zoroastrian background is quietly assumed.
The Magi read the sky, the omen marked a birth, the royal light legitimized a king.
In that ideology, the “keeper of light” was a priestly role at court, not a later Christian flourish.
Matthew does not name *khvarenah*; he simply uses the system’s moves: omen → embassy → obeisance → investiture gifts.

Even Matthew’s Greek reads like a dossier.
Key terms are administrative: \textit{ἀνατολή} (rising), \textit{προσκυνέω} (royal obeisance), \textit{θησαυροί} (state treasure), \textit{δῶρα} (formal gifts).
He places the scene in a \textit{οἰκία} (house) with a \textit{παιδίον} (child), not a manger tableau.
It reads as a recognition scene of a dynastic infant.

The symbolism still surfaces in coronation ceremonies.
When a new Pope is crowned, he receives the golden ring and is incensed with frankincense.
When a grandmaster of a military order is chosen, he is invested with myrrh.
The survival of these patterns across millennia is what one expects from state liturgy, not from improvised legend.

What is most striking is that these names preserve authentic Eastern court titles, yet the tradition that transmitted them offers no explanation of their meaning.
Had a Latin author with remarkable knowledge of ancient Eastern tradition in the sixth century invented them, he would almost certainly have drawn out the symbolism—stating plainly that Caspar was a treasurer, Melchior a priest and keeper of light, and Balthazar a commander or guardian.
Instead, the names were passed on in silence, their significance left unarticulated, as though even the transmitters no longer understood them.
Indeed, that significance was never explained for centuries after the names first appeared in the West.
This very lack of commentary is the strongest evidence that the names were not late fabrications but vestiges of genuine diplomatic memory—fragments of a tradition already older than the Church Fathers who repeated it.

It is remarkable Matthew’s Magi story coheres as a compressed report of Eastern diplomatic recognition: astronomer–priests identify a royal birth, an embassy is dispatched, obeisance is rendered, and investiture gifts are presented.
The later names only confirm the pattern: treasurer, priest of light, and guard-commander, each matched with the appropriate gift.
This coherence is what sets the story apart from folklore.

\paragraph{5.5.
The other brothers of Jesus, James, Simon and Judas, also prominently feature in the New Testament}\label{par:the-other-brothers-of-jesus-james-simon-and-judas-also-prominently-feature-in-the-new-testament}

The importance of brothers taking over is a very common characteristic of dynasties, not religious movements.

Sanhedrin 67a -- Mentions a figure called Ben Stada, who was accused of sorcery and brought from Egypt.


\paragraph{8.
Herod dies in 10AD, not 4BC, and the gospels of Matthew and Luke are not in agreement on the date of Jesus's birth.}\label{par:herod-dies-in-10ad-not-4bc-and-the-gospels-of-matthew-and-luke-are-not-in-agreement-on-the-date-of-jesuss-birth.}

There is a small difficulty in dating the Jesus birth to 7BC, common date used in modern scholarship.

There is an apparent contradiction between the gospels of Matthew and Luke, as the gospel of Matthew states that Jesus was born during the reign of Herod the Great, who died in 4BC, while the gospel of Luke states that Jesus was born during the reign of Quirinius, who was governor of Syria from 6AD to 12AD .

Even if we date the gospels very late to late second century, it is still hard to explain how the authors could be so right about so many historical background details, yet so wrong about such a critical part of the story.
Other than birth, Luke would have to be wrong about Jesus age at the time of baptism, which is also a very critical part of the story.
Adding to this, in this book we have a further discrepancy of Caspar, Melchior, and Balthazar journey in 7BC would make no sense as the Greek throne was still held by Strato III and this kingdom was not at the brink of collapse just yet.
Mainstream scholars dismiss contradictions between Matthew and Luke as proof of invention, but the eclipse correction to 12 AD shows both can be harmonized — meaning both traditions may preserve genuine memory rather than fabrication.

It is very overstated what we really know historically about this date.
We know the Quirinius was governor of Syria from 6AD to 12AD .
It is worth emphasizing that Quirinius being a governor of Syria does not imply Judea was annexed the second day he came to power.
It does not violate any historical records to assume Judea was annexed just before he left the office in 12AD .
We know the census happened during his governorship, but we do not know when exactly.
Josephus does not claim the census occurred after the death of Herod the Great.
We know Josephus placed the death of Herod the Great at the lunar eclipse just before the Passover.
Note that he specifically emphasized he died after a long illness, which is consistent with the fact he may have spent a few years with the power shifted to his son Archelaus.

And here already comes a substantial mainstream scholarship error.
The passage from Josephus that is used to date the death of Herod the Great states ``the eclipse of the moon, which happened before his death, took place on the 10th day of the month Lous, which is Nisan.'' What is not admitted is that there were 51 lunar eclipses in the years 7BC to 12AD .
Out of these, there were three that happened at nighttime in Nisan and would have been prominent enough to be seen by the people.
The scholarly consensus is that this passage refers to the lunar eclipse of 4BC, however, upon closer inspection, in 4BC the lunar eclipse happened in Adar, a whole month earlier.

The only possible dates for the lunar eclipse that happened close to 10 Nisan between 4pm UTC and 4am UTC, so could have been seen by the people in Jerusalem is: 0012 Apr 24--22:38:24 10410 -24585 41 N -a 1.4693 0.1335 -0.8101 96.4 - - 11S 64E

Notably the lunar eclipse always happens at full moon and the 15 Nisan is the full moon of the month of Nisan.
However, as the lunar calendars do not align with solar, Jewish calendars introduce a second month of Adar, immediately before Nisan on leap years.
And it so happens that in year 12AD, the extra month of Adar moved the 15th of Nisan a few days later, which explains why Josephus would have said the lunar eclipse happened on the 10th of Nisan.
While on all other possible dates the lunar eclipse happened within a day of the 15th of Nisan.

This readjustment actually solves a lot of problems with the dating of Jesus's birth and the census.
And then the gospels would be in perfect agreement with each other.
Quirinius may have conducted his census in 6AD, Jesus could have been both in 6AD .
There was enough time to spend some time in Jerusalem in the first 2 years of Jesus's life, visiting the temple and being circumcised.
The magi could have come to visit him in 8AD when he was a very young child, right at the time of the fall of Greco-Indian kingdom.
Herod would have the time to order the slaughter of the innocents in 8AD, and Jesus could have fled to Egypt staying there for the next 4 years.
And then settling in Nazareth once Herod Antipas took over the throne of Judea.

With this new dating, there come some apparent contradictions with the records.
Most notably the traditional dating of the reign of Herod Archelaus, who was the son of Herod the Great and was the ruler of Judea from 4BC to 6AD .
Do note that Herod the Great was a tetrarch of Judea and not a king, and the title subsumed Judea proper Galilee Perea Idumea Samaria And other parts adjacent to Nabataea and the Decapolis.

At that time Archelaus was given the title of ethnarch, which was only a king of the Jews.
Josephus also quite clearly states that Archelaus ruled while Herod the Great was still alive.

Heir Territory Title Archelaus Judea, Samaria, Idumea Ethnarch (not King) Herod Antipas Galilee, Perea Tetrarch Philip Gaulanitis, Batanea, Trachonitis, Iturea Tetrarch

It is also important to highlight the tenures of the Roman governors of Judea is also actually tied to the reign of Herod Archelaus and the know dates are not attested from other sources.
So if the reign of Herod Archelaus ended in 12AD, there is still plenty of time to plausibly fit Coponius, Ambivulus, Annius Rufus and Valerius Gratus, the predecessors of Pontius Pilate, into the timeline.



\paragraph{10.
Jesus was very close to and heavily influenced by his cousin John the Baptist, who christened Jesus and declared him a Christ, the rightful ruler of a Greek kingdom.}\label{par:jesus-was-very-close-to-and-heavily-influenced-by-his-cousin-john-the-baptist-who-christened-jesus-and-declared-him-a-christ-the-rightful-ruler-of-a-greek-kingdom.}

John mentions that he was not worthy to untie the sandals of Jesus, which was a common Greek expression for a disciple.
The gospel of John also mentions that Jesus was the lamb of God

\paragraph{11.
John was known to be in close contact with the religious movements in the desert and learned from them extensively.}\label{par:john-was-known-to-be-in-close-contact-with-the-religious-movements-in-the-desert-and-learned-from-them-extensively.}

``I am the voice of one crying out in the wilderness: `Make straight the way of the Lord,' as the prophet Isaiah said.''

\paragraph{12.
Under the influence of his mother Mary, Jesus was still a believer in the Greek way of life and always taught others the teachings of the Greek philosophers.}\label{par:under-the-influence-of-his-mother-mary-jesus-was-still-a-believer-in-the-greek-way-of-life-and-always-taught-others-the-teachings-of-the-greek-philosophers.}

The philosopher king was a very common concept in the Greek world, and Jesus was educated to become a true philosopher king.
He learned from the stoics, the cynics, and the epicureans, and his teachings were heavily influenced by them.
Do onto others as you would have them do onto you, is a very common stoic principle.

\paragraph{13.
He opposed the covenant of old testament and the preachings of the pharisees and the sadducees.}\label{par:he-opposed-the-covenant-of-old-testament-and-the-preachings-of-the-pharisees-and-the-sadducees.}

Jesus said pharisees were very hypocritical.

\paragraph{14.
He saw these groups and the Jews as taking what was rightfully his, the kindom of Judea.}\label{par:he-saw-these-groups-and-the-jews-as-taking-what-was-rightfully-his-the-kindom-of-judea.}

He got very angry and overturned the tables of the money changers in the temple.

\paragraph{15.
The first historical mention of Jesus is in the writings of Mar Bar Serapion, a stoic philosopher, who wrote a letter to his son from prison.}\label{par:the-first-historical-mention-of-jesus-is-in-the-writings-of-mar-bar-serapion-a-stoic-philosopher-who-wrote-a-letter-to-his-son-from-prison.}

The testimony of Serapion of Syria is a very important piece of evidence for the existence of Jesus.
The testimony from 73AD reads: ``What advantage did the Athenians gain from putting Socrates to death?
Famine and plague came upon them as a judgment for their crime.
Or the people of Samos for burning Pythagoras?
In one moment their country was covered by sand.
Or the Jews by murdering their wise king?
After that their kingdom was abolished.''

\paragraph{16.
Testimonium Flavianum - Josephus, a Jewish historian, mentions Jesus in his Antiquities}\label{par:testimonium-flavianum---josephus-a-jewish-historian-mentions-jesus-in-his-antiquities}

Jesper Flavius, a Jewish historian, mentions Jesus in his Antiquities, which was written in 93 AD .
The passage reads: ``At this time there appeared Jesus, a wise man.
For he was a doer of startling deeds, a teacher of people who receive the truth with pleasure.
And he gained a following both among many Jews and among the Greeks.
He was the Christ.'' This passage has been heavily debated by scholars, but the evidence for its authenticity is very strong.
The question is why would a Jewish historian known to have no sympathy for Christianity call Jesus a Christ?
The only plausible explanation is that he was a Christ, a rightful heir to the Greek empire, and not a Jewish Messiah.

This passage poses a very strong conundrum for the scholars who believe that Jesus was a Jewish Messiah and not a Greek Christ.
The typical explanation is that the passage was interpolated by later Christian scribes is not supported by evidence and would require an extraordinary high level of conspiracy.

The conundrum is further deepened by the fact that in another passage that is also considered authentic by the scholars and much less likely to be an interpolation, Josephus mentions that Jesus the Christ was the brother of James, who was the high priest of the temple in Jerusalem.

\paragraph{17.
Cornelius Tacitus, a one of the most prominent Roman historians, mentions Jesus in his Annals, which was written in 116 AD.}\label{par:cornelius-tacitus-a-one-of-the-most-prominent-roman-historians-mentions-jesus-in-his-annals-which-was-written-in-116-ad.}

The passage reads: ``The founder of this name, Christ, was put to death by Pontius Pilate, procurator of Judea in the reign of Tiberius.'' The overwhelming majority of scholars consider this passage to be authentic, and evidence against a possible interpolation is very strong.
This is both because of the style of writing and the difficulty of the passage to be interpolated given the high popularity and importance of his works.
A single church father would have had a very hard time to interpolate such a wide-spead work.
Given how skilled Tacitus was at historical writing it is an extraordinary corroborating evidence for Jesus being a Christ.

It is thus very compelling that the three earliest mentions of Jesus are from a stoic philosopher, a Jewish historian, both of whom had no reason to lie about Jesus's identity and all of whom are know not to believe in Jesus's divinity clearly state that Jesus was a Christ and a king of the Jews.

\paragraph{18.
Pliny the Younger, a Roman governor, mentions Jesus in his letters to the emperor Trajan, which were written in 112 AD.}\label{par:pliny-the-younger-a-roman-governor-mentions-jesus-in-his-letters-to-the-emperor-trajan-which-were-written-in-112-ad.}

The passage reads: ``But they asserted that the sum and substance of their fault or error had been that they were accustomed to meet on a fixed day before dawn and sing responsively a hymn to Christ as to a god.'' The authenticity of this letter is sometimes debated, but regardless of its authenticity, it highlights the fact that what was known about Jesus was that he was a Christ.
So even though this passage adds little to the historical evidence of existence of Jesus, it does add to the evidence that Jesus was a Christ.
Note that once again, it is not Jesus the wise man or Jesus of Nazareth, but Jesus the Christ.

\paragraph{19.
Suetonius, a Roman historian, mentions Jesus in his Life of Claudius, which was written in 121 AD.}\label{par:suetonius-a-roman-historian-mentions-jesus-in-his-life-of-claudius-which-was-written-in-121-ad.}

Suetonius writes briefly about Jesus as an instigator.
This is notable as highlights Jesus was not just a preacher but a political figure.

\paragraph{20.
Lucian of Samosata, a satirist, mentions Jesus in his work The Passing of Peregrinus, which was written in 170 AD.}\label{par:lucian-of-samosata-a-satirist-mentions-jesus-in-his-work-the-passing-of-peregrinus-which-was-written-in-170-ad.}

The part of the passage reads ``and they worship the same crucified sophist and live after his laws.'' In here Jesus is called a sophist, which was a common term for a philosopher in the Greek world.

This passage is much later, but highlights the fact that Jesus was considered a philosopher by the Greeks which would corroborate Jesus's noble Greek lineage.
Again, we are talking about cultural ties here, someone who was closely related to and promoting Greek culture.
The non-Christian sources never call him ‘the prophet’, ‘the rabbi’, or ‘the Messiah’.
They consistently call him Christos or ‘king’.
This unanimity is otherwise inexplicable unless the dominant cultural understanding was dynastic.

\paragraph{21.
Clark Kent argument}\label{par:clark-kent-argument}

There is something to be said about the fact of the wealth of historical records surrounding Jesus Christ and his apostles.
``In total, we have 42 sources dating to less than 150 years after Jesus's death that mention his existence, 9 of which are non-Christian.
In comparison, regarding Julius Caesar, only five sources report his military operations.''

The argument goes that if Jesus was an apocalyptic preacher, or just a wise man, why would there be so many sources mentioning him?
There were no books about Clark Kent, but there are many about Superman.
For this argument we have to consider both independent sources and the sources that are not Christian.

The wealth of the available records can would be hard to explain by Jesus without supernatural powers or not being a very prominent political figure of major significance.
In fact the only non-supernatural plausible explanation may be that Christianity was a loyalist movement attempting to restore the Greek empire.
By contrast, dozens of other would-be prophets and rebels of the era (Theudas, Judas the Galilean, Athronges) leave only scraps in Josephus, not a chorus across Roman, Jewish, and Greek literature. Why Jesus? Only because he was a royal claimant.

\paragraph{20.5.
Silence in Jewish archives}\label{par:silence-in-jewish-archives}

It is striking that no Gospel manuscript or Christian text has been found in the Dead Sea Scrolls or in any Jewish library deposits, despite their preservation of countless minor sectarian writings. This silence suggests Christianity did not originate as a Jewish sect but as a Greek-aligned movement outside Pharisaic or Essene circles.

\paragraph{22.
Ossuary of James}\label{par:ossuary-of-james}

The James Ossuary, which reads ``James son of Joseph, brother of Jesus,'' has long been the subject of authenticity debates.
However, when examined in the context of the Talpiot Tomb, it becomes a strong data point for Jesus having been a historical figure of high lineage --- not a legendary peasant.

Over 1,000 ossuaries have been excavated from the Jerusalem area, dated within a few decades of Jesus's death.
The practice of bone collection into ossuaries was limited primarily to urban, upper-class Jewish families due to cost and ritual precision.
Most ossuaries found have no inscriptions --- only a minority were inscribed, and among those, only prominent individuals typically had names written.

Critics of the Talpiot Tomb theory have claimed that the names found --- Yeshua (Jesus), Yosef (Joseph, father of Jesus), Maria (Mary), Yose (a diminutive of Joseph), Matya (Matthew), Yehuda bar Yeshua (Judas son of Jesus), and Mariamne --- were all common in 1st-century Judea.
And the critics are right to question all published statistical analyses as they are full of statistical and logical errors that critics very rightfully point out.
Almost all the analysis already fail on even asking the right questions to judge the authenticity of the tomb.
The right question to ask is what is the probability that this specific tomb could have existed at that time with these specific names in it and not be related to Jesus.
P(this tomb exists \textbar{} not-Jesus) --- the probability this cluster arises in a non-Jesus family tomb, purely by chance.

Here we will re-do the analysis correctly and using the priors of the theory that Jesus and his family were a prominent family deserving an inscription on their ossuaries.

But this objection fails to consider two key statistical distortions:

Only a small fraction of ossuaries has inscriptions, and the more elite the individual, the more likely the ossuary was inscribed.
This filters the name sample to a specific social class, not the general population.
Only one other known ossuary inscription among over 1,000 ever includes the phrase ``brother of'' --- a highly unusual addition.
The James Ossuary is the only one pairing ``brother of'' with names that correspond precisely to the Jesus of the Gospels.
This feature alone radically shifts the statistical significance: the rarity of ``brother of'' in ossuary inscriptions is what transforms this find from coincidental to highly suggestive.
Furthermore, geochemical analysis has shown that the patina of the James Ossuary matches that of the ossuaries in the Talpiot tomb, suggesting it was originally part of the same tomb.
If so, then the presence of a ``James son of Joseph, brother of Jesus'' ossuary from that family tomb further increases the probability that this is indeed the burial site of the historical Jesus and his immediate family.

This convergence of:

Rare inscription format, Clustered familial names, Matching archaeological context, And elite-class burial indicators \ldots invalidates assumptions of random name coincidence and suggests high plausibility that we are looking at a dynastic burial, not a later legend.

Mariamnou is a fairly uncommon variant of the name Mary but held by Mary Magdalene.

Some experts claim the change of these names with familial connections is 1 in a few hundred.
That is simply very poor math skills of the mainstream scholars that everyone repeats and nobody mathematically oriented bothers to check.
We know the change of Mary is about 25\%, and the change of Joseph is about 10\%, Jesus is about 1.5\%, Jehuda is about 1.5\%, and the change of James is about 1.5\%.

So when we combine the odds of Jesus son of Joseph, brother of James, two Marys with a Mariamnou variant.

Very notably this is a greek variant of the name Mary and that is because as a Herodian court member Mary Magdalene would have been a Greek speaker or at least very heavy hellenized.
Finding that name in a tomb in Jerusalem would likely not be even close to 0.01, but for this analysis we will assume it is 0.01.

The way to do the math correctly is to estimate how many tombs with inscribed ossuaries were there in Jerusalem around the time of Jesus.

We can estimate around 1000 tombs, and generously 200 tombs with inscribed ossuaries.

The Talpiot tomb had 6 inscribed ossuaries, we add James ossuary to it as it was a chemical match.

We have no record of Juda son of Jesus, and Mathew so we treat them as random names, plausibly close family members that we cannot assign with much probability to any particular person in any text or tradition.

The statistical question is what is the probability of finding these 5 names within a tomb of 7 people given the number of tombs being 200.

P(Jesus,Joseph,Jacob,Mary,Mariamnou in a group of 7)=0.015×0.10×0.015×0.25×0.01 * 21, In 200 tombs this gives us 0.00023625

Now from here, we need to account for Jesus and Jacob being sons of Joseph and Jesus brother of James.
Given 3 known male names that are very likely to be brothers or sons of each other, that gives us a factor of 1 in 7.
So interesting the strongest arguments given for the strength of the statistics which is brother-father relations only give a relatively minor boost to the probability.
And even more interestingly, Jesus listed brother of James is thought to be a major factor in the probability of the ossuary being authentic, but actually it does not even change the probability calculation at all.
Both Jesus and James are marked sons of Joseph already, so knowing Jesus and James being brother is already fully accounted for.

But now finally out of 1000 inscribed ossuaries we have only one case of brother of, so having the brother of Jesus is 1 in 1000.

That leaves us with a staggering 3.375e-8 probability.
However, we cannot stop here.
We need to consider the fact we would have also been stunned if we found another combination of 5 names.

So for that Jesus and Mary are pretty much required, but the other three people could plausibly be Joanna, Suzanna, Salome, Martha (other women mentioned in the gospels, could be romantic interests, but also sisters), Simon, Judas (Thaddeus), John the Beloved disciple, Joseph the father of Jesus

Overall we should estimate about up to 10 people filling up the other two names, albeit with some adjustment because Juda Thaddeus and Suzanna are not as closely tied to Jesus as James and Mary Magdalene.
For that we account for about 30 or so plausible combinations of people we would have plausibly identified as Jesus family members.

There is one caveat that Jose in the tomb is Joseph the brother of Jesus, and not Joseph the father of Jesus.
Hence, it is referred to by a slightly different name variant than Joseph the father of Jesus.

Note that although Joseph the father may actually not be in the ossuary in favor od the brother, he still needs to enter the statistics in a similar factor as Joseph the father is clearly inscribed as the father of Jesus and James.

Joseph in the tomb bing the father of Jesus is actually the statistically much more probable case, and so for the authenticity calculation we only need to consider that case (Joseph the father of Jesus makes this one in a million as this accident being possible to occur, Joesph the brother of Jesus makes this less than one in 10 million chance).

And that gives us almost exactly one in a million chance a tomb exited with names and relations we would be leaning to identify as Jesus family.
This is close to 5 sigma which is considered very strong evidence in natural sciences.

Finally, in a bayesian sense to answer if this tomb is authentic this needs to be adjusted with a prior of Jesus and some of his family having a tomb in Jerusalem.
In the scenario of this book, this is a very likely scenario, close to one, but of course if we assign a large probability to Jesus was only a preacher or Jesus really ascending bodily to heaven, then the probability of this tomb being authentic will need to get adjusted by that factor.

\paragraph{23.
δεσπόσυνοι}\label{par:par:ux3b4ux3b5ux3c3ux3c0ux3ccux3c3ux3c5ux3bdux3bfux3b9}

Although it is said Jesus never had a son Yehuda, that is necessarily true.

The blood relatives of Jesus or δεσπόσυνοι are mentioned by Hegesippus, a prominent early Christian historian were brought to the attention of the Roman emperor Domitian.

The people brought forward to the emperor were the descendents of Jude, not Jesus, not James.
So although Jude is not directly identified as a son of Jesus, there is a possibility the Jude discussed by Hegesippus was a son of Jesus, not Jesus's brother.

Note that James, Simon and Jude, sons of Joesph, are mentioned in the gospels as brothers of Jesus, but likely sons of Joseph, from a previous marriage, while Joseph was likely only an adoptive father of Jesus, not a biological father.

Hence, δεσπόσυνοι could likely not refer to the brother of Jude, but instead Jesus's son, who was also named Jude.

Regardless of if Jude was the son or brother or step-brother, Why would the emperor be interested in the blood relatives of Jesus if Jesus did not have a royal lineage?
Domitian’s interest makes no sense if Jesus were a village teacher.
Emperors feared dynasties, not prophets.
The bloodline mattered because it was still viewed as politically dangerous.
