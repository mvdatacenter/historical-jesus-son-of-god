\section{Chapter 3 - The Early Dating of Gospels}\label{sec:chapter-3---the-early-dating-of-gospels}

Acknowledging Jesus as the rightful Christos and his companions as highly powerful and educated people can completely change the way we view the gospels.
Most notably in this framework all arguments for early gospels become much more plausible and the arguments for late dating lose almost all of their plausibility.

Moreover with the dating of the gospels, we can also revisit the dating of the large number of early apocryphal texts.
The apocryphal texts are typically dated to the second century, but with extreme bias of past scholars placing them consistently later than the gospels regardless of the evidence.

\subsection{John Gospel}\label{subsec:john-gospel}

\paragraph{0.
Dating of the Gospel of John}\label{par:dating-of-the-gospel-of-john}

Famously gospel of John is attested the earliest and has the most prolific earliest sources, yet it is dated the latest, purely based on bias coming from invalid assumptions about ``christology''.
This is one of these 19th century assumptions of early biblical scholars that falls apart completely when under any detailed scrutiny, but remains a dogma of the modern biblical scholarship.

By attested, we have a papyrus P52 of John, which is dated to 125AD.
The allegedly oldest Mark, Matthew and Luke are attested by Irenaeus in 180AD together with several other gospels that are deemed by Irenaeus as heretical, and the earliest manuscript of Mark is dated to late 3rd century.
Although the oldest mention of Marcion is from Tertullian, Tertullian himself gives credible attestation of Marcion in 140AD.
So, by current manuscript and reference chronology, the order of attestation is: John first, followed by Marcion, then Mary Magdalene, Thomas, Judas, and only later Mark, Matthew, and Luke.

This alone can be explained by the accident of history, most early sources were lost, and the oldest was John, just by accident.
However, we need to consider this gospel of John was found not alone, but as a part of a large library with many texts, including other gospels, and many documents of all kinds, including dated tax documents.
The Oxyrhynchus papyrus P52, was unequivocally written in a script style that matched the style of dated to around 125AD, but the texts of other gospels in Oxyrhynchus were certainly at least 25-50 years older.

Although the dating can be disputed, on the basis of geography, in Oxyrhynchus, Egypt, we have gospels that were written likely in the same room by the same group of people, so the relative dating of the gospels as they arrived to Oxyrhynchus is rather certain.

Certainly, it could be the case that the synoptic gospels arrived to Oxyrhynchus much later just by accident, however, the same pattern is repeated in the second oldest source of manuscripts, the Nag Hammadi library, where the Gospel of John is also the oldest and the most prolific text, and the synoptic gospels are not present at all.
Nag Hammadi manuscripts are a bit older, but the originals allegedly date to mid 2nd century texts.

The third source fairly securely dated to second century and likely early second century is the Bodmer papyrus, P66 with the Gospel of John.
As in the other cases, the Gospel of John pre-dates the synoptic Gospels in the Bodmer collection by at least 25-50 years.

This is essentially it for Gospel references pre-140AD.
The only possible earlier references to Gospels are Papias, Polycarp and Clement, who plausibly wrote before 140AD.
In case of Papias, he refers to gospels not written in Greek, so they are almost certainly not any of the canonical gospels nor any of the widely known apocryphal gospels.
Clement and Polycarp are quoting Jesus as in the Gospels, but regardless of whether Jesus really did say the words, there is no indication whether they quoted the gospels or the gospels from them or mostl likely they simply quoted the same text or oral tradition.
The next reference is Justin Martyr, and he again is most likely quoting the Gospel of John, notably not any quote of Jesus recorded in the Gospel of John, but a known fragment of John.

Then the later references from Tertullian, Irenaeus, and Origen all contain unambiguous references to all the canonical gospels as already mature and wide-spread texts, but at this point we are already in the year 180AD and later.

To be very clear, this does not indicate John was written in 125AD, nor that Marcion was written in 140AD.
John may have been written in right after the events described, or it may have been written in 100AD.

What that does show is that based on the dates of manuscripts and references, it should be John to have the highest probability of being the earliest gospel, and not Mark, Matthew or Luke.

\paragraph{1.
The Gospel of John is widely accepted to be one gospel that indicates it was written by an eyewitness.}\label{par:the-gospel-of-john-is-widely-accepted-to-be-one-gospel-that-indicates-it-was-written-by-an-eyewitness.}

The gospel itself claims to be written by the disciple whom Jesus loved, which is widely accepted to be John.
Notably all the other gospels themselves had ample opportunity to claim to be written by an eyewitness, but none of them do.
Mark portrays himself as an omniscient narrator, telling a story clearly in third person, does not seem correct about the geography of the area where the events took place.
Luke claims to have compiled the stories from the eyewitnesses, which is a clear admission that he was not an eyewitness himself.
Matthew is clealy copying Mark and a source like Luke, while also writing in very detached third person, not something an eyewitness conceivably would do.

\paragraph{2.
ὁ μαθητὴς ὃν ἠγάπα ὁ Ἰησοῦς}\label{par:ux1f41-ux3bcux3b1ux3b8ux3b7ux3c4ux1f74ux3c2-ux1f43ux3bd-ux1f20ux3b3ux3acux3c0ux3b1-ux1f41-ux1f30ux3b7ux3c3ux3bfux1fe6ux3c2}

The frequent reference to the disciple whom Jesus loved, may also be an indication the disciple was not one of the twelve apostles, as the gospel of John does not mention the names of the apostles at all.

Notably the gospel of John does name the Twelve on multiple occasions, but the disciple whom Jesus loved does not seem to be one of them, while it would have been logical to note that if that were the case.
The disciple whom Jesus loved seems separate from the Peter's group.
Most notably all gospels seem to strongly indicate all male apostles fled after the arrest of Jesus, and the only ones who stayed were women.
``There were also women looking on from a distance\ldots{} Mary Magdalene, Mary the mother of James\ldots{} and many other women\ldots'' This means we are faced with either a very major contradiction between John and all three Synoptics where there was no reason to introduce one, or --- far more plausibly --- the Beloved Disciple was not one of the Twelve, and was a woman, preserved anonymously but recognized within the early community as an authoritative witness.

\paragraph{3.
Γύναι, ἰδοὺ ὁ υἱός σου}\label{par:ux3b3ux3cdux3bdux3b1ux3b9-ux1f30ux3b4ux3bfux1f7a-ux1f41-ux3c5ux1f31ux3ccux3c2-ux3c3ux3bfux3c5}

The famous words of Jesus on the cross --- ``Woman, behold your son'' (John 19:26) --- have traditionally been interpreted as referring to the Beloved Disciple, but the wording itself is deliberately ambiguous.
The Greek phrase υἱός σου means ``your child'' or ``your son,'' but it is not inherently gender-specific in context.
Some scholars, such as Ramon K.
Jusino (1998), have argued that this may point to Mary Magdalene, especially given her presence at the crucifixion and her prominent role in resurrection narratives.

It is also entirely possible that the author is one of the other women from the court of Herod, like Joanna, who was another woman who named in the gospel of Luke as a follower of Jesus, and who was one of three women who visited the tomb of Jesus to anoint his body.
Many scholars link Joanna to the court of Herod, and she was the wife of Chuza, the manager of Herod's household, which would make her a very powerful and influential.
She is also frequently linked to Junia.
As the name Junia is not mentioned in the gospels, yet Paul considers her to be the foremost among all the apostles, the connection to the Jesus's beloved disciple and a biographer of his life seems very plausible.
Finally, this would also explain the authorship of the gospel as John not disputed by the early church fathers, as the author would be a disciple of Jesus, Joanna.

\paragraph{4.
When the authors of the apocryphal gospel of Mary or other early texts try to argue who the beloved disciple is, they argue that it is Mary Magdalene rather than Peter.}\label{par:when-the-authors-of-the-apocryphal-gospel-of-mary-or-other-early-texts-try-to-argue-who-the-beloved-disciple-is-they-argue-that-it-is-mary-magdalene-rather-than-peter.}

No source says it was Mary Magdalene over John, which is what the sources would say the the prevalent view was that John was the most beloved one.

\paragraph{5.
Finally, a woman writer would have been very likely to describe themselves based on the feelings towards others, not many male authors write like this.}\label{par:finally-a-woman-writer-would-have-been-very-likely-to-describe-themselves-based-on-the-feelings-towards-others-not-many-male-authors-write-like-this.}

She says Jesus loved Lazarus, not in a romantic way, but in a way a female writer would describe close friendship.
Jesus is weeping, which is a thing a female would write about Jesus, not a true man.

\paragraph{6.
Jesus interactions towards Nicodemus are also very feminine, close and personal, Jesus is talking about being born again.}\label{par:jesus-interactions-towards-nicodemus-are-also-very-feminine-close-and-personal-jesus-is-talking-about-being-born-again.}

This very likely is not necessarily suggesting Jesus had some homosexual tendencies, but rather that the author of the gospel had some romantic feelings towards Jesus.

\paragraph{7.
The Samaritan Woman (John 4:1-42) -- An Intimate Encounter at the Well}\label{par:the-samaritan-woman-john-41-42-an-intimate-encounter-at-the-well}

Jesus meets her alone at a well, a setting often symbolizing romantic or covenant relationships in the Old Testament (e.g., Isaac and Rebekah, Jacob and Rachel).
The conversation is deeply personal because Jesus reveals he knows her past (five husbands) yet does not condemn her.
He gradually leads her to recognize him as the Messiah, and she becomes a key figure in spreading his message.
The emotional shift---from skepticism to joy---makes this one of the most transformative personal encounters in the Gospels.

\paragraph{8.
The detailed description of the passion seems to add credibility to the claim that the gospel was written by an eyewitness, or at least someone portraying themselves as an eyewitness.}\label{par:the-detailed-description-of-the-passion-seems-to-add-credibility-to-the-claim-that-the-gospel-was-written-by-an-eyewitness-or-at-least-someone-portraying-themselves-as-an-eyewitness.}

The gospels seem to indicate the apostles fled and for a good reason, as they were likely rightfully scared to be accused of supporting Jesus in his alleged crime against the Roman state.
It is far more likely though women would have been allowed to stay and not prosecuted by the Romans, who were from a far more patriarchal society and would be very unlikely to conceive of a woman being a serious threat to the state.

\paragraph{9.
Mary Magdalene (John 20:11-18) -- A Moment of Pure Devotion}\label{par:mary-magdalene-john-2011-18-a-moment-of-pure-devotion}

Mary is weeping alone at the empty tomb, showing deep grief.
Jesus appears, but she doesn't recognize him at first, thinking he's the gardener.
The moment becomes deeply personal when Jesus simply says her name, ``Mary'', and she immediately recognizes him.
She calls him ῥαββωνί (Teacher), showing a deeply personal bond.
He then entrusts her with the first announcement of the Resurrection, making her the first witness of Easter.

If you know one thing about girls is that they have a thing for teachers.
The teacher makes her special by entrusting her with the first announcement of the resurrection.

\paragraph{10.
The Bridegroom Motif}\label{par:the-bridegroom-motif}

John the Baptist calls Jesus the bridegroom (John 3:29).
The first miracle is at a wedding (John 2), which is unusual as an opening scene for a Gospel.
The idea of Jesus as a bridegroom has strong symbolic meaning in Jewish and early Christian traditions, sometimes linking to the idea of a divine marriage (God and Israel, Christ and the Church).
Some Gnostic and esoteric traditions later emphasized Mary Magdalene's connection to Jesus, seeing her as part of this bridal imagery.

Please genuinely check how many male writers like to fantasize about weddings and marriage.

\paragraph{11.
The anointing of Jesus by a woman with expensive perfume (John 12:3, Mark 14:3-9)}\label{par:the-anointing-of-jesus-by-a-woman-with-expensive-perfume-john-123-mark-143-9}

Please genuinely check how many male writers like to provide details on the perfume selection.

Similarly, smell is emphasized when Jesus raises Lazarus from the dead, something that is very unlikely to be emphasized by a male writer.

\paragraph{12.
The gospel of John passes the Bechdel test.}\label{par:the-gospel-of-john-passes-the-bechdel-test.}

The Bechdel test is almost universally determining if a story is written by a woman.

Notably the gospel of Luke also passes the Bechdel test even better, but in that case the author is focusing on many female characters may be due to the sources and sponsors he or she has.
Luke also mentions unusually frequently the words like womb, nursing, mother, birth and much higher female agency including annunciation to Mary and not Joseph.
Whereas the gospel of John seems to be an original text, Luke is very clearly a compilation of many sources as admitted by the author, and overwhelmingly agreed upon by scholars.

\paragraph{12.
Note the women traveling with Jesus and the disciples are also linked to the royal family of Herod.}\label{par:note-the-women-traveling-with-jesus-and-the-disciples-are-also-linked-to-the-royal-family-of-herod.}

``The Twelve were with him, and also some women who had been cured of evil spirits and diseases: Mary (called Magdalene) from whom seven demons had come out; Joanna the wife of Chuza, the manager of Herod's household; Susanna; and many others.
These women were helping to support them out of their own means.'' Mary Magdalene, Joanna, and Susanna are specifically named as financial supporters.
Joanna was connected to Herod's court, meaning she had access to wealth and influence.
The phrase ``out of their own means'' implies they personally funded Jesus' ministry.
Note, much like Sepphoris, Migdala is also within the same township as Nazareth.
If Mary Magdalene was also highly positioned in the Herod's court, then it would make sense that she would have been a skilled writer, yet perhaps not as skilled as the most prominent classics.
This explains the very deep knowledge of royal greek cult, while still maintaining some lack of literary sophistication.

\paragraph{13.
The authorship of Mary is typically dismissed as a possibility because of the use of male pronouns in the gospel.}\label{par:the-authorship-of-mary-is-typically-dismissed-as-a-possibility-because-of-the-use-of-male-pronouns-in-the-gospel.}

This is just skill issues of the scholars.
A fully female woman calling herself a beloved disciple would still use male grammatical forms.
This is because the pronoun is determined by the gender of the word μαθητής (``disciple''), and even if a female version of the μαθήτρια exists, the woman would still typically choose to use the default gender of the word if she does not want to emphasize her gender.
And if that is the case the grammatical forms all over the gospel would still be male, even though they are completely talking about a female person.
This is not just a very obscure feature of the ancient greek language, but also the right way to speak in many modern languages like Polish.
For example in Polish, one would always say ``Jan Ewangelista był kobietą'', John the evangelist was (male) a woman and not ``Jan Ewangelista była kobietą'' John the evangelist was (female) a woman.

\paragraph{14.
The Beloved Disciple is thought to have played an important role in the early Christian community.}\label{par:the-beloved-disciple-is-thought-to-have-played-an-important-role-in-the-early-christian-community.}

Some scholars argue that John, as a male apostle, would have had more influence in the early church, and thus, it seems unlikely that a woman like Mary Magdalene would have been the one to write or serve as a central figure.
The early church's view of Mary Magdalene is much more complex than sometimes assumed.
Though she was initially marginalized, she held an important role in early Christian traditions.
The Gospel of Mary, a non-canonical text, suggests that Mary Magdalene had an influential role, potentially even as a leader among the disciples.
Furthermore, early Christian communities were not as rigid in terms of gender roles as sometimes portrayed, and there is evidence that women held significant positions in the early church (e.g., Priscilla, Phoebe, Thecla).

\paragraph{15.
The Gospel of John emphasizes the Jesus was the only-begotten son of God.}\label{par:the-gospel-of-john-emphasizes-the-jesus-was-the-only-begotten-son-of-god.}

Unique Relationship with God: Jesus is depicted as having a unique and intimate relationship with God the Father, one that is unlike any other.
In John 1:14, 18, and 3:16, Jesus is referred to as the ``only-begotten Son'' (Greek: μονογενής, monogenēs).
This term emphasizes the uniqueness of Jesus as the Son, as no other figure shares this same divine sonship.

\paragraph{16.
Here there is a very interesting perspective linking the gnostic gospel of Thomas with John:}\label{par:here-there-is-a-very-interesting-perspective-linking-the-gnostic-gospel-of-thomas-with-john}

``Jesus said, `If they say to you, \texttt{Where\ did\ you\ come\ from?} say to them, \texttt{We\ came\ from\ the\ light,\ the\ place\ where\ the\ light\ came\ into\ being\ by\ itself,\ established\ itself\ and\ became\ manifest\ through\ their\ image.}If they say to you, \texttt{Who\ are\ you?} say, \texttt{We\ are\ its\ children,\ and\ we\ are\ the\ elect\ of\ the\ living\ Father.}'\,'' There are a few places where Thomas seems to indicate ``we'' are all children of God.
The uniqueness of the sonship seems to be very heavily emphasized in John to the point where it almost seems to become the core concept of the gospel.
Interestingly simultaneously John is making Thomas into the non-conformist and disbeliever.
ch 11.
All the disciples are afraid to go to Jerusalem, but Thomas says, ``Let us also go, that we may die with him.'' ch 14.
Jesus mentioning he will go to his father.
Thomas doubts.
Then Jesus says, ``I am the way, the truth, and the life.
No one comes to the Father except through me.'' You can only come to the father through the only-begotten son.
ch 20.
Thomas recognizes.
You are my Lord and my God.
This shows Thomas despite being Jesus brother or twin, likely holding a meaningful claim himself, John may try to emphasize Thomas could have contested Jesus at first, but then recognized him as the only rightful heir.

14 John 12:31 (NIV): ``Now is the time for judgment on this world; now the prince (archon) of this world will be driven out.'' In this verse, ``archon'' refers to Rome or the prince of this world.
It signifies authority and power over the world, but it's used in a negative sense to describe the ruler of evil.

\paragraph{17.
Temple destruction is typically used as a marker for the dating of the gospels.}\label{par:temple-destruction-is-typically-used-as-a-marker-for-the-dating-of-the-gospels.}

This argument is usually applied to the synoptic gospels, but needs a bit of a different treatment for John.
One note though that in case of John ``Destroy this temple, and I will raise it again in three days.'' Unlike in Olivet discourse, the identity and the prophecy of the destruction of the particular temple in Jerusalem in 70AD is much more questionable.
And that if we do not overemphasize the Mosaic beliefs of the gospel of John, then the temple could be one of the many many many other temples destroyed, not necessarily the main temple in Jerusalem.
Note that the temple in Jerusalem was already destroyed by the Babylonians and then rebuilt and already besieged by romans in 63BC.
It must not have been a very big stretch it could be destroyed again.
Note that Sepphoris was also destroyed by the Roman army during the life of Jesus, and Jesus was heavily involved in the reconstruction of Sepphoris.

\paragraph{18.
The females of the greek period were very empowered but not in roman and jewish culture.}\label{par:the-females-of-the-greek-period-were-very-empowered-but-not-in-roman-and-jewish-culture.}

True---there were multiple female rulers in both the Ptolemaic dynasty and the Hasmonean kingdom.
The Ptolemies had several Cleopatras, Berenices, and Arsinoës ruling independently or as co-rulers.
Even in the Hasmonean period, Salome Alexandra (76--67 BCE) ruled as queen and was one of the most powerful figures of her time.
This definitely weakens the argument that a woman couldn't have written John just because of gender.
If elite women in those circles had power, wealth, and influence, then a female gospel author (like Mary Magdalene or another court-connected woman) wouldn't be impossible.
In John 11:27, Martha says: ``ναί, κύριε· ἐγὼ πεπίστευκα ὅτι σὺ εἶ ὁ Χριστὸς ὁ υἱὸς τοῦ θεοῦ ὁ εἰς τὸν κόσμον ἐρχόμενος.'' She uses ``Χριστός'' (Christos), not ``Μεσσίας'' (Messiah).
This is significant because John does use ``Μεσσίας'' in other places (John 1:41 and John 4:25), but here, Martha is speaking in Greek terminology.
This aligns with John's overall Greek imperial framework, rather than a purely Jewish messianic expectation.

John never uses Μεσσίας (Messiah) alone---he immediately translates it as Χριστός, meaning his audience wouldn't naturally recognize ``Messiah'' as a meaningful title.
John 1:41 -- ``He first finds his own brother Simon and says to him, `We have found the Messiah' (which is translated Christos).'' (εὑρήκαμεν τὸν Μεσσίαν, ὅ ἐστιν μεθερμηνευόμενον Χριστός).
John 4:25 -- ``The woman said to him, `I know that Messiah is coming' (who is called Christos).'' (Οἶδα ὅτι Μεσσίας ἔρχεται, ὁ λεγόμενος Χριστός).
Both times, it appears in direct speech, meaning these are Jewish characters (Andrew and the Samaritan woman) using a Jewish term, not the narrator or author.
This is the case where the greeks tried to convince the Jews to accept Jesus as their rightful ruler as well to additionally bolster his claim and gather more support.

Nathanael's declaration: ``Rabbi, you are the Son of God!
You are the King of Israel!'' This means Nathanael is recognizing that Jesus, who is acknowledged as Christos (a Greek royal title), is not only the ruler of the Greek world (Christos) but also the king of the Jews, in a more localized context.
The title ``King of Israel'' acknowledges the Jewish concept of kingship, which would have been understood as a ruler over Israel, distinct from the broader Greek context of kingship.
Andrew's declaration to Simon: ``We have found the Messiah.'' Andrew is indicating that Jesus, who is referred to as Christos (the Greek royal title), is not just the Greek king (Christos) but also the Messiah, a title that resonates specifically with Jewish expectations of a king.
He's telling Simon that Jesus fulfills both roles---the Greek Christos and the Jewish Messiah---and both Jewish and Greek audiences should recognize him as the ruler of both realms.

\paragraph{19.
In that time, kings and emperors were often called divine, not necessarily because people believed they were literal gods, but because divinity was part of royal language and legitimacy.}\label{par:in-that-time-kings-and-emperors-were-often-called-divine-not-necessarily-because-people-believed-they-were-literal-gods-but-because-divinity-was-part-of-royal-language-and-legitimacy.}

For Mary Magdalene, calling Jesus her ``God'' wouldn't necessarily mean she thought he was some abstract deity but rather that he was her true, anointed ruler---her Christos---just as emperors were called divine.
She saw him as the rightful king, the one destined to rule, and her personal devotion to him would naturally be expressed in the highest possible terms.
So, when she calls him ``Lord'' or when Thomas says ``My Lord and my God'' in John, it could reflect the same kind of royal language used for emperors---deep respect, loyalty, and recognition of his ultimate authority---rather than a fully developed theological doctrine of divinity.

\paragraph{20.
Watching him die would have been unbearable.}\label{par:watching-him-die-would-have-been-unbearable.}

She had placed all her faith, love, and devotion in him---not just as a ruler, but as the center of her world.
When the Romans crucified him, it wasn't just the death of a man; it was the destruction of everything she believed in.
Grief can do powerful things to the mind.
Faced with such a loss, she might have experienced visions, dreams, or an overwhelming sense that he couldn't really be gone.
The idea that he had risen wouldn't have started as a theological claim---it would have been a desperate, emotional response to the unbearable pain of losing him.
At that moment, she wasn't just an eyewitness; she was the one who couldn't let go.

\paragraph{21.
The resurrection story in John isn't just a theological claim---it's a deeply personal, emotional experience that reflects the profound impact Jesus had on her life.}\label{par:the-resurrection-story-in-john-isnt-just-a-theological-claimits-a-deeply-personal-emotional-experience-that-reflects-the-profound-impact-jesus-had-on-her-life.}

This explains why the resurrection account in John is so intimate and personal, focusing on Mary's grief, confusion, and joy.
This explains why it is only Mary who sees Jesus first, and why she doesn't recognize him at first---her vision is clouded by tears and despair.

\paragraph{22.
Description of male nudity}\label{par:description-of-male-nudity}

There is an unusual detail of Simon Peter being naked in the presence of the disciple whom Jesus loved.
This detail seems to have no place in the narrative, and it is not clear why it is included.

Although more sophisticated readings of this passage have been proposed, the most direct explanation stands out.
A woman, emotionally close to Peter or Jesus, remembering this odd, slightly intimate moment: ``He was basically naked, saw Jesus, and scrambled to cover up before jumping in.'' Peter being ``γυμνός'' (naked or underdressed) is a kind of detail a woman would be more likely to remember vividly and include in her account, especially if she was reminiscing the stories of her youth.
Feels less like theology, more like memory.

And then finally the Jesus's resurrection is described by the shroud being left in the tomb with the implication that Jesus was naked when he met Mary Magdalene.
Portraying Jesus as naked meeting a woman would be a very fit conclusion for a female author romance writer, but would make very little sense in any other scenario.

\paragraph{23.
The lack of description of physical appearance of Jesus in John seems to contradict the idea that the author was an eyewitness a woman and a lover of Jesus.}\label{par:the-lack-of-description-of-physical-appearance-of-jesus-in-john-seems-to-contradict-the-idea-that-the-author-was-an-eyewitness-a-woman-and-a-lover-of-jesus.}

Given all the other tropes of female writing, it is something very surprising at first.
One explanation of this is that Philo of Alexandria, from whom the author of John seems to have borrowed a lot of ideas, also actively avoided describing the physical appearance of anyone in his works.
It is possible that Philo conveyed this unusual to us idea to the author of John, and the author of John tried to follow the best practices of their teacher.

\paragraph{24.
John leaves out the institution of the Eucharist.}\label{par:john-leaves-out-the-institution-of-the-eucharist.}

The Last Supper is a key moment in the synoptic gospels, where Jesus breaks bread and shares wine with his disciples, establishing the Eucharist.
In John, this moment is replaced with the foot-washing scene (John 13:1-17), which emphasizes servanthood and humility rather than the sacramental act of communion.

In the context of Jesus being an observer of Greek religious traditions, the Eucharist would have been already a set tradition.
Unlike in Greek feasts, a Jewish passover meal would have been centered around the lamb, and bread and wine rituals would not have been the center of the meal.
Most notably the widespread around that time, Dionysian feasts would have been centered around the wine and bread rituals.
While drinking metaphorical blood and body of Dionysus would have been a widespread practice, the drinking of the blood and body in a Jewish banquet would have been considered a major blasphemy.

Even if we discount synoptic gospels and the letter to Corinthians, the Didache, a very early Christian text, also mentions the Eucharist as a well-established tradition.

If the Eucharist was a well-established tradition, and already a centerpiece of the Christian faith by the late first century, then how can one possibly explain it just slipped John's mind?

Also noteworthy, John describes Jesus talking about flesh and blood in the context of logos, so a modernized version of the Greco-Roman feast.
This is exactly a modern perspective Philo would have embraced, and it is very likely that the author of John was trying to convey the same message.

To note, Justin Martyr and Tertullian both accuse followers of Mythras of copying the Eucharist from the Christians.
This only gives a string indication that the ritual was already prevalent in the pre-Jesus Greek world.

\subsubsection{The Message}\label{par:the-message}

Finally, the traditional view of John being a man.
We have to keep in mind that most scholars agree the very early Christian communities had very strong influence from women.
Figures like Mary Magdalene, Tecla, Priscilla were all at the forefront of the early church.
However, when the gospels were brought to the Roman writers, text had to be interpolated to the modern audience.
Women leaders were erased from history, Mary Magdalene made into a prostitute, and Saint Paul 50 years after his death stared to tell all good Christian women to be silent and let their husbands do the talking.
In that context do we really have any doubt a gospel written by a Joanna or Mary Magdalene would have been attributed to a John?
There is always a concern that John the beloved disciple is not mentioned in the synoptic gospels, but maybe the beloved disciple is actually one of the women mentioned in the synoptic gospels.

Keep in mind that there were many Eastern Roman empresses such as Theodora, Aelia Eudocia, and Zoe Porphyrogenita.
The most famous scholar of the Byzantine empire, Hypatia was a woman, one possibly the only meaningful innovator astronomical and navigation tools for many centuries.

Whereas in the Latin culture the most prominent women had power only by the virtue of indirect influence on their husband or son.

\subsubsection{Miracles in John}\label{par:miracles-in-john}

There are seven miracles in John, which are often called the ``signs'' because they point to Jesus' identity and mission.
It is very worth pointing out that the miracles are not just random show of magic, but all are conveying the royal authority of Jesus.
Also, the miracles may not be completely a metaphor.
Jesus as a ruler very well learned could have been a skilled healer, or perceived as a skilled healer with the power of confirmation bias.
Notably it is not hard to believe that if Jesus visited a sick person and the person got better from just a placebo effect, then the person writing his royal biography would have been very keen to write about it.
Similarly, if Jesus had the resources to provide food for the hungry, and he showed the high generosity of a ruler, then the person writing his royal biography would have been very keen to write about it.

The seven signs in John are as follows: \#\#\#\# 1.
Water into wine (John 2:1-11) The transformation of water into wine signifies the arrival of a new covenant, just as royal feasts marked the legitimacy of kings.
\#\#\#\# 2.
Healing the official's son in Herod's court (John 4:46-54) The healing of the official's son and the paralytic demonstrate his power over life and sickness, which in the ancient world were under divine royal prerogative.
\#\#\#\# 3.
Healing the paralytic (John 5:1-9) \#\#\#\# 4.
Feeding the 5,000 (John 6:1-14) Feeding the 5,000 mirrors the provision of a ruler to his people, akin to divine kingship in Hellenistic and Jewish traditions.
\#\#\#\# 5.
Walking on water (John 6:16-21) Walking on water recalls dominion over the elements, a sign of imperial and divine control in both Greek and Jewish thought.
\#\#\#\# 6.
Healing the man born blind (John 9:1-12) \#\#\#\# 7.
Raising Lazarus from the dead (John 11:1-44) Healing the blind and raising Lazarus emphasize his authority over fate itself---challenging even the most immutable conditions of human existence.

\paragraph{8.
The non-miracle at the death of Jesus.}\label{par:the-non-miracle-at-the-death-of-jesus.}

It is also interesting to note that the gospel of John is the only gospel that does not mention the miracles at the death of Jesus.
As a gospel written by an eyewitness, it is very likely that the author was present at the crucifixion of Jesus, and would have been able to see the miracles that happened at the time.

\paragraph{9.
Conclusion}\label{par:conclusion}

The lack of mention of the miracles at the death of Jesus in the gospel of John is a strong indication that the author was actually attempting to write an accurate account of the events that happened at the time, and not just a theological treatise.
As if you were to introduce a miracle for theological reasons this would be the number one place to do it.
So although the seven miracles are not necessarily completely historically accurate, there is no compelling reason to believe they were not simply slightly exaggerated versions of the truth, the highlight of actual events where Jesus showed his leadership skills, knowledge of medicine, and generosity, just embellished as would have been expected from a royal biography, especially if the author was in love with Jesus.

What is really worth considering is that the events of the seven signs were really reported by the beloved disciple by Jesus as accurate.
For I tell you, let him who never exaggerated their manly deeds to their love interest be the one to cast the first stone.

\paragraph{10.
The Gospel of John point}\label{par:the-gospel-of-john-point}

While we discuss the historical Jesus and historical John, there is one thought that keeps coming back to me.

Nearly every discussion of the Gospel of John is about its theology.
We argue over doctrine and vocabulary and historic accuracy, but maybe we actually miss the single most important thing in the text.

We have lost the author.
We have lost the community that first carried it.
We have lost the Jesus they knew, and the beliefs of the first church.
What survives are copies of copies of copies, shaped by centuries of shifting theology and changing cultures.

But perhaps the best evidence for the true authorship of the Gospel of John is not the in any of the individual hard evidence, but rather the overall message of the Gospel itself.
What if the message of love is so strong, so deep, that it actually became the root of Christianity.
Perhaps when the billions of Christians read the Gospel of John, they may not realize, but they are subconsciously feeling the love that the author had for Jesus.
Perhaps an ancient letter about a lost love has become the most powerful and enduring message of love in human history.
Perhaps it is true that love is the only thing we're capable of perceiving that transcends the dimensions of time and space.

\subsection{Synoptic Gospels}\label{par:synoptic-gospels}

\paragraph{1.
The main argument for the dating of the synoptic gospels to after year 70AD is the destruction of the temple by the Romans.}\label{par:the-main-argument-for-the-dating-of-the-synoptic-gospels-to-after-year-70ad-is-the-destruction-of-the-temple-by-the-romans.}

\paragraph{2.
Most scholars argue the Olivet Discourse in the synoptic gospels is a prophecy of the destruction of the temple.}\label{par:most-scholars-argue-the-olivet-discourse-in-the-synoptic-gospels-is-a-prophecy-of-the-destruction-of-the-temple.}

\paragraph{3.
At the same time the textual analysis of Mark shows that these verses were added later.}\label{par:at-the-same-time-the-textual-analysis-of-mark-shows-that-these-verses-were-added-later.}

which is strikingly more formal and technical than the surrounding Markan language.
In the rest of the Gospel, Mark tends to favor more direct and accessible expressions of Jesus' teachings and prophecies, often focusing on clear, vivid imagery.
The ``abomination of desolation'' is a very specific, technical term that feels more aligned with prophetic, liturgical, or later theological writing.
This suggests it could be an editorial addition, introduced to reflect later theological concerns (like the destruction of the Temple in 70 CE) or to align the passage with more familiar Old Testament imagery.

\paragraph{\texorpdfstring{4.
``But in those days, after that tribulation\ldots{}'' (Mark 13:24)}{4.
``But in those days, after that tribulation\ldots'' (Mark 13:24)}}\label{par:but-in-those-days-after-that-tribulation-mark-1324}

The phrase ``after that tribulation'' introduces a more formal, structured apocalyptic expression that contrasts with the more immediate and direct tone of earlier sections in Mark.
Mark's Gospel typically uses simple, conversational language, but this phrase carries a more stylized apocalyptic tone, characteristic of later prophetic writings.
It introduces a period of cosmic upheaval, which is a prominent feature of later apocalyptic literature (including the Book of Revelation and some Jewish apocalyptic texts), but less common in the earlier parts of the Gospel.

\paragraph{5.
``When you see the desolating sacrilege set up where it ought not to be'' (Mark 13:14)}\label{par:when-you-see-the-desolating-sacrilege-set-up-where-it-ought-not-to-be-mark-1314}

Language Issue: This passage contains complex terminology like ``desolating sacrilege'' and the phrase ``set up where it ought not to be''.
While the ``abomination of desolation'' in earlier prophecies is a well-established term, this variation feels more convoluted and distant from the language of Mark's usual teachings.
It introduces an additional layer of complexity that seems more fitting to later apocalyptic writers, who used this terminology to indicate ritualistic desecration in a manner distinct from Mark's usual straightforward style.

\paragraph{6.
``Heaven and earth will pass away, but my words will not pass away'' (Mark 13:31)}\label{par:heaven-and-earth-will-pass-away-but-my-words-will-not-pass-away-mark-1331}

Language Issue: The eternal nature of Jesus' words is a theme more common in Johannine literature (John 1:1, 6:68, 14:23).
While Mark presents Jesus as authoritative, this statement emphasizes a long-lasting theological development that might be at odds with Mark's more human-centric portrayal of Jesus.
This phrase might also echo themes found in later Christian writings that focus on the eternal validity of Jesus' teachings in a way that is somewhat different from the earthly mission Mark emphasizes.

\paragraph{7.
``But concerning that day or that hour, no one knows, not even the angels in heaven, nor the Son, but only the Father'' (Mark 13:32)}\label{par:but-concerning-that-day-or-that-hour-no-one-knows-not-even-the-angels-in-heaven-nor-the-son-but-only-the-father-mark-1332}

Language Issue: The emphasis on the ignorance of the Son regarding the coming of the Kingdom contrasts with the high Christology that develops in later texts like John, where Jesus has full knowledge and authority.
While it is possible that this reflects Mark's emphasis on Jesus' humanity, the way the phrase is worded feels slightly out of sync with the rest of the Gospel, where Jesus is shown to have a strong authority over time and knowledge of God's plans.
The focus on ignorance is strikingly different from the confident tone found in later Christian reflections on Jesus.

\paragraph{8.
``Let the reader understand'' (Mark 13:14)}\label{par:let-the-reader-understand-mark-1314}

This phrase alone is already an indication of an insert.
This is a phrase a later scribe would typically use to clarify a point or to provide additional context for a passage that might be unclear to the reader.

\paragraph{9.
Now the question is, if Mark 13 and Mark 16 were added to the gospel known to us, then when would the edits have been made?}\label{par:now-the-question-is-if-mark-13-and-mark-16-were-added-to-the-gospel-known-to-us-then-when-would-the-edits-have-been-made}

Well, likely it was quite soon after the event when the event was still fresh in the minds of the people.
As so many events relevant to global politics, Jewish and Christian history happened at the time, the edits were done say decades later, likely different events would have been on top of the minds of the editors to make the edits.
Based on this alone, we could put the upper end for the dating of the original manuscript of Mark to before 70AD.

\paragraph{10.
The other arguments for the lower bound of the dating of the gospels focus on the need to develop the theology of the early church.}\label{par:the-other-arguments-for-the-lower-bound-of-the-dating-of-the-gospels-focus-on-the-need-to-develop-the-theology-of-the-early-church.}

In the context of Jesus being an heir to the greek empire, the gospel of Mark would not require any development of the theology as it is a historical account of the final downfall of the greek empire.
The theological part of it would be the resurrection of Jesus, which was not present in the original gospel of Mark.
As such there is little reason to expect much time to pass between the historical events and the writing of the gospel of Mark.

\paragraph{11 The gospel of Mark starts with the baptism of Jesus by John the Baptist.}\label{par:the-gospel-of-mark-starts-with-the-baptism-of-jesus-by-john-the-baptist.}

The baptism of Jesus is a very important event in the gospels, as it is the event that marks the beginning of Jesus' reign as the rightful king of the greek empire.
Courtly scribes often start at the beginning of the reign of the king, and the baptism of Jesus is likely the event that would have been considered the beginning of his reign.
Traditional scholarship does not provide a good explanation for why the story of Jesus would start with the baptism by John the Baptist, but in the context of Jesus being the rightful heir to the greek empire, it makes perfect sense.
The story of baptism is also portrayed as a very public event, which would have been a very important event for the people to know about.
In all four Gospels, during Jesus' baptism, a voice from heaven declares: ``This is my Son, whom I love; with him, I am well pleased'' (Matthew 3:17, Mark 1:11, Luke 3:22, John 1:34).
This is significant because, in ancient Jewish and Hellenistic traditions, the declaration of someone as a ``son of God'' often had royal connotations, signifying divine legitimacy or appointment to a kingship.
The title ``Son of God'' was used for kings in the Hebrew scriptures (e.g., 2 Samuel 7:14) and in Hellenistic traditions for emperors and rulers.

In the context of ancient Jewish and Hellenistic practices, the act of baptism, particularly with water, was used in various rites of initiation or purification.
However, in the case of Jesus, the combination of the heavenly voice, the dove, and the symbolic actions surrounding the baptism could be seen as a formal coronation---especially given the royal language and imagery associated with it.

Finally, a well known earlier variant of the baptism quote is ``Today I have begotten you'' (Psalm 2:7), This phrase (Psalm 2:7), as used in the Gospel of the Hebrews during the baptism narrative, is not just about divine sonship.
It is a Davidic enthronement formula.
Psalm 2 is a royal psalm, spoken by God to the newly installed king of Israel (originally for Davidic kings).
In Second Temple Judaism, it was widely understood as Messianic, referring to the anointed one (Christos) who would rule the nations.
The Gospel of the Hebrews' use of this verse explicitly identifies baptism as the moment Jesus is declared king---not metaphorically, but in the tradition of Hellenistic and Jewish coronation rituals.

10.3 A very interesting but little known fact is that there exists a common early variant of Luke saying ``You are my Son, today I have begotten you.'' This is a very important variant as it is a direct reference to jesus actually becoming the king or the son of God at the baptism.
Much like the references to temple destruction and additional resurrection stories this variant was likely erased from the gospel of Luke when the understanding changed from more political to a religious one.

\paragraph{12.
Gospel of the Hebrews variants of the baptism story}\label{par:gospel-of-the-hebrews-variants-of-the-baptism-story}

In some fragments, Jesus is portrayed as baptizing others, not merely receiving baptism.

This reverses the standard narrative and places Jesus in the position of the initiator, the one who commands and empowers.
That act reflects imperial privilege---a king is not merely initiated, he is the one who now administers the rites of his reign.
This underscores Jesus as king over a new order, already exercising regal authority.

The Gospel of the Hebrews uniquely calls the Holy Spirit Jesus's ``Mother,'' and she lifts him up by his hair and carries him to Mount Tabor.
This echoes ancient royal investiture scenes, where a deity (or female embodiment of Wisdom) reveals or blesses a new ruler.
Mount Tabor is symbolic --- a ``high place'' often used in imperial and divine appearances in both Jewish and Hellenistic traditions.
The motif of being lifted and carried by the Spirit recalls imperial enthronements where the new king is symbolically raised.

\paragraph{12.
Royal Psalms:}\label{par:royal-psalms}

The Psalms contain numerous references to the king being God's anointed (e.g., Psalm 2:7, Psalm 45:7), and the baptism scene echoes these royal motifs.
By identifying Jesus as the Son of God and anointing him with the Spirit, the Gospel writers could be alluding to these royal traditions, suggesting a coronation-like event rather than merely a religious cleansing.

\paragraph{13 The dove descending on Jesus at his baptism is a symbol of divine approval and empowerment.}\label{par:the-dove-descending-on-jesus-at-his-baptism-is-a-symbol-of-divine-approval-and-empowerment.}

The Dove as a Divine Symbol: In Greek and Hellenistic cultures, the dove was often associated with various divine manifestations, particularly linked to gods like Zeus and Apollo.
A dove descending from heaven was a common symbol of divine approval or empowerment, as in the case of Zeus transforming into a dove in some myths.
In Hellenistic royal ideology, the idea of divine signs, such as a dove appearing at the moment of a king's or leader's coronation, was not unusual.
Dove and the ``Anointing'' of Rulers: Greek rulers and Roman emperors were often seen as divinely chosen, and symbols of divine approval---such as a dove---would serve to emphasize this idea.
The use of a dove in the baptism of Jesus could be interpreted as a sign that Jesus is divinely anointed for kingship, in line with how Hellenistic rulers were often anointed by the gods.

\paragraph{14.
The gospel of Matthew includes the details of the birth of Jesus with the visit of the Magi.}\label{par:the-gospel-of-matthew-includes-the-details-of-the-birth-of-jesus-with-the-visit-of-the-magi.}

The visit of the Magi has been universally accepted as an allegory but as mentioned in the previous chapter, the titles of the Magi are exceedingly hard to fabricate by even the most knowledgeable of scholars in the Roman empire that were not greek empire loyalists.
The Herod's massacre of the innocents is also universally accepted as allegory, but as outlined in the Jesus Dynasty, the massacre of the innocents likely refers to Herod's killing of multiple of his own children and their families.
In the context that Joseph was likely too of the Hasmonean family, giving birth in the city of the Hasmonean dynasty, Bethlehem, is again a lot more likely to be a real event than an allegory.
Similarly, the flight to Egypt is likely a real event as a noble family would have had more than enough resources to flee to and be welcomed in Egypt for a few years.

\paragraph{15.
Finally, the gospel of Matthew includes the great commission, which is likely an edit to the original gospel of Mark.}\label{par:finally-the-gospel-of-matthew-includes-the-great-commission-which-is-likely-an-edit-to-the-original-gospel-of-mark.}

πορευθέντες οὖν μαθητεύσατε πάντα τὰ ἔθνη Make disciples of all nations -- Spread the teachings of Jesus to people all over the world.
Here the word ``nations'' is a translation of the Greek word ``ethne,'' which can also be translated as ``Gentiles'' or ``non Baptize them in the name of the Father, the Son, and the Holy Spirit -- Initiate them into the faith through baptism.
Teach them to obey everything I have commanded you -- Continue teaching and guiding new believers to follow Jesus' teachings.
Jesus promises his presence --''And surely I am with you always, to the very end of the age.'' The great commission is essentially what the emissaries of the greek empire would have been doing, spreading the decrees of the king across the empire.

\paragraph{16.
The term ἔθνη is typically translated as ``nations'' but can also be translated as ``Gentiles'' or ``non-Jews.''}\label{par:the-term-ux1f14ux3b8ux3bdux3b7-is-typically-translated-as-nations-but-can-also-be-translated-as-gentiles-or-non-jews.}

In the context of the greek empire, the term would have been used to refer to all the greek people across the empire, all the nations, kingdoms or tribes that were part of the greek empire.

\paragraph{17.
The gospel of Matthew also includes the genealogy of Jesus, which is widely accepted as not-genuine or lost to history.}\label{par:the-gospel-of-matthew-also-includes-the-genealogy-of-jesus-which-is-widely-accepted-as-not-genuine-or-lost-to-history.}

However, with the acceptance of the theory that Jesus was the grandson of Herod the Great, the genealogy of Jesus through Joseph as his adopted father would require a more thorough examination.
One notable figure from this lineage are Zerubbabel, who was a governor of the Persian province of Yehud, or Judea, and can be dated to around 520BC.
Making best guess estimates to the years of the other figures in the genealogy, the Eleazar fits very well with son Eleazar of Onias I, who was the high priest of the temple in Jerusalem.
From these we can fairly obviously link Matthan to Mattathias Hasmonean, Father of Judas Maccabee, leader of Maccabean Revolt.

\begin{verbatim}
| Name      | Possible Historical Identity | Estimated Lifespan (BC) | Significance                                      |
|-----------|------------------------------|-------------------------|--------------------------------------------------|
| Zerubbabel| Zerubbabel                   | ~520 BC                 | Governor under Persian rule                      |
| Abiud     | Unknown                      | ~480 BC                 | Persian period                                    |
| Eliakim   | Unknown                      | ~440 BC                 | Persian period                                    |
| Azor      | Unknown                      | ~400 BC                 | Late Persian rule                                 |
| Zadok     | Possibly High Priest Zadokite line | ~360 BC           | Hellenization begins                              |
| Achim     | Possibly Onias I             | ~320 BC                 | Early Ptolemaic rule                              |
| Eliud     | Possibly Simon I the Just    | ~280 BC                 | Respected Jewish leader under Ptolemies           |
| Eleazar   | Eleazar, son of Onias I      | ~260–245 BC             | Jewish High Priest                                |
| Matthan   | Mattathias Hasmonean         | ~190–160 BC             | Father of Judas Maccabee, leader of Maccabean Revolt |
| Jacob     | Possibly Alexander Jannaeus | ~120–75 BC       | Expanded Hasmonean territory                      |
| Joseph    | Possibly linked to late Hasmonean elite | ~60 BC–10 AD  | Era of Herodian dominance                         |
| Jesus     | Himself                      | ~4 BC–30/33 AD          | Claimed rightful kingship                         |
\end{verbatim}

Based on the estimated lifespans of the figures in the genealogy, the Jacob mentioned in the genealogy would be Alexander Jannaeus, whose unusual name is also mentioned in the lineage of Mary in the Luke gospel may have been recorded by Matthew as a more familiar sounding Jacob.
Regardless of the exact identities of all figures, there is enough data here to conclude that the genealogy of Jesus in the gospel of Matthew is not a complete fabrication, but a genuine attempt to trace the lineage of Jesus through Joseph.
It is true we may never find the true identity of Abuid, but for the argument to be valid, all we need to accept is that Matthew attempted to trace Jesus lineage though Hasmonean dynasty through some historical figures and focusing on more prominent ones.
There were more figures between Mattathias and Jesus, but the genealogy is by design listing only the most prominent ones to show Jesus descent from the well known powerful kings, and not an attempt to list every single person in the family.

Again, gospels are starting with baptism of Jesus, state Jesus's lineage and show Jesus teaching with parables, and perform miracles and forgave sin.
This is precisely what you expect from royal chronicles of the time.
Start add the lineage of the king, describe coronation, describe the reign when the rules show wisdom and power control over nature and human fate.
Then later version may add omens and prophecies of the king's birth, and the death.
There is absolutely nothing special about gospels here.

\paragraph{18.
Possibly the most prominent element of the gospel of Matthew is the reference to the Jewish scriptures and the fulfillment of the prophecies at seemingly every turn.}\label{par:possibly-the-most-prominent-element-of-the-gospel-of-matthew-is-the-reference-to-the-jewish-scriptures-and-the-fulfillment-of-the-prophecies-at-seemingly-every-turn.}

As such the gospel seems to be with a very high probability written by a prominent Jewish priest, who was also a greek empire loyalist.
There is a theory with substantial textual evidence that the gospel of Matthew was written by the Jewish priest Mattathias ben Theophilus.
That theory is also not generally accepted as it requires a pre-70AD dating of the gospel of Matthew, however, withing the context of discussed in this book, that argument is not valid, and based on other observations made in the framework that theory becomes the one more likely than the traditional dating.

\paragraph{19.
The gospel of Luke also includes a number of edits, most notably the birth narrative of Jesus.}\label{par:the-gospel-of-luke-also-includes-a-number-of-edits-most-notably-the-birth-narrative-of-jesus.}

The birth narrative seems to be full of internal contradictions and is at odds with the gospel of Matthew the events of which have some of the extremely remarkable facts corroborating to the rest of this theory.
The genealogy of Jesus in the gospel of Luke, notably has even more genuine figures from the Hasmonean dynasty down to the father of Mary, Heli, which is a short form of Eliakim, which is the same name as Joachim, which is the father of Mary in the protoevangelium of James.
If we accept that the gospel of Luke was written as a correspondence with the author of the gospel of Matthew, the dating of it would also not be far removed.

\paragraph{20.
The gospel of Marcion deserves another look in this context.}\label{par:the-gospel-of-marcion-deserves-another-look-in-this-context.}

Even though the gospel of Marcion is dated to the 2nd century, the gospel is also remarkably missing the pre-cognition of the destruction of the temple.
Much of modern textual analysis places the gospel not as an edited version of the gospel of Luke, but as an earlier version of the gospel of Luke, due to the simpler language and the lack of the birth narrative of Jesus.

\paragraph{21.
The gospel of Luke describes how Jesus engaged with temple leaders and the Jewish elite when he was a child.}\label{par:the-gospel-of-luke-describes-how-jesus-engaged-with-temple-leaders-and-the-jewish-elite-when-he-was-a-child.}

We are told Joseph and Mary settled in Galilee, but they traveled to Jerusalem for the Passover festival every year.
This is consistent with Mary being a member of the Hasmonean dynasty, as the Hasmoneans were known to have a strong connection to the temple in Jerusalem.
This theory fully explains how Jesus would realistically be able to engage with the temple leaders and the Jewish elite at such a young age.
As per his wise man reputation, it is very likely that Jesus was a child prodigy, and would have been eager and able to engage with the temple leaders about scriptures and the law at a very young age.
Likely if Jesus was a child prodigy, he would have been studying Jewish and Greek scriptures and philosophy in Egypt and in Galilee.
Galilee libraries in Sepphoris would have been precisely well stocked with the works of the Greek philosophers and the Jewish scriptures fully accounting for Jesus's teachings.

\paragraph{22.
Luke is using greek historical writing style, and the gospel is written, not a religious text.}\label{par:luke-is-using-greek-historical-writing-style-and-the-gospel-is-written-not-a-religious-text.}

Ancient Greek Historians' Opening Formula: Luke's opening in Luke 1:1-4 mirrors a well-established convention in Greek historical writing.
The common formula in works by historians like Herodotus and Thucydides begins with a clear intention to provide a factual account and a claim of reliable sources.
The historian often says that he will provide a narrative based on careful research and interviews with eyewitnesses.
Similarly, Luke introduces his gospel by stating that many have already written accounts, and he aims to write ``an orderly account'' based on what he has investigated carefully (Luke 1:3).
This intro not only appeals to the authority of eyewitness testimony but establishes the writer's credibility---a trait highly valued in Greek historiography.

By establishing this historian's formula, Luke is distancing his account from mythical or legendary stories.
This is similar to the way Herodotus sets himself apart from storytellers or poets, aiming for historical accuracy.
His explicit claim of factual investigation is Greek historiographical technique.

He says many have undertaken to draw up an account of the things that have been fulfilled among us, just as they were handed down to us by those who from the first were eyewitnesses and servants of the word.
(note he was a servant of the word)

Emphasis on Historical Accuracy: Luke consistently anchors his narrative with real, known political figures and chronological details.
For example, he refers to the reign of Caesar Augustus (Luke 2:1), the governorship of Quirinius (Luke 2:2), and the rule of Pontius Pilate (Luke 23:1-25).
This is highly reminiscent of Greek historians, especially Thucydides, who grounds his narrative in precise dates and political actors (e.g., the years of the Peloponnesian War).
Public Acts and Decrees: The Roman census (Luke 2:1-3) is a key event that ties the story into the broader imperial political world.
This is a hallmark of Greek historical writing, where historians would use official decrees or military campaigns as turning points in the narrative.
Similarly, Acts uses references to imperial trials and Roman legal processes (e.g., Paul's trial before Gallio, Acts 18:12-17) as a means to legitimize Christianity and explain its spread through the empire.
Roman Emperors and Local Leaders: The mention of Herod the Great in the narrative helps place the events within a concrete political framework.
Like Polybius, who uses the actions of local leaders as windows into the larger workings of an empire, Luke does the same.
He mentions Herod as a regional figure whose reign is significant not just in its own right but as a piece of the greater political puzzle.

Luke's precise use of geographical locations, like Judea, Galilee, and Capernaum, adds to this historical framework, providing context for the actions of Jesus and the early church in a way that closely resembles the method of ancient historians who anchored their stories in known historical settings.

Unified Narrative: Luke and Acts together form a single, continuous narrative.
The structure itself is highly methodical, resembling the way historians like Polybius wrote histories that were divided into books or volumes that followed the chronological development of key events.
Luke's work is divided between the story of Jesus' life and ministry (Luke) and the early history of the Christian church (Acts).
The transition from the life of Jesus to the apostles' missions parallels how ancient historians would follow a ruler's life and then shift to his empire's activities.

Political and Religious Movements: Just as Polybius traces the rise and fall of Rome, or Suetonius traces the lives of emperors, Luke similarly focuses on political and religious figures and their movements through the empire.
His narrative of Paul's missionary journeys (in Acts) mirrors the structure of royal chronicles, detailing the challenges and legal trials faced by a prominent political figure (Paul) on his travels through the empire.

Speech as Political Tool: Luke often uses speeches by Jesus and Paul to express core theological and political ideas.
These speeches resemble the formal rhetoric found in Greek histories, where kings or political figures deliver speeches to clarify their goals and assert their legitimacy.
In Acts, for instance, Paul's speeches before kings (Agrippa, Acts 26) or the Sanhedrin (Acts 23) are presented in a historical context where the figure is asserting his political and religious identity.
Strategic Use of Legal Trials and Defenses: Much like Plutarch or Suetonius, who present key speeches in the trials of emperors or important political figures, Luke uses trials and defenses (such as Paul's before Festus and Agrippa) to emphasize the legitimacy of the Christian message.
These trials are strategically placed in the narrative to reflect not only political struggles but also theological defense---showing that the new movement aligns with or challenges imperial policy in ways that are deeply significant.

The Divine Favor of Jesus and Paul: The portrayal of Jesus' divine mission and Paul's apostolic authority mirrors the way Greek historians often depicted kings or emperors as possessing divine favor.
Luke's consistent depiction of Jesus' miracles and Paul's visions and divine encounters frames the early Christian movement as divinely ordained, much like the way kings and rulers were seen as divinely chosen or supported in Greek histories.

\paragraph{23.
Luke places the lineage of Jesus at the baptism, not at the birth.}\label{par:luke-places-the-lineage-of-jesus-at-the-baptism-not-at-the-birth.}

Scholars argue that Jesus lineage placed after birth narrative is an evidence that the birth narrative was added as a scribal edit.
However, this is not a valid argument, as if the scribe went to the effort of adding major piece of text and blend it well into the narrative and the core idea of the author, how could he have just omitted moving the lineage to the beginning of the text?

The obvious answer is that the lineage was not intended to be at the beginning of the text, and was given at the baptism of Jesus when proclaiming him as the rightful king.

\paragraph{24.
Later gospels and apocryphal texts}\label{par:later-gospels-and-apocryphal-texts}

It is important to note that to judge the significance of the dating of the texts, or really any claims in any historical work, it is important to address all evidence available, not just the evidence supporting the claim.
The emphasis is that while some texts like the gospel of Hebrews, the gospel of Marcion, the protoevangelium of James, gospel of Thomas, are indeed very likely very tied so independent source testimonies, there are a large number of writings, although seemingly expanded narrative, seem to be fully derived from other texts and do not bring any new source.
One example is the very common gospel of Nicodemus or the Acts of Pilate, which, although it adds a lot of human interactions, does not seem to bring up any fresh or contradictory information at all that would not be a natural embellishment or interpolation of the original texts.
Thus, it is very likely only composed based on other known texts and not on an unknown source or a direct testimony of first or second hand witness.
