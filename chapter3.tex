As outlined in the previous chapter, there is overwhelming evidence that Jesus was of royal standing.
Yet if he were only a claimant to the throne of Jerusalem, we would not see the rapid spread of Christianity across the entire Greek empire.
Jesus and his story must have been much more extraordinary than that.

Royal pretenders attract their own nation; they do not instantly win foreigners.
While we know Jesus had a connection to Jerusalem, many try to view his life and the early Christian movement purely through the lens of Second Temple Judaism or Jewish apocalypticism.
But this lens is too narrow.
The movement spoke Greek, wrote Greek, and framed Jesus in the political–religious idiom Greeks already used for rulers.
Second Temple Judaism shows almost no sign of Jesus or his followers — no converts, no manuscripts, no trace in its libraries.
Jesus’ public ministry unfolded in Galilee, a borderland threaded by the Via Maris, looking west to the Phoenician ports of Tyre and Sidon and north and east toward Syria and the Decapolis.
Its urban centers, Sepphoris and Tiberias, were Greek-facing administrative hubs, and village life traded into those markets.
The Gospels themselves place him in that cross-cultural zone (for example, his travel toward Tyre and Sidon and the encounter with the Syrophoenician woman).
And the Gospel of Matthew preserves a tradition of an early sojourn in Egypt, locating his infancy within wider Mediterranean geography.
From the beginning, the horizon of his movement was broader than Jerusalem and its sectarian disputes.

When the earliest proclamation hailed Jesus as Christos, Son of God, Lord, and Savior, Greeks did not hear sectarian jargon; they heard the vocabulary of sovereignty reassigned.
Jews, for whom such claims were theologically charged and politically intolerable, did not follow in large numbers.
Greeks, for whom such claims defined kingship, did.

Christos was not the only title given to him.
He was hailed as Son of God, King of Kings, Lord of Lords, Savior of the World, Light of the World, Prince of Peace, Lamb of God, Good Shepherd, Way, Truth, Life, Alpha, and Omega.
His earliest followers proclaimed that he died for the sins of others so that those who believed in him might not perish but have eternal life.
These phrases are often read as theological inventions, but in fact they were political terms already established in the Greek and Roman world.

In the Hellenistic and Roman order divinity was the grammar of rule.
Alexander the Great claimed descent from Zeus–Ammon, and his image appeared with the horns of Ammon on coins that circulated across the empire.
The Ptolemies in Egypt were styled as “saviors” and “manifest gods,” their names carved in temples and honored in civic cults.
Seleucid monarchs in Syria adopted epithets such as \emph{Epiphanēs} (“God Manifest”) to advertise divine mandate.
Coins, decrees, and festivals constantly reminded subjects that the ruler’s authority was both divine and political.
Rome inherited and deepened this tradition.
Augustus presented himself as \emph{divi filius}, “son of the divine Julius,” and this title appeared on coins and in civic festivals.
Cities hailed emperors as \emph{sōtēr} (“savior”) for providing grain, protection, and peace.
Even the Hasmonean ancestors of Jesus followed the same pattern.
John Hyrcanus was remembered as beloved of God, possessing prophetic gifts and ruling by divine favor.
His sons Aristobulus and Alexander Jannaeus combined priesthood and kingship, and Alexander struck coins inscribed “Jonathan the King” in the same Hellenistic style as the Seleucids.
Josephus describes them with titles such as “savior” and “chosen of God,” categories which in the wider Greek world carried unmistakable divine resonance.
In effect, the Hasmoneans claimed the same political–theological ground as their Hellenistic neighbors: rule itself as a form of divinity.

To call Jesus “Son of God,” “Savior,” “Lord,” or “Light” was therefore to reuse the civic language of kingship, not to invent a new theology.
These were not novel Christian creations but existing titles of political-theological power, instantly recognizable in the empire’s public culture.

On this reading Jesus’ “divinity” flows from his royal status in the imperial idiom.
He was treated as divine because he was asserted as the rightful ruler in a world where rule was by nature divine.
In what follows we take these titles one by one and show how each belonged to Greek and Roman kings before they were given to Jesus, and how that transfer explains both the movement’s speed and its map.

\section{Jewish Silence}\label{sec:jewish-silence}

The first-century Jewish world had multiple active repositories of texts.
Qumran alone preserved over nine hundred manuscripts from ~250 BC to 70 AD .
The Temple precincts in Jerusalem held archives of laws, genealogies, decrees, and sacred writings until 70 AD .
Alexandrian Judaism, represented above all by Philo, produced and preserved extensive Greek exegesis before 50 AD .
Philo also describes the Therapeutae near Lake Mareotis as a community devoted to copying and storing sacred books.
The schools of Hillel and Shammai generated halakhic material that fed later rabbinic compilations.

Across these repositories there is no Gospel and no early Christian treatise.
Not one.
This is striking because conceptually there was much overlap.
Strict monotheism was shared.
Motifs such as the “voice in the wilderness” from Isaiah circulated widely in Jewish thought, appearing at Qumran and in the Gospels’ portrait of John the Baptist.
Apocalyptic writings, messianic hopes, and salvation histories tied to Israel’s past were common across multiple Jewish sects.
In Alexandria, Philo’s doctrine of the Logos sits close to the Johannine Prologue in form and claims.
With this level of shared themes, some recognition—approval, rebuttal, or even polemic—would be expected.
Instead, the Jewish archives are silent.

There was a Jewish nucleus of the Jesus movement at the start.
The Gospels, Acts, and Paul all assume it.
Josephus mentions “James, the brother of Jesus who was called Christ” and records his execution in 62 AD .
But this nucleus left no manuscripts in Jewish repositories.
Rabbinic literature catalogs disputes with Sadducees, Samaritans, and sectarians, yet it preserves no first-century engagement with Christian writings.

Later Jewish-Christian groups such as the Ebionites and Nazarenes appear only in second-century notices, small and marginal.
They represent fragments of that original Jerusalem community rather than evidence of broad Jewish conversion.

The record we actually have is entirely Greek.
Paul’s letters from the 50s AD are addressed to Greek-speaking assemblies in Corinth, Thessalonica, Galatia, Philippi, Rome, and elsewhere.
When synagogues rejected the message, Paul “turned to the Gentiles,” and the texts reflect that reality.
The earliest Christian writings to survive are Greek: Paul’s letters; then the \emph{Didache}, 1 Clement, Ignatius, and the Gospels themselves.
For the first two centuries, every extant Christian manuscript is Greek.
There are no securely dated Hebrew or Aramaic Christian texts.

Put simply: the Christian movement existed from the beginning in a Greek-speaking form, while Judaism preserved no record of it.
The abundance of Jewish texts alongside the complete absence of Christian writings is positive evidence that Christianity’s center lay from the start in Greek urban networks and in claims of rule that Jewish communities neither shared nor transmitted.

\section{Was God the Father the God of Moses or the God of Plato?}\label{sec:was-god-the-father-the-god-of-moses-or-the-god-of-plato}

In modern Christianity it is assumed without question that God the Father is YHWH, the God of Abraham, Isaac, Jacob, and Moses.
Few realize that this identification was not fixed in the first generations of the movement.

Paul himself describes the Father in Greek categories.
Preaching in Athens he quotes the hymn to Zeus: “For we are indeed his offspring” (Acts 17:28).
He continues in the idiom of Greek philosophy: “The God who made the world and everything in it does not live in temples made by man … we ought not to think that the divine being is like gold or silver … in him we live and move and have our being” (Acts 17:24–28).
This is the language of Plato and the Stoics, not of the Mosaic covenant.

The Gospel of John likewise presents Jesus as the \emph{Logos}, the divine Word, the very term used by Greek thinkers for the creative principle.
Philo of Alexandria had already developed a theology of the Logos before the New Testament was written, and his influence is visible in both Johannine and Pauline language.

Later Christian writers expressed the synthesis openly.
Clement of Alexandria wrote: “God is one and the same, the universal Father, being known under many names.”
Even popular piety reflected this blending.
A Christian epitaph from the fifth century prays:
“May the ruler of Olympus’ height
Give rest to these members with the noble sign of the cross,
proclaiming an heir of Christ.”
Here God is invoked both as ruler of Olympus and as the Christian Lord.

By the second century Christianity existed in three large branches: Gnostics, Marcionites, and the community that evolved into the Church of Rome.
That Roman stream remained deliberately ambiguous, able to speak of the Father in both Jewish and Greek terms without settling the question.
The other two branches were explicit: both insisted that the Father was not YHWH.
Gnostics described the God of Moses as a lower or ignorant power, while Marcionites went further, calling him evil.
If Jesus himself had proclaimed YHWH as his Father, such views could not have become dominant currents of Christian belief.
The fact that they did shows that the earliest proclamation of God the Father was received in Greek categories, not tied to the covenant name of Israel.
Only later did the Roman church impose the identification of the Father with YHWH as official doctrine.

\subsection{Pater Noster}\label{subsec:pater-noster}
While we may try to argue which God portrayed in the New Testament could have been later distorted by the authors trying to convert the readers, what Jesus himself taught about the God he called “Father” is much harder to dismiss.

In catechisms and commentaries the Pater Noster is presented as a quintessentially Jewish prayer.
Two Gospel forms survive—Matthew 6:9–13 (Sermon on the Mount) and Luke 11:2–4 (disciple request).
Its usual gloss runs like this: • Our Father in heaven'' echoes synagogue formulas (e.g., the later Kaddish: Exalted and hallowed be His great Name’’).
• Hallowed be thy Name'' = sanctifying YHWH's Name (already holy in Israel).
• Thy kingdom come / Thy will be done on earth as in heaven’’ = Israel’s hope from Daniel and the Prophets.
• Give us today our daily bread'' = manna typology or Psalmic providence.
• Forgive us our debts\ldots{} as we forgive’’ = Jubilee/Leviticus ethic.
• Lead us not into temptation, but deliver us from evil'' = moral temptations; God's protection from sin.
On this reading the prayer is Second-Temple Jewish’’ through and through.
\begin{enumerate}
    \def\labelenumi{\arabic{enumi})}
    \setcounter{enumi}{1}
    \item
    Why that neat picture collapses
\end{enumerate}
Read closely, the prayer pointedly avoids Israel’s covenant markers.
There is no Sinai, no Torah, no Zion, no Abraham, no Sabbath, no sacrifices—none of the themes that anchored Second Temple Jewish piety.
Let's note why the prayer doesn't fit the standard Jewish reading:
• A universal Father in the heavens, not the covenant God ``who brought you out of Egypt.’’
• A sanctified Name without the Tetragrammaton or Temple.
• A Kingdom that descends from heaven to earth (cosmic axis), not the restoration of David on Mount Zion.
• A petition for daily bread does not echo manna (which was not daily) or psalmic providence (which is not daily).
• A trial (Greek peirasmos) and rescue from the Evil One that ring like an eschatological ordeal with a devouring adversary, not like a generic plea about private temptations.
The standard reading ``works’’ only by importing background the text doesn’t supply, while ignoring the imagery it does supply.

What is fascinating is that when read in the light of Egyptian solar-royal liturgy, every clause falls into place without strain.
\begin{enumerate}
    \def\labelenumi{\arabic{enumi})}
    \setcounter{enumi}{2}
    \item
    The Egyptian solar–royal reading (what fits cleanly, line by line)
\end{enumerate}
\begin{enumerate}
    \def\labelenumi{(\alph{enumi})}
    \item
    Our Father in the heavens'' Egyptian hymns to Aten and Amun-Ra address the high god as father of all; the Pharaoh is the son of the Sun. The address is cosmic, not ethnic. It's the right register for the prayer.
    \item Hallowed be thy Name’’ (ἁγιασθήτω τὸ ὄνομά σου) Egyptian piety centers on the Name (rèn).
    Amun literally means the Hidden (One)''; his hidden Name is praised and protected. Refrains like Your name is Amun—Amun, Amun’’ are liturgical.
    This is precisely a sanctified Name without pronouncing it—a far tighter fit than Moses’ tetragrammaton practices as usually described for lay prayer.
    \item
    Thy kingdom come. Thy will be done on earth as in heaven.'' Every dawn the Sun restores Maʿat (order) in the heavens and, through the king, on earth. Solar kingship is a heaven-to-earth pipeline of will/order. That is exactly the structure of this petition.
    \item Give us today our daily bread’’ (τὸν ἄρτον\ldots{} τὸν ἐπιούσιον) The Great Hymn to Aten praises the god who daily makes bread for humankind.'' Egyptian offering formulas (bread and beer, daily’’) are standard temple language.
    This line is almost a quotation in sense.
    \item
    Forgive us our debts as we forgive our debtors.'' Egypt frames justice as weight at judgment---the heart weighed against the feather of Maʿat. Being set right’’ (absolved of moral weight/debt) is the difference between survival and obliteration.
    The ethical turn (as we forgive'') binds the worshiper to enact Maʿat socially. This is far closer to Egyptian moral weight/debt than to later juridical hair-splitting.
    \item Do not bring us into the time of trial, but deliver us from the Evil One.’’ (μὴ εἰσενέγκῃς\ldots{} εἰς πειρασμόν\ldots{} ῥῦσαι ἀπὸ τοῦ πονηροῦ) Peirasmos = trial/ordeal, not chiefly temptation.'' Read apocalyptically, it is the great ordeal; read visually, it is the judgment scene. And the Evil One is not an abstraction: in Egyptian iconography the failed soul is devoured by Ammit (crocodile-lion-hippo) as the scales tip. Deliver us from the devourer’’ is exactly how the scene works.
    \item
    For thine is the kingdom, and the power, and the glory, for ever and ever.'' This is the quintessential royal acclamation, the kind of formula shouted in honor of Pharaohs, Ptolemies, and emperors. In temple liturgy and civic festivals, crowds proclaimed the ruler’s dominion, power, and radiant glory as everlasting—language mirrored on Hellenistic and Roman inscriptions. Appending this doxology plants the prayer squarely in royal–solar ideology, where kingship, might, and glory are affirmed without limit of time.
    \item
    Amen.'' Before we pretend ``amen’’ is safely, uniquely Hebrew, note the liturgical practice: Egyptian hymns and responses end with acclamations of Amun; congregational call-and-response reinforces the Name. In both Hebrew and Egyptian scripts, the word was originally written only with the consonants *MN* (Hebrew: אמן; Egyptian: jmn), without vowels. In Hebrew this yields “amen” (firm/true); in Egyptian, “Amun” (the Hidden One). The phonetic overlap is not accidental but reflects a shared acclamatory base. Alexander the Great claimed descent from Zeus-Ammon; Ptolemaic kings styled themselves similarly. To say “Amen” thus seals the prayer in the register of Amun’s hidden Name—the god underwriting royal legitimacy. Functionally and phonetically, “All say Amun/Amen!” is the logic.
\end{enumerate}

Bottom line: every clause fits solar-royal liturgy without strain.
While one can argue that perhaps Jesus or Matthew simply borrowed a few phrases from Egyptian hymns unknowingly, there is a smoking gun this is not a coincidence.
Matthew frames almsgiving, prayer, and fasting with the same refrain: "your Father who sees in secret will reward you" (6:4, 6:6, 6:18).
This is not incidental --- it is the precise epithet of Amun, "the Hidden One" and precise the way Amun's theology works.
The Lord's Prayer immediately follows these three sayings, so the prayer is introduced with the description of God as the hidden Father.
Thus the prayer begins by addressing the hidden Father in heaven, and ends by sealing it with the acclamation of Amun/Amen --- a perfectly consistent solar-royal liturgy.

The ``Jewish only’’ reading must paper over the prayer’s cosmic grammar; the Egyptian reading doesn’t.

The visual counterpart to this liturgy is the radiant halo—the sunburst crown seen on Alexander, the Ptolemies, and later Roman emperors—imagery that early Christian art freely transferred to Jesus as Son of the cosmic Father.
\begin{enumerate}
    \def\labelenumi{\arabic{enumi})}
    \setcounter{enumi}{3}
    \item
    The historical pipeline (why this imagery would still be alive)
\end{enumerate}
This isn’t Bronze-Age dust accidentally stuck to a 1st-century text.
It’s continuous culture: • Egypt in Canaan (c.~1500–1150 BC).
For centuries the southern Levant was an Egyptian province.
Jerusalem appears in the Amarna archive (\textasciitilde1350 BC), with its ruler Abdi-Heba writing Pharaoh as my Sun.'' Egyptian garrisons and cult stood at Beth-Shean, Jaffa, Deir el-Balah, etc. • Davidic psalmic core (c.~1150--970 BC). The linguistically older psalms are drenched in sun, light, heaven, earth, kingship, divine rule (Ps 19; 29; 68; 84; 104). These read like Hebrew adaptations of solar hymns, not Torah homilies. • Aten's revolution and Amun-Ra's supremacy. Akhenaten's Aten-monotheism collapses, but Amun-Ra returns stronger; the solar-monotheist pressure never disappears. • Ptolemaic Egypt (3rd--1st c.~BC). The dynasty crafts Serapis/Isis cult and keeps solar divine kingship explicit. Cleopatra's death (30 BC) is within grandparent memory of Jesus' generation. • Jesus' milieu. A king of the Jews’’ claim sits inside Roman Syria-Palestine, saturated with Helios/Sol imagery.
Early Christian art happily paints Christ as Helios; the holy day is Sunday.
The solar-royal idiom is not alien—it is the water everyone swims in.
Seen through this pipeline, the Pater Noster doesn’t borrow a few Egyptian phrases; it belongs to the solar-royal register that ran from Aten → Amun-Ra → Ptolemaic kingship straight into the first century. In cosmopolitan hubs like Alexandria and Antioch, prayers themselves circulated and fused idioms; the Lord’s Prayer fits this world of shared solar-royal language intelligible across traditions.

\begin{enumerate}
    \def\labelenumi{\arabic{enumi})}
    \setcounter{enumi}{4}
    \item
    ``But isn’t it still Jewish?’’ — the honest reconciliation
\end{enumerate}
Yes, the prayer can be prayed in a Jewish key (and was).
Luke embeds it in a lesson on dependence; Matthew frames it within piety and forgiveness.
The wording genuinely overlaps later synagogue language (``hallowed be His Name’’).
But that overlap proves adaptability, not origin.
Crucially, the prayer: • avoids covenant particulars, • speaks in universal solar-cosmic grammar, and • lands perfectly inside Egyptian/Ptolemaic royal theology.
The better model is fusion: Israel’s high-God devotion absorbed, translated, and reused the dominant solar-royal grammar everyone understood.
If Jesus stands—as our thesis argues—as a royal claimant in the wake of a just-collapsed Ptolemaic world, then this prayer reads not as a synagogue formula but as a dynastic solar-royal prayer: heavenly Father (Sun), kingdom descending, daily sustenance, righteous scales, rescue from the devourer—Amen.

\begin{enumerate}
    \def\labelenumi{\arabic{enumi})}
    \setcounter{enumi}{5}
    \item
    Clause-by-clause gloss (for the reader who wants the map) • Father in heaven → Solar source (Aten/Amun-Ra) and royal sonship.
    • Hallowed Name → the hidden Name exalted (Amun’s rèn).
    • Kingdom come / will be done → Maʿat restored from heaven to earth via the king.
    • Daily bread → the sun-god who daily provides bread.
    • Forgive debts → lighten the moral weight at the scales; enact Maʿat with others.
    • Do not bring into the trial → spare us the ordeal/judgment.
    • Deliver from the Evil One → rescue from the devourer who consumes the failed.
    • Amen → the communal seal, functionally identical to Amun-acclamation.
\end{enumerate}
Why this matters for David and Jesus • David (1150–970 BC) sits close enough to the Amarna horizon for Egyptian solar kingship to be living memory; the oldest psalms sound like it because they grew in it.
• Jesus (early 1st c.~AD) stands within living memory of Ptolemaic solar monarchy.
If he is (as our book argues) a royal figure inside that political theology, the Pater Noster is exactly the kind of solar-royal prayer a claimant would teach: it translates Egypt’s oldest grammar of kingship into a form his followers can pray anywhere.
That is the reading that explains everything the text actually says—without importing Sinai—and explains why the prayer crossed languages and empires so effortlessly.

\paragraph{Sunday.} Finally we should also look at the Lord’s Day, Sunday, the day of the Sun.
Romans in the early empire still followed the eight-day cycle marked A to H, with market day every eighth day.
Jews kept a seven-day week but did not use planetary names, numbering the days from the Sabbath.
The seven-day planetary week, created in Ptolemaic Alexandria, was the framework Christians adopted.
Philo of Alexandria reflects the Jewish numbered week, while contemporary Egyptian calendars already used the planetary scheme.
And it was Christians who carried this seven-day week into the Roman world, with Sun and Lord identified together — not the other way around.

\paragraph{Our Father in Heaven.}
People have always recognized the Sun gives them everything they need: heat, light, food, flowers in the park, reflections on the lake.
The Sun has been recognized as the father of life in ancient and modern times.
The Sun as the father of all people or Sun as the father of the ruling dynasty motif can be found in many contexts and religions.
"We are all star stuff", the quote from Carl Sagan, repeated by modern scientists, is a modern way of expressing we are children of the Sun.
Amun-Ra's main adjective was “it nṯrw," the father of the gods, with many prayers calling him "the one who makes mankind, who formed all that lives, father of the fathers of all the gods."
The hymn to Aten calls the Sun "the father of all that he made."
In the Inca Empire, Inti was considered both the Sun God and the ancestor of all Incas.
Similarly, in the Solar dynasty of India, kings traced their lineage to Surya, the Sun God.
In the Japanese Shinto religion, Amaterasu is the Sun Goddess and the mythical ancestress of the Japanese imperial family.
In Greece, Zeus, the father of gods and mankind, was extensively called our Father.
While not strictly God of the Sun, Zeus representing the heaven, brings in the father motif for exactly the same reason.

\paragraph{Halo.}
In this context we should not forget that throughout the history of Jesus, Mary and all saints have been portrayed with a halo, a circle of light around their heads, symbolizing their holiness.
But what the halo very overtly and unmistakenly represents is the disk of the Sun.
This was already the imagery of Ra, a figure with a solar disk above his head.

The divine and holy figures in the Indian subcontinent, where the halo symbol was equally or even more ancient than in Egypt.
The prabhāmaṇḍala has been even more prominent and identical to the one used in Christianity, symbolizing the divine light, holiness, and spiritual power of the figure portrayed, but has also always been unmistakenly identified as the radiance of the Sun.

The pharaohs until even the time of Jesus were portrayed with a solar disk above their heads as well.
Even though most present-day depictions of Cleopatra do not show her with a solar disk, her multiple contemporary Egyptian depictions show it very prominently.

The solar disk to denote the Son of God executing the will of the Father did not even disappear with the rise of Christianity, as it was used in the depictions of Constantine the Great, Justinian I and even Charlemagne.
While it is highly unlikely portrait makers in the Middle Ages consciously thought of the solar disk, it shows how deeply rooted the imagery of the solar disk was in the Christian consciousness.

None of these figures were Saints, but all were portrayed with a halo — a solar disc representation of God the Father, symbolizing the divine authority standing behind their earthly rule.

\subsection{Jesus Christ referring to Our Father in Heaven and YHWH}\label{subsec:jesus-christ-referring-to-our-father-in-heaven-and-yhwh}
Jesus’ speech consistently distinguishes \emph{the Father in the heavens} (whose essence is mercy, generosity, forgiveness) from the \emph{Kyrios} invoked in Israel’s scriptures and civic religion. The Gospels preserve this distinction at three levels: (1) Jesus’ own address to God as “my Father” and the teaching of “our Father”; (2) his polemics with Judeans who claim God as their Father; (3) his quoting of Scripture where the Tetragrammaton (YHWH) appears as \emph{Kyrios} in Greek, alongside his reassignment of \emph{Kyrios} as a royal title to himself.

\paragraph{A. The Father in heaven: the ethic of universal mercy (quotes).}
\begin{itemize}
    \item \textbf{Love for enemies as filial likeness:} “Love your enemies and pray for those who persecute you, \emph{so that you may be sons of your Father who is in heaven}; for \emph{he makes his sun rise on the evil and on the good}, and sends rain on the just and on the unjust.” (Matt 5:44–45).
    \item \textbf{Be merciful as the Father:} “He is kind to the ungrateful and the evil. \emph{Be merciful, even as your Father is merciful.}” (Luke 6:35–36).
    \item \textbf{Perfection = indiscriminate goodness:} “You shall be perfect, as \emph{your heavenly Father} is perfect.” (Matt 5:48).
    \item \textbf{Forgiveness as the Father’s policy:} “If you forgive others their trespasses, \emph{your heavenly Father} will also forgive you…” (Matt 6:14–15).
    \item \textbf{The Hidden Father:} “When you give to the needy… so that your giving may be in secret; and \emph{your Father who sees in secret will reward you}.” (Matt 6:3–4). Likewise, “pray to your Father who is in secret; and \emph{your Father who sees in secret will reward you}.” (Matt 6:6). Here the Father is literally called “hidden” (ἐν τῷ κρυπτῷ), resonating with Amun’s epithet “the Hidden One.”
    \item \textbf{Provision without partiality:} “\emph{Your heavenly Father} knows that you need them all.” (Matt 6:32); “How much more will \emph{your Father who is in heaven} give good things to those who ask him!” (Matt 7:11).
    \item \textbf{Jesus’ own address and mission:} “\emph{Abba, Father}, all things are possible for you…” (Mark 14:36). “God did not send the Son into the world to condemn the world, but… to save the world.” (John 3:17); “I did not come to judge the world but to save the world.” (John 12:47).
    \item \textbf{The cross as intercession, not retribution:} “\emph{Father, forgive them}…” (Luke 23:34).
\end{itemize}

\paragraph{B. Jesus’ polemic: denial that Israel’s claim (“God is our Father”) stands as-is.}
\begin{quote}
    “They said to him, ‘We have one Father—God himself.’ Jesus said to them, ‘\emph{If God were your Father, you would love me}… \emph{You are of your father the devil}.’” (John 8:41–44).
\end{quote}

\paragraph{C. Where \emph{Kyrios} appears: Scripture citation and royal transfer.}
\begin{enumerate}
    \item \textbf{Scripture citation (YHWH → Kyrios in Greek):} Deut 8:3 // Matt 4:4; Deut 6:16 // Matt 4:7; Deut 6:13 // Matt 4:10; Mark 12:29 (Shema). Jesus quotes Israel’s Scriptures \emph{without} adopting the Tetragrammaton, which appears as \emph{Kyrios} in Greek.
    \item \textbf{Royal transfer of \emph{Kyrios}:} “The Son of Man is \emph{Lord of the Sabbath}” (Matt 12:8). “Not everyone who says to me, \emph{‘Lord, Lord,’}… but the one who does the will of \emph{my Father} who is in heaven.” (Matt 7:21).
    \item \textbf{Two ‘Lords’ text (Ps 110:1):} “\emph{The Lord said to my Lord}…” (Mark 12:36, \emph{Eipen Kyrios tō Kyriō mou}).
\end{enumerate}

\paragraph{D. The Name and the Amen.}
\begin{itemize}
    \item \textbf{Name:} “\emph{Holy Father}, keep them \emph{in your name}…” (John 17:11); “I have \emph{manifested your name}…” (John 17:6). This pairs with “Hallowed be your Name” and Egyptian \emph{r\`en} piety (hidden Name/Amun), without pronouncing YHWH.
    \item \textbf{Amen as oath-formula:} Jesus uniquely fronts sayings with “\emph{Amen, amen}, I say to you …” (esp. in John). This turns the congregational seal (*MN* → Amen/Amun) into his signature of authority, aligning the Father’s truth/hiddenness with the Son’s proclamation. A striking parallel appears in the Greek Magical Papyri from Roman Egypt (PGM IV, 1st–3rd c.~AD): “ἀληθῶς ἀληθῶς, κύριε, εἰσάκουσόν μου” — “Truly, truly, Lord, hear me.” Here a solar hymn addresses the high god (Helios/Serapis) with the same doubled truth-formula. The correspondence is chronological: these invocations belong to the same world as Jesus, not to distant antiquity. Where Egyptian hymns cry “Truly, truly, Lord, hear,” Jesus claims “Amen, amen, I say to you.” The liturgical structure is the same, but Jesus moves from invocation of the Lord to speaking as the Lord, transferring the solar-hymnic formula into his own mouth as signature of divine authority.
    \item \end{itemize}

\paragraph{E. Synthesis.}
(1) \emph{Father}-language clusters around mercy, indiscriminate benevolence, forgiveness—linked to solar provision (Matt 5:45). (2) \emph{Kyrios}-language appears in Scripture citation and royal prerogatives. (3) Jesus denies that the bare claim “God is our Father” stands without recognition of the Son (John 8). (4) The Lord’s Prayer centers the Father’s Name, Kingdom, Bread, Forgiveness, Rescue—the cosmic-royal grammar traced earlier. The visual corollary is the radiant halo: the Father’s light embodied in the royal Son.

\subsubsection{Acts and the Epistles: crystallizing the Father vs. Kyrios distinction}\label{subsubsec:acts-epistles-father-kyrios}
\begin{itemize}
    \item \textbf{Acts (cosmic Father; Kyrios to Jesus):} Acts 2:36 — “God has made him both \emph{Kyrios} and Christ, this Jesus whom you crucified.” Acts 17:28–29 (Areopagus) — “For we are indeed his offspring… we ought not to think that the divine being is like gold or silver or stone.” Paul quotes Greek poets (Aratus/Cleanthes) to assert universal Fatherhood, not a tribal deity.
    \item \textbf{Paul’s creedal split:} 1 Cor 8:5–6 — “For us there is \emph{one God, the Father}, from whom are all things… and \emph{one Lord, Jesus Christ}, through whom are all things…” Philippians 2:11 — “Every tongue confess that Jesus Christ is \emph{Kyrios}, to the glory of God the Father.” Distinct roles: Father = source; Jesus = \emph{Kyrios}.
    \item \textbf{Abba and adoption beyond Torah:} Gal 4:6 — “God has sent the Spirit of his Son into our hearts, crying, ‘Abba! Father!’” Romans 3:29 — “Is God the God of Jews only? Is he not the God of Gentiles also? Yes, of Gentiles also.”
    \item \textbf{Johannine hiddenness and love:} 1 John 4:8 — “God is love.” 1 John 4:12 — “No one has ever seen God.” The unseen/hidden Father coheres with Matt 6’s “who sees in secret,” while Jesus (the \emph{Kyrios}) is manifest.
\end{itemize}
\paragraph{Conclusion.} Acts and the letters do not collapse Father into YHWH; they formalize the lived Gospel distinction: \emph{one God, the Father} (universal, often hidden, source of mercy) and \emph{one Lord, Jesus Christ} (royal \emph{Kyrios}), with solar-royal imagery naturally migrating to the Son as the Father’s visible regency.
\subsection{Jesus Christ believed in true immortal soul}\label{subsec:jesus-christ-believed-in-true-immortal-soul}
Sometimes overlooked, the idea of soul in greek and jewish beliefs is very different.
Do not fear those who kill the body but cannot kill the soul'' ψυχή (psyche) ≠ body (Matt 10:28) What does it profit a man to gain the world and forfeit his soul?’’ (Matt 16:26) ``Today you will be with me in Paradise’’ (Luke 23:43)
The belief in post mortem conscious existence as a continuation of soul is not a belief of the Old Testament, but it is a common belief in the Greek world.
The Jews of that time believed in a resurrection of the body, but not in the immortality of the soul.
Within Judaism itself the debate divided Pharisees (affirming resurrection and often an ongoing soul-life) and Sadducees (denying it); Jesus’ teaching aligns more with the Hellenized Pharisaic/Greco-philosophical view than with Torah-literalist denial.

\subsection{Jesus Opposition to the Pharisees and Sadducees}\label{subsec:jesus-opposition-to-the-pharisees-and-sadducees}
He opposed the covenant of old testament and the preachings of the pharisees and the sadducees.
The only story of Jesus from his childhood is when he was 12 years old and he was found in the temple discussing with the teachers of the law.
He saw these groups and the Jews as taking what was rightfully his, the kindom of Judea.
Jesus said pharisees were very hypocritical.
He got very angry and overturned the tables of the money changers in the temple.


\paragraph{10.
Jesus was very close to and heavily influenced by his cousin John the Baptist, who christened Jesus and declared him a Christ, the rightful ruler of a Greek kingdom.}\label{par:jesus-was-very-close-to-and-heavily-influenced-by-his-cousin-john-the-baptist-who-christened-jesus-and-declared-him-a-christ-the-rightful-ruler-of-a-greek-kingdom.}

John mentions that he was not worthy to untie the sandals of Jesus, which was a common Greek expression for a disciple.

\section{Titles of Jesus}\label{sec:titles-of-jesus}

The titles of Jesus are deliberate, not accidental.
Some titles are more uniquely Jewish, like Messiah and Emmanuel.
Scholars dissect every echo of Jewish scripture in the gospels, but many obvious other references are ignored.
Titles like Son of Man or Good Shepherd come from Jewish scripture but also appear in other Near Eastern texts.
Many other ones are Greek or Egyptian---Soter, Epiphanes, Logos, Son of God.
These were already the language of kings, stamped on coins, decrees, and civic cults.
To call Jesus by these names was to claim him as Christos of all nations, including Israel, Greece, and Egypt.
The gospels did not invent the terms; they used the royal vocabulary already in place and fixed it on one man.

\paragraph{Jesus is the Christ.}\label{par:jesus-is-the-christ.}
In the gospels and all the earliest materials we can see by far the most common title given to Jesus is Christ.
Overwhelmingly in the early sources the title is \emph{Christ} (Χριστός), not Messiah (Μεσσίας).
In the very few cases the word Messiah is used, the gospels clearly state that Messiah is the Hebrew translation of the word Christ.
Christ is the title used by Josephus, Christ is used by Pliny the Younger, Christ is used by Tacitus, Christ is used by Suetonius, Christ is used by Paul.
Christ is the only term used by Jesus own highly educated, greek speaking brother James.
That strongly contradicts the common narrative that Jesus was originally called the Messiah and that was later translated into Christ.
The gospels just use the term Messiah to explain to the Hebrews what the Christ was.
If Jesus used the term Messiah, we would expect exactly the opposite, explaining Christ is a translation of Messiah to non-Hebrews audiences.
We would also expect the term Messiah to appear in James or the other epistles if it had been the original title, but it never does.
We should look at the usage of the word Christ in the context of the greek empire and not purely in the context of the Hebrew apocalyptic literature.

\paragraph{Jesus is the Son of God.}\label{par:jesus-is-the-son-of-god.}
Alexander was the son of God and so were the Seleucids, Ptolemys.
The title was used for the rulers of the greek empire.
Millenia of pharaohs had literally been called the son of Ra as their title.
And not only one of many titles, the "Son of Ra" actually preceded the name of the ruler known to us.
“Son of Ra, Tutankhamun, ruler of Heliopolis of the South."
"Son of Ra, Amenhotep, ruler of Thebes."
"Son of Ra, Ramses, ruler of the South and the North."
This filial-divine status connected rulers to the cosmic order, reinforcing their role as intermediaries between the divine and the earthly realms, beyond mere dynastic lineage.
In other words, “sonship” named a cosmological-political vocation: the ruler embodied heaven’s order among humans.

\paragraph{Jesus is the Son Of Father.}\label{par:jesus-is-the-son-of-father.}

The well known ruler of Egypt, Cleopatra, was given the title father loving goddess.
Seleucus IV Philopator was also given the title father loving god.
This or similar title was used by countless other rulers in the greek world.
In this context the multiple references to being the only-begotten son of the father signify the rightful heir, and the only rightful claim to the throne.
In Hellenistic dynastic ideology, “only-begotten of the father” marks exclusive succession and undisputed inheritance of sovereignty.

\paragraph{Jesus is the Logos, the Word of God.}\label{par:jesus-is-the-logos-the-word-of-god.}
The concept of Logos was a central theme debated across various philosophical schools, including Heraclitus, who first introduced it as a principle of order and knowledge, the Stoics who saw it as the rational structure of the universe, and the Middle Platonists who integrated it into their metaphysical systems.
This widespread philosophical discourse permeated popular rhetoric, setting the stage for Philo of Alexandria’s interpretations.
The Stoics, particularly Zeno of Citium (c.~334–262 BCE) and later Chrysippus, played a significant role in further developing the idea of Logos.

In Stoic thought, Logos became the rational principle that permeates and organizes the cosmos.
It was understood as a divine rationality that was present in the universe, giving it structure and coherence.
Very notably, Philo of Alexandria who was closely related to the Herodian and Hasmonean dynasties and contemporary to Jesus, described the Logos as a mediator between God and the world.
Thus, if Jesus and some of the other members of the court were indeed visiting Alexandria in Jesus’s youth, they must have had close exposure to the idea of logos.
Based on the age difference and the family relations, Jesus would have likely treated Philo as his distinguished uncle, and so would John the Evangelist.
Notably in the context of the greek empire, the Logos was a divine principle that was both God’s expression and as such the ruler would have been considered to speak the Logos of the God.
This is consistent with the idea of multiple quotes as nobody can get to the father except through me.
Notably Philo described the Logos as the firstborn son of God, and so the rightful heir to the throne and the intermediary between God and the world.
Even though countless philosophers claimed rulers should follow the Logos and called the Logos the divine, the title of the Logos itself does not appear to have been given to any ruler prior to John’s gospel.
And as the idea of Logos in John appears so closely to the evolution of the Logos idea from Philo, it increases the likelihood that indeed John was in a close contact with Philo.
John’s Gospel thus stands squarely within this philosophical trajectory, translating Logos-speculation into royal christology.

\paragraph{Jesus is the God manifested in the flesh.}\label{par:jesus-is-the-god-manifested-in-the-flesh.}
Epiphanes is a title given to multiple rulers, such as Antiochus IV and Ptolemy V .
The title signify the ruler is the god manifested in the flesh.
This epiphanic logic saturated Hellenistic royal propaganda and civic cult: the ruler’s body was the god’s appearing.
“Emmanuel” is the Jewish version of this same political theology: the divine presence embodied in and with the ruler.

\paragraph{Jesus is the Lamb of God who takes away the sins of the world.}\label{par:jesus-is-the-lamb-of-god-who-takes-away-the-sins-of-the-world.}
The Gospel of John proclaims: “Behold, the Lamb of God, who takes away the sin of the world” (John 1:29).
This title frames Jesus both as sacrificial offering and as the ruler who redeems his people.
The Lamb of God indeed matches very well with the quintessential Jewish idea of the sacrificial lamb, and atonement for sins.
But we should also note that tradition is by no means unique to Israel, and in fact was very prevalent in many, if not most, other Near Eastern cultures.
Notably, the Jewish Day of Atonement, where the sins of the people are taken away, refers to the quintessential Egyptian idea of the Book of the Dead and the Book of the Living.
But while admittedly while the Jewish parallel here is possible, the Egyptian parallel is much more direct and explicit.
Amun, the hidden father, was also portrayed as a ram or lamb.
It is not a one time vague reference, but a motif so widespread throughout ancient world and many centuries, even Pliny the Elder wrote about them in his Natural History.
Alexander the Great was portrayed with the ram horns of Amun on his coins.
And Amun was the who forgave the sins of the people.
During confession of sins in ancient Egypt, the following was said:
"
You are Amun, the Lord of the silent, who comes at the voice of the poor; when I call to you in my distress You come and rescue me.
Though the servant was disposed to do evil, the Lord is disposed to forgive.
The Lord of Thebes spends not a whole day in anger;
His wrath passes in a moment; none remains.
His breath comes back to us in mercy ...
May your ka be kind; may you forgive; It shall not happen again.
"

\paragraph{Jesus is the Good Shepherd.}\label{par:jesus-is-the-good-shepherd.}
While the good shepherd is a well-known image in the Old Testament, the term was also already prevalent in the Near East for millennia.
Hammurabi, in the prologue to his law code, called himself “Enlil’s chosen shepherd.”
Assyrian rulers such as Shalmaneser~III likewise styled themselves as faithful shepherds who led their people.
Even earlier, Sumerian kings such as Shulgi of Ur praised themselves as shepherds of their people.
This long pedigree shows that “Good Shepherd” was not merely a rustic image but a royal title with philosophical and political weight across the Mediterranean and Near East.
Let's also note that Plato taught that a true philosopher-king is a good shepherd.

\paragraph{Jesus is the Savior of the World.}\label{par:jesus-is-the-savior-of-the-world.}
Possibly the most common title for a ruler of the Greek kingdoms was \emph{Soter}, which means savior.
The first Ptolemy was given the title \emph{Soter}, and so was the last king of the Greek kingdom, Strato~II~Soter.
Antiochus~III was even given the title “Theos Soter,” “God the Savior.”
The language of \emph{Soter} also resonated with city-level benefactor cults, where rulers were honored as saviors for providing concrete benefits like grain and peace, thus giving the Christian title a civic–imperial resonance.
Cities enacted this in festivals, decrees, and statues, so “Savior” named both theology and public policy.

\paragraph{Jesus is the Lord (Kyrios).}\label{par:jesus-is-the-lord-kyrios}
\emph{Kyrios}, meaning Lord, was the standard honorific: “Kyrios Kaisar,” Lord Caesar.
YHWH appears as \emph{Kyrios} in the Greek translation of the Hebrew scriptures.
Early Christians often called Jesus as lord of lords, or kyrios kyriōn, designating the highest authority.
Similarly, King of Kings and Lord of Lords was a title used for the rulers of empires, not only greeks, but also of India and Persia.
This title highlighted the absolute claim of emperors who absorbed subordinate kings into a universal hierarchy, a precedent set by Persian and Hellenistic rulers.
When early Christians hailed Jesus as \emph{Kyrios}, they proclaimed a rival sovereignty, transferring the empire’s own honorific to another king.

\paragraph{Jesus is the Word, Logos.}\label{par:jesus-is-the-word.}
The Stoics, particularly Zeno of Citium (c.~334–262 BCE) and later Chrysippus, played a significant role in further developing the idea of Logos.
In Stoic thought, Logos became the rational principle that permeates and organizes the cosmos.
It was understood as a divine rationality that was present in the universe, giving it structure and coherence.
The Stoics taught that humans should live in accordance with the Logos, which represented divine reason and the natural law of the universe.
For the Stoics, the Logos was both immanent and cosmic, meaning it was part of everything and governed all things.
For Philo, the Logos was a mediator between God and the world.
He described the Logos as a divine principle that was both God’s expression and the agent of creation.
Philo’s Logos was similar to the Stoic idea, but with a more direct connection to the Jewish monotheistic tradition.

\paragraph{Jesus is the Prince of Peace.}\label{par:jesus-is-the-prince-of-peace.}
Ptolemy V Epiphanes famous of the Rosetta stone, was called the bringer of peace.

\paragraph{Jesus is the Light of the World.}\label{par:jesus-is-the-light-of-the-world.}
Here the title relates to the cave from Plato’s Republic.
The journey from darkness to light symbolizes the philosopher’s ascent from ignorance to knowledge, particularly knowledge of the Good. Solar metaphors were also prevalent in mystery cults and imperial propaganda, especially under Augustus and his successors, who were portrayed as bringers of light and peace.

\paragraph{Alexander the Great (356–323 BC).}
Claimed sonship of Zeus-Ammon after visiting the oracle at Siwa in Egypt.
Depicted with ram’s horns of Ammon on coinage.
Arrian and Plutarch testify to Alexander’s divine self-understanding and the cult that followed him.
Cities honored him as a god during his lifetime.
The cult of Alexander was maintained under the Diadochi.

\paragraph{Ptolemaic Dynasty (323–30 BC).}
Ptolemy I Soter received the epithet “Soter” (Savior) and initiated the cult of Serapis.
Ptolemy II Philadelphus and his sister-wife Arsinoë II were hailed as \textit{Theoi Adelphoi} (Sibling Gods).
Ptolemy III Euergetes was called “Benefactor” and linked to new stars in the sky (Canopus Decree).
Ptolemy V Epiphanes (Rosetta Stone, 196 BC) hailed as “Theos Epiphanes Eucharistos” — the Manifest God who brings peace.
Later Ptolemies, including Cleopatra VII, were worshiped as incarnations of Isis and divine queens.

\paragraph{Seleucid Dynasty (312–63 BC).}
Seleucus I Nicator claimed descent from Apollo, a story reported by Appian and other sources.
Antiochus I Soter established ruler cult with Zeus.
Antiochus III Megas hailed as “Theos Megas” and “Soter.”
Antiochus IV Epiphanes explicitly styled himself as “God Manifest.”
Seleucid coins routinely show radiate crowns, symbolizing solar divinity, and Zeus with a star.

\paragraph{Attalid Dynasty of Pergamon (281–133 BC).}
Attalus I and successors styled themselves as “Soter” and established divine cults.
Pergamon’s Great Altar and monuments present their kings as divinely sanctioned, defenders of cosmic order against the giants.
Royal cult and civic cult were inseparable.

\paragraph{Other Hellenistic dynasts.}
Strato II of Bactria bore the epithet “Soter.”
Kings of Pontus, notably Mithridates VI Eupator, claimed descent from both Dionysus and Alexander, presenting themselves as semi-divine world rulers.
Inscriptions and coins proclaim Mithridates as a godlike liberator against Rome.
The Indo-Greek kings (e.g. Menander I Soter) fused Greek and Indian imagery, styling themselves as savior-kings.

\paragraph{Evidence from inscriptions and coins.}
The Rosetta Stone and Canopus Decree explicitly call Ptolemaic monarchs gods and peacemakers.
Seleucid coins name rulers as Epiphanes (manifest god).
Pergamene inscriptions honor kings as divine protectors.
Greek cities such as Athens and Rhodes voted cult statues and sacrifices to Hellenistic rulers.
Thus the language of “god-king” was not metaphorical but political theology enacted in law, coinage, and liturgy.

\subsubsection{15.
Jesus birth was represented as a new star in the sky.}\label{subsubsec:jesus-birth-was-represented-as-a-new-star-in-the-sky.}
This is a common trope in the greek world, notably used for the birth of Alexander the Great.
The Ptolemies of Egypt (Greek rulers after Alexander) often linked their divine status to stars and celestial phenomena.
Ptolemy III (246–222 BC) was honored with a new star appearing, supposedly confirming his divine favor.
The tradition actually was transferred to the Roman empire, where the birth of Augustus was also represented as a new star in the sky, and so was the death and deification of Julius Caesar.
The story of Ptolemy III (246–222 BC) and a celestial sign is linked to the Canopus Decree (238 BC), an inscription issued by Egyptian priests during his reign.
This decree honors Ptolemy III and his wife, Berenice II, and includes references to astronomical phenomena associated with his rule.
The Canopus Decree (238 BC) This decree was issued by Egyptian priests to honor Ptolemy III for his military campaigns and religious patronage.
It mentions a new star appearing in the sky, likely in connection with his divine status.
The decree also orders the addition of a leap day to the Egyptian calendar, demonstrating Ptolemy III’s association with astronomical knowledge.
Callimachus (Greek Poet, 3rd Century BC) In his lost work, Aetia, Callimachus possibly referenced Berenice’s Lock, a constellation myth that was linked to a celestial omen for Ptolemy III .
The myth suggests that Berenice II dedicated a lock of her hair for her husband’s victory, which disappeared and was later seen as a new star in the sky (Coma Berenices).
Manetho (Egyptian Historian, 3rd Century BC) Though most of Manetho’s works are lost, later writers reference his accounts of omens, stars, and divine portents during the reign of Ptolemy III .
Antiochus III (the Great, 222–187 BC) was said to have had a new star appear before his greatest campaigns.
Seleucid coins often depicted Zeus with a star, symbolizing divine rule.
Portents and omens were integral to the legitimation of rule across various cultures, situating the Gospel story within a broader tradition of celestial signs affirming divine favor.

\paragraph{Synthesis.}
From Alexander through the Ptolemies, Seleucids, Attalids, and others, the Hellenistic world lived under rulers who were literally hailed as gods.
The \textit{Basileia tou Theou} — “Kingdom of God” — was the technical term for such divine monarchies.
Philo of Alexandria, Josephus, and inscriptions confirm that by the first century AD this language still carried the weight of centuries of Greek imperial theology.
When Jesus proclaimed the arrival of the “Kingdom of God,” his audience would have understood it not as an abstract heaven, but as a claim that the divine royal order — the Greek god-kingship — was being restored on earth.

\section{Holy Mary, Mother of God, Perpetual Virgin, Queen of Heaven}\label{sec:holy-mary-mother-of-god-perpetual-virgin-queen-of-heaven}
Not only Jesus but also Mary, the mother of Jesus, can be associated with Greek royal imagery and titles.
\paragraph{Mary was the rightful heiress to the Hasmonean dynasty and so the royal titles we know very well today also correspond to the titles of the Greek rulers.}\label{par:mary-was-the-rightful-heiress-to-the-hasmonean-dynasty-and-so-the-royal-titles-we-know-very-well-today-also-correspond-to-the-titles-of-the-greek-rulers.}
There were countless greek female rulers with highly distinguished, divine titles, such as Cleopatra, who was also a common name in the Hasmonean dynasty.
\paragraph{Holy Mary mother of God.}\label{par:holy-mary-mother-of-god.}
As Jesus was the Son of God, as a rightful heir, Mary as the mother of future ruler would have been considered the Mother of God.
Mother of the Lord'' (Μήτηρ τοῦ Κυρίου) -- Found in Luke 1:43, where Elizabeth calls Mary the Mother of my Lord.’’ This implies royal status since Lord'' (Kyrios) could mean a divine or kingly ruler.
Elizabeth greets Mary: And why is this granted to me, that the mother of my Lord should come to me?’’
\paragraph{Mary was a perpetual virgin.}\label{par:mary-was-a-perpetual-virgin.}
A very common trope in the greek world was that the royal women were pure often called perpetual virgins.
This is something that fits really well with the idea of Mary as royal but seems at odds with the idea of Mary as a mother to Jesus.
\paragraph{Mary was the Queen of Heaven.}\label{par:mary-was-the-queen-of-heaven.}
The earliest known hymns and prayers to Mary refer to her as Βασίλισσα τῶν Οὐρανῶν – explicit royal status.
The Woman Clothed with the Sun'' -- Revelation 12:1 A great sign appeared in heaven: a woman clothed with the sun, with the moon under her feet, and a crown of twelve stars on her head.’’ While this passage refers to Israel, early Christian writings (Hippolytus, 3rd century AD) link it to Mary as a royal mother figure. The Isis tradition was not abstract: Hellenistic queens, including Cleopatra, embodied divine motherhood as political theology; Mary’s elevation continues this dynastic-feminine model in Christian form.
\paragraph{Mary was Blessed among all women}\label{par:mary-was-blessed-among-all-women}
Highlighting Mary’s royal lineage
\paragraph{Mary was the New Eve}\label{par:mary-was-the-new-eve}
New Eve'' -- Early Church Fathers (Justin Martyr, Irenaeus) described Mary as the new Eve, implying a role in a divine dynasty.
Protoevangelium of James (c.~2nd century AD) -- While emphasizing her perpetual virginity, it also hints at a priestly and royal lineage, calling her set apart for the Lord.’’

\section{The Kingdom of God}\label{sec:the-kingdom-of-god}
The phrase “Kingdom of God” (\textgreek{Βασιλεία τοῦ θεοῦ}) did not emerge in a vacuum.
It was part of a long Hellenistic tradition where rulers styled themselves as divine kings, embodying heaven’s rule on earth.
The Greek dynasties of the post-Alexander world repeatedly presented their monarchs as “sons of god,” “manifest gods,” “saviors,” and “epiphanies.”
Listing them and their evidence helps clarify how the title “Kingdom of God” would have resonated to contemporaries of Jesus.

\subsection{The Kingdom of God as Crown of God: common imperial usage, earthly restoration, heavenly mandate.}\label{subsec:the-kingdom-of-god-as-crown-of-god-common-imperial-usage-earthly-restoration-heavenly-mandate}
“Kingdom of God” was not a Christian invention but a phrase already alive in the public vocabulary of the Greek imperial world.
Etymologically, \emph{basileia} comes from \emph{basileus}, king, and its primary sense is kingship, reign, sovereignty—the condition of wearing the crown.
Only secondarily, by extension, does it mean the territory ruled.
The lexica preserve this order: LSJ lists “kingship, dominion, monarchy” as the first meaning and “realm” as a derivative; BDAG does the same, giving “reign, rule, sovereignty” priority before “realm.”
Thus when first-century hearers heard “\emph{basileia tou theou},” they thought not of a place but of the crown itself.

The word was standard for the monarchies of the Hellenistic age—the Ptolemaic \emph{basileia}, the Seleucid \emph{basileia}, the Antigonid \emph{basileia}—as a matter of common speech, and in inscriptions these were praised as orders sanctioned by the gods.
Philo of Alexandria could write in Greek that “the \emph{basileia} of God is the sovereignty over all things” (\emph{Spec. Leg}.~4.164), and Josephus records that Rome “granted the \emph{basileia} to Agrippa” (\emph{Ant}.~19.343), using the same word for the crown-right, not a map of land.
The Septuagint likewise speaks of the God of heaven “raising up a \emph{basileia} that shall never be destroyed” (Dan LXX 2:44) and praises “Your \emph{basileia} is a \emph{basileia} for all ages” (Ps 145:13 LXX).
This is the usage inherited by Jesus and his hearers.
When he announced “the \emph{basileia} of God has drawn near” (Mark 1:15), they heard not a novel metaphor of piety but the familiar imperial claim of a crown restored by divine mandate.

The weight of the New Testament points overwhelmingly to this earthly restoration.
“The \emph{basileia} of God is among you” (Luke 17:21).
“Your kingdom come, your will be done, on earth as in heaven” (Matt 6:10).
“The kingdom of the world has become the kingdom of our Lord and of his Christ” (Rev 11:15).
These are declarations of sovereignty transferred in history, not invitations to escape into another realm.
Even Jesus’ parable of the nobleman presumes this logic: “he went to receive for himself a \emph{basileia} and then return” (Luke 19:12), coronation and comeback, not emigration.

Some fragments do present a heavenly register.
Paul names “the Jerusalem above” (Gal 4:26).
Hebrews speaks of “the city of the living God, the heavenly Jerusalem” (Heb 12:22).
Revelation climaxes with “the holy city, New Jerusalem, coming down out of heaven from God” (Rev 21:2).
These passages show heaven as the secure seat of divine kingship, with its own capital-city imagery, and later Christian tradition rightly developed them into the language of keys, gates, and heavenly admission.
Yet even here the emphasis is not on relocation but on descent and union: the city comes down, the crown held in heaven is bestowed on earth.
Heaven supplies the mandate and validates the stewards’ acts—“whatever you bind on earth shall be bound in heaven, whatever you loose on earth shall be loosed in heaven” (Matt 16:19; 18:18)—but the enactment is in the earthly arena.

Therefore the “kingdom of God” in the New Testament is best understood as a claim to the restoration of divine kingship on earth.
The heavenly city strand affirms that the crown is secure in heaven and dramatizes its final descent.
The majority voice of the Gospels and Epistles insists that Christos already bears the crown here, that Rome’s rule is passing, and that the Greek kingdoms are restored under the name of God’s anointed.
This is not a mystical metaphor but the central proclamation of sovereignty: the crown belongs to God, and it is given to his Christos to rule the nations.

Here we should note that Apostles were a common term for the emissaries of the greek empire.
Ptolemaic and Seleucid Kings often sent out ἀπόστολοι as official emissaries or envoys on diplomatic missions.
The emissaries represented the kings’ political will and spread decrees across the empire.
In Hellenistic courts, these emissaries were not only political envoys but also served as cultic representatives, thereby intertwining religion and governance.

\subsection{Kingdom of God was a common term for the Greek empire and the Greek kingdoms.}\label{subsec:kingdom-of-god-was-a-common-term-for-the-greek-empire-and-the-greek-kingdoms.}

The Hellenistic rulers were not merely kings---they were divine monarchs ruling by the will of the gods.
\emph{Basileia tou theou} (Βασιλεία τοῦ θεοῦ) was a pre-Christian Greek expression for such divinely sanctioned rule, and it appears in Philo of Alexandria, who lived in the same Alexandrian world that shaped Jesus and the first Christian writers.
This royal ideology was fused from Greek, Persian, and Egyptian traditions and had been entrenched for centuries.
To speak of restoring the ``Kingdom of God'' on earth therefore most naturally meant restoring a concrete Greek-style empire under God’s Son, not inventing an abstract inner spirituality.
This is not an anachronism; it was fresh in the mind of Philo and in the horizon of Jesus.

We must distinguish ``Kingdom of God'' (Βασιλεία τοῦ θεοῦ) from ``Kingdom of Heaven'' (Βασιλεία τῶν οὐρανῶν).
And we must also acknowledge that the Gospels sometimes use them interchangeably or in parallel.
Matthew overwhelmingly prefers ``heaven'' while Mark and Luke consistently say ``God,'' yet the same sayings appear in both registers:
Luke 6:20 has ``yours is the \emph{kingdom of God},'' where Matthew 5:3 reads ``theirs is the \emph{kingdom of heaven}.''
Luke 13:29 speaks of reclining at table in the \emph{kingdom of God}, while Matthew 8:11 has the scene in the \emph{kingdom of heaven}.
Luke 14:15 blesses those who eat in the \emph{kingdom of God}, while Matthew 22:2 compares the \emph{kingdom of heaven} to a royal banquet.
Most decisively, Matthew 19:23--24 places them back-to-back: ``enter the \emph{kingdom of heaven}'' (v.\,23) and then ``enter the \emph{kingdom of God}'' (v.\,24).
So there is interchangeability, but there is also a clear pattern.

The pattern is this: ``Kingdom of Heaven'' is the cosmic realm above, the gate in the sky, the court of God.
``Kingdom of God'' is the name of the nation, the basileia claimed for God on earth under his anointed.
When Jesus speaks of afterlife, he uses the cosmic register---``Today you will be with me in \emph{paradise}'' (Luke 23:43).
When he announces his mission in history, he uses the national register---``The time is fulfilled, and the \emph{kingdom of God} is at hand; repent and believe'' (Mark 1:15).

Although universally it is said Jesus avoided political claims as is shown in ``Render to Caesar the things that are Caesar's, and to God the things that are God's'' (Mark 12:17), he also said he spoke in parables ``so that hearing they may not understand' '' (Mark 4:12).
And so we see many quotes and parables from Jesus that talk about spiritual wisdom, but at the same time are profoundly used in political contexts over and over again.
``A house divided against itself cannot stand`` (Mark 3:24) was used as the title in one of the most famous speeches of political unity.
The ``greater love has no one than this, that someone lay down his life for his friends'' (John 15:13) is used as the epitaph on countless war memorials and used in many wartime speeches.
``Truth will set you free'' (John 8:32) was used as the motto of multiple universities and organizations.
``Blessed are the peacemakers, for they shall be called sons of God'' (Matt 5:9) was used as the motto of the United Nations.

Viewed through this lens, a deeper thought resurfaces that modern readings have flattened.
When Jesus says, ``The \emph{Kingdom of God} is within/among you'' (Luke 17:21), he is not describing a vague spiritual metaphor.
He is saying: our empire may be lost on the map, but it lives in us; the nation is carried by its people.
As has been sung, ``As long as a Jewish soul lives in the heart, our country is not lost, and hope endures.''
The Kingdom survives by the allegiance and trust of its citizens.

Nobody is also arguing the message of equality and social justice for all is the very core of Jesus’s message.
“Blessed are you poor, for yours is the kingdom of God” (Luke 6:20).
“He has sent me to proclaim liberty to the captives and to set at liberty those who are oppressed” (Luke 4:18).
“Whatever you did for one of the least of these brothers and sisters of mine, you did for me” (Matt 25:40).

The political description of the Kingdom of God is thus clear and consistent:
One nation, under God, indivisible, with liberty and justice for all.

And the point of Jesus was not to beat Rome by sword but to beat them with love,
and convince them to give the independence by the flail of his arguments.

What Jesus proclaimed was not to endure Rome but to overcome it.
Not by replacing Caesar with another warlord, but by restoring the Kingdom of God as a just nation.
And yet Jesus never called for armed revolt; the Gospels are unanimous on this point.
The very core of Jesus’s teaching is to love your enemy, turn the other cheek, and forgive.
History knows some of the biggest empires in history fell not by force of arms, but by the spirit alone.

Gandhi broke the British Empire not with guns but with truth-force.
When he marched 240 miles to the sea and lifted a pinch of salt in 1930, the whole world saw an empire humiliated by a handful of brine.
Crowds followed him not with rifles but with bare feet, and British jails filled with ordinary people whose only crime was refusing to obey unjust laws.
In that moment the British realized they could not hold India by force, for the governed no longer believed in their legitimacy.
The empire collapsed not because it was defeated in battle, but because it lost the will to rule.

And to most Polish people the Soviet Union fell when the Pope visited Poland in 1979 and announced:
\emph{“Let your Spirit descend and renew the face of the earth; this land!”}
It was the moment when the Soviets realized they could not hold Poland by force.
By domino effect within two years, the entire Eastern Bloc dissolved in possibly the greatest peaceful revolution in history.

Later figures such as Martin Luther King Jr. and Nelson Mandela brought over major social changes in their countries precisely by asking their followers to be less violent and more forgiving.


