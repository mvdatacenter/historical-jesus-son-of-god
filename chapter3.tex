
Next we re-investigate Jesus’s royal claim within the context of the greek empire.
Christos was not the only title given to Jesus.
He was given many other titles: Son of God, King of Kings, Lord of Lords, Savior of the World, Light of the World, Prince of Peace, Lamb of God, Good Shepherd, Way, Truth, Life, Alpha, and Omega.
His earliest followers proclaimed that he died for the sins of others so that those who believed in him might not perish but have eternal life.
In this book we will explore if this may have similarly been a political statement rather than a theological one.
We want to show that all the titles given to Jesus were titles given to the rulers of the greek empire.
That the events surrounding Jesus’s life were consistent with the events surrounding the rulers of the greek empire and were deeply rooted in greek imperial cult.
From Alexander onward, the boundaries between religion and politics dissolved into a single imperial theology.
Alexander’s claim as son of Zeus–Ammon was not just personal piety but the framework that bound army, temple, and polis together.
Successor dynasties carried forward the pattern: Ptolemies as “saviors” and “manifest gods,” Seleucids as heirs of divine mandate.
Coins, temple inscriptions, and public rituals constantly reminded subjects that the ruler’s authority was simultaneously divine and political.
Under Rome, this tradition only deepened: Augustus as \emph{divi filius}, “son of the divine [Julius],” marked coins and festivals alike.
Cities hailed emperors as \emph{soter} (savior) for bringing food and peace.
To call Jesus “Son of God,” “Savior,” “Lord,” or “Light” is to reuse this shared civic language of kingship.
These were not novel Christian inventions but existing titles of political-theological power.

\paragraph{0.
Jesus is the Christ.}\label{par:jesus-is-the-christ.}
In the gospels and all the earliest materials we can see by far the most common title given to Jesus is Christ.
It is not the Messiah, but the Christ.
In the very few cases the word Messiah is used, the gospels clearly state that Messiah is the Hebrew translation of the word Christ.
Christ is the title used by Josephus, Christ is used by Pliny the Younger, Christ is used by Tacitus, Christ is used by Suetonius, Christ is used by Paul.
Christ is the only term used by Jesus own highly educated, greek speaking brother James.
It is not the other way around, Jesus was not actually called the Messiah, which later morphed into Christ.
The gospels just use the Messiah to explain to the Hebrews what the Christ was, not the other way around.
We should look at the usage of the word Christ in the context of the greek empire and not purely in the context of the Hebrew apocalyptic literature.

\paragraph{1.
Jesus is the Son of God.}\label{par:jesus-is-the-son-of-god.}
Alexander was the son of God and so were the Seleucids, Ptolemys.
The title was used for the rulers of the greek empire.
This filial-divine status connected rulers to the cosmic order, reinforcing their role as intermediaries between the divine and the earthly realms, beyond mere dynastic lineage.
In other words, “sonship” named a cosmological-political vocation: the ruler embodied heaven’s order among humans.

\paragraph{2.
Jesus is the Son Of Father.}\label{par:jesus-is-the-son-of-father.}

The well known ruler of Egypt, Cleopatra, was given the title father loving goddess.
Seleucus IV Philopator was also given the title father loving god.
This or similar title was used by countless other rulers in the greek world.
In this context the multiple references to being the only-begotten son of the father signify the rightful heir, and the only rightful claim to the throne.
In Hellenistic dynastic ideology, “only-begotten of the father” marks exclusive succession and undisputed inheritance of sovereignty.

\paragraph{3.
Jesus is the Logos, the Word of God.}\label{par:jesus-is-the-logos-the-word-of-god.}
The concept of Logos was a central theme debated across various philosophical schools, including Heraclitus, who first introduced it as a principle of order and knowledge, the Stoics who saw it as the rational structure of the universe, and the Middle Platonists who integrated it into their metaphysical systems.
This widespread philosophical discourse permeated popular rhetoric, setting the stage for Philo of Alexandria’s interpretations.
The Stoics, particularly Zeno of Citium (c.~334–262 BCE) and later Chrysippus, played a significant role in further developing the idea of Logos.

In Stoic thought, Logos became the rational principle that permeates and organizes the cosmos.
It was understood as a divine rationality that was present in the universe, giving it structure and coherence.
Very notably, Philo of Alexandria who was closely related to the Herodian and Hasmonean dynasties and contemporary to Jesus, described the Logos as a mediator between God and the world.
Thus, if Jesus and some of the other members of the court were indeed visiting Alexandria in Jesus’s youth, they must have had close exposure to the idea of logos.
Based on the age difference and the family relations, Jesus would have likely treated Philo as his distinguished uncle, and so would John the Evangelist.
Notably in the context of the greek empire, the Logos was a divine principle that was both God’s expression and as such the ruler would have been considered to speak the Logos of the God.
This is consistent with the idea of multiple quotes as nobody can get to the father except through me.
Notably Philo described the Logos as the firstborn son of God, and so the rightful heir to the throne and the intermediary between God and the world.
Even though countless philosophers claimed rulers should follow the Logos and called the Logos the divine, the title of the Logos itself does not appear to have been given to any ruler prior to John’s gospel.
And as the idea of Logos in John appears so closely to the evolution of the Logos idea from Philo, it increases the likelihood that indeed John was in a close contact with Philo.
John’s Gospel thus stands squarely within this philosophical trajectory, translating Logos-speculation into royal christology.

\paragraph{3.
Jesus is the God manifested in the flesh.}\label{par:jesus-is-the-god-manifested-in-the-flesh.}
Epiphanes is a title given to multiple rulers, such as Antiochus IV and Ptolemy V .
The title signify the ruler is the god manifested in the flesh.
This epiphanic logic saturated Hellenistic royal propaganda and civic cult: the ruler’s body was the god’s appearing.

\paragraph{4.
Jesus is the Savior of the World.}\label{par:jesus-is-the-savior-of-the-world.}
Essentially every other ruler of the greek empire was given Soter as a title, which means savior.
The first Ptolemy was given the title Soter, and so was the last king of the greek kingdom Strato II Soter.
Antiochus III was even given the title The God the Savior’’.
The language of “soter” also resonated with city-level benefactor cults, where rulers were honored as saviors for providing concrete benefits like grain and peace, thus giving the Christian title a civic-imperial resonance.
Cities enacted this in festivals, decrees, and statues, so “Savior” named both theology and public policy.
\paragraph{4a.
Jesus is the Lord (Kyrios).}\label{par:jesus-is-the-lord-kyrios}
\emph{Kyrios} was the standard imperial honorific: “Kyrios Kaisar,” Lord Caesar. To call the emperor \emph{Kyrios} was a liturgical act of allegiance embedded in civic oaths and sacrifices. When early Christians hailed Jesus as \emph{Kyrios}, they proclaimed a rival sovereignty, transferring the empire’s own honorific to another king. “Lord of lords” heightens the claim by placing him at the pinnacle of the imperial hierarchy.
\paragraph{4b.
Jesus is Emmanuel (“God with us”).}\label{par:jesus-is-emmanuel}
Hellenistic titulature abounds in epiphanic claims—\emph{Epiphanes} (“manifest”), \emph{Theos} (“god”), \emph{Soter} (“savior”).
“Emmanuel” is the Semitic-facing version of this same political theology: the divine presence embodied in and with the ruler.
\paragraph{5.
Jesus is the Good Shepherd.}\label{par:jesus-is-the-good-shepherd.}
Plato taught that a true philosopher king is a good shepherd.
Likely it stuck as a title for Jesus.
The image of the shepherd also has deep roots in Near Eastern kingship traditions, such as those of the Assyrians and Babylonians, reinforcing its long royal pedigree.

\paragraph{6.
Jesus is the Light of the World.}\label{par:jesus-is-the-light-of-the-world.}
Here the title relates to the cave from Plato’s Republic.
The journey from darkness to light symbolizes the philosopher’s ascent from ignorance to knowledge, particularly knowledge of the Good. Solar metaphors were also prevalent in mystery cults and imperial propaganda, especially under Augustus and his successors, who were portrayed as bringers of light and peace.
\paragraph{7.
Jesus is the King of Kings and Lord of Lords.}\label{par:jesus-is-the-king-of-kings-and-lord-of-lords.}
Similarly, King of Kings and Lord of Lords was a title used for the rulers of empires, not only greeks, but also of India and Persia. This title highlighted the absolute claim of emperors who absorbed subordinate kings into a universal hierarchy, a precedent set by Persian and Hellenistic rulers.
\paragraph{8.
Jesus is the Prince of Peace.}\label{par:jesus-is-the-prince-of-peace.}

Ptolemy V Epiphanes famous of the Rosetta stone, was called the bringer of peace.
\paragraph{9.
Jesus is the Word.}\label{par:jesus-is-the-word.}
The Stoics, particularly Zeno of Citium (c.~334–262 BCE) and later Chrysippus, played a significant role in further developing the idea of Logos.
In Stoic thought, Logos became the rational principle that permeates and organizes the cosmos.
It was understood as a divine rationality that was present in the universe, giving it structure and coherence.
The Stoics taught that humans should live in accordance with the Logos, which represented divine reason and the natural law of the universe.
For the Stoics, the Logos was both immanent and cosmic, meaning it was part of everything and governed all things.
For Philo, the Logos was a mediator between God and the world.
He described the Logos as a divine principle that was both God’s expression and the agent of creation.
Philo’s Logos was similar to the Stoic idea, but with a more direct connection to the Jewish monotheistic tradition.
\paragraph{10.
Jesus has 12 apostles.}\label{par:jesus-has-12-apostles.}
Apostles were a common term for the emissaries of the greek empire.
Ptolemaic and Seleucid Kings often sent out ἀπόστολοι as official emissaries or envoys on diplomatic missions.
The emissaries represented the kings’ political will and spread decrees across the empire.
In Hellenistic courts, these emissaries were not only political envoys but also served as cultic representatives, thereby intertwining religion and governance.
\paragraph{11.
Jesus was crucified for being the King of The Jews}\label{par:jesus-was-crucified-for-being-the-king-of-the-jews}
This is the actual question that was asked to Jesus by the Roman governor Pontius Pilate.
If Jesus really had the title of the last rightful heir to the greek empire, then he would have indeed posed a threat to the Roman empire.
Generally apocalyptic preachers would not have been given this punishment, but a rightful heir to the greek empire would have been.
Typically, the punishment was reserved for those who posed a threat to the Roman empire.
Roman governors viewed claimants to kingship as existential threats to stability within the client-king system, necessitating severe measures.
\paragraph{12.
The writing on the cross was ``The King of the Jews’’}\label{par:the-writing-on-the-cross-was-the-king-of-the-jews}
In the context of crucifixion being a punishment for those who posed a threat to the Roman empire, it would make sense that Romans would post a note like this not to mock Jesus, but to warn others against rebelling against the Roman empire.
\paragraph{13.
Jesus was not left on the cross to be eaten by scavengers.}\label{par:jesus-was-not-left-on-the-cross-to-be-eaten-by-scavengers.}
Typically, the bodies of the crucified were left on the cross to be eaten by scavengers, but Jesus was taken down from the cross and buried in a tomb.
This is consistent with the Roman being harsh but ultimately not trying to overstep their bounds.
In a case of a slave committing a crime, Jesus would be left on the cross, but as they could expect more rebellion if their cruelty had no bounds, it is conceivable that they would have allowed Jesus to be buried.
Crucifixion served as a deterrent spectacle, and making an exception for Jesus indicates a political calculation to avoid inciting further unrest.
\paragraph{13.
Parables of Jesus.}\label{par:parables-of-jesus.}
There are many parables of Jesus that seem to hint at him being a good ambassador of a future kingdom.
The parable of the talents, the parable of the sower, the parable of the lost sheep, the parable of the prodigal son, the parable of the good samaritan, the parable of the wedding feast, the parable of the ten virgins, the parable of the wise and foolish builders, the parable of the rich fool, the parable of the barren fig tree, the parable of the great banquet, the parable of the mustard seed, the parable of the yeast, the parable of the hidden treasure, the parable of the pearl, the parable of the net, all seem to hint at a future kingdom that Jesus would rule over.

\subsubsection{15.
The Kingdom of God as Crown of God: common imperial usage, earthly restoration, heavenly mandate.}\label{subsubsec:the-kingdom-of-god-as-crown-of-god-common-imperial-usage-earthly-restoration-heavenly-mandate}
“Kingdom of God” was not a Christian invention but a phrase already alive in the public vocabulary of the Greek imperial world.
Etymologically, \emph{basileia} comes from \emph{basileus}, king, and its primary sense is kingship, reign, sovereignty—the condition of wearing the crown.
Only secondarily, by extension, does it mean the territory ruled.
The lexica preserve this order: LSJ lists “kingship, dominion, monarchy” as the first meaning and “realm” as a derivative; BDAG does the same, giving “reign, rule, sovereignty” priority before “realm.”
Thus when first-century hearers heard “\emph{basileia tou theou},” they thought not of a place but of the crown itself.

The word was standard for the monarchies of the Hellenistic age—the Ptolemaic \emph{basileia}, the Seleucid \emph{basileia}, the Antigonid \emph{basileia}—as a matter of common speech, and in inscriptions these were praised as orders sanctioned by the gods.
Philo of Alexandria could write in Greek that “the \emph{basileia} of God is the sovereignty over all things” (\emph{Spec. Leg}.~4.164), and Josephus records that Rome “granted the \emph{basileia} to Agrippa” (\emph{Ant}.~19.343), using the same word for the crown-right, not a map of land.
The Septuagint likewise speaks of the God of heaven “raising up a \emph{basileia} that shall never be destroyed” (Dan LXX 2:44) and praises “Your \emph{basileia} is a \emph{basileia} for all ages” (Ps 145:13 LXX).
This is the usage inherited by Jesus and his hearers.
When he announced “the \emph{basileia} of God has drawn near” (Mark 1:15), they heard not a novel metaphor of piety but the familiar imperial claim of a crown restored by divine mandate.

The weight of the New Testament points overwhelmingly to this earthly restoration.
“The \emph{basileia} of God is among you” (Luke 17:21).
“Your kingdom come, your will be done, on earth as in heaven” (Matt 6:10).
“The kingdom of the world has become the kingdom of our Lord and of his Christ” (Rev 11:15).
These are declarations of sovereignty transferred in history, not invitations to escape into another realm.
Even Jesus’ parable of the nobleman presumes this logic: “he went to receive for himself a \emph{basileia} and then return” (Luke 19:12), coronation and comeback, not emigration.

Some fragments do present a heavenly register.
Paul names “the Jerusalem above” (Gal 4:26).
Hebrews speaks of “the city of the living God, the heavenly Jerusalem” (Heb 12:22).
Revelation climaxes with “the holy city, New Jerusalem, coming down out of heaven from God” (Rev 21:2).
These passages show heaven as the secure seat of divine kingship, with its own capital-city imagery, and later Christian tradition rightly developed them into the language of keys, gates, and heavenly admission.
Yet even here the emphasis is not on relocation but on descent and union: the city comes down, the crown held in heaven is bestowed on earth.
Heaven supplies the mandate and validates the stewards’ acts—“whatever you bind on earth shall be bound in heaven, whatever you loose on earth shall be loosed in heaven” (Matt 16:19; 18:18)—but the enactment is in the earthly arena.

Therefore the “kingdom of God” in the New Testament is best understood as a claim to the restoration of divine kingship on earth.
The heavenly city strand affirms that the crown is secure in heaven and dramatizes its final descent.
The majority voice of the Gospels and Epistles insists that Christos already bears the crown here, that Rome’s rule is passing, and that the Greek kingdoms are restored under the name of God’s anointed.
This is not a mystical metaphor but the central proclamation of sovereignty: the crown belongs to God, and it is given to his Christos to rule the nations.

\subsubsection{18. God-Kings of Greece and the Kingdom of God}\label{subsubsec:god-kings-of-greece-and-the-kingdom-of-god}
The phrase “Kingdom of God” (\textgreek{Βασιλεία τοῦ θεοῦ}) did not emerge in a vacuum.
It was part of a long Hellenistic tradition where rulers styled themselves as divine kings, embodying heaven’s rule on earth.
The Greek dynasties of the post-Alexander world repeatedly presented their monarchs as “sons of god,” “manifest gods,” “saviors,” and “epiphanies.”
Listing them and their evidence helps clarify how the title “Kingdom of God” would have resonated to contemporaries of Jesus.

\subsubsection{14.
Kingdom of God was a common term for the Greek empire and the Greek kingdoms.}\label{subsubsec:kingdom-of-god-was-a-common-term-for-the-greek-empire-and-the-greek-kingdoms.}

The Hellenistic rulers were not merely kings---they were divine monarchs ruling by the will of the gods.
\emph{Basileia tou theou} (Βασιλεία τοῦ θεοῦ) was a pre-Christian Greek expression for such divinely sanctioned rule, and it appears in Philo of Alexandria, who lived in the same Alexandrian world that shaped Jesus and the first Christian writers.
This royal ideology was fused from Greek, Persian, and Egyptian traditions and had been entrenched for centuries.
To speak of restoring the ``Kingdom of God'' on earth therefore most naturally meant restoring a concrete Greek-style empire under God’s Son, not inventing an abstract inner spirituality.
This is not an anachronism; it was fresh in the mind of Philo and in the horizon of Jesus.

We must distinguish ``Kingdom of God'' (Βασιλεία τοῦ θεοῦ) from ``Kingdom of Heaven'' (Βασιλεία τῶν οὐρανῶν).
And we must also acknowledge that the Gospels sometimes use them interchangeably or in parallel.
Matthew overwhelmingly prefers ``heaven'' while Mark and Luke consistently say ``God,'' yet the same sayings appear in both registers:
Luke 6:20 has ``yours is the \emph{kingdom of God},'' where Matthew 5:3 reads ``theirs is the \emph{kingdom of heaven}.''
Luke 13:29 speaks of reclining at table in the \emph{kingdom of God}, while Matthew 8:11 has the scene in the \emph{kingdom of heaven}.
Luke 14:15 blesses those who eat in the \emph{kingdom of God}, while Matthew 22:2 compares the \emph{kingdom of heaven} to a royal banquet.
Most decisively, Matthew 19:23--24 places them back-to-back: ``enter the \emph{kingdom of heaven}'' (v.\,23) and then ``enter the \emph{kingdom of God}'' (v.\,24).
So there is interchangeability, but there is also a clear pattern.

The pattern is this: ``Kingdom of Heaven'' is the cosmic realm above, the gate in the sky, the court of God.
``Kingdom of God'' is the name of the nation, the basileia claimed for God on earth under his anointed.
When Jesus speaks of afterlife, he uses the cosmic register---``Today you will be with me in \emph{paradise}'' (Luke 23:43).
When he announces his mission in history, he uses the national register---``The time is fulfilled, and the \emph{kingdom of God} is at hand; repent and believe'' (Mark 1:15).

Although universally it is said Jesus avoided political claims as is shown in ``Render to Caesar the things that are Caesar's, and to God the things that are God's'' (Mark 12:17), he also said he spoke in parables ``so that hearing they may not understand' '' (Mark 4:12).
And so we see many quotes and parables from Jesus that talk about spiritual wisdom, but at the same time are profoundly used in political contexts over and over again.
``A house divided against itself cannot stand`` (Mark 3:24) was used as the title in one of the most famous speeches of political unity.
The ``greater love has no one than this, that someone lay down his life for his friends'' (John 15:13) is used as the epitaph on countless war memorials and used in many wartime speeches.
``Truth will set you free'' (John 8:32) was used as the motto of multiple universities and organizations.
``Blessed are the peacemakers, for they shall be called sons of God'' (Matt 5:9) was used as the motto of the United Nations.

Viewed through this lens, a beautiful thought resurfaces that modern readings have flattened.
When Jesus says, ``The \emph{Kingdom of God} is within/among you'' (Luke 17:21), he is not describing a vague spiritual metaphor.
He is saying: our empire may be lost on the map, but it lives in us; the nation is carried by its people.
As has been sung, ``As long as a Jewish soul lives in the heart, our country is not lost, and hope endures.''
The Kingdom survives by the allegiance and trust of its citizens.

Nobody is also arguing the message of equality and social justice for all is the very core of Jesus’s message.
“Blessed are you poor, for yours is the kingdom of God” (Luke 6:20).
“He has sent me to proclaim liberty to the captives and to set at liberty those who are oppressed” (Luke 4:18).
“Whatever you did for one of the least of these brothers and sisters of mine, you did for me” (Matt 25:40).

The political description of the Kingdom of God is thus clear and consistent:
One nation, under God, indivisible, with liberty and justice for all.

And the point of Jesus was not to beat Rome by sword but to beat them with love,
and convince them to give the independence by the flail of his arguments.

What Jesus proclaimed was not to endure Rome but to overcome it.
Not by replacing Caesar with another warlord, but by restoring the Kingdom of God as a just nation.
And yet Jesus never called for armed revolt; the Gospels are unanimous on this point.
History knows some of the biggest empires in history fell not by force of arms.

Gandhi broke the British Empire not with guns but with truth-force.
When he marched 240 miles to the sea and lifted a pinch of salt in 1930, the whole world saw an empire humiliated by a handful of brine.
Crowds followed him not with rifles but with bare feet, and British jails filled with ordinary people whose only crime was refusing to obey unjust laws.
In that moment the British realized they could not hold India by force, for the governed no longer believed in their legitimacy.
The empire collapsed not because it was defeated in battle, but because it lost the will to rule.

And to most Polish people the Soviet Union fell when the Pope visited Poland in 1979 and announced:
\emph{“Let your Spirit descend and renew the face of the earth; this land!”}
It was the moment when the Soviets realized they could not hold Poland by force.
By domino effect within two years, the entire Eastern Bloc dissolved in possibly the greatest peaceful revolution in history.

\paragraph{Alexander the Great (356–323 BC).}
Claimed sonship of Zeus-Ammon after visiting the oracle at Siwa in Egypt.
Depicted with ram’s horns of Ammon on coinage.
Arrian and Plutarch testify to Alexander’s divine self-understanding and the cult that followed him.
Cities honored him as a god during his lifetime.
The cult of Alexander was maintained under the Diadochi.

\paragraph{Ptolemaic Dynasty (323–30 BC).}
Ptolemy I Soter received the epithet “Soter” (Savior) and initiated the cult of Serapis.
Ptolemy II Philadelphus and his sister-wife Arsinoë II were hailed as \textit{Theoi Adelphoi} (Sibling Gods).
Ptolemy III Euergetes was called “Benefactor” and linked to new stars in the sky (Canopus Decree).
Ptolemy V Epiphanes (Rosetta Stone, 196 BC) hailed as “Theos Epiphanes Eucharistos” — the Manifest God who brings peace.
Later Ptolemies, including Cleopatra VII, were worshiped as incarnations of Isis and divine queens.

\paragraph{Seleucid Dynasty (312–63 BC).}
Seleucus I Nicator claimed descent from Apollo, a story reported by Appian and other sources.
Antiochus I Soter established ruler cult with Zeus.
Antiochus III Megas hailed as “Theos Megas” and “Soter.”
Antiochus IV Epiphanes explicitly styled himself as “God Manifest.”
Seleucid coins routinely show radiate crowns, symbolizing solar divinity, and Zeus with a star.

\paragraph{Attalid Dynasty of Pergamon (281–133 BC).}
Attalus I and successors styled themselves as “Soter” and established divine cults.
Pergamon’s Great Altar and monuments present their kings as divinely sanctioned, defenders of cosmic order against the giants.
Royal cult and civic cult were inseparable.

\paragraph{Other Hellenistic dynasts.}
Strato II of Bactria bore the epithet “Soter.”
Kings of Pontus, notably Mithridates VI Eupator, claimed descent from both Dionysus and Alexander, presenting themselves as semi-divine world rulers.
Inscriptions and coins proclaim Mithridates as a godlike liberator against Rome.
The Indo-Greek kings (e.g. Menander I Soter) fused Greek and Indian imagery, styling themselves as savior-kings.

\paragraph{Evidence from inscriptions and coins.}
The Rosetta Stone and Canopus Decree explicitly call Ptolemaic monarchs gods and peacemakers.
Seleucid coins name rulers as Epiphanes (manifest god).
Pergamene inscriptions honor kings as divine protectors.
Greek cities such as Athens and Rhodes voted cult statues and sacrifices to Hellenistic rulers.
Thus the language of “god-king” was not metaphorical but political theology enacted in law, coinage, and liturgy.

\paragraph{Synthesis.}
From Alexander through the Ptolemies, Seleucids, Attalids, and others, the Hellenistic world lived under rulers who were literally hailed as gods.
The \textit{Basileia tou Theou} — “Kingdom of God” — was the technical term for such divine monarchies.
Philo of Alexandria, Josephus, and inscriptions confirm that by the first century AD this language still carried the weight of centuries of Greek imperial theology.
When Jesus proclaimed the arrival of the “Kingdom of God,” his audience would have understood it not as an abstract heaven, but as a claim that the divine royal order — the Greek god-kingship — was being restored on earth.

\subsubsection{15.
Jesus birth was represented as a new star in the sky.}\label{subsubsec:jesus-birth-was-represented-as-a-new-star-in-the-sky.}
This is a common trope in the greek world, notably used for the birth of Alexander the Great.
The Ptolemies of Egypt (Greek rulers after Alexander) often linked their divine status to stars and celestial phenomena.
Ptolemy III (246–222 BC) was honored with a new star appearing, supposedly confirming his divine favor.
The tradition actually was transferred to the Roman empire, where the birth of Augustus was also represented as a new star in the sky, and so was the death and deification of Julius Caesar.
The story of Ptolemy III (246–222 BC) and a celestial sign is linked to the Canopus Decree (238 BC), an inscription issued by Egyptian priests during his reign.
This decree honors Ptolemy III and his wife, Berenice II, and includes references to astronomical phenomena associated with his rule.
The Canopus Decree (238 BC) This decree was issued by Egyptian priests to honor Ptolemy III for his military campaigns and religious patronage.
It mentions a new star appearing in the sky, likely in connection with his divine status.
The decree also orders the addition of a leap day to the Egyptian calendar, demonstrating Ptolemy III’s association with astronomical knowledge.
Callimachus (Greek Poet, 3rd Century BC) In his lost work, Aetia, Callimachus possibly referenced Berenice’s Lock, a constellation myth that was linked to a celestial omen for Ptolemy III .
The myth suggests that Berenice II dedicated a lock of her hair for her husband’s victory, which disappeared and was later seen as a new star in the sky (Coma Berenices).
Manetho (Egyptian Historian, 3rd Century BC) Though most of Manetho’s works are lost, later writers reference his accounts of omens, stars, and divine portents during the reign of Ptolemy III .
Antiochus III (the Great, 222–187 BC) was said to have had a new star appear before his greatest campaigns.
Seleucid coins often depicted Zeus with a star, symbolizing divine rule.
Portents and omens were integral to the legitimation of rule across various cultures, situating the Gospel story within a broader tradition of celestial signs affirming divine favor.
\subsubsection{16.
Jesus was buried in a tomb.}\label{subsubsec:jesus-was-buried-in-a-tomb.}
No other crucified person was buried in a tomb.
They were left on the cross to be eaten by scavengers.
A regular revolutionary would have been left on the cross, but someone with a royal lineage could be given an extraordinary exception.
As the Romans likely could have conceived of Jesus being more divine by his royal lineage, they may have already be afraid of Gods wrath at the time of crucifixion.
\subsubsection{17.
Jesus was crucified on Wednesday in 31 AD.}\label{subsubsec:jesus-was-crucified-on-wednesday-in-31-ad.}
It is commonly believed that Jesus was crucified on Friday, but all the Christian sources actually very directly say that Jesus was crucified on Wednesday.
This is not a new idea, but for some unknown reason it is ignored by nearly all scholars.
The Friday crucifixion is a later liturgical development and not present in the bible or earliest Christian sources.
All gospels agree that Jesus was crucified on the day of preparation for the Passover.
The common misunderstanding is that the day of preparation is the day before the Sabbath, which is Saturday, but in this context the passover is also a Sabbath.
So for the texts to be consistent, the day of preparation must be Thursday.
If we assume Friday we have trials before Annas, Caiaphas, Sanhedrin, Herod, and Pilate, mocking, beatings, travel, flogging, crucifixion, death, burial before sunset all supposedly occurred on the same morning, which is highly implausible.
There is obviously the extremely common statement that Jesus was in the tomb for three days and three nights which only works if we assume Jesus was crucified on Thursday.
To summarize: Wednesday crucifixion in 34 AD (April 21 Julian) fits all the evidence: John 19:31 explicitly distinguishes the Sabbath following Jesus’ death as a high day'' (i.e., not a regular Saturday Sabbath).
Leviticus 23:7 establishes that Nisan 15 is a mandatory Sabbath regardless of weekday---this is the Feast of Unleavened Bread.
Matthew 28:1 does indeed say after the Sabbaths’’ (plural: σαββάτων), and this is not a scribal error.
Mark 16:1 – When the Sabbath was past, Mary Magdalene, and Mary the mother of James, and Salome, bought spices\ldots'' → This refers to buying spices after the first (Thursday) Sabbath, so it must be Friday.
Three days and three nights’’ (Matthew 12:40) cannot be forced into a Friday-to-Sunday window without manipulating Jewish idiom or chronology.
Luke 23:56 – ``They returned and prepared spices and ointments; and rested on the Sabbath according to the commandment.’’ → This refers to preparing spices on Friday, and then resting on Saturday.
Day Event Here’s the count if Jesus died Wednesday afternoon: Buried before sunset Wednesday Night 1: Wednesday night Day 1: Thursday (High Sabbath) Night 2: Thursday night Day 2: Friday (spice preparation day) Night 3: Friday night, third night in the tomb Day 3: Saturday (Weekly Sabbath), third day in the tomb Night 4: Jesus resurrected at the sundown of Saturday, that is the beginning of Sunday in Jewish calendar Day 4: Sunday (first day of the week) the tomb was found empty If resurrection occurred just after sunset on Saturday, it is exactly three days and three nights.
Leading to Thursday, Friday, and Saturday being the three days and three nights, and the evening of Saturday when the resurrection allegedly occurred is the beginning of Sunday.
So this is fully consistent with the numerous references to the resurrection on Sunday.
Why the tomb was found empty Sunday: Because the women waited until the Sabbath ended (Saturday evening), then came at dawn Sunday (Matthew 28:1, Luke 24:1).
That’s when the resurrection was discovered — not when it happened.
So: Three full days and nights: yes.
Resurrection not seen, only tomb found empty Sunday: yes.
Fits Jewish counting and Gospel timeline: yes.
Then the crucifixion must have happened on Wednesday, April 21, 31 AD .
Note that for example, the possibly oldest preserved Christian text, the Didache, explicitly states that fast on Wednesday and Friday.
Friday is likely linked to Sabbath, but Wednesday is a new addition and most logically it would be the day of crucifixion.
Notably the earliest Christian writers outside the gospels and before Tertullian (c.~160–220 AD) that talk about the resurrection are Justin Martyr (c.~100–165 AD) and Barnabas (c.~70–135 AD), both of whom place the resurrection on at the start of Sunday, but do not list the day of crucifixion.
\subsubsection{17.
Jesus survived crucifixion}\label{subsubsec:jesus-survived-crucifixion}
In this context it is not even inconceivable that the Romans would have allowed Jesus to be picked up from the cross before death.
It is highly speculative, but it has a really strong explanation power to it.
Perhaps something as trivial as lightnings and thunders could have already made the Roman soldiers and the crowds to superstitiously believe he truly was the son of God and got scared.
Joseph of Arimathea and Nicodemus did receive an agreement from Pilate to pull him from the cross early, and Jesus could have simply survived the trauma and barely alive.
And on the third day he got so much better that he was able to walk around and talk to the apostles and show his wounds.
Then Jesus died fifty days later due to infection and all started to believe he was resurrected and ascended to heaven.
Notably all burial care was done at the time of death.
It was not Sabbath yet, and they had allegedly plenty of time to do bury Jesus on Friday.
Yet why would they still tend to Jesus body on Sunday morning?
This could have been a medical care, not just continuation of unfinished burial process.
The eventual death of Jesus from infection, especially given the severe wounds he suffered, adds a realistic angle to the story.
After his brief recovery, it would be plausible for his body to succumb to the damage sustained during the crucifixion.
This would also explain why the apostles continued to believe in his resurrection, even after his eventual death.
They might have interpreted his survival and brief recovery as divine intervention and seen his later death as part of a larger divine plan.
Perhaps all the doubting really did happen as the apostles were certain that Jesus was dead as they were not eyewitnesses to the event itself.
Further support to the story is In Mark 15:44, Pilate is described as being surprised by the news of Jesus’ death, as he expected Jesus to have been on the cross longer.
Mark states: ``Pilate was surprised to hear that he was already dead.
Summoning the centurion, he asked him if Jesus had already died.
When he learned from the centurion that it was so, he gave the body to Joseph.’’ This detail suggests that Jesus’ death was unexpectedly quick.
Crucifixion was a prolonged form of execution designed to last for hours, if not days, as the condemned person typically died from a combination of blood loss, exposure, and suffocation.
For Pilate to be surprised, it could imply that Jesus’ death occurred more quickly than usual, which is significant because:
Jesus cries out loudly before dying (Mark 15:37, Luke 23:46): Crucifixion victims typically die slowly, often suffocating, with fading strength.
A loud cry right before death is unusual and may imply he still had significant strength—suggesting he was not yet at death’s door.
A burst of strength like this would point more towards a theatrical performance to convince the others the death was real.
Roman soldiers were typically experienced in carrying out executions, and the death on the cross was intended to be slow and torturous.
The standard time for death was several hours, and for someone to die within less than six hours, as Jesus did, would have been unusual.
Pilate’s surprise may indicate that Jesus’ death was significantly quicker than expected.
It may be that Jesus wasn’t fully dead at the time he was taken down from the cross.
It seems likely small omens in the sky mixed with fear in the crowd and even among the soldiers made Pilate more receptive to Joseph of Arimathea’s request without carefully checking Jesus fully passed away.
Note that Joseph of Arimathea was a member of the Sanhedrin, and he was likely a person of influence making it even more likely Pilate would have been more receptive to his request.
When the Roman soldiers pierced Jesus’ side with a spear, blood and water flowed out, which is often interpreted as a sign of death.
Instead, one of the soldiers pierced Jesus' side with a spear, bringing a sudden flow of blood and water.'' (John 19:34) Trauma, scourging, and prolonged stress could have caused a buildup of fluid around the lungs (pleural effusion) or around the heart (pericardial effusion). If pierced, these fluids could flow out as a mix of clear fluid (water’’) and blood.
If Jesus were completely dead, the blood would have likely clotted in his body, and the wound wouldn’t produce such a sudden flow.'' The fact that both blood and water flowed out’’ immediately suggests the body still had some circulatory activity, meaning the heart might not have fully stopped yet.
Pleural or pericardial effusion does NOT mean the person is already dead—it can happen before death in cases of extreme shock or injury.
It would be likely more than enough to convince the centurion that Jesus was dead and pass the news to Pilate.
Joseph of Arimathea taking Jesus to his own garden, according to the gospel of Peter, to bury him there, is also highly suspect.
If Jesus were truly dead, it would be strange for Joseph, not his family, to take initiative.
But if alive, placing him in a personal tomb under control of a sympathizer makes sense.
The excuse that Jesus had to be buried quickly before the Sabbath in a temporary grave may actually be a plot to hide the fact that Jesus was not dead yet.
Aloes and myrrh for treatment, not burial (John 19:39): 75 pounds of myrrh and aloes were brought by Nicodemus.
That’s far more than needed for burial alone and both have known medicinal properties—especially for healing wounds.
The quantity hints at treatment, not embalming.
It is also worth noting that using armed guards to protect the tomb is not a common practice.
Armed guards when the person is still alive would make a lot more sense.
For example some of the opponents of Jesus suspected a foul play and wanted to make sure Jesus was really dead and not taken down from the cross alive.
Ancient medicine and religious practice overlapped heavily: aloes and myrrh were recognized treatments for wounds, but their abundance could easily be read as ritual-sacramental. Survival under such conditions would have appeared miraculous, reinforcing divine aura even if natural causes played the larger role.
Many scholars point out that Arimathea is a place that does not exist, and so it is likely a made up name.
However, Ar-Ram, also known as Ramathaim, today better known as Ramallah or Ram Allah, is an ancient town a few kilometers north of Jerusalem that is more than likely to be that place.
We actually already know Ramallah from the Old Testament, where it is mentioned as the birthplace of Samuel.
Notably early versions of Septuagint translated the birthplace as ``Αρμαθαιμ’’ in 1 Samuel 1:1, while hebrew text used Ramathaim.
It is a natural translation of the name into Greek, and the Septuagint should be considered as a strong evidence that Arimathea is indeed Ramallah.
It is also notable enough and close enough to Jerusalem that it would have been very logical origin place for a prominent member of the Sanhedrin.
It is also notable enough to have been mentioned in the gospels as a place elevating the status of Joseph of Arimathea and familiar enough to the people of the story to not be needing any further explanation.
The doubting found in all the gospels is also highly natural.
Simply Jesus being tortured and left for near-certain death but then eventually surviving would have still been treated as a miracle.
Most likely the apostles really did not believe much in miracles and were not expecting Jesus to survive.
Finally, many scholars point out that the victim of crucifixion were always left on the cross to be eaten by scavengers in every known record.
However, even Philo of Alexandria, featured in many discussions in this book for unrelated reasons, described a case of numerous Jew insurrectionists in Alexandria in 38AD were actually taken down from the cross in exchange for a bribe.
It is not completely clear from the text, but the more plausible reading is that some of the victims may have been taken down from the cross before they died.
In this light it should not be considered as implausible that a member of the Sanhedrin would have been able to bargain with the centurion to convince Pilate Jesus already died.
Josephus, Jewish War 4.5.2 (333): He says he recognized three of his acquaintances being crucified, asked for their removal, and one of them survived.
Indications do not fully end at the burial, Thomas’ request to touch wounds (John 20:27) only makes sense if the wounds are fresh and still healing—rather than glorified.
Summarizing, although partially speculative, there is a lot of speculation that matches unexpectedly well on close inspection.
Addition of so much of this detail We need to consider the possibilities that Jesus really died and then numerous propaganda stories were created with strictly controlled narrative.
The non-resurrection theory actually does suffer very substantially from the problem of consistent narrative.
The ungrounded claims would not be corroborated by everyone in the same way.
There is bound to be more serious discrepancies in the story and more variants of the story.
To witness the fact, there is famously very serious discrepancy between all the gospels as to how the resurrection was discovered.
This actually strongly corroborates the idea that a lot of earlier highly consistent narratives were actually independently attested in the gospels, while the discovery of the missing body must have been a deliberate attempt to cover up the actual story.
The other alternative is that there really was empty tomb and misunderstanding.
For example, Joseph and Nicodemus really did use a temporary grave and then moved Jesus to a different grave without telling anyone.
Then the women came to the grave and found it empty and spread the news to Peter and John and so the story started to spread.
Then the empty grave was undeniable but everyone doubted the resurrection as they had no certainty that Jesus was resurrected or simply his body was secretly moved.
So here we need to consider the plausibility that Jesus survived because of luck or a conspiracy of Joseph of Arimathea and Nicodemus, and contrast it with the alternative of the event being completely fabricated.
\subsection{Was God the Father the God of Moses or the God of Plato?}\label{subsec:was-god-the-father-the-god-of-moses-or-the-god-of-plato}
In modern Christianity it is completely beyond doubt that God the Father is YHWH, the God of Moses, the God of Abraham, Isaac and Jacob.
Not many are aware that this is actually a surprisingly recent development in Christianity.
Many believe that the distinction of God the Father from the God of Moses was the heresy of Marcion and one Marcion has been excommunicated, the Church has always believed that God the Father is the God of Moses.
However, this is not the case.
Marcion’s theory went further than that, he believed that the God of Moses was an evil god, and that the God of Jesus was a good god.
This theory was far more radical than the idea that God the Father is not the God of Moses, and that was what was condemned by other thinkers of the early Church.
In the Gospel of John, Jesus is described as the Logos, the Word of God with is very unequivocal reference to the Greek creator deity and not the God of Moses.
A lot more of the philosophy and theology of Logos has been described by Philo who seems to have had very substantial influence on the New Testament texts.
Saint Paul refers to the God the Father by For we are indeed his offspring'' which was one of the most common phrases related to Zeus.
Just a moment later he uses the common greek thought The God who made the world and everything in it\ldots{} does not live in temples\ldots{} we should not think the divine being is like gold or silver\ldots{} In him we live and move and have our being.’’
Perhaps the most common perspective was expressed by Clement of Alexandria ``God is one and the same, the universal Father, being known under many names.’’
An epitaph from the 5th century AD, written by deeply Christian family, reads: ``Weak at birth with defence of crosses walled, Guiltless of any dark stain of sin, Little Theodosius, who with pure mind parents Chose to dip in the sacred baptismal font Cruel death seized.
May the ruler of Olympus’ height Give rest to these members with the noble sign Of the cross marked, proclaiming an heir of Christ.’’
Then we obviously have the images of God
\subsubsection{Pater Noster}\label{subsubsec:pater-noster}
In catechisms and commentaries the Pater Noster is presented as a quintessentially Jewish prayer.
Two Gospel forms survive—Matthew 6:9–13 (Sermon on the Mount) and Luke 11:2–4 (disciple request).
Its usual gloss runs like this: • Our Father in heaven'' echoes synagogue formulas (e.g., the later Kaddish: Exalted and hallowed be His great Name’’).
• Hallowed be thy Name'' = sanctifying YHWH's Name (already holy in Israel). • Thy kingdom come / Thy will be done on earth as in heaven’’ = Israel’s hope from Daniel and the Prophets.
• Give us today our daily bread'' = manna typology or Psalmic providence. • Forgive us our debts\ldots{} as we forgive’’ = Jubilee/Leviticus ethic.
• Lead us not into temptation, but deliver us from evil'' = moral temptations; God's protection from sin. On this reading the prayer is Second-Temple Jewish’’ through and through.
\begin{enumerate}
\def\labelenumi{\arabic{enumi})}
\setcounter{enumi}{1}
\item
Why that neat picture collapses
\end{enumerate}
Read closely, the prayer pointedly avoids Israel’s covenant markers.
There is no Sinai, no Torah, no Zion, no Abraham, no Sabbath, no sacrifices—nothing that nails it to Moses.
Instead, the vocabulary is cosmic, royal, solar: • A universal Father in the heavens, not the covenant God ``who brought you out of Egypt.’’ • A sanctified Name without the Tetragrammaton or Temple.
• A Kingdom that descends from heaven to earth (cosmic axis), not the restoration of David on Mount Zion.
• A petition for daily bread does not echo manna (which was not daily) or psalmic providence (which is not daily).
• A trial (Greek peirasmos) and rescue from the Evil One that ring like an eschatological ordeal with a devouring adversary, not like a generic plea about private temptations.
The standard reading ``works’’ only by importing background the text doesn’t supply, while ignoring the imagery it does supply.
\begin{enumerate}
\def\labelenumi{\arabic{enumi})}
\setcounter{enumi}{2}
\item
The Egyptian solar–royal reading (what fits cleanly, line by line)
\end{enumerate}
\begin{enumerate}
\def\labelenumi{(\alph{enumi})}
\item
Our Father in the heavens'' Egyptian hymns to Aten and Amun-Ra address the high god as father of all; the Pharaoh is the son of the Sun. The address is cosmic, not ethnic. It's the right register for the prayer.
\item Hallowed be thy Name’’ (ἁγιασθήτω τὸ ὄνομά σου) Egyptian piety centers on the Name (rèn).
Amun literally means the Hidden (One)''; his hidden Name is praised and protected. Refrains like Your name is Amun—Amun, Amun’’ are liturgical.
This is precisely a sanctified Name without pronouncing it—a far tighter fit than Moses’ tetragrammaton practices as usually described for lay prayer.
\item
Thy kingdom come. Thy will be done on earth as in heaven.'' Every dawn the Sun restores Maʿat (order) in the heavens and, through the king, on earth. Solar kingship is a heaven-to-earth pipeline of will/order. That is exactly the structure of this petition.
\item Give us today our daily bread’’ (τὸν ἄρτον\ldots{} τὸν ἐπιούσιον) The Great Hymn to Aten praises the god who daily makes bread for humankind.'' Egyptian offering formulas (bread and beer, daily’’) are standard temple language.
This line is almost a quotation in sense.
\item
Forgive us our debts as we forgive our debtors.'' Egypt frames justice as weight at judgment---the heart weighed against the feather of Maʿat. Being set right’’ (absolved of moral weight/debt) is the difference between survival and obliteration.
The ethical turn (as we forgive'') binds the worshiper to enact Maʿat socially. This is far closer to Egyptian moral weight/debt than to later juridical hair-splitting.
\item Do not bring us into the time of trial, but deliver us from the Evil One.’’ (μὴ εἰσενέγκῃς\ldots{} εἰς πειρασμόν\ldots{} ῥῦσαι ἀπὸ τοῦ πονηροῦ) Peirasmos = trial/ordeal, not chiefly temptation.'' Read apocalyptically, it is the great ordeal; read visually, it is the judgment scene. And the Evil One is not an abstraction: in Egyptian iconography the failed soul is devoured by Ammit (crocodile-lion-hippo) as the scales tip. Deliver us from the devourer’’ is exactly how the scene works.
\item
For thine is the kingdom, and the power, and the glory, for ever and ever.'' This is the quintessential royal acclamation, the kind of formula shouted in honor of Pharaohs, Ptolemies, and emperors. In temple liturgy and civic festivals, crowds proclaimed the ruler’s dominion, power, and radiant glory as everlasting—language mirrored on Hellenistic and Roman inscriptions. Appending this doxology plants the prayer squarely in royal–solar ideology, where kingship, might, and glory are affirmed without limit of time.
\item
Amen.'' Before we pretend ``amen’’ is safely, uniquely Hebrew, note the liturgical practice: Egyptian hymns and responses end with acclamations of Amun; congregational call-and-response reinforces the Name. In both Hebrew and Egyptian scripts, the word was originally written only with the consonants *MN* (Hebrew: אמן; Egyptian: jmn), without vowels. In Hebrew this yields “amen” (firm/true); in Egyptian, “Amun” (the Hidden One). The phonetic overlap is not accidental but reflects a shared acclamatory base. Alexander the Great claimed descent from Zeus-Ammon; Ptolemaic kings styled themselves similarly. To say “Amen” thus seals the prayer in the register of Amun’s hidden Name—the god underwriting royal legitimacy. Functionally and phonetically, “All say Amun/Amen!” is the logic.
\end{enumerate}

Bottom line: every clause fits solar-royal liturgy without strain.
The ``Jewish only’’ reading must paper over the prayer’s cosmic grammar; the Egyptian reading doesn’t.

The visual counterpart to this liturgy is the radiant halo—the sunburst crown seen on Alexander, the Ptolemies, and later Roman emperors—imagery that early Christian art freely transferred to Jesus as Son of the cosmic Father.
\begin{enumerate}
\def\labelenumi{\arabic{enumi})}
\setcounter{enumi}{3}
\item
The historical pipeline (why this imagery would still be alive)
\end{enumerate}
This isn’t Bronze-Age dust accidentally stuck to a 1st-century text.
It’s continuous culture: • Egypt in Canaan (c.~1500–1150 BC).
For centuries the southern Levant was an Egyptian province.
Jerusalem appears in the Amarna archive (\textasciitilde1350 BC), with its ruler Abdi-Heba writing Pharaoh as my Sun.'' Egyptian garrisons and cult stood at Beth-Shean, Jaffa, Deir el-Balah, etc. • Davidic psalmic core (c.~1150--970 BC). The linguistically older psalms are drenched in sun, light, heaven, earth, kingship, divine rule (Ps 19; 29; 68; 84; 104). These read like Hebrew adaptations of solar hymns, not Torah homilies. • Aten's revolution and Amun-Ra's supremacy. Akhenaten's Aten-monotheism collapses, but Amun-Ra returns stronger; the solar-monotheist pressure never disappears. • Ptolemaic Egypt (3rd--1st c.~BC). The dynasty crafts Serapis/Isis cult and keeps solar divine kingship explicit. Cleopatra's death (30 BC) is within grandparent memory of Jesus' generation. • Jesus' milieu. A king of the Jews’’ claim sits inside Roman Syria-Palestine, saturated with Helios/Sol imagery.
Early Christian art happily paints Christ as Helios; the holy day is Sunday.
The solar-royal idiom is not alien—it is the water everyone swims in.
Seen through this pipeline, the Pater Noster doesn’t borrow a few Egyptian phrases; it belongs to the solar-royal register that ran from Aten → Amun-Ra → Ptolemaic kingship straight into the first century. In cosmopolitan hubs like Alexandria and Antioch, prayers themselves circulated and fused idioms; the Lord’s Prayer fits this world of shared solar-royal language intelligible across traditions.

\begin{enumerate}
\def\labelenumi{\arabic{enumi})}
\setcounter{enumi}{4}
\item
``But isn’t it still Jewish?’’ — the honest reconciliation
\end{enumerate}
Yes, the prayer can be prayed in a Jewish key (and was).
Luke embeds it in a lesson on dependence; Matthew frames it within piety and forgiveness.
The wording genuinely overlaps later synagogue language (``hallowed be His Name’’).
But that overlap proves adaptability, not origin.
Crucially, the prayer: • avoids covenant particulars, • speaks in universal solar-cosmic grammar, and • lands perfectly inside Egyptian/Ptolemaic royal theology.
The better model is fusion: Israel’s high-God devotion absorbed, translated, and reused the dominant solar-royal grammar everyone understood.
If Jesus stands—as our thesis argues—as a royal claimant in the wake of a just-collapsed Ptolemaic world, then this prayer reads not as a synagogue formula but as a dynastic solar-royal prayer: heavenly Father (Sun), kingdom descending, daily sustenance, righteous scales, rescue from the devourer—Amen.

\begin{enumerate}
\def\labelenumi{\arabic{enumi})}
\setcounter{enumi}{5}
\item
Clause-by-clause gloss (for the reader who wants the map) • Father in heaven → Solar source (Aten/Amun-Ra) and royal sonship.
• Hallowed Name → the hidden Name exalted (Amun’s rèn).
• Kingdom come / will be done → Maʿat restored from heaven to earth via the king.
• Daily bread → the sun-god who daily provides bread.
• Forgive debts → lighten the moral weight at the scales; enact Maʿat with others.
• Do not bring into the trial → spare us the ordeal/judgment.
• Deliver from the Evil One → rescue from the devourer who consumes the failed.
• Amen → the communal seal, functionally identical to Amun-acclamation.
\end{enumerate}
Why this matters for David and Jesus • David (1150–970 BC) sits close enough to the Amarna horizon for Egyptian solar kingship to be living memory; the oldest psalms sound like it because they grew in it.
• Jesus (early 1st c.~AD) stands within living memory of Ptolemaic solar monarchy.
If he is (as our book argues) a royal figure inside that political theology, the Pater Noster is exactly the kind of solar-royal prayer a claimant would teach: it translates Egypt’s oldest grammar of kingship into a form his followers can pray anywhere.
That is the reading that explains everything the text actually says—without importing Sinai—and explains why the prayer crossed languages and empires so effortlessly.

\subsubsection{Jesus Christ referring to Our Father in Heaven and YHWH}\label{subsubsec:jesus-christ-referring-to-our-father-in-heaven-and-yhwh}

\paragraph{Thesis and method.}
Jesus’ speech consistently distinguishes \emph{the Father in the heavens} (whose essence is mercy, generosity, forgiveness) from the \emph{Kyrios} invoked in Israel’s scriptures and civic religion. The Gospels preserve this distinction at three levels: (1) Jesus’ own address to God as “my Father” and the teaching of “our Father”; (2) his polemics with Judeans who claim God as their Father; (3) his quoting of Scripture where the Tetragrammaton (YHWH) appears as \emph{Kyrios} in Greek, alongside his reassignment of \emph{Kyrios} as a royal title to himself.

\paragraph{A. The Father in heaven: the ethic of universal mercy (quotes).}
\begin{itemize}
  \item \textbf{Love for enemies as filial likeness:} “Love your enemies and pray for those who persecute you, \emph{so that you may be sons of your Father who is in heaven}; for \emph{he makes his sun rise on the evil and on the good}, and sends rain on the just and on the unjust.” (Matt 5:44–45).
  \item \textbf{Be merciful as the Father:} “He is kind to the ungrateful and the evil. \emph{Be merciful, even as your Father is merciful.}” (Luke 6:35–36).
  \item \textbf{Perfection = indiscriminate goodness:} “You shall be perfect, as \emph{your heavenly Father} is perfect.” (Matt 5:48).
  \item \textbf{Forgiveness as the Father’s policy:} “If you forgive others their trespasses, \emph{your heavenly Father} will also forgive you…” (Matt 6:14–15).
  \item \textbf{The Hidden Father:} “When you give to the needy… so that your giving may be in secret; and \emph{your Father who sees in secret will reward you}.” (Matt 6:3–4). Likewise, “pray to your Father who is in secret; and \emph{your Father who sees in secret will reward you}.” (Matt 6:6). Here the Father is literally called “hidden” (ἐν τῷ κρυπτῷ), resonating with Amun’s epithet “the Hidden One.”
  \item \textbf{Provision without partiality:} “\emph{Your heavenly Father} knows that you need them all.” (Matt 6:32); “How much more will \emph{your Father who is in heaven} give good things to those who ask him!” (Matt 7:11).
  \item \textbf{Jesus’ own address and mission:} “\emph{Abba, Father}, all things are possible for you…” (Mark 14:36). “God did not send the Son into the world to condemn the world, but… to save the world.” (John 3:17); “I did not come to judge the world but to save the world.” (John 12:47).
  \item \textbf{The cross as intercession, not retribution:} “\emph{Father, forgive them}…” (Luke 23:34).
\end{itemize}

\paragraph{B. Jesus’ polemic: denial that Israel’s claim (“God is our Father”) stands as-is.}
\begin{quote}
“They said to him, ‘We have one Father—God himself.’ Jesus said to them, ‘\emph{If God were your Father, you would love me}… \emph{You are of your father the devil}.’” (John 8:41–44).
\end{quote}

\paragraph{C. Where \emph{Kyrios} appears: Scripture citation and royal transfer.}
\begin{enumerate}
  \item \textbf{Scripture citation (YHWH → Kyrios in Greek):} Deut 8:3 // Matt 4:4; Deut 6:16 // Matt 4:7; Deut 6:13 // Matt 4:10; Mark 12:29 (Shema). Jesus quotes Israel’s Scriptures \emph{without} adopting the Tetragrammaton, which appears as \emph{Kyrios} in Greek.
  \item \textbf{Royal transfer of \emph{Kyrios}:} “The Son of Man is \emph{Lord of the Sabbath}” (Matt 12:8). “Not everyone who says to me, \emph{‘Lord, Lord,’}… but the one who does the will of \emph{my Father} who is in heaven.” (Matt 7:21).
  \item \textbf{Two ‘Lords’ text (Ps 110:1):} “\emph{The Lord said to my Lord}…” (Mark 12:36, \emph{Eipen Kyrios tō Kyriō mou}).
\end{enumerate}

\paragraph{D. The Name and the Amen.}
\begin{itemize}
  \item \textbf{Name:} “\emph{Holy Father}, keep them \emph{in your name}…” (John 17:11); “I have \emph{manifested your name}…” (John 17:6). This pairs with “Hallowed be your Name” and Egyptian \emph{r\`en} piety (hidden Name/Amun), without pronouncing YHWH.
  \item \textbf{Amen as oath-formula:} Jesus uniquely fronts sayings with “\emph{Amen, amen}, I say to you …” (esp. in John). This turns the congregational seal (*MN* → Amen/Amun) into his signature of authority, aligning the Father’s truth/hiddenness with the Son’s proclamation. A striking parallel appears in the Greek Magical Papyri from Roman Egypt (PGM IV, 1st–3rd c.~AD): “ἀληθῶς ἀληθῶς, κύριε, εἰσάκουσόν μου” — “Truly, truly, Lord, hear me.” Here a solar hymn addresses the high god (Helios/Serapis) with the same doubled truth-formula. The correspondence is chronological: these invocations belong to the same world as Jesus, not to distant antiquity. Where Egyptian hymns cry “Truly, truly, Lord, hear,” Jesus claims “Amen, amen, I say to you.” The liturgical structure is the same, but Jesus moves from invocation of the Lord to speaking as the Lord, transferring the solar-hymnic formula into his own mouth as signature of divine authority.
\item \end{itemize}

\paragraph{E. Synthesis.}
(1) \emph{Father}-language clusters around mercy, indiscriminate benevolence, forgiveness—linked to solar provision (Matt 5:45). (2) \emph{Kyrios}-language appears in Scripture citation and royal prerogatives. (3) Jesus denies that the bare claim “God is our Father” stands without recognition of the Son (John 8). (4) The Lord’s Prayer centers the Father’s Name, Kingdom, Bread, Forgiveness, Rescue—the cosmic-royal grammar traced earlier. The visual corollary is the radiant halo: the Father’s light embodied in the royal Son.

\subsubsection{Acts and the Epistles: crystallizing the Father vs. Kyrios distinction}\label{subsubsec:acts-epistles-father-kyrios}
\begin{itemize}
  \item \textbf{Acts (cosmic Father; Kyrios to Jesus):} Acts 2:36 — “God has made him both \emph{Kyrios} and Christ, this Jesus whom you crucified.” Acts 17:28–29 (Areopagus) — “For we are indeed his offspring… we ought not to think that the divine being is like gold or silver or stone.” Paul quotes Greek poets (Aratus/Cleanthes) to assert universal Fatherhood, not a tribal deity.
  \item \textbf{Paul’s creedal split:} 1 Cor 8:5–6 — “For us there is \emph{one God, the Father}, from whom are all things… and \emph{one Lord, Jesus Christ}, through whom are all things…” Philippians 2:11 — “Every tongue confess that Jesus Christ is \emph{Kyrios}, to the glory of God the Father.” Distinct roles: Father = source; Jesus = \emph{Kyrios}.
  \item \textbf{Abba and adoption beyond Torah:} Gal 4:6 — “God has sent the Spirit of his Son into our hearts, crying, ‘Abba! Father!’” Romans 3:29 — “Is God the God of Jews only? Is he not the God of Gentiles also? Yes, of Gentiles also.”
  \item \textbf{Johannine hiddenness and love:} 1 John 4:8 — “God is love.” 1 John 4:12 — “No one has ever seen God.” The unseen/hidden Father coheres with Matt 6’s “who sees in secret,” while Jesus (the \emph{Kyrios}) is manifest.
\end{itemize}
\paragraph{Conclusion.} Acts and the letters do not collapse Father into YHWH; they formalize the lived Gospel distinction: \emph{one God, the Father} (universal, often hidden, source of mercy) and \emph{one Lord, Jesus Christ} (royal \emph{Kyrios}), with solar-royal imagery naturally migrating to the Son as the Father’s visible regency.
\subsection{Jesus Christ believed in true immortal soul}\label{subsec:jesus-christ-believed-in-true-immortal-soul}
Sometimes overlooked, the idea of soul in greek and jewish beliefs is very different.
Do not fear those who kill the body but cannot kill the soul'' ψυχή (psyche) ≠ body (Matt 10:28) What does it profit a man to gain the world and forfeit his soul?’’ (Matt 16:26) ``Today you will be with me in Paradise’’ (Luke 23:43)
The belief in post mortem conscious existence as a continuation of soul is not a belief of the Old Testament, but it is a common belief in the Greek world.
The Jews of that time believed in a resurrection of the body, but not in the immortality of the soul.
Within Judaism itself the debate divided Pharisees (affirming resurrection and often an ongoing soul-life) and Sadducees (denying it); Jesus’ teaching aligns more with the Hellenized Pharisaic/Greco-philosophical view than with Torah-literalist denial.

\subsection{Holy Mary, Mother of God, Perpetual Virgin, Queen of Heaven}\label{subsec:holy-mary-mother-of-god-perpetual-virgin-queen-of-heaven}
\paragraph{1.
Mary was the rightful heiress to the Hasmonean dynasty and so the royal titles we know very well today also correspond to the titles of the Greek rulers.}\label{par:mary-was-the-rightful-heiress-to-the-hasmonean-dynasty-and-so-the-royal-titles-we-know-very-well-today-also-correspond-to-the-titles-of-the-greek-rulers.}
There were countless greek female rulers with highly distinguished, divine titles, such as Cleopatra, who was also a common name in the Hasmonean dynasty.
\paragraph{2.
Holy Mary mother of God.}\label{par:holy-mary-mother-of-god.}
As Jesus was the Son of God, as a rightful heir, Mary as the mother of future ruler would have been considered the Mother of God.
Mother of the Lord'' (Μήτηρ τοῦ Κυρίου) -- Found in Luke 1:43, where Elizabeth calls Mary the Mother of my Lord.’’ This implies royal status since Lord'' (Kyrios) could mean a divine or kingly ruler. Elizabeth greets Mary: And why is this granted to me, that the mother of my Lord should come to me?’’
\paragraph{3.
Mary was a perpetual virgin.}\label{par:mary-was-a-perpetual-virgin.}
A very common trope in the greek world was that the royal women were pure often called perpetual virgins.
This is something that fits really well with the idea of Mary as royal but seems at odds with the idea of Mary as a mother to Jesus.
\paragraph{4.
Mary was the Queen of Heaven.}\label{par:mary-was-the-queen-of-heaven.}
The earliest known hymns and prayers to Mary refer to her as Βασίλισσα τῶν Οὐρανῶν – explicit royal status.
The Woman Clothed with the Sun'' -- Revelation 12:1 A great sign appeared in heaven: a woman clothed with the sun, with the moon under her feet, and a crown of twelve stars on her head.’’ While this passage refers to Israel, early Christian writings (Hippolytus, 3rd century AD) link it to Mary as a royal mother figure. The Isis tradition was not abstract: Hellenistic queens, including Cleopatra, embodied divine motherhood as political theology; Mary’s elevation continues this dynastic-feminine model in Christian form.
\paragraph{5.
Mary was Blessed among all women}\label{par:mary-was-blessed-among-all-women}
Highlighting Mary’s royal lineage
\paragraph{6.
Mary was the New Eve}\label{par:mary-was-the-new-eve}
New Eve'' -- Early Church Fathers (Justin Martyr, Irenaeus) described Mary as the new Eve, implying a role in a divine dynasty. Protoevangelium of James (c.~2nd century AD) -- While emphasizing her perpetual virginity, it also hints at a priestly and royal lineage, calling her set apart for the Lord.’’
