\section{Chapter 4 - Pauline Epistles to All Nations}\label{par:chapter-4---pauline-epistles-to-all-nations}

πορευθέντες οὖν μαθητεύσατε πάντα τὰ ἔθνη ``Go, therefore, and make disciples of all nations.''

The key point missed by most scholars is that panta ta ethnē does not refer to ``all nations'' universally, but rather to the nations of the Greek world.
The Septuagint consistently uses ethnē to describe the nations under God's rule---those invited into covenant---not every nation globally.

Καὶ εὐλογηθήσονται ἐν τῷ σπέρματί σου πάντα τὰ ἔθνη τῆς γῆς ``And in your seed, all the nations of the earth shall be blessed.'' This is a key passage used in early Christianity (Galatians 3:8) to show that all the nations were meant to be part of God's covenant, not just outsiders.

Βασιλεῖς τῆς γῆς καὶ πάντες λαοὶ, ἄρχοντες καὶ πάντες κριταὶ γῆς\ldots{} ὑψώσεως κεράτων λαοῦ αὐτοῦ, ὕμνος πᾶσι τοῖς ὁσίοις αὐτοῦ, τοῖς υἱοῖς Ἰσραὴλ, λαῷ ἐγγίζοντι αὐτῷ.
``Kings of the earth and all peoples, rulers and all judges of the earth\ldots{} a hymn for all His saints, the sons of Israel, a people near to Him.'' The ``kings of the earth and all peoples'' are brought into God's rule, but the focus remains on God's people.

All nations referred to God's people, not necessarily only Greek ethnicity, but all nations who proclaimed the rule of God.
So that would also include Jews and

\paragraph{1.
The early dating of the gospels can make the letters of Paul more plausible as it seems Paul already has the knowledge of at least one of the gospels and the acts.}\label{par:the-early-dating-of-the-gospels-can-make-the-letters-of-paul-more-plausible-as-it-seems-paul-already-has-the-knowledge-of-at-least-one-of-the-gospels-and-the-acts.}

Many scholars dispute the existence of Paul based on the striking contradiction in mainstream scholarship that authors of Pauline epistles seem to have the knowledge of the gospels and the acts, and yet the gospels and the acts are unanimously dated to be written after the Pauline epistles.
In here the existence of Paul can once more be reconsidered if we acknowledge the early dating of the gospels and the gospel of John in being written by an eyewitness of Jesus's life.

\paragraph{2.
What is quite striking is that Paul and others write to so many different churches over the short period of time.}\label{par:what-is-quite-striking-is-that-paul-and-others-write-to-so-many-different-churches-over-the-short-period-of-time.}

There is a major challenge for the traditional timeline of the apostles establishing so many churches in such a short period of time.
These churches would have to all be established, grow, keep up to date with the fastly shifting theology, and then do nearly nothing for the next 100 years.
These churches immediately showed up in every single major city in the former greek empire, and no churches showed up anywhere else.
All of the correspondence and scripture was written in Greek, and no other languages.
It is important to point out that greek was absolutely not the lingua franca in any part of the Roman empire, that was not recently part of the greek empire.
The lingua franca of the Roman empire was Latin, and that was the only language that was used in the administration and the primary language used by the authors.
If the apostles were to establish churches everywhere in the Roman empire, and not just in the former greek empire, then we would have had the epistles to the extremely prominent cities of Mediolanum, Lutetia, Aquilea, Lugdunum, Memphis, and Londinium.
The truth is that no matter how we model the growth of the early church, it is not possible to explain the the patterns we observe.
and then for the next 100 years do not add any new churches.

There were essentially no prominent cities in the former greek empire that were not mentioned in the acts and the epistles:

Judea and Surrounding Regions: \textbar{} Region \textbar{} City \textbar{} Reference(s) \textbar{} Notes \textbar{} \textbar---------------------------\textbar--------------\textbar---------------------------------------\textbar-----------------------------------------------------------------------\textbar{} \textbar{} Judea and Surrounding Regions \textbar{} Jerusalem \textbar{} Acts 1:4, 2:5, 8:1 \textbar{} Center of the early Christian movement and where Pentecost occurred.
\textbar{} \textbar{} \textbar{} Bethany \textbar{} Acts 1:12 \textbar{} Near Jerusalem, where Jesus ascended to heaven.
\textbar{} \textbar{} \textbar{} Joppa \textbar{} Acts 9:36 \textbar{} A port city where Peter stayed and raised Tabitha from the dead.
\textbar{} \textbar{} \textbar{} Caesarea \textbar{} Acts 8:40, 10:1 \textbar{} Where Philip preached and where Cornelius, a Gentile, was baptized by Peter.
\textbar{} \textbar{} \textbar{} Antioch \textbar{} Acts 11:19 \textbar{} A key city for early Christian missions and where followers of Jesus were first called Christians.
\textbar{} \textbar{} \textbar{} Nazareth \textbar{} Acts 2:22 \textbar{} Hometown of Jesus, mentioned in the sermon at Pentecost.
\textbar{} \textbar{} Asia Minor (Modern-day Turkey) \textbar{} Tarsus \textbar{} Acts 9:11 \textbar{} The birthplace of Paul (Saul).
\textbar{} \textbar{} \textbar{} Lystra \textbar{} Acts 14:6 \textbar{} Where Paul healed a crippled man and was nearly stoned.
\textbar{} \textbar{} \textbar{} Derbe \textbar{} Acts 14:6 \textbar{} Where Paul and Barnabas preached and made many disciples.
\textbar{} \textbar{} \textbar{} Iconium \textbar{} Acts 14:1 \textbar{} Where Paul preached and faced opposition from the local Jewish authorities.
\textbar{} \textbar{} \textbar{} Ephesus \textbar{} Acts 18:19 \textbar{} A major city in Asia Minor where Paul spent a significant amount of time preaching and establishing the church.
\textbar{} \textbar{} \textbar{} Miletus \textbar{} Acts 20:15 \textbar{} Where Paul met with the Ephesian elders on his way to Jerusalem.
\textbar{} \textbar{} \textbar{} Smyrna, Pergamum, Thyatira, Sardis, Philadelphia, Laodicea \textbar{} Revelation letters \textbar{} Acts doesn't mention directly but likely influenced early Christian activity.
\textbar{} \textbar{} Greece \textbar{} Philippi \textbar{} Acts 16:12 \textbar{} Where Paul and Silas were imprisoned and where the first European Christian church was founded.
\textbar{} \textbar{} \textbar{} Thessalonica \textbar{} Acts 17:1 \textbar{} A major city where Paul preached and faced opposition.
\textbar{} \textbar{} \textbar{} Berea \textbar{} Acts 17:10 \textbar{} Where Paul went after Thessalonica and found the Bereans to be more receptive to the gospel.
\textbar{} \textbar{} \textbar{} Athens \textbar{} Acts 17:16 \textbar{} Where Paul preached on Mars Hill and engaged with philosophers about the ``unknown god.'' \textbar{} \textbar{} \textbar{} Corinth \textbar{} Acts 18:1 \textbar{} Where Paul stayed and established a Christian community, and where he later wrote 1 and 2 Corinthians.
\textbar{} \textbar{} Macedonia and the Surrounding Areas \textbar{} Neapolis \textbar{} Acts 16:11 \textbar{} Port city in Macedonia, where Paul and his companions arrived after sailing from Troas.
\textbar{} \textbar{} \textbar{} Philippi \textbar{} Acts 16:12 \textbar{} Mentioned earlier in Greece.
\textbar{} \textbar{} Egypt \textbar{} Alexandria \textbar{} Acts 6:9, 18:24 \textbar{} Apollos was from here; a major Jewish and early Christian hub.
\textbar{} \textbar{} Libya (North Africa) \textbar{} Cyrene \textbar{} Acts 2:10, 11:20 \textbar{} Home of Simon of Cyrene; some early Christians were from here.
\textbar{} \textbar{} Italy and Rome \textbar{} Puteoli \textbar{} Acts 28:13 \textbar{} Port city in Italy where Paul arrived after sailing from Malta.
\textbar{} \textbar{} \textbar{} Rome \textbar{} Acts 28:16 \textbar{} Where Paul was taken as a prisoner and spent two years under house arrest.
\textbar{} \textbar{} Other Notable Cities \textbar{} Cyprus \textbar{} Acts 13:4 \textbar{} Where Paul and Barnabas first traveled for missionary work.
\textbar{} \textbar{} \textbar{} Salamis \textbar{} Acts 13:5 \textbar{} A city in Cyprus where Paul preached.
\textbar{} \textbar{} \textbar{} Paphos \textbar{} Acts 13:6 \textbar{} A city in Cyprus where Paul encountered the sorcerer Elymas.
\textbar{} \textbar{} \textbar{} Patara \textbar{} Acts 21:1 \textbar{} A port city in Lycia where Paul caught a ship to Phoenicia.
\textbar{} \textbar{} \textbar{} Tyre \textbar{} Acts 21:3 \textbar{} A city in Phoenicia where Paul stopped to meet the disciples.
\textbar{}

Regions and Cities in the Epistles:

\begin{longtable}[]{@{}
  >{\raggedright\arraybackslash}p{(\linewidth - 4\tabcolsep) * \real{0.1429}}
  >{\raggedright\arraybackslash}p{(\linewidth - 4\tabcolsep) * \real{0.1333}}
  >{\raggedright\arraybackslash}p{(\linewidth - 4\tabcolsep) * \real{0.7238}}@{}}
\toprule\noalign{}
\begin{minipage}[b]{\linewidth}\raggedright
Location
\end{minipage} & \begin{minipage}[b]{\linewidth}\raggedright
Mentioned In
\end{minipage} & \begin{minipage}[b]{\linewidth}\raggedright
References
\end{minipage} \\
\midrule\noalign{}
\endhead
\bottomrule\noalign{}
\endlastfoot
Rome & Acts, Romans, Philippians, 2 Timothy & Acts 28, Romans 1:7, 1:15, Philippians 1:13, 2 Timothy 4:16-17 \\
Corinth & Acts, 1 Corinthians, 2 Corinthians & 1 Corinthians 1:2, 2 Corinthians 1:1 \\
Ephesus & Acts, Ephesians & Ephesians 1:1 \\
Galatia & Galatians & Galatians 1:2 \\
Philippi & Acts, Philippians & Philippians 1:1 \\
Thessalonica & 1 Thessalonians, 2 Thessalonians & 1 Thessalonians 1:1, 2 Thessalonians 1:1 \\
Colossae & Colossians & Colossians 1:2 \\
Laodicea & Colossians, Revelation & Colossians 4:13-16, Revelation 3:14-22 \\
Crete & Titus & Titus 1:5 \\
Cyprus & Galatians & Galatians 4:13 \\
Pontus, Galatia, Cappadocia, Asia, Bithynia & 1 Peter & 1 Peter 1:1 \\
Macedonia & 2 Corinthians, Philippians & 2 Corinthians 8:1, Philippians 4:15 \\
Miletus & 2 Timothy & 2 Timothy 4:20 \\
Antioch & Acts, Galatians, 1 Corinthians & Galatians 2:11, 1 Corinthians 9:6 \\
Tarsus & Acts, 2 Corinthians & Acts 9:11, 2 Corinthians 11:22 \\
Syria & 1 Corinthians, Galatians, 2 Corinthians & 1 Corinthians 16:3, Galatians 1:21, 2 Corinthians 11:9 \\
Asia & 1 Corinthians, 2 Corinthians, Revelation & 1 Corinthians 16:19, 2 Corinthians 1:8 \\
Troas & Acts, 2 Timothy & Acts 20:6, 2 Timothy 4:13 \\
Berea & Acts, 1 Thessalonians & Acts 17:10, 1 Thessalonians 1:7 \\
Paphos & Acts, Titus & Titus 1:5 \\
Puteoli & Romans & Romans 16:3-4 \\
\end{longtable}

It is important to actually visualize the locations mentioned in the acts and the epistles to see the striking pattern of the locations mentioned in the acts and the epistles being the same as the most significant greek speaking cities.

\begin{figure}
\centering
\pandocbounded{\includegraphics[keepaspectratio,alt={Description of the image}]{locations_map.png}}
\caption{Description of the image}
\end{figure}

For those geographically inclined you can spot the near perfect correlation with the borders of Eastern Roman Empire.
Of note is the trip to Rome, which was substantially different in nature to the other trips.
\href{https://en.wikipedia.org/wiki/Byzantine_Empire_under_the_Theodosian_dynasty\#/media/File:4KTHEODOSIAN.png}{Rome Map}

\paragraph{3.
The striking statistics of the cities mentioned in the acts and the epistles are that they are all in the former greek empire, and not one mention of a city in the Roman empire that was not part of the former greek empire.}\label{par:the-striking-statistics-of-the-cities-mentioned-in-the-acts-and-the-epistles-are-that-they-are-all-in-the-former-greek-empire-and-not-one-mention-of-a-city-in-the-roman-empire-that-was-not-part-of-the-former-greek-empire.}

This fact makes any theory that deems Christianity as a religious and not a political movement immediately highly implausible.

\paragraph{4.
Finally we consider the apparent minimal resistance to the acceptance of the new religion.}\label{par:finally-we-consider-the-apparent-minimal-resistance-to-the-acceptance-of-the-new-religion.}

The new religion was accepted by the masses in the former greek empire, and not a single mention of any resistance to the new religion.
This can be explained if this was already a pre-existing imperial cult.

\paragraph{5.
We should also consider that even though the religion was so successful at converting the masses, it still had all the conspiratorial parts to it.}\label{par:we-should-also-consider-that-even-though-the-religion-was-so-successful-at-converting-the-masses-it-still-had-all-the-conspiratorial-parts-to-it.}

Early christians used secret symbols to identify each other, they frequently met in secret, often at night in the catacombs.

\paragraph{6.
Consider why the religion was seemingly much more prosecuted than any other religion in the Roman empire.}\label{par:consider-why-the-religion-was-seemingly-much-more-prosecuted-than-any-other-religion-in-the-roman-empire.}

The Roman empire was very tolerant of other religions, and the only time they would prosecute a religion was if it was a threat to the empire.
There are some claims that other religions were not exclusive and people could believe in multiple gods, and give offerings to Zeus, but very little has ever been produced to support this very dubious claim.

\paragraph{7.
The phrase ``soldiers of Christ'' is not used explicitly in the Gospels, but it appears prominently in the Pauline Epistles, particularly in the context of the Christian life being compared to a military struggle or a spiritual battle.}\label{par:the-phrase-soldiers-of-christ-is-not-used-explicitly-in-the-gospels-but-it-appears-prominently-in-the-pauline-epistles-particularly-in-the-context-of-the-christian-life-being-compared-to-a-military-struggle-or-a-spiritual-battle.}

The most famous reference to ``soldiers of Christ'' comes from 2 Timothy 2:3-4, where Paul uses military imagery to describe the commitment and discipline required for Christian ministry: 2 Timothy 2:3-4 (NIV): ``Join with me in suffering, like a good soldier of Christ Jesus.
No one serving as a soldier gets entangled in civilian affairs, but rather tries to please his commanding officer.'' This metaphor emphasizes the dedication and discipline expected from Christians, likening them to soldiers who are focused on their mission and loyalty to their leader, which in this case is Christ Jesus.

\paragraph{8.
Paul barely mentions the life of Jesus, and almost never quotes him.}\label{par:paul-barely-mentions-the-life-of-jesus-and-almost-never-quotes-him.}

It is frequently claimed that Paul's religion is not the religion of Jesus, but the religion about Jesus.
There is a shocking lack of references to any of the teachings of Jesus, the Jewish law, and any of the events surrounding Jesus's life and death.
So we may go one step further.
It is a religion focusing on restoring the kingdom of God by resurrecting the office of the Christos, the rightful king of the kingdom of God.
And so to Paul and all the early Christian, it was a all about resurrecting a Christ not teaching of the particular Jesus Christ.
The idea that the God will once again send a king that will restore the greek empire, the kingdom of God, headed by Christos, the rightful earthly king of the kingdom of God.

\paragraph{9.
Using this conspiratorial language clearly worked}\label{par:using-this-conspiratorial-language-clearly-worked}

The Rome did not even realize the new religion's goal was to restore the Eastern Empire until it actually happened.

\paragraph{10.
Alexandria was the capital of the Greek Empire and the center of the Hellenistic world and yet there are no missions or letters to Alexandria.}\label{par:alexandria-was-the-capital-of-the-greek-empire-and-the-center-of-the-hellenistic-world-and-yet-there-are-no-missions-or-letters-to-alexandria.}

The absence of Alexandria in the New Testament is striking, especially considering its significance and proximity to Jesus's life and the early Christian movement.
Alexandria was erased from the text, as it was the origin city of Apollos, as well as some of the other companions or contacts of Paul in like Mark, Demas, and Luke.
(Mark and Luke that are often presumed to be the evangelists).
Although we do not know if Mark mentioned in Pauline epistles is the same Mark as the one who wrote the Gospel of Mark, and that he really came from Alexandria, cities of origin is something we can be more trusting about as people making up tradition are less likely to make up the city of origin.

We are looking here at Alexandria missing as the largest city while the next 30 cities or regions are mentioned.
An interpretation withing this theory was that Alexandria was the center of the Greek Empire and the apostles came from there to other nations to spread the word of God, the word that originally came from Alexandria.
So they were envoys from Alexandria to all nations.
At the same time we know the Christian movement was very strong in Alexandria from the very oldest records on early Christianity we have.
When we look at some of the most influential early church fathers, we see Origen, Clement of Alexandria, and Athanasius of Alexandria.

Since Alexandria was such an integral part of the Greek imperial world, the apostles or early missionaries didn't have to make the same kind of outreach there that they did in other regions.
Their mission was not to convince Alexandria to join the cause, but to spread it elsewhere in the Greek world, with the understanding that Alexandria was already aligned with the goal of restoring the empire.

This is not too dissimilar from how Ptolemaic empire is almost completely missing from the Old Testament even though it was where the story of the Old Testament took place.

The absence of Alexandria in the New Testament is very hard to explain within any common theories of historical Jesus.
We see a lot of activity and communication with the second and third most significant cities which are Antioch and Ephesus, that also had a very strong Jewish presence, also heavily influential, also deeply philosophical and well educated for the era, so it was not like apostles tried to prioritize major cities less, or cities with a strong Jewish presence or cities with where convincing population to change the views could be harder.

\subsection{Acts of the Apostles}\label{par:acts-of-the-apostles}

Is called the Acts of the Apostles, not acts of the disciples.
Apostles doing imperial work of letting all nations of the empire know the will of the God king.

\paragraph{10.
Acts opens with a royal enthronement}\label{par:acts-opens-with-a-royal-enthronement}

Acts 1:6 --- ``Lord, will you at this time restore the kingdom to Israel?'' This is not a spiritual question.
It implies Jesus had a claim to political kingship.
Your theory: Jesus was understood as the rightful monarch of a revived kingdom---a successor to the Herodian or Hasmonean thrones under Greek imperial ideals.

\paragraph{10.
Jesus is taken up like an emperor}\label{par:jesus-is-taken-up-like-an-emperor}

Acts 1:9--11 --- The Ascension mimics apotheosis scenes (e.g., Alexander, Roman emperors).
It frames Jesus in imperial terms, being enthroned in heaven---like a divine emperor.
This matches your view that Christianity was about loyalty to the ``Christ Emperor.''

\paragraph{10.
The Pentecost scene mimics an imperial inauguration}\label{par:the-pentecost-scene-mimics-an-imperial-inauguration}

Acts 2 --- Multilingual miracle and mass conversion reflects the imperial ideal of uniting nations under one divine king.
This isn't random spiritualism---it reflects a Hellenistic, cosmopolitan imperial theology.
The language of ``tongues'' is political: the emperor's message is for all nations.

\paragraph{10.
Acts 5: The trial of the apostles}\label{par:acts-5-the-trial-of-the-apostles}

Gamaliel references past revolutionary figures---Theudas and Judas the Galilean.
This acknowledges that messianic revolts were political, and that Jesus' movement was seen in similar terms.
It subtly affirms that Jesus was considered a royal claimant and threat.

\paragraph{10.
Stephen's speech in Acts 7 is anti-Temple}\label{par:stephens-speech-in-acts-7-is-anti-temple}

Stephen attacks the Temple and Mosaic tradition, echoing Philo and Stoic-influenced criticisms of Jewish legalism.
This supports your view that early Christianity rejected the Mosaic religion and aligned more with philosophical monotheism.

\paragraph{10.
Paul as imperial envoy}\label{par:paul-as-imperial-envoy}

Paul appeals to Caesar, travels through Greek cities, and preaches to Hellenized elites.
His speeches (e.g., Acts 17 in Athens) are clearly political-philosophical, not sectarian Jewish.
Acts frames Paul as a kind of philosopher-diplomat for the Christ-emperor, aligning with your imperial theology model.

\paragraph{11.
Acts ends without resolution}\label{par:acts-ends-without-resolution}

The book ends in Rome, with Paul freely preaching ``the kingdom of God.'' It lacks a narrative climax because its real message is that the empire is now Christian.
It presumes a pre-existing audience that sees Christianity as a political-theological force.

\subsubsection{James the Just}\label{par:james-the-just}

\paragraph{1.
James the Just also wrote an epistle to all nations.}\label{par:james-the-just-also-wrote-an-epistle-to-all-nations.}

He was the brother of Jesus, and the next in line to the throne.

Much like Jesus Christ the Soter, James also held a royal title, the Just.

James the Just also wrote an epistle to all nations, which is included in the New Testament.
Interestingly, James, like John, refers to the same understanding of Logos as Philo of Alexandria.

The writing style of James and John also bear a striking resemblance to the writing style of Philo of Alexandria.

In Greek, James 1:21 reads as: ``Διὸ ἀποθέμενοι πάσαν ἀκαθαρσίαν καὶ περισσείαν κακίας ἐν πραΰτητι δέξασθε τὸν ἐμφυτον λόγον, ὃς δύναται σῶσαι τὰς ψυχὰς ὑμῶν.'' Transliteration: ``Dio apothemenoi pasan akatharsian kai perisseian kakias en prautēti dexasthē ton emphuton logon, hos dynatai sōsai tas psychas hymōn.'' A literal translation would be: ``Therefore, putting away all filthiness and the overflow of wickedness, with meekness receive the implanted word, which is able to save your souls.''

It should go without saying that the advanced writing style of James and John is not something that would be expected from a simple fisherman or a son of a carpenter.
