With a deeper understanding of the historical Jesus, the interpretation of the events that followed his death can change quite dramatically as well.

In the context of an influential royal figure with a large following and with early texts in circulation, the early Christian movement can be seen in a new light.

If we acknowledge that Jesus was framed in Greek and Egyptian royal cult, then Christianity appears from the very beginning as an extension of that tradition.
Jesus did not emerge in a vacuum with completely new ideas.
He was a continuation of the recently fallen Greek empire and inherited a movement that already spanned a major part of Rome.

In this chapter we will try to view the events of the New Testament after Jesus’ death in the light of what we found in previous chapters.
As the Gospel of Matthew closes: πορευθέντες οὖν μαθητεύσατε πάντα τὰ ἔθνη — “Go, therefore, and make disciples of all nations.”

\section{Who were the gentiles?}\label{sec:who-were-the-gentiles}

The word “gentiles” has long been misapplied in the context of the New Testament.
It is almost universally taken to mean “anyone who is not Jewish.”
The confusion comes from the Hebrew word \textit{goyim}, which in many contexts does carry this meaning.
While in the Septuagint, \textit{ethnē} often is used as a translation of \textit{goyim}, it does not show any sign of the same meaning in the context of the New Testament or other Hellenistic works and inscriptions.
The key point missed by most interpreters is that \textit{panta ta ethnē} does not refer to \textit{goyim} or to all peoples universally, but to the nations of the Greek world.
In Hellenistic inscriptions, the phrase “all the nations” (\textit{panta ta ethnē}) is repeatedly used for the \textit{oikoumene}, the civilized world subject to the emperor.
It was imperial language, not global anthropology.
Thus, Matthew’s Great Commission is not an open-ended summons to the Americas or India, but to the Greek-speaking nations of the empire.

Καὶ εὐλογηθήσονται ἐν τῷ σπέρματί σου πάντα τὰ ἔθνη τῆς γῆς — “And in your seed all the nations of the earth shall be blessed” (Gen 22:18 LXX).
This verse is quoted in Galatians 3:8 to argue that “all the nations” were meant to be part of God’s covenant.
Here the nations are not portrayed as outsiders, but as peoples folded into God’s promise.

Βασιλεῖς τῆς γῆς καὶ πάντες λαοὶ, ἄρχοντες καὶ πάντες κριταὶ γῆς… ὕμνος πᾶσι τοῖς ὁσίοις αὐτοῦ, τοῖς υἱοῖς Ἰσραήλ, λαῷ ἐγγίζοντι αὐτῷ — “Kings of the earth and all peoples, rulers and all judges of the earth… a hymn for all His saints, the sons of Israel, a people near to Him” (Ps 148:11, 14 LXX).
The kings and peoples of the earth are included in God’s rule, yet the focus remains on Israel as the center.


“All the nations” in these texts does not mean “non-Jews.”
It refers to all peoples under God’s sovereignty, which in a Hellenistic and imperial context meant the nations integrated into the covenantal and political order of the \textit{oikoumene}.
That included Jews as well as Greeks and others within the world of the recently fallen Greek empire.

\section{The early dating of the gospels can make the letters of Paul more plausible as it seems Paul already has the knowledge of at least one of the gospels and the acts.}\label{sec:the-early-dating-of-the-gospels-can-make-the-letters-of-paul-more-plausible-as-it-seems-paul-already-has-the-knowledge-of-at-least-one-of-the-gospels-and-the-acts.}

Many scholars dispute the existence of Paul based on the striking contradiction in mainstream scholarship that authors of Pauline epistles seem to have the knowledge of the gospels and the acts, and yet the gospels and the acts are unanimously dated to be written after the Pauline epistles.
In here the existence of Paul can once more be reconsidered if we acknowledge the early dating of the gospels and the gospel of John in being written by an eyewitness of Jesus's life.

\subsection{What is quite striking is that Paul and others write to so many different churches over the short period of time.}\label{subsec:what-is-quite-striking-is-that-paul-and-others-write-to-so-many-different-churches-over-the-short-period-of-time.}

There is a major challenge for the traditional timeline of the apostles establishing so many churches in such a short period of time.
These churches would have to all be established, grow, keep up to date with the fastly shifting theology, and then do nearly nothing for the next 100 years.
These churches immediately showed up in every single major city in the former Greek empire, and no churches showed up anywhere else.
All of the correspondence and scripture was written in Greek, and no other languages.
It is important to point out that Greek was absolutely not the lingua franca in any part of the Roman empire that was not recently part of the Greek empire.
The lingua franca of the Roman empire was Latin, and that was the only language that was used in the administration and the primary language used by the authors.
If the apostles were to establish churches everywhere in the Roman empire, and not just in the former Greek empire, then we would have had the epistles to the extremely prominent cities of Mediolanum, Lutetia, Aquilea, Lugdunum, Memphis, and Londinium.
The truth is that no matter how we model the growth of the early church, it is not possible to explain the the patterns we observe.
And then for the next 100 years do not add any new churches.

Even more striking is the total absence of Aramaic or Hebrew letters.
If the apostles’ mission were Jewish in nature, at least some correspondence would survive in the languages of Judea.
Instead, all the letters are in Greek, to Greek assemblies, proving that the movement was imperial and Hellenistic at its core.

\subsection{Paul’s letters as state correspondence.}\label{subsec:pauls-letters-as-state-correspondence.}

The epistles resemble circular letters of the Hellenistic and Roman administrations.
They are addressed to \textit{ekklesiai}, the same word used for political assemblies of citizens in Greek cities.
Paul is not writing to small house cults, but to recognized civic assemblies under a higher king.
This gives the Pauline corpus the character of imperial decrees rather than private exhortations.

There were essentially no prominent cities in the former Greek empire that were not mentioned in the Acts and the epistles:
These letters name couriers and order public reading and inter-city circulation, exactly as civic circulars do (cf.\ the exchange instructions for neighboring assemblies).
They carry greetings from co-workers the way chancery letters list co-signatories.
They include directives, adjudications, financial instructions, muster language, and discipline protocols.
They function as decrees to be read \emph{in assembly} and enacted across a network of poleis.
Nothing here looks like a private house cult.
Everything reads like state correspondence to the existing \textit{ekklesiai} of the Greek cities, continuing their business under the banner of the \textit{Christos}.

\subsection{Ekklesia — the civic assembly continued}\label{subsec:ekklesia-the-civic-assembly-continued}

The word ἐκκλησία did not name a new religious club.
It was the same civic assembly that governed the Greek city before the Roman conquest.
It was the same sovereign body of citizens that passed laws, heard proclamations, and received official dispatches.
After conquest the population and magistrates sought continuity, not reinvention.
They kept their assemblies, their procedures, and their language.
Only the allegiance changed.
The assembly now recognized the rightful \textit{Christos} instead of Rome.
This is why all the correspondence is in Greek.
This is why the addressees are the historic poleis of the Greek world.
Paul is not inventing a “church.”
He is addressing the standing political assemblies of the Greek cities as they realign themselves to the true king.

There were essentially no prominent cities in the former greek empire that were not mentioned in the acts and the epistles:

Judea and Surrounding Regions: \textbar{} Region \textbar{} City \textbar{} Reference(s) \textbar{} Notes \textbar{} \textbar---------------------------\textbar--------------\textbar---------------------------------------\textbar-----------------------------------------------------------------------\textbar{} \textbar{} Judea and Surrounding Regions \textbar{} Jerusalem \textbar{} Acts 1:4, 2:5, 8:1 \textbar{} Center of the early Christian movement and where Pentecost occurred.
\textbar{} \textbar{} \textbar{} Bethany \textbar{} Acts 1:12 \textbar{} Near Jerusalem, where Jesus ascended to heaven.
\textbar{} \textbar{} \textbar{} Joppa \textbar{} Acts 9:36 \textbar{} A port city where Peter stayed and raised Tabitha from the dead.
\textbar{} \textbar{} \textbar{} Caesarea \textbar{} Acts 8:40, 10:1 \textbar{} Where Philip preached and where Cornelius, a Gentile, was baptized by Peter.
\textbar{} \textbar{} \textbar{} Antioch \textbar{} Acts 11:19 \textbar{} A key city for early Christian missions and where followers of Jesus were first called Christians.
\textbar{} \textbar{} \textbar{} Nazareth \textbar{} Acts 2:22 \textbar{} Hometown of Jesus, mentioned in the sermon at Pentecost.
\textbar{} \textbar{} Asia Minor (Modern-day Turkey) \textbar{} Tarsus \textbar{} Acts 9:11 \textbar{} The birthplace of Paul (Saul).
\textbar{} \textbar{} \textbar{} Lystra \textbar{} Acts 14:6 \textbar{} Where Paul healed a crippled man and was nearly stoned.
\textbar{} \textbar{} \textbar{} Derbe \textbar{} Acts 14:6 \textbar{} Where Paul and Barnabas preached and made many disciples.
\textbar{} \textbar{} \textbar{} Iconium \textbar{} Acts 14:1 \textbar{} Where Paul preached and faced opposition from the local Jewish authorities.
\textbar{} \textbar{} \textbar{} Ephesus \textbar{} Acts 18:19 \textbar{} A major city in Asia Minor where Paul spent a significant amount of time preaching and establishing the church.
\textbar{} \textbar{} \textbar{} Miletus \textbar{} Acts 20:15 \textbar{} Where Paul met with the Ephesian elders on his way to Jerusalem.
\textbar{} \textbar{} \textbar{} Smyrna, Pergamum, Thyatira, Sardis, Philadelphia, Laodicea \textbar{} Revelation letters \textbar{} Acts doesn't mention directly but likely influenced early Christian activity.
\textbar{} \textbar{} Greece \textbar{} Philippi \textbar{} Acts 16:12 \textbar{} Where Paul and Silas were imprisoned and where the first European Christian church was founded.
\textbar{} \textbar{} \textbar{} Thessalonica \textbar{} Acts 17:1 \textbar{} A major city where Paul preached and faced opposition.
\textbar{} \textbar{} \textbar{} Berea \textbar{} Acts 17:10 \textbar{} Where Paul went after Thessalonica and found the Bereans to be more receptive to the gospel.
\textbar{} \textbar{} \textbar{} Athens \textbar{} Acts 17:16 \textbar{} Where Paul preached on Mars Hill and engaged with philosophers about the ``unknown god.'' \textbar{} \textbar{} \textbar{} Corinth \textbar{} Acts 18:1 \textbar{} Where Paul stayed and established a Christian community, and where he later wrote 1 and 2 Corinthians.
\textbar{} \textbar{} Macedonia and the Surrounding Areas \textbar{} Neapolis \textbar{} Acts 16:11 \textbar{} Port city in Macedonia, where Paul and his companions arrived after sailing from Troas.
\textbar{} \textbar{} \textbar{} Philippi \textbar{} Acts 16:12 \textbar{} Mentioned earlier in Greece.
\textbar{} \textbar{} Egypt \textbar{} Alexandria \textbar{} Acts 6:9, 18:24 \textbar{} Apollos was from here; a major Jewish and early Christian hub.
\textbar{} \textbar{} Libya (North Africa) \textbar{} Cyrene \textbar{} Acts 2:10, 11:20 \textbar{} Home of Simon of Cyrene; some early Christians were from here.
\textbar{} \textbar{} Italy and Rome \textbar{} Puteoli \textbar{} Acts 28:13 \textbar{} Port city in Italy where Paul arrived after sailing from Malta.
\textbar{} \textbar{} \textbar{} Rome \textbar{} Acts 28:16 \textbar{} Where Paul was taken as a prisoner and spent two years under house arrest.
\textbar{} \textbar{} Other Notable Cities \textbar{} Cyprus \textbar{} Acts 13:4 \textbar{} Where Paul and Barnabas first traveled for missionary work.
\textbar{} \textbar{} \textbar{} Salamis \textbar{} Acts 13:5 \textbar{} A city in Cyprus where Paul preached.
\textbar{} \textbar{} \textbar{} Paphos \textbar{} Acts 13:6 \textbar{} A city in Cyprus where Paul encountered the sorcerer Elymas.
\textbar{} \textbar{} \textbar{} Patara \textbar{} Acts 21:1 \textbar{} A port city in Lycia where Paul caught a ship to Phoenicia.
\textbar{} \textbar{} \textbar{} Tyre \textbar{} Acts 21:3 \textbar{} A city in Phoenicia where Paul stopped to meet the disciples.
\textbar{}

Regions and Cities in the Epistles:

\begin{longtable}[]{@{}p{0.15\linewidth} p{0.15\linewidth} p{0.7\linewidth}@{}}
    \toprule\noalign{}
    \begin{minipage}[b]{\linewidth}\raggedright
    Location
    \end{minipage} & \begin{minipage}[b]{\linewidth}\raggedright
    Mentioned In
    \end{minipage} & \begin{minipage}[b]{0.7\linewidth}\raggedright
    References
    \end{minipage} \\
    \midrule
    \endhead
    \bottomrule
    \endlastfoot
    Rome & Acts, Romans, Philippians, 2 Timothy & Acts 28, Romans 1:7, 1:15, Philippians 1:13, 2 Timothy 4:16-17 \\
    Corinth & Acts, 1 Corinthians, 2 Corinthians & 1 Corinthians 1:2, 2 Corinthians 1:1 \\
    Ephesus & Acts, Ephesians & Ephesians 1:1 \\
    Galatia & Galatians & Galatians 1:2 \\
    Philippi & Acts, Philippians & Philippians 1:1 \\
    Thessalonica & 1 Thessalonians, 2 Thessalonians & 1 Thessalonians 1:1, 2 Thessalonians 1:1 \\
    Colossae & Colossians & Colossians 1:2 \\
    Laodicea & Colossians, Revelation & Colossians 4:13-16, Revelation 3:14-22 \\
    Crete & Titus & Titus 1:5 \\
    Cyprus & Galatians & Galatians 4:13 \\
    Pontus, Galatia, Cappadocia, Asia, Bithynia & 1 Peter & 1 Peter 1:1 \\
    Macedonia & 2 Corinthians, Philippians & 2 Corinthians 8:1, Philippians 4:15 \\
    Miletus & 2 Timothy & 2 Timothy 4:20 \\
    Antioch & Acts, Galatians, 1 Corinthians & Galatians 2:11, 1 Corinthians 9:6 \\
    Tarsus & Acts, 2 Corinthians & Acts 9:11, 2 Corinthians 11:22 \\
    Syria & 1 Corinthians, Galatians, 2 Corinthians & 1 Corinthians 16:3, Galatians 1:21, 2 Corinthians 11:9 \\
    Asia & 1 Corinthians, 2 Corinthians, Revelation & 1 Corinthians 16:19, 2 Corinthians 1:8 \\
    Troas & Acts, 2 Timothy & Acts 20:6, 2 Timothy 4:13 \\
    Berea & Acts, 1 Thessalonians & Acts 17:10, 1 Thessalonians 1:7 \\
    Paphos & Acts, Titus & Titus 1:5 \\
    Puteoli & Romans & Romans 16:3-4 \\
\end{longtable}

It is important to actually visualize the locations mentioned in the Acts and the epistles to see the striking pattern of the locations mentioned in the Acts and the epistles being the same as the most significant Greek speaking cities.

\begin{figure}[ht]
    \centering
    \includegraphics[width=\textwidth, keepaspectratio]{assets/locations_map}
    \caption{Map of all the locations mentioned in the Acts and the epistles.}
    \label{fig:figure}
\end{figure}

For those geographically inclined you can spot the near perfect correlation with the borders of Eastern Roman Empire.
Of note is the trip to Rome, which was substantially different in nature to the other trips.
\href{https://en.wikipedia.org/wiki/Byzantine_Empire_under_the_Theodosian_dynasty\#/media/File:4KTHEODOSIAN.png}{Rome Map}

\subsection{The striking statistics of the cities mentioned in the Acts and the epistles are that they are all in the former Greek empire, and not one mention of a city in the Roman empire that was not part of the former Greek empire.}\label{subsec:the-striking-statistics-of-the-cities-mentioned-in-the-acts-and-the-epistles-are-that-they-are-all-in-the-former-greek-empire-and-not-one-mention-of-a-city-in-the-roman-empire-that-was-not-part-of-the-former-greek-empire.}

This fact makes any theory that deems Christianity as a religious and not a political movement immediately highly implausible.

\subsection{Finally we consider the apparent minimal resistance to the acceptance of the new religion.}\label{subsec:finally-we-consider-the-apparent-minimal-resistance-to-the-acceptance-of-the-new-religion.}

The new religion was accepted by the masses in the former Greek empire, and not a single mention of any resistance to the new religion from the Greeks themselves.
The only real opposition comes from the Temple authorities in Jerusalem.
This is consistent with Christianity being a continuation of Greek imperial philosophy, which the Hellenized population already accepted.

\subsection{We should also consider that even though the religion was so successful at converting the masses, it still had all the conspiratorial parts to it.}\label{subsec:we-should-also-consider-that-even-though-the-religion-was-so-successful-at-converting-the-masses-it-still-had-all-the-conspiratorial-parts-to-it.}

Early Christians used secret symbols to identify each other, they frequently met in secret, often at night in the catacombs.
This is not unlike the initiation practices of other imperial soldier cults such as Mithras.
The Pauline language of ``soldiers of Christ'' fits this conspiratorial military model.

\subsection{Consider why the religion was seemingly much more prosecuted than any other religion in the Roman empire.}\label{subsec:consider-why-the-religion-was-seemingly-much-more-prosecuted-than-any-other-religion-in-the-roman-empire.}

The Roman empire was very tolerant of other religions, and the only time they would prosecute a religion was if it was a threat to the empire.
Christians were persecuted not for worshipping a strange god, but for refusing the Roman emperor cult while insisting that only their Christ was king.
This was a political treason, not theological quibbling.

\subsection{The phrase ``soldiers of Christ'' is not used explicitly in the Gospels, but it appears prominently in the Pauline Epistles.}\label{subsec:the-phrase-soldiers-of-christ-is-not-used-explicitly-in-the-gospels-but-it-appears-prominently-in-the-pauline-epistles-particularly-in-the-context-of-the-christian-life-being-compared-to-a-military-struggle-or-a-spiritual-battle.}

The metaphor emphasizes loyalty, discipline, and readiness for conflict under a divine king.
It reflects the military cultic environment of the empire, not a rural Jewish sect.

\subsection{Paul barely mentions the life of Jesus, and almost never quotes him.}\label{subsec:paul-barely-mentions-the-life-of-jesus-and-almost-never-quotes-him.}

It is frequently claimed that Paul's religion is not the religion of Jesus, but the religion about Jesus.
There is a shocking lack of references to any of the teachings of Jesus, the Jewish law, and any of the events surrounding Jesus's life and death.
So we may go one step further.
It is a religion focusing on restoring the kingdom of God by resurrecting the office of the Christos, the rightful king of the kingdom of God.
And so to Paul and all the early Christians, it was all about resurrecting a Christ, not teachings of the particular Jesus Christ.
The idea that God will once again send a king that will restore the Greek empire, the kingdom of God, headed by Christos, the rightful earthly king of the kingdom of God.

\subsection{7a.
Paul fuses army and court: enlistment under the Christ and absorption into his imperial bureaucracy.}\label{subsec:pau-fuses-army-and-court-enlistment-under-the-christ-and-absorption-into-his-imperial-bureaucracy.}

This is not angelic poetry; it is administrative vocabulary.
Paul lists “thrones, dominions, principalities, and authorities” (\emph{thronoi, kyriotētes, archai, exousiai}) as if reading out a court register (Col 1:16; Eph 1:21).
\emph{Thronoi} are sovereign seats—the royal chair and council thrones.
\emph{Kyriotētes} are lordships—jurisdictional dominions held by high lords.
\emph{Archai} are magistracies—governorships and command offices that execute policy.
\emph{Exousiai} are delegated jurisdictions—legal powers to act, enforce, and judge.
Christ is seated “at the right hand” above every such office (Eph 1:20–21).
Creation itself is said to be structured as this bureaucracy and to be re-chartered “in him” (Col 1:16–18).
He does not abolish the chain of command; he reassigns it to himself.

The soldier metaphor slots into that same constitution.
“Share in suffering as a good soldier of Christ Jesus” names enlistment, not mysticism (2 Tim 2:3).
“The armor of God” equips imperial troops, not village pietists (Eph 6:10–17).
“I bear the stigmata of Jesus” marks the branded body of a sworn man, not a private devotion (Gal 6:17).
“Sealed with the Spirit” is the bureaucratic seal of a new jurisdiction, not a feeling (Eph 1:13; 2 Cor 1:22).
Confessing “Jesus is Lord” functions as an oath of allegiance, not a personal mantra (Rom 10:9; Phil 2:11).

The civic layer is explicit.
“Our \emph{politeuma} is in heaven” declares citizenship in a capital city, not a mood (Phil 3:20).
The \emph{ekklesia} is the assembly of that polity, not a synagogue proxy.
\emph{Episkopoi} and \emph{diakonoi} are administrative titles—overseers and service officials—of the king’s house (Phil 1:1).
“God has appointed, first apostles, second prophets, third teachers…” is rank order, not poetry (1 Cor 12:28).
“The saints will judge the world… we will judge angels” grants appellate competence to citizens, not hobbyists (1 Cor 6:2–3).
“Christ is head over every \emph{archē} and \emph{exousia}” is constitutional supremacy, not metaphor (Col 2:10).
“He disarmed the \emph{archai} and \emph{exousiai}” is a change of control of offices, not a ghost story (Col 2:15).
“Let every soul be subject to the \emph{exousiai}” shows Paul using the same word for earthly magistrates, proving the term’s legal register (Rom 13:1).

Once you see the court chart, Paul’s silence on Jesus’s daily sayings stops being puzzling and becomes programmatic.
He is not compiling a \emph{bios}.
He is promulgating a charter.
His gospel installs the \emph{Christos} as sovereign and drafts citizens and soldiers into that new order.
Biography is secondary when the constitution is being proclaimed.

\subsection{Using this conspiratorial language clearly worked.}\label{subsec:using-this-conspiratorial-language-clearly-worked}

Rome did not even realize the new religion's goal was to restore the Eastern Empire until it actually happened.

\subsection{10.
Alexandria was the capital of the Greek Empire and the center of the Hellenistic world and yet there are no missions or letters to Alexandria.}\label{subsec:alexandria-was-the-capital-of-the-greek-empire-and-the-center-of-the-hellenistic-world-and-yet-there-are-no-missions-or-letters-to-alexandria.}

The absence of Alexandria in the New Testament is striking, especially considering its significance.
Alexandria was erased from the text, as it was the origin city of Apollos, as well as some of the other companions of Paul such as Mark, Demas, and Luke.
The omission is best explained as deliberate suppression, precisely because Alexandria was already the center of the movement.
Just as the Old Testament largely omits the Ptolemaic empire despite its centrality, so the New Testament minimizes Alexandria while still presuming its leadership.

\section{Acts of the Apostles}\label{subsec:acts-of-the-apostles}

Is called the Acts of the Apostles, not acts of the disciples.
Apostles doing imperial work of letting all nations of the empire know the will of the God king.

\subsection{10.
Acts opens with a royal enthronement}\label{subsec:acts-opens-with-a-royal-enthronement}

Acts 1:6 --- ``Lord, will you at this time restore the kingdom to Israel?'' This is not a spiritual question.
It implies Jesus had a claim to political kingship.
Your theory: Jesus was understood as the rightful monarch of a revived kingdom---a successor to the Herodian or Hasmonean thrones under Greek imperial ideals.

\subsection{10.
Jesus is taken up like an emperor}\label{subsec:jesus-is-taken-up-like-an-emperor}

Acts 1:9--11 --- The Ascension mimics apotheosis scenes (e.g., Alexander, Roman emperors).
It frames Jesus in imperial terms, being enthroned in heaven---like a divine emperor.
This matches your view that Christianity was about loyalty to the ``Christ Emperor.''

\subsection{10.
The Pentecost scene mimics an imperial inauguration}\label{subsec:the-pentecost-scene-mimics-an-imperial-inauguration}

Acts 2 --- Multilingual miracle and mass conversion reflects the imperial ideal of uniting nations under one divine king.
The language of ``tongues'' is political: the emperor's message is for all nations.

\subsection{10.
Acts 5: The trial of the apostles}\label{subsec:acts-5-the-trial-of-the-apostles}

Gamaliel references past revolutionary figures---Theudas and Judas the Galilean.
This acknowledges that messianic revolts were political, and that Jesus' movement was seen in similar terms.

\subsection{10.
Stephen's speech in Acts 7 is anti-Temple}\label{subsec:stephens-speech-in-acts-7-is-anti-temple}

Stephen attacks the Temple and Mosaic tradition, echoing Philo and Stoic-influenced criticisms of Jewish legalism.
This supports the view that early Christianity rejected the Mosaic religion and aligned more with philosophical monotheism.

\subsection{10.
Paul as imperial envoy}\label{subsec:paul-as-imperial-envoy}

Paul appeals to Caesar, travels through Greek cities, and preaches to Hellenized elites.
His speeches (e.g., Acts 17 in Athens) are clearly political-philosophical, not sectarian Jewish.
Acts frames Paul as a philosopher-diplomat for the Christ-emperor.

\subsection{11.
Acts ends without resolution}\label{subsec:acts-ends-without-resolution}

The book ends in Rome, with Paul freely preaching ``the kingdom of God.''
It lacks a narrative climax because its real message is that the empire is now Christian.
It presumes a pre-existing audience that sees Christianity as a political-theological force.

\subsubsection{James the Just}\label{subsec:james-the-just}

\subsection{James the Just also wrote an epistle to all nations.}\label{subsec:james-the-just-also-wrote-an-epistle-to-all-nations.}

He was the brother of Jesus, and the next in line to the throne.
Much like Jesus Christ the Soter, James also held a royal title, the Just.

James the Just also wrote an epistle to all nations, which is included in the New Testament.
James, like John, refers to the same understanding of Logos as Philo of Alexandria.
The writing style of James and John also bears a striking resemblance to the writing style of Philo of Alexandria.

In Greek, James 1:21 reads as:
``Διὸ ἀποθέμενοι πάσαν ἀκαθαρσίαν καὶ περισσείαν κακίας ἐν πραΰτητι δέξασθε τὸν ἐμφυτον λόγον, ὃς δύναται σῶσαι τὰς ψυχὰς ὑμῶν.''
Transliteration: ``Dio apothemenoi pasan akatharsian kai perisseian kakias en prautēti dexasthē ton emphuton logon, hos dynatai sōsai tas psychas hymōn.''
A literal translation: ``Therefore, putting away all filthiness and the overflow of wickedness, with meekness receive the implanted word, which is able to save your souls.''

It should go without saying that the advanced writing style of James and John is not something that would be expected from a simple fisherman or a son of a carpenter.
Rather, it confirms Alexandrian philosophical schooling at the very center of the Greek imperial world.

\subsubsection{The Epistles of John}\label{subsec:the-epistles-of-john}

\subsection{A disproof of existing models}\label{subsec:a-disproof-of-existing-models}

The scholarly literature usually explains the rise of Christianity through one of three frameworks: the missionary diffusion model, the Judaic sect model, or the imperial cult competition model.
All three assume Christianity began as a small religious group and gradually expanded through organic processes.
The actual data show this assumption is impossible.

\textbf{1. Missionary Diffusion Model (Harnack, Stark, etc.).}
This model imagines apostles traveling city by city, converting households, and founding local congregations which then grew over time.
It predicts: (a) gradual spread across regions, (b) uneven adoption, (c) multilingual documentation, and (d) continuous growth into the second century.
\emph{Observed:} a sudden network of churches in every major polis of the Greek East, exclusively in Greek, followed by a century of stagnation.
\emph{Conclusion:} the model cannot account for this. Gradual spread does not yield instantaneous empire-wide presence and then silence.

\textbf{2. Judaic Sect Model (Eisenman, Sanders, Vermes, etc.).}
This model portrays Christianity as a messianic reform within Judaism that later opened to Gentiles.
It predicts: (a) early writings in Aramaic or Hebrew, (b) reliance on Mosaic law and Temple tradition, (c) expansion via diaspora synagogues, and (d) later translation into Greek.
\emph{Observed:} not a single Aramaic or Hebrew letter, no synagogue-based diffusion, hostility to Mosaic law, and universal Greek correspondence.
\emph{Conclusion:} the expected Jewish framework is entirely absent. The movement is Greek and imperial from its inception.

\textbf{3. Imperial Cult Competition Model (Friesen, Harland, etc.).}
This model treats Christianity as another civic cult competing in the Roman religious marketplace.
It predicts: (a) incremental adoption across both Latin West and Greek East, (b) local diversity of practice, (c) syncretism with Roman emperor worship, and (d) gradual growth from private associations to public cult.
\emph{Observed:} no spread into the Latin West, no random scatter across Roman cities, but perfect coverage of the former Greek empire with uniform claims of one royal Christ.
\emph{Conclusion:} the data negate this model. Instead of scattered cult competition, the pattern is coherent and centralized.

\textbf{Final judgment.}
Every alternative model collapses.
Their predictions do not merely “fit poorly” — they are completely impossible to reconcile with the actual evidence.
It is therefore certain that early Christianity was not the product of missionary growth, sectarian Judaism, or cult competition.
The churches were not newly founded congregations but the continuing \textit{ekklesiai} of the Greek world, reorganized under Christos.
The sequence of events — zero presence, immediate imperial coverage, then long stagnation — admits only one explanation: continuation of the Greek empire under a new royal proclamation.

\subsection{The Corpus of the New Testament}\label{subsec:the-corpus-of-the-new-testament}

What is actually in the New Testament, and how does it map onto the four imperial poles we have identified?
Below we make the mapping explicit \emph{and} anchor it with specific ancient evidence, especially for the epistles and early authorship claims.
Note well: in antiquity the attributions of the four canonical gospels to \textit{Matthew, Mark, Luke, John} are uniformly asserted by the Fathers.
No alternative apostolic names are recorded for them.

\subsection{Patristic attestations (authorship in antiquity).}
Before the mid–third century the picture is consistent.
\textbf{Papias} (via Eusebius, \textit{Hist.\ Eccl.} 3.39) — Mark wrote as Peter’s interpreter.
Matthew compiled the \textit{logia}.
\textbf{Irenaeus} (\textit{Adv.\ Haer.} 3.1.1) — explicitly lists and defends the four: Matthew, Mark, Luke (Paul’s companion), John.
The \textbf{Muratorian Fragment} (late 2nd c.) names Luke and John as the third and fourth gospels and recognizes a Pauline corpus.
\textbf{Clement of Alexandria} and \textbf{Origen} repeat these attributions.
\textbf{Eusebius} later summarizes them as received.
For epistles: \textbf{1~Clement} (c.~96) cites 1~Corinthians.
\textbf{Ignatius} (c.~110) presupposes Pauline churches and diction.
\textbf{Polycarp} (c.~135) quotes or echoes multiple Pauline letters and 1~Peter.
\textbf{Irenaeus} uses 1–2~Peter, 1~John, and James as apostolic.
Discussion about Hebrews’ author and the lateness of 2~Peter appears later.
Crucially, there is no early counter–attribution of the \emph{four gospels} to other names.

\subsection{Papyri and early circulation (snapshot).}
\textbf{P\textsuperscript{46}} (c.~200) preserves a Pauline collection (Rom, 1–2~Cor, Gal, Eph/Col, Phil, 1~Thess, Heb), showing the letters already circulated as a corpus.
\textbf{P\textsuperscript{66}} (c.~200) and \textbf{P\textsuperscript{75}} (early 3rd c.) are major early witnesses to John (and Luke/John).
\textbf{P\textsuperscript{9}} (3rd c.) preserves 1~John at Oxyrhynchus.
\textbf{P\textsuperscript{72}} (3rd/4th c.) contains 1–2~Peter and Jude.
\textbf{P\textsuperscript{13}} (3rd c.) Hebrews.
\textbf{P\textsuperscript{45}} (3rd c.) Gospels/Acts.
These confirm early circulation and natural clustering (Pauline set; Petrine/Jude; Johannine) that mirrors our four-pole architecture.
Egyptian findspots reflect preservation bias.
The \emph{composition} clustering still tracks the imperial map.

\subsection{Johannine (Egypt / Ptolemies).}
\textbf{Gospel of John} is the Alexandrian gospel, written in the idiom of Logos theology.
In our framework it preserves the testimony of the Beloved Disciple (the woman whom Jesus loved) and originated in Egypt.
The \textbf{Epistles of John} (1–3~John) extend the same idiom (light/dark, truth/lie, the Word manifested).
They consolidate Egyptian \textit{ekklesiai} under the royal claim of Christ.
They insist on embodied loyalty (“what we have heard, seen, handled”) and police schism.
\emph{Early evidence:} Irenaeus quotes John and 1~John as apostolic.
P\textsuperscript{66} and P\textsuperscript{75} (John) and P\textsuperscript{9} (1~John) are among our earliest witnesses.
The Muratorian Fragment includes Johannine Catholic epistles.
\emph{We deliberately do not bundle Revelation here.}
It will be treated separately.

\subsection{Matthean (Judea / Herodians).}
\textbf{Gospel of Matthew} is the Judean gospel: Jesus as true King of the Jews and new Moses, fulfilling and surpassing Torah.
Partner writings: \textbf{James} (by James the Just, the royal brother who led Jerusalem) and \textbf{Jude} (self-identified brother of James).
\emph{Early evidence:} Papias on Matthew.
Irenaeus and the Muratorian Fragment list Matthew.
Origen and Eusebius affirm Jacobean authorship.
Early Catholic lists include Jude.
James’ concise Greek moral style fits Alexandrian/Jerusalem schooling at the imperial center.

\subsection{Lukan–Pauline (Aegean / Greeks).}
\textbf{Gospel of Luke} is the Aegean gospel: historiographic prologue (\textit{kratiste Theophile}), polished Greek.
It continues in \textbf{Acts}.
The corpus pairs with the \textbf{Pauline epistles} — letters to \textit{ekklesiai} in the Aegean and Asia Minor: \textit{Romans} (from Corinth), \textit{1–2~Corinthians}, \textit{Galatians} (interior Anatolia), \textit{Philippians} (Macedonia), \textit{1–2~Thessalonians} (Macedonia), \textit{Ephesians}/\textit{Colossians} (Asia), \textit{Philemon}.
\emph{Early evidence:} 1~Clement refers to Paul’s Corinthian correspondence.
Ignatius addresses Pauline cities and adopts Pauline diction.
Polycarp quotes Paul repeatedly.
The Muratorian Fragment enumerates a Pauline corpus.
P\textsuperscript{46} attests a bound collection.
Luke’s tie to Paul is ancient: Col 4:14; Phlm 24; 2~Tim 4:11; and the “we” sections in Acts.

\subsection{Markan–Petrine (Seleucid East / Syria–Anatolia).}
\textbf{Gospel of Mark} is the Seleucid gospel, proclamation in rough Greek with Aramaic traces, suited to Antioch–Syria–Anatolia.
It preserves Peter’s preaching in written form.
The partner epistle is \textbf{1~Peter}, addressed explicitly to the Seleucid frontier provinces (“Pontus, Galatia, Cappadocia, Asia, Bithynia,” 1~Pet 1:1).
\emph{Early evidence:} Papias on Mark as Peter’s interpreter.
Irenaeus locates Mark after Peter and Paul.
1~Peter’s destination list maps our Seleucid frontier.
P\textsuperscript{72} later transmits 1–2~Peter with Jude in an eastern Catholic collection.

\subsection{Epistolary map (evidence at a glance).}
\begin{center}
\begin{tabular}{@{}p{3.1cm}p{3.3cm}p{3.8cm}p{4.0cm}@{}}
\toprule
\textbf{Epistle} & \textbf{Addressees / Region} & \textbf{Earliest patristic attestation} & \textbf{Earliest papyrus witness} \\
\midrule
Romans & Rome (from Corinth) / Aegean network & 1~Clement; Ignatius; Polycarp & P\textsuperscript{46} \\
1–2 Corinthians & Corinth / Aegean & 1~Clement 47 cites 1~Cor; Polycarp & P\textsuperscript{46} \\
Galatians & Anatolian interior & Irenaeus; Polycarp echoes & P\textsuperscript{46} \\
Ephesians/Colossians & Asia (circular) & Ignatius (Eph); Irenaeus & P\textsuperscript{46} \\
Philippians & Macedonia & Polycarp (to Philippi) & P\textsuperscript{46} \\
1–2 Thessalonians & Macedonia & Irenaeus; Polycarp & P\textsuperscript{46} (1~Thess) \\
Philemon & Colossae (Asia) & Ignatius; Polycarp & P\textsuperscript{46} \\
Hebrews* & (Alexandrian homily) & Clement/Origen discuss; cited early & P\textsuperscript{13}; P\textsuperscript{46} \\
James & Diaspora (Judea/Jerusalem pole) & Origen; Irenaeus alludes & — (early codices) \\
1–2 Peter & Seleucid provinces & Polycarp; Irenaeus & P\textsuperscript{72} \\
Jude & Judea/Jerusalem pole & Irenaeus (catalogues); Origen & P\textsuperscript{72} \\
1–3 John & Egyptian pole (Catholic) & Irenaeus; Muratorian & P\textsuperscript{9} (1~John); later P\textsuperscript{74} \\
\bottomrule
\end{tabular}
\end{center}
\noindent *Hebrews is anonymous in antiquity.
We treat it as Alexandrian in style and reception rather than bundle it under Paul.

\paragraph{Regional canons and libraries.}
Ancient “regional canons” mirror this fourfold map.
\textbf{Marcion} (c.~140) circulated an Aegean canon (an edited Luke and a Pauline \textit{Apostolikon}).
In Egypt, the strongest early \textbf{Johannine} cluster (Gospel and Epistles) appears in papyrus caches.
The pattern matches the imperial reading.
Each capital curates its banner gospel and administrative letters.

\paragraph{Scholarly alignment on audience focus.}
While details vary, leading scholars broadly support that each gospel targets a distinct audience and matrix.
\textit{Matthew} for a Judean/Jewish–Christian setting.
\textit{Luke–Acts} for Hellenistic Gentiles.
\textit{Mark} as Petrine proclamation for mixed non–Judean believers.
\textit{John} shaped by a distinctive community with Logos categories recognizable in Alexandria.
Representative voices: R.~Bauckham on eyewitness foundations.
M.~Hengel on the early fixed fourfold gospel.
R.~E.~Brown on the Johannine community.
L.~T.~Johnson and J.~M.~G.~Barclay on Pauline urban assemblies.
L.~W.~Hurtado on early royal/devotional Christology.
Our contribution is to make the \emph{imperial–regional} mapping explicit and to show how the epistolary evidence supports it.

\subsection{Bridge: Why Revelation enters the canon now.}\label{subsec:bridge-why-revelation-enters-the-canon-now}
Without Revelation the corpus has proclamation and administration but no consummation.
Without Revelation the program lacks a legal verdict against Rome and an operational green light.
Revelation is the missing third act: judgment, mobilization, succession.

\subsection{Revelation as Imperial Restoration by Revolt from Rome.}\label{subsec:revelation-as-imperial-restoration-by-revolt-from-rome}
Revelation is not a handbook of private piety.
Revelation is a war order.
The Beast is Rome.
Babylon is Rome’s capital ideology.
The Dragon is the supra-political power that enthroned Rome over the Greek world.
Christos appears not as rabbi but as βασιλεύς with diadems and a στρατεῦμα.
The goal is not escape from earth but transfer of sovereignty on earth.
The charter is “He shall shepherd the nations with a rod of iron.”
That is imperial language, not synagogue language.

The seven messages are not devotional essays.
They are seven readiness reports.
Ephesus, Smyrna, Pergamum, Thyatira, Sardis, Philadelphia, Laodicea form a coastal-inland logistics arc.
This is the trunk line of the former Greek empire in Asia.
Each letter ends with a victory formula (ὁ νικῶν), not a therapy formula.
Victory is the category of empire, not of retreat.

The seals, trumpets, and bowls are not random catastrophes.
They are a three-stage campaign plan.
Seven seals: legal unsealing of the succession.
Seven trumpets: mobilization signals and shock operations.
Seven bowls: terminal judgments that collapse Roman capacity to govern.
Four horsemen are the opening phase of destabilization.
Two witnesses are state envoys who assert jurisdiction and are assassinated by the occupying power.
Their resurrection is the sign that the occupying power has lost the mandate.

The mark (χάραγμα) is not about personal sin.
It is about economic allegiance to Rome’s cultic state.
No mark, no buying and selling, because Rome fuses market and worship to police loyalty.
The counter-mark is the seal on the servants of God (σφραγίς).
This is counter-state administration under a rival king.

The Woman clothed with the sun is not a private mystic.
She is the mother of the royal line.
The Child is the claimant who is caught up to God and to his throne.
Rome cannot seize the throne, only the earth; the succession escapes and returns with mandate intact.
That is imperial succession theology, not apocalyptic despair.

The Beast’s number encodes the emperor, but the point is not arithmetic cleverness.
The point is prosecutorial clarity: name the regime, strip its aura, announce its doom.
Seven heads and ten horns are not zoology.
They are a map of Roman sovereignty claims and vassal teeth.

The Rider on the white horse does not preach forgiveness to the Beast.
He makes war in righteousness.
He bears many diadems because he is not another client king.
He is the suzerain.
His name is Λόγος τοῦ Θεοῦ because the decree that creates worlds now deposes empires.
This is the coronation the Gospels began and the Epistles administrated in exile.

The New Jerusalem is not an ethereal cloud city.
It is a capital.
It descends.
It has walls, gates, measurements, foundations named for apostles, and a throne.
It displaces Rome’s urban theology with a rival polis.
Kings of the earth bring their glory into it, which presumes kings still exist and are re-aligned under Christos.

Therefore Revelation belongs in the canon precisely because the project is imperial restoration.
The Gospels proclaim the king.
The Epistles organize his assemblies.
Revelation authorizes revolt from Rome and narrates Rome’s removal.
The end is not the end of history but the end of Roman hegemony over the Greek world and its replacement by the basileia of Christ.
