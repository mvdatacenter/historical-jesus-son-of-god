A common unassailable but certainly wrong assumption of modern scholarship is that all early christian literature was written in Greek because Greek was the lingua franca of the Roman empire.
According to this claim we would expect Julius Caesar, the People and the Senate of Rome, Virgil, Seneca to all be fluent and write and speak extensively in Greek.
In fact, it is correct Greek was the administrative language of nearly all of the former Greek empire which later became known as the Eastern Roman Empire.
Notably nearly all Christian writers including Clement of Rome, Ignatius of Antioch, Polycarp, Justin Martyr, Irenaeus, Origen wrote in Greek.
The first non-greek church father is Tertullian close to the end of second century.
In this chapter we want to highlight that Christianity was a Greek-only religion in its literary and intellectual foundation, while there were many mentions of Jesus in non-greek sources, and very notably, religious sources written in Coptic were not considered the same religion as Christianity.
In this chapter we look at the writings of the church fathers from the prism of the Byzantine Empire resurgence.

As the early Christians wrote:
``
There is a bird which is called the Phoenix.
This is the only one of its kind and lives five hundred years.
When the time of its dissolution draws near, it makes for itself a coffin of frankincense and myrrh and other spices, and when the time is fulfilled it enters it and dies.
But as its flesh decays, a worm is produced, which is nourished by the moisture of the dead creature and puts forth wings.
Then, when it has grown strong, it takes up that coffin and flies from the land of Arabia to Egypt, to the city of Heliopolis, and, in the daytime, in the sight of all, it places itself on the altar of the sun.
``
What if this story is not not a myth of resurrection, but a reference to the restoration of the Byzantine Empire, which was considered the Phoenix rising from the ashes?

\paragraph{0.
Alexandria Egypt suffered enormous hardship and persecution during the Roman Empire.}\label{par:alexandria-egypt-suffered-enormous-hardship-and-persecution-during-the-roman-empire.}

The 200-year struggle of Rome to conquer all the Greek world ended in enormous tax burden on the newly conquered territories.
Almost all tax revenue of the Roman Empire at the time of Jesus and shortly after was coming from the newly conquered territories of the Greeks and nearly half of it came from Egypt.
It is often misunderstood that Egypt was simply so wealthy and more developed than the rest of the world that it could account for such a large share of the tax revenue.
Egypt was indeed the richest, but by no means by such a large margin.
The tax was simply so high to transfer the wealth of the Greeks to Rome and destroy the heartland of the Greek world and their capability to resist in the future.
Given the hardships that the Greeks suffered, there should be no doubt they would seek to restore their kingdom of God and not be ruled by the beast of Rome.
Egypt was considered property of the emperor and not subject to the normal senatorial and imperial governance system.
The suppression of Greek autonomy by Roman taxation created not only resentment but also a theological-political hope: that the empire of the Greeks, now crushed under Roman tribute, would rise again as the true kingdom of God.
This is why Christianity was entirely born in Greek literary forms---every epistle, every gospel, every apologetic treatise---because it belonged to the intellectual and political heirs of Alexander, not the heirs of Moses.
The Roman destruction of Jerusalem (70 AD) left Judaism fragmented, but the Greek cities of the East---Alexandria, Antioch, Ephesus, Smyrna---became the centers of Christian theology.

\paragraph{1.
Many other early Christian texts were written to seemingly promote the restoration of the Eastern Roman Empire.}\label{par:many-other-early-christian-texts-were-written-to-seemingly-promote-the-restoration-of-the-eastern-roman-empire.}

The Clement of Alexandria, a very prominent early Christian theologian, wrote in his book ``Stromata'' that the true philosophy was the Greek philosophy, and that the Greek philosophy was the true philosophy.
He wrote of the bird Phoenix, which was a symbol of the Eastern Roman Empire, and that the bird was purple, which was the color of the Eastern Roman Empire.
Universally the phoenix has been assumed to be a symbol of the resurrection of Jesus, but the phoenix was a symbol of the Eastern Roman Empire.
Coins of late Roman emperors depict the phoenix explicitly as the symbol of \emph{Rome renewed}.
The Christian allegory of resurrection fit seamlessly into this political symbol.
The color purple, used in martyr literature as the ``purple of blood,'' was the same imperial purple reserved for emperors, showing that Christians consistently merged their identity with imperial ideology.

\paragraph{2.
The very famous shields with the Chi-Rho symbol were a symbol of the creation of the Eastern Roman Empire.}\label{par:the-very-famous-shields-of-with-the-chi-rho-symbol-were-a-symbol-of-the-creation-of-the-eastern-roman-empire.}

The creation of the Eastern Roman Empire as a kingdom of God ruled by the Christ by a Christian army of Christian emperor Constantine was a very important event in the history of the world.
At that point the symbol of the soldiers was not a cross.
The symbol of the soldiers was the Chi-Rho symbol, which was a symbol of the Christos.
Note the shields only signified the Christos on them and not Jesus directly, which may have been a designation that ``we are the soldiers of the Christ,'' rightful king of the kingdom of God, and the battles for the restoration of the kingdom of the Christos.
If the war was thought for religious reasons primarily, there would have likely been a lot more emphasis on the cross and mass conversions to Christianity, and we do not see that being strongly emphasized in the historical records of this particular war.
We do however see the large emphasis on political changes and restoring the rule in the East.
Constantine’s use of Chi-Rho on coins, shields, and banners was not a personal devotion to Jesus of Nazareth, but the creation of a Christic imperial monogram---a sign that the empire itself now belonged to the Christos, the anointed ruler.
Eusebius (\emph{Life of Constantine}, Book 1) insists that the empire was renewed under the Christ, making it clear that Christianity functioned as the legitimizing religion of a new imperial order.

\paragraph{3.
Rome converting to Christianity matched in time with Rome splitting into two empires, the Western Roman Empire and the Eastern Roman Empire.}\label{par:rome-converting-to-christianity-matched-in-time-with-rome-splitting-into-two-empires-the-western-roman-empire-and-the-eastern-roman-empire.}

The restoration of the Eastern Roman Empire was the actual goal of the christians and christians taking over corresponded with the restoration of the Eastern Roman Empire.
Eusebius explicitly calls Constantinople the ``New Rome'' aligned with God’s will, presenting the division of empire and the Christianization of the East as one and the same event.

\paragraph{4.
Hippolytus of Rome (c.~170--235 AD) and other early Christian thinkers believed that the fall of Rome was part of God's plan for the eventual establishment of Christ's eternal empire.}\label{par:hippolytus-of-rome-c.-170235-ad-and-other-early-christian-thinkers-believed-that-the-fall-of-rome-was-part-of-gods-plan-for-the-eventual-establishment-of-christs-eternal-empire.}

\paragraph{5.
The Apocalyptic Restoration - The End of Time and the Empire:}\label{par:the-apocalyptic-restoration---the-end-of-time-and-the-empire}

In the Apocalyptic literature, particularly found in Revelation, Christians looked forward to a new heaven and new earth (Revelation 21).
The New Jerusalem would come down from heaven, and it was often interpreted as both a spiritual and literal kingdom that would restore what was lost with the fall of the world.
Eschatological visions were tied to the restoration of an empire under Christ's rule, not just in a spiritual sense, but in a political and cosmic sense.
The Roman Empire was viewed by some as a precursor to the final divine kingdom.

\paragraph{6.
Similarly, Origen (c.~185--254 AD) also held that Christianity could restore the world order.}\label{par:similarly-origen-c.-185254-ad-also-held-that-christianity-could-restore-the-world-order.-he-saw-the-future-reign-of-christ-as-the-ultimate-restoration-of-order-justice-and-peace-in-a-new-universal-kingdom-where-christianity-would-reign.}

He saw the future reign of Christ as the ultimate restoration of order, justice, and peace in a new universal kingdom, where Christianity would reign.

\paragraph{7.
Irenaeus (c.~130--202 AD)}\label{par:irenaeus-c.-130202-ad---in-his-work-against-heresies-irenaeus-speaks-about-the-role-of-the-roman-empire-in-gods-providence-and-the-eventual-victory-of-christianity.-he-hints-at-a-future-unity-and-a-cosmic-victory-which-could-be-seen-as-the-restoration-of-the-world-through-christs-reign.}

In his \emph{Against Heresies}, Irenaeus focuses on defending orthodox Christian beliefs, but there are also passages where he highlights the role of the Roman Empire in God's plan.
He stresses that the empire's rule is part of God's providence and suggests that its peaceful reign is a way of preparing the world for Christ's return, offering a sense of the empire's importance in Christian restoration.

\paragraph{8.
Tertullian (c.~155--240 AD)}\label{par:tertullian-c.-155240-ad---in-his-writings-such-as-apology-and-on-the-resurrection-of-the-flesh-tertullian-often-implies-the-eventual-triumph-of-christianity-within-the-roman-empire-framing-it-as-part-of-a-divine-plan.-while-he-doesnt-directly-speak-of-the-restoration-of-the-empire-there-is-a-sense-of-christianity-fulfilling-the-destiny-of-the-roman-state.}

In his writings, such as \emph{Apology} and \emph{On the Resurrection of the Flesh}, Tertullian often implies the eventual triumph of Christianity within the Roman Empire, framing it as part of a divine plan.
While he doesn't directly speak of the ``restoration'' of the empire, there is a sense of Christianity fulfilling the destiny of the Roman state.

\paragraph{9.
Eusebius of Caesarea (c.~260--340 AD)}\label{par:eusebius-of-caesarea-c.-260340-ad}

In his work \emph{Ecclesiastical History} and \emph{Life of Constantine}, Eusebius explicitly presents the rise of Constantine and the establishment of Christianity as the fulfillment of God's plan for the Roman Empire.
He sees Constantine's reign as a ``restoration'' of the empire, aligning it with divine will.
This reflects the idea that Christianity would not only restore the empire but also bring it to its true, Christian purpose.

\paragraph{10.
Athanasius of Alexandria (c.~296--373 AD)}\label{par:athanasius-of-alexandria-c.-296373-ad}

In his writings, particularly in his defense against Arianism and his theological works, Athanasius talks about the cosmic victory of Christ over evil, which has implications for the empire's restoration.
He often frames the Christian emperor as the rightful ruler under divine guidance, which could be seen as linking the restoration of the empire to Christ's victory.

\paragraph{11.
Victorinus of Pettau (c.~250--303 AD)}\label{par:victorinus-of-pettau-c.-250303-ad}

In his \emph{Commentary on the Apocalypse}, Victorinus draws connections between the Roman Empire and the eventual triumph of Christianity.
Like many of his contemporaries, he believes that the empire is part of God's plan and that its ultimate transformation into a Christian empire would bring about the fulfillment of prophecy.
This can be seen as a form of ``restoration'' through the christianization of the empire.

\paragraph{12.
The revolt is not militaristic---Rome chose to spiritually convert to Christianity, and the Eastern Roman Empire was restored peacefully while the Christian writers started to praise Rome.}\label{par:the-revolt-is-not-militaristic-rome-chose-to-spiritually-convert-to-christianity-and-the-eastern-roman-empire-was-restored-peacefully-while-the-christian-writers-started-to-praise-rome.}

Lactantius, writing shortly before Constantine’s victory, declared in \emph{Divine Institutes} that Rome’s destiny was to become Christian, and that the empire itself was God’s instrument to unify the world.
Augustine of Hippo (354--430 AD)---Expanded In \emph{City of God}, Augustine offers a vision where the fall of the Roman Empire is viewed through a Christian lens.
He argues that the decline of the empire does not indicate the failure of divine providence.
While he focuses on the spiritual aspects of empire, he acknowledges the empire's role in preparing the world for the Christian kingdom, suggesting a ``restoration'' of the Roman Empire as a Christian entity in the future.

\paragraph{13.
The Shepherd of Hermas (c.~100--160 AD)}\label{par:the-shepherd-of-hermas-c.-100-160-ad}

This early Christian text, written by Hermas, is a visionary work similar in nature to Revelation.
It contains visions and allegories about the future of the Church and the world.
The \emph{Shepherd} speaks about the coming of the end times and the restoration of the Church through repentance and faithfulness, much like how Revelation portrays the establishment of the New Jerusalem and God's final victory over evil.
Hermas depicts the hope of restoration for the Church and its ultimate triumph, paralleling Revelation's theme of the faithful being vindicated at the end of time.

\paragraph{14.
Clement of Alexandria's Exhortation to the Greeks (c.~190 AD)}\label{par:clement-of-alexandrias-exhortation-to-the-greeks-c.-190-ad}

In his writings, Clement combines Christian eschatology with Greek philosophy, advocating for the return of the ``Logos'' and the eventual restoration of humanity to divine harmony.
While not strictly apocalyptic in the sense of a Revelation-style vision, his vision of the future aligns with the idea of cosmic renewal.
Clement's apocalyptic themes include the eventual restoration of the world through the Logos, an idea that ties back to the restoration of divine order similar to the eschatological views found in Revelation.

\paragraph{15.
Cyprian of Carthage's The Lapsed (c.~250 AD)}\label{par:cyprian-of-carthages-the-lapsed-c.-250-ad}

Cyprian writes about the persecution of Christians and the imminent return of Christ.
His works anticipate the final judgment, the victory of the righteous, and the establishment of a divine kingdom.
Like other early Christian apocalyptic writers, Cyprian believed that the Church would be restored and triumph over its persecutors, reflecting the broader apocalyptic hope seen in Revelation.
