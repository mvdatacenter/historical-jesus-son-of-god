This is the great mystery of faith:
We Christ has died, Christ has risen, Christ will come again.

This is the core of Christian belief.
Christians taught love and forgiveness, and promised eternal life to all who believed in Christ.
That attracted many converts.
Then as the converts grew in number, the mistery was revealed.
Christ will come again.
Not just as a spiritual savior, but as the new emperor of the greek world.
And so Greeks slowly grew the Christian community penetrating the Roman Empire, lower classes, urban centers, and the ruling elites.
Until such time that the Roman Empire itself was Christianized and the former Greek empire of Alexander the Great was restored as the kingdom of God on Earth.

A common unassailable but certainly wrong assumption of modern scholarship is that all early christian literature was written in Greek because Greek was the lingua franca of the Roman empire.
According to this claim we would expect Julius Caesar, the People and the Senate of Rome, Virgil, Seneca to all be fluent and write and speak extensively in Greek.
In fact, it is correct Greek was the administrative language of nearly all of the former Greek empire which later became known as the Eastern Roman Empire.
Notably nearly all Christian writers including Clement of Rome, Ignatius of Antioch, Polycarp, Justin Martyr, Irenaeus, Origen wrote in Greek.
The first non-greek church father is Tertullian close to the end of second century.
In this chapter we want to highlight that Christianity was a Greek-only religion in its literary and intellectual foundation, while there were many mentions of Jesus in non-greek sources, and very notably, religious sources written in Coptic were not considered the same religion as Christianity.
In this chapter we look at the writings of the church fathers from the prism of the Byzantine Empire resurgence.

\section{The Phoenix: Imperial Symbolism in Early Christian Literature}\label{sec:phoenix-symbolism}

One of the earliest Christian writings outside the New Testament canon, \emph{1 Clement} (c.~96 AD), addressed from Rome to Corinth, contains an extended passage on the phoenix:

\begin{quote}
There is a bird which is called the Phoenix.
This is the only one of its kind and lives five hundred years.
When the time of its dissolution draws near, it makes for itself a coffin of frankincense and myrrh and other spices, and when the time is fulfilled it enters it and dies.
But as its flesh decays, a worm is produced, which is nourished by the moisture of the dead creature and puts forth wings.
Then, when it has grown strong, it takes up that coffin and flies from the land of Arabia to Egypt, to the city of Heliopolis, and, in the daytime, in the sight of all, it places itself on the altar of the sun.
(\emph{1 Clement} 25:1-5)
\end{quote}

Modern scholarship invariably interprets this passage as an allegory of Christ's resurrection.
But this reading ignores the political context in which the phoenix functioned as imperial iconography.
The phoenix was not a Christian invention; it was an established symbol of \emph{imperium renovatum}---``empire renewed.''
Roman coinage from the reigns of Antoninus Pius, Commodus, Severus Alexander, and Probus depicted the phoenix with the inscription AETERNITAS AUGUSTI (``The Eternity of the Emperor'') or SAECVLVM NOVVM (``A New Age'').
Coins struck under Hadrian (117--138 AD) and later emperors explicitly used the phoenix to symbolize the cyclical renewal of Roman power (RIC II, Hadrian 246--247).

The critical detail is that the phoenix flies to \emph{Heliopolis in Egypt}, not to Rome.
Heliopolis was the center of the Greek solar cult in Egypt, and Egypt was the symbolic heart of the former Ptolemaic empire.
The phoenix's journey from Arabia to Egypt and its offering on the altar of the sun is not a random myth; it is a geographic and political itinerary mapping the restoration of the Greek-speaking East.

Clement of Alexandria, writing around 190 AD, describes the phoenix as \emph{purple} (πορφυροῦς), explicitly linking it to imperial purple, the color reserved for emperors (\emph{Stromata} 6.138).
Christian martyr literature consistently uses ``the purple of blood'' (τὸ πορφυροῦν αἷμα) to describe martyrs, merging Christian sacrifice with imperial symbolism.
This was not accidental: Christians were appropriating imperial imagery to encode their own restoration theology.

The phoenix allegory, read politically, is transparent: the empire that once died under Roman conquest will rise again, renewed, and return to its homeland in the Greek East.
This is not a story about individual resurrection; it is a narrative of imperial restoration, written into Christian literature from its earliest extant texts.

\section{Alexandria Egypt suffered enormous hardship and persecution during the Roman Empire.}\label{sec:alexandria-egypt-suffered-enormous-hardship-and-persecution-during-the-roman-empire.}

The 200-year struggle of Rome to conquer all the Greek world ended in enormous tax burden on the newly conquered territories.
Almost all tax revenue of the Roman Empire at the time of Jesus and shortly after was coming from the newly conquered territories of the Greeks and nearly half of it came from Egypt.
It is often misunderstood that Egypt was simply so wealthy and more developed than the rest of the world that it could account for such a large share of the tax revenue.
Egypt was indeed the richest, but by no means by such a large margin.
The tax was simply so high to transfer the wealth of the Greeks to Rome and destroy the heartland of the Greek world and their capability to resist in the future.
Given the hardships that the Greeks suffered, there should be no doubt they would seek to restore their kingdom of God and not be ruled by the beast of Rome.
Egypt was considered property of the emperor and not subject to the normal senatorial and imperial governance system.
The suppression of Greek autonomy by Roman taxation created not only resentment but also a theological-political hope: that the empire of the Greeks, now crushed under Roman tribute, would rise again as the true kingdom of God.
This is why Christianity was entirely born in Greek literary forms---every epistle, every gospel, every apologetic treatise---because it belonged to the intellectual and political heirs of Alexander, not the heirs of Moses.
The Roman destruction of Jerusalem (70 AD) left Judaism fragmented, but the Greek cities of the East---Alexandria, Antioch, Ephesus, Smyrna---became the centers of Christian theology.

\section{What exactly happen near year 70 AD?}\label{sec:what-exactly-happen-near-year-70-ad}

The year 70 AD is very critical in the study of early Christianity.
It is the anchor year on which the gospels are dated and the point where Christianity is thought to have changed from a failed apocalyptic movement into a spiritual religion.
But we need to ask why exactly this year is considered so decisive.
It was the year when the Temple in Jerusalem was destroyed by the Romans.
Yet all Christian writers of the time, including Paul, lived far away from Jerusalem and the Temple and did not maintain any real attachment to its priestly religion.
Therefore, the common assumption that the destruction of the Temple forced an entire theological shift in Christianity makes little sense.
It is not remotely plausible that this single event, in a city where almost none of the early authors lived, would have caused the wholesale rewriting of all gospels, epistles, and traditions, or the supposed burning of their original texts.

What we need to realize is that the destruction of the Temple in Jerusalem was only one of many violent events that marked the same decade.
The late 60s and early 70s AD were a time of massive instability across the Greek East of the Roman Empire.
In Galilee, Josephus fought the Romans as commander of the northern army in the \textit{First Jewish–Roman War} (66–73 AD).
At the same time, a fierce \textbf{Greek–Jewish civil war} broke out in Alexandria in 66 AD, when the Greek and Jewish quarters of the city erupted into mutual slaughter; the legions sent by Vespasian to restore order eventually razed much of the city (Josephus, \textit{War} 2.497–507; Philo, \textit{In Flaccum}).
Similar \textbf{inter-ethnic riots} followed in Antioch and Damascus, where Greek and Jewish citizens massacred each other in the thousands (Josephus, \textit{War} 7.43–45).
Meanwhile, in Ephesus, Sardis, and throughout Asia Minor, there were \textbf{anti-Roman uprisings} over taxation and forced conscription, recorded by Tacitus (\textit{Histories} 2.81) and Dio Cassius.
In Cyrene and Egypt, the \textbf{Kitos War} (beginning around 70 AD and flaring up again under Trajan) continued the same unrest.
That is essentially every major Greek city in the Roman Empire, all in revolt against Roman rule, all at the same time.
Even in Rome itself, several senators and equestrians were accused of complicity with eastern rebels and executed for treason.
This was not an isolated Jewish war—it was a general Greek and oriental rebellion against Roman rule, in which local ethnic conflicts and anti-imperial revolts fused into one massive civil breakdown.

And it was precisely at this date when historians put the best estimates for the execution of Paul (c.~64–67 AD), Peter (c.~64–67 AD), and James (c.~62 AD).
All three were killed in Rome, the heart of the empire, far from Jerusalem.
All were executed for sedition, not for religious heresy.

And here emerges the critical point.
Josephus, Paul, Peter, and James all played on the same team.
They did not follow the same rituals or the same philosophical schools, but they all belonged to the same political world: the imperial restoration movement that united Greek, Jewish, and eastern elites against Roman occupation.
Their different “religions” were branches of the same royal cause, adapted to different audiences.
Josephus, in his \textit{Testimonium Flavianum}, describes Jesus as a leader who attracted many followers.
In the royal reading this statement is not paradoxical at all.
Josephus remained faithful to the religion of Moses, but he also recognized Jesus as a \textit{Christos}—a royal leader of Galilee—whose movement, though defeated, formed part of the wider rebellion that shook the entire Greek world around the year 70 AD.

\subsection{The Geographic Correlation: Ekklesia Network as Political Infrastructure}

The connection between Chapter 5's geographic findings and the year 70 AD revolts is inescapable.
Chapter 5 demonstrated that early Christian \emph{ekklesiai} appeared in \emph{every} major Greek city---Alexandria, Antioch, Ephesus, Sardis, Smyrna, Corinth, Philippi, Thessalonica---and in \emph{zero} major Latin-speaking cities.
This chapter shows that in the late 60s and early 70s AD, \emph{precisely those same cities}---Alexandria, Antioch, Ephesus, Sardis, and throughout Asia Minor---erupted in coordinated anti-Roman revolts.

This cannot be coincidence.
The \emph{ekklesia} network was not a religious association that happened to align with political boundaries; it was the organizational infrastructure for coordinated political action.
When we map the cities mentioned in Acts and the Epistles onto the cities that revolted in 66--73 AD, the overlap is nearly perfect.
The assemblies that Paul addressed in Corinth, Ephesus, Philippi, and Thessalonica were not passive religious communities; they were nodes in a restoration network that, when activated, launched a coordinated rebellion across the Greek-speaking East.

The statistical improbability of this pattern cannot be overstated.
If early Christian communities were simply missionary churches spreading by word-of-mouth evangelism, we would expect random geographic distribution, gradual diffusion across linguistic and political boundaries, and no correlation with political events.
Instead, we observe: (1) exclusive concentration in former Greek empire territories, (2) simultaneous appearance in all major Greek cities by the 60s AD, (3) coordinated revolts in those exact cities in 66--73 AD, and (4) immediate collapse of the network after the revolts were crushed, followed by (5) the restoration of the network only when imperial patronage shifted toward the East under Constantine.

This is not the pattern of a religious movement.
It is the pattern of a political restoration program using civic assemblies (\emph{ekklesiai}) as the organizational mechanism, activating them for coordinated revolt, and ultimately succeeding three centuries later when the Eastern Roman (Byzantine) Empire was established as a Christian empire.

\section{Martyrdom Literature: The Universal Literature of Occupied Peoples}\label{sec:martyrdom-literature-occupied-peoples}

Modern scholarship treats the massive abundance of martyrdom literature in early Christianity as a religious phenomenon---evidence of faith, piety, and willingness to die for doctrine.
But this interpretation ignores the most obvious comparative context: martyrdom literature is the standard literary production of \emph{all} occupied peoples resisting imperial domination.

Wherever we examine occupied countries with strong national identity---Poland under partition, Vietnam under French colonial rule, Algeria under French occupation, Ireland under English domination---we find the same four elements: abundant martyrdom literature, a program of national liberation through moral superiority over the oppressor, the veneration of spiritual leaders and national heroes from both recent and distant history, and intensified religious organization centered on community gatherings.

The martyrs are celebrated not for personal holiness but as symbols of collective resistance.
Their deaths prove the nation still lives and will be restored.
But martyrdom alone is insufficient: the occupied people must also \emph{become better} than their oppressors---more virtuous, more disciplined, more worthy of sovereignty.
They must look to past leaders---ancient kings, prophets, warriors, and sages---as proof of former glory and models for restoration.
And they must gather regularly in churches, temples, community centers, or clandestine assemblies to maintain collective identity when direct political organizing is suppressed.
This is the standard program of national liberation movements: moral and civic transformation, inspired by heroic ancestors, sustained through religious community gatherings, as preparation for political restoration.

Poland, partitioned and occupied from 1795 to 1918, provides the clearest parallel.
Polish Romantic literature---especially the works of Adam Mickiewicz, Juliusz Słowacki, and Zygmunt Krasiński---created a massive corpus of martyrdom narratives that became foundational to Polish national identity.
The concept of Poland as the ``Christ of Nations'' (\emph{Chrystus narodów})---a nation that suffers crucifixion under foreign rule but will rise again---pervaded Polish poetry, drama, and prose throughout the occupation period.
This was not religious mysticism; it was political literature encoded in Christian imagery.
Martyrs like Tadeusz Kościuszko, executed or exiled for leading resistance, became secular saints whose suffering proved Poland's continued existence as a nation.
After independence was restored in 1918, the genre's political function largely ceased.

Vietnamese and Algerian resistance literature similarly framed their struggles as moral purification preparing the way for independence.
The pattern is universal: occupied peoples produce martyrdom narratives and preach self-improvement not as ends in themselves but as the path to national restoration.

Early Christian martyrdom literature fits this pattern exactly.
The \emph{Acts of the Martyrs}, the \emph{Martyrdom of Polycarp}, the \emph{Passion of Perpetua and Felicity}, and countless other texts were not primarily theological documents; they were political literature produced by an occupied people to maintain hope in restoration.
The martyrs died not for abstract doctrine but for refusing to acknowledge Roman imperial authority as ultimate.
They were executed for \emph{maiestas}---treason against the emperor---the same charge under which Paul, Peter, and James were killed.

The critical detail is that Christian martyrdom literature was produced almost exclusively in the Greek-speaking East, the very territories of the former Greek empire documented in Chapter 5.
Latin martyrologies appear much later and are demonstrably derivative.
The original martyrdom corpus comes from Smyrna, Antioch, Alexandria, Asia Minor---the cities that revolted in 66--73 AD and would later form the core of the Byzantine Empire.

This is not coincidence.
Martyrdom literature flourishes where political restoration remains a living hope.
When the Eastern Roman Empire was successfully Christianized under Constantine, the genre shifted: martyrs were no longer rebels against Rome but defenders of orthodoxy against heresy.
The political function of martyrdom---to sustain hope of liberation under occupation---disappeared once liberation was achieved.

The sheer volume of Christian martyrdom literature in the first three centuries is therefore not evidence of unusual piety but of sustained political resistance.
It is the literature of a people who believed their kingdom would be restored, who saw every martyr as proof that God had not abandoned them, and who eventually succeeded in establishing the Christian Byzantine Empire.
The martyrs were not dying for heaven; they were dying for the restoration of the kingdom of God to the Greek-speaking world.

\section{The Chi-Rho Symbol and Constantine's Eastern Imperial Restoration}\label{sec:chi-rho-constantinan-restoration}

The creation of the Eastern Roman Empire as a kingdom of God ruled by the Christ by a Christian army of Christian emperor Constantine was a very important event in the history of the world.
At that point the symbol of the soldiers was not a cross.
The symbol of the soldiers was the Chi-Rho symbol, which was a symbol of the Christos.
Note the shields only signified the Christos on them and not Jesus directly, which may have been a designation that ``we are the soldiers of the Christ,'' rightful king of the kingdom of God, and the battles for the restoration of the kingdom of the Christos.
If the war was thought for religious reasons primarily, there would have likely been a lot more emphasis on the cross and mass conversions to Christianity, and we do not see that being strongly emphasized in the historical records of this particular war.
We do however see the large emphasis on political changes and restoring the rule in the East.
Constantine's use of Chi-Rho on coins, shields, and banners was not a personal devotion to Jesus of Nazareth, but the creation of a Christic imperial monogram---a sign that the empire itself now belonged to the Christos, the anointed ruler.
Eusebius (\emph{Life of Constantine}, Book 1) insists that the empire was renewed under the Christ, making it clear that Christianity functioned as the legitimizing religion of a new imperial order.

The timing is critical: Rome's Christianization coincided precisely with the division of the empire into Western and Eastern halves.
The restoration of the Eastern Roman Empire was the explicit goal of this movement, and Eusebius's writings confirm it.
He calls Constantinople the ``New Rome'' aligned with God's will, presenting the division of the empire and the Christianization of the East as a single, unified event---the fulfillment of the restoration theology that had animated early Christian writings for three centuries.

\section{Revelation: The Beast of Rome and the Kingdom Restored}\label{sec:revelation-political-restoration}

The Book of Revelation (c.~95 AD) is conventionally read as spiritual allegory about persecution and heavenly vindication.
But the political content is explicit and undeniable.
Revelation identifies Rome as ``the great city that rules over the kings of the earth'' (Rev 17:18), called ``Babylon'' in coded language to avoid direct sedition charges.
The beast with seven heads represents ``seven kings'' (Rev 17:9-10), universally recognized by scholars as Roman emperors.
The whore of Babylon sits on ``many waters,'' explained as ``peoples, multitudes, nations, and languages'' (Rev 17:15)---the imperial \emph{oikoumene}.

The climactic vision is not spiritual escapism but political restoration:

\begin{quote}
Then I saw a new heaven and a new earth, for the first heaven and the first earth had passed away, and the sea was no more.
And I saw the holy city, new Jerusalem, coming down out of heaven from God, prepared as a bride adorned for her husband.
And I heard a loud voice from the throne saying, ``Behold, the dwelling place of God is with man. He will dwell with them, and they will be his people, and God himself will be with them as their God.'' (Revelation 21:1-3)
\end{quote}

This is not metaphor; it is restoration theology.
The ``new Jerusalem'' descending from heaven is the reestablishment of the kingdom of God on earth, replacing the beastly empire of Rome.
The vision describes a \emph{politeia}---a political order where God rules through his anointed king (\emph{Christos}) over a restored kingdom.
Revelation 11:15 makes this explicit: ``The kingdom of the world has become the kingdom of our Lord and of his Christ, and he shall reign forever and ever.''

This is imperial language.
The promise is not escape to heaven but the replacement of Roman imperium with divine imperium---exactly what occurred when Constantine established the Christian Byzantine Empire three centuries later.

\section{Origen of Alexandria (c.~185--254 AD): Universal Kingdom Through the Logos}\label{sec:origen-universal-kingdom}

Origen, writing from Alexandria---the intellectual center of the Greek East---articulated a vision of cosmic restoration through the Logos that merges Greek philosophy with Christian eschatology.
In \emph{Contra Celsum}, he explicitly defends Christianity against accusations of sedition by arguing that Christians seek not the overthrow of Rome by force but its transformation through divine truth.
But transformation toward what end?

Origen writes in \emph{Contra Celsum} 8.68:

\begin{quote}
We do say that there will eventually be, when the Word has prevailed over the entire rational creation and has converted every soul to His perfection, a time when He shall deliver up the kingdom to God the Father, when all rational beings shall have been moulded by the Word of God into one perfection.
\end{quote}

This is not otherworldly escapism.
Origen envisions a time when ``the Word has prevailed over the entire rational creation''---the \emph{oikoumene}, the civilized world.
The ``kingdom'' that Christ will ``deliver up to God the Father'' is a political reality: the universal kingdom unified under divine rule.

Origen's concept of \emph{apokatastasis} (ἀποκατάστασις)---the restoration of all things---is often spiritualized by modern interpreters, but his language is explicitly political.
He describes the eventual submission of all \emph{archontes} (ἄρχοντες, rulers) and \emph{exousiai} (ἐξουσίαι, authorities) to Christ (\emph{De Principiis} 3.6.6), using the vocabulary of imperial administration.
This is restoration theology: the reconstitution of a divinely ordered political cosmos, centered in the Greek-speaking East where Origen wrote.

\section{Irenaeus of Lyons (c.~130--202 AD): Providence and the Coming Kingdom}\label{sec:irenaeus-providence-kingdom}

Irenaeus, bishop of Lyons (a Greek-speaking city in Gaul), articulates a millennialist eschatology in \emph{Against Heresies} that envisions a literal earthly kingdom ruled by Christ.
Writing around 180 AD, Irenaeus interprets Daniel's vision of four kingdoms (Dan 2:31-45) as prophecy of the succession of empires---Babylonian, Median, Persian, and finally the Roman Empire---with the kingdom of God as the fifth and final kingdom that will crush and replace all earthly powers.

In \emph{Against Heresies} 5.26.1, Irenaeus writes:

\begin{quote}
In the end, the stone cut without hands [from Daniel] shall smite the image upon the iron and clay feet and shall break them to pieces, and shall itself fill the whole earth.
This refers to the kingdom of our Lord...which shall break in pieces and bring to an end all kingdoms, and which shall itself endure forever.
\end{quote}

This is not spiritual allegory---it is explicit political prophecy.
The ``stone'' that smites the feet of iron and clay (Rome) is the kingdom of Christ, which will physically ``fill the whole earth'' and ``break in pieces...all kingdoms.''
Irenaeus interprets this as the establishment of a millennial kingdom on earth where the risen saints will reign with Christ (5.32.1--5.35.2), directly paralleling Revelation 20:4-6.

Crucially, Irenaeus's vision is geographically centered: the kingdom will be restored in Jerusalem and the surrounding lands, but Jerusalem here functions as the symbolic capital of the renewed oikoumene---the civilized world now under divine rule, not Roman imperium.

\section{Tertullian (c.~155--240 AD)}\label{sec:tertullian-c.-155240-ad---in-his-writings-such-as-apology-and-on-the-resurrection-of-the-flesh-tertullian-often-implies-the-eventual-triumph-of-christianity-within-the-roman-empire-framing-it-as-part-of-a-divine-plan.-while-he-doesnt-directly-speak-of-the-restoration-of-the-empire-there-is-a-sense-of-christianity-fulfilling-the-destiny-of-the-roman-state.}

In his writings, such as \emph{Apology} and \emph{On the Resurrection of the Flesh}, Tertullian often implies the eventual triumph of Christianity within the Roman Empire, framing it as part of a divine plan.
While he doesn't directly speak of the ``restoration'' of the empire, there is a sense of Christianity fulfilling the destiny of the Roman state.

\section{Eusebius of Caesarea (c.~260--340 AD)}\label{sec:eusebius-of-caesarea-c.-260340-ad}

In his work \emph{Ecclesiastical History} and \emph{Life of Constantine}, Eusebius explicitly presents the rise of Constantine and the establishment of Christianity as the fulfillment of God's plan for the Roman Empire.
He sees Constantine's reign as a ``restoration'' of the empire, aligning it with divine will.
This reflects the idea that Christianity would not only restore the empire but also bring it to its true, Christian purpose.

\section{Athanasius of Alexandria (c.~296--373 AD)}\label{sec:athanasius-of-alexandria-c.-296373-ad}

In his writings, particularly in his defense against Arianism and his theological works, Athanasius talks about the cosmic victory of Christ over evil, which has implications for the empire's restoration.
He often frames the Christian emperor as the rightful ruler under divine guidance, which could be seen as linking the restoration of the empire to Christ's victory.

\section{Victorinus of Pettau (c.~250--303 AD)}\label{sec:victorinus-of-pettau-c.-250303-ad}

In his \emph{Commentary on the Apocalypse}, Victorinus draws connections between the Roman Empire and the eventual triumph of Christianity.
Like many of his contemporaries, he believes that the empire is part of God's plan and that its ultimate transformation into a Christian empire would bring about the fulfillment of prophecy.
This can be seen as a form of ``restoration'' through the christianization of the empire.

\section{The revolt is not militaristic---Rome chose to spiritually convert to Christianity, and the Eastern Roman Empire was restored peacefully while the Christian writers started to praise Rome.}\label{sec:the-revolt-is-not-militaristic-rome-chose-to-spiritually-convert-to-christianity-and-the-eastern-roman-empire-was-restored-peacefully-while-the-christian-writers-started-to-praise-rome.}

Lactantius, writing shortly before Constantine’s victory, declared in \emph{Divine Institutes} that Rome’s destiny was to become Christian, and that the empire itself was God’s instrument to unify the world.
Augustine of Hippo (354--430 AD)---Expanded In \emph{City of God}, Augustine offers a vision where the fall of the Roman Empire is viewed through a Christian lens.
He argues that the decline of the empire does not indicate the failure of divine providence.
While he focuses on the spiritual aspects of empire, he acknowledges the empire's role in preparing the world for the Christian kingdom, suggesting a ``restoration'' of the Roman Empire as a Christian entity in the future.

\section{The Shepherd of Hermas: The Most Popular Christian Text and the Nostalgia for Greek Glory}\label{sec:the-shepherd-of-hermas-c.-100-160-ad}

The \emph{Shepherd of Hermas} (c.~100--160 AD) was not merely one early Christian text among many---it was \emph{the most widely read Christian work} in the second and third centuries, more popular than most texts that later became canonical.
Manuscript evidence, patristic citations, and early Christian library catalogs demonstrate that \emph{Hermas} circulated more widely than the Gospels of Mark or John in many regions, was included in some early New Testament collections (Codex Sinaiticus), and was treated as Scripture by Irenaeus, Tertullian, Origen, and Clement of Alexandria.

This extraordinary popularity demands explanation.
Modern scholarship treats \emph{Hermas} as a bland moral allegory about repentance and church discipline, hardly the stuff of mass appeal.
But the political reading reveals why it resonated so powerfully with Greek-speaking audiences under Roman occupation: \emph{Hermas} is an extended allegory of imperial restoration, encoding Greek nostalgia for lost sovereignty in the imagery of an aging woman transformed back into youth, a ruined tower rebuilt to former glory, and a people purified and restored to their rightful place.

\subsection{The Vision of the Old Woman Transformed: Restoration of Greek Vigor}

The central vision of \emph{Hermas} (Vision 1--4) features an old woman who progressively becomes younger.
In the first vision, she appears elderly, seated in a chair, holding a book.
In subsequent visions, she grows younger, her face more vigorous, until in the fourth vision she appears as a radiant bride.
Hermas asks the Shepherd who this woman is, and the answer given is: ``the Church.''

But \emph{which} church, and \emph{what kind} of transformation?

Modern interpreters spiritualize this as the Church's moral renewal through repentance.
But the imagery is explicitly political: an old woman becoming young again is not moral improvement but \emph{restoration of youthful vigor}---precisely the language used in Greek and Roman political discourse for imperial renewal.
The Latin \emph{renovatio imperii} and Greek \emph{ananeōsis tēs arkhēs} (ἀνανέωσις τῆς ἀρχῆς) describe empire restored to its former strength after a period of decline.

The old woman represents the Greek-speaking world under Roman domination: aged, weakened, conquered.
Her transformation into a youthful bride represents the restoration of that world to its former glory---the re-establishment of Greek sovereignty, the renewal of Hellenistic imperial power.
This is not abstract ecclesiology; it is political allegory of the kind that circulated widely under occupation, where direct speech was dangerous but symbolic literature could encode resistance.

The woman's progressive rejuvenation mirrors the structure of resistance movements: first, the recognition of decline (the old woman, conquered and subjugated); then, the hope of restoration (her gradual youthing); finally, the vision of full restoration (the radiant bride).
This is the exact narrative arc of occupied peoples' literature: loss, resistance, and promised restoration.

\subsection{The Tower Vision: Rebuilding the Greek Oikoumene}

The most sustained allegory in \emph{Hermas} is the vision of the tower being built (Vision 3, Similitude 9).
The tower is constructed of stones, some of which are rejected, some of which require shaping and purification before they can be fitted into the structure.
The tower represents ``the Church,'' but the architectural imagery is unmistakably Greek: the tower (\emph{pyrgos}, πύργος) built of carefully fitted stones (\emph{lithoi}, λίθοι) echoes the monumental architecture of Hellenistic city-building.

Greek cities across the former Ptolemaic and Seleucid empires---Alexandria, Antioch, Ephesus, Pergamum---were distinguished by their monumental towers, gates, and colonnaded streets.
Under Roman rule, many of these structures fell into disrepair, their maintenance neglected as tribute flowed to Rome instead of local civic projects.
The vision of a tower being rebuilt, stone by stone, with careful selection and purification of materials, is a transparent allegory for the rebuilding of the Greek-speaking world: each stone a city or region, each act of purification a preparation for political restoration.

Critically, the stones that are rejected or discarded are those that cannot be fitted into the structure---those who refuse to participate in the restoration program.
This is not about individual salvation; it is about collective membership in a restored political body.
The language of purification and repentance (\emph{metanoia}) in \emph{Hermas} is civic, not merely moral: the call is to become worthy stones in the rebuilt tower, to prepare for the restoration of the kingdom.

\subsection{Why Was Hermas the Most Popular Christian Text?}

The extraordinary popularity of \emph{Hermas} in the Greek East becomes comprehensible once we read it as restoration allegory.
It was popular precisely because it articulated, in vivid and emotionally resonant imagery, the central political hope of occupied Greeks: that their world---aged and defeated---would be restored to youthful vigor, that their cities---ruined and neglected under Roman rule---would be rebuilt in glory, and that they themselves would be purified and made worthy of this restoration.

The text does not advocate immediate armed rebellion.
It preaches \emph{metanoia}---repentance, transformation, civic virtue---as the path to restoration.
This is the standard program of resistance movements under occupation: moral and civic renewal as preparation for political liberation.
The vision is gradualist, not revolutionary, but the end goal is explicit: the tower will be completed, the woman will be fully restored, and the faithful will reign in the renewed kingdom.

This is why \emph{Hermas} circulated more widely than most canonical texts in the second and third centuries.
It was not a theological treatise; it was the popular literature of Greek imperial restoration, written in allegory to evade Roman censorship but transparent to its intended audience.
The Greeks who copied, circulated, and venerated \emph{Hermas} understood exactly what it promised: the restoration of their kingdom, the rebuilding of their cities, and the vindication of their hope that God had not abandoned them to permanent subjugation.

When Constantine established the Christian Byzantine Empire in the fourth century, \emph{Hermas}' popularity began to wane.
The vision had been fulfilled: the tower was rebuilt as Constantinople, the old woman was restored as the youthful Christian empire, and the stones were fitted into place as the cities of the Greek East were reintegrated into a Christian imperial order.
The text had served its purpose, and its political function---to sustain hope under occupation---was no longer necessary.

\section{Clement of Alexandria's Exhortation to the Greeks (c.~190 AD)}\label{sec:clement-of-alexandrias-exhortation-to-the-greeks-c.-190-ad}

In his writings, Clement combines Christian eschatology with Greek philosophy, advocating for the return of the ``Logos'' and the eventual restoration of humanity to divine harmony.
While not strictly apocalyptic in the sense of a Revelation-style vision, his vision of the future aligns with the idea of cosmic renewal.
Clement's apocalyptic themes include the eventual restoration of the world through the Logos, an idea that ties back to the restoration of divine order similar to the eschatological views found in Revelation.

\section{Cyprian of Carthage's The Lapsed (c.~250 AD)}\label{sec:cyprian-of-carthages-the-lapsed-c.-250-ad}

Cyprian writes about the persecution of Christians and the imminent return of Christ.
His works anticipate the final judgment, the victory of the righteous, and the establishment of a divine kingdom.
Like other early Christian apocalyptic writers, Cyprian believed that the Church would be restored and triumph over its persecutors, reflecting the broader apocalyptic hope seen in Revelation.
