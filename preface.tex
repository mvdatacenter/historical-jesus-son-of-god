Few figures in history have generated as many competing portraits as Jesus of Nazareth.
Few movements have been as dissected, debated, and reimagined as early Christianity.
Across centuries of inquiry, scholars and theologians have offered radically different answers to the same question: Who was Jesus, and what was the origin of the movement that carried his name?

Some see him as an apocalyptic prophet, a moral teacher, a revolutionary zealot, a Cynic philosopher, a mystic, a Jewish messiah, a prophet of Allah, a purely mythical creation, or even an actual supernatural being.
Every one of these portraits points to real evidence, and every one has shaped entire schools of thought.

And yet none has proven definitive.
Each reconstruction captures something real and powerful, but each leaves contradictions unresolved when tested against the evidence as a whole.
Even the model often treated as the scholarly default in recent decades—the apocalyptic prophet—faces relentless criticism, with formidable counter-arguments advanced by a wide range of historians, theologians, and philosophers.
Far from being settled, the ``quest for the historical Jesus'' has had to be renewed again and again for over two centuries: not because the question is trivial, but because no single solution has yet succeeded in fitting all the data.

The paradox is sharp: Jesus and early Christianity are among the most richly documented figures and movements of antiquity, yet no reconstruction has managed to hold the evidence together without breaking at key points.
This book takes a different approach.
Its foundation is the use of artificial intelligence to synthesize an archive larger than any single human could absorb: two millennia of historical writing, theological speculation, textual criticism, archaeology, and cultural commentary.
Each generation of scholars left insights, but also blind spots.
AI allows us to read them all at once—to uncover patterns, highlight contradictions, and recover questions that have been hiding in plain sight.

With AI-assisted synthesis we are able to continuously revise large numbers of detailed hypotheses about Jesus and early Christianity.
Every hypothesis can quickly be tested against the entire body of data.
As the laws of logic state, the moment a contradiction appears, the hypothesis must be abandoned: two mutually exclusive claims cannot both be a part of the same coherent model.
Only the model that survives these collisions---accounting for more evidence and leaving no contradictions---deserves to stand.
While even the best AI systems are prone to major hallucinations, in this project they serve only as tools to help us read and identify patterns in the vast literature, not to make decisions or draw conclusions.

This is not a shortcut.
It is the opposite: the cumulative labor of countless scholars, disciplined by the logic of scientific testing, brought into a single field of vision.
AI lets us read the whole archive and keep instead what survives cross-examination.
In practice, time, incentives, and language barriers filter what gets read.
Even valuable non-religious historical works, or scholarship from other countries and cultures, often remain untouched.
We read everything, preserve what proves sound, and discard the rest.
Devout Christians tend to skip books critical of religion.
Secular scholars tend to skip arguments framed theologically.
Scientific thinkers tend to skip anything branded “fringe.”
At the same time works labeled as “fringe” often have a poor central premise, but very frequently contain profound insights or questions overlooked by the mainstream.
Across all camps—devout, secular, or unconventional—we find deeply insightful observations and powerful arguments.
We also confront accepted points whose consequences have not been fully explored.
The result is a sharper, more demanding inquiry into Jesus and the birth of Christianity than has been possible before.
And it leads to a striking possibility: the answer may have been hiding in plain sight all along.

This is not a theological work, and it does not aim to change the reader’s religious beliefs.
Rather, it seeks to offer a new framework that both deeply committed Christians and those with no religious affiliation—whether casual readers or professional scholars—may find intellectually valuable.
Jesus and early Christianity can be studied historically with the same rigor applied to any other figure of antiquity.
We do not seek to challenge faith, but we do seek to challenge assumptions widely held about the story of Jesus and his early followers.
Our aim is to invite readers to confront new and important questions about the historical Jesus and the origins of Christianity.

Beyond identifying questions, AI tools are also highly effective at spotting contradictions and inconsistencies in traditional narratives.
In particular, we revisit mainstream consensus on issues such as the dating and authorship of the Gospels and the original structure of the earliest Christian movement.
Examples include:
* **Johannine primacy and authorship** — We will show that what we know as John’s Gospel was likely the earliest, based on a source who was an eyewitness woman, most likely Mary Magdalene.
* **Late dating of Jesus’s birth** — We show that the birth narratives are far more consistent and historically grounded than most historians or theologians have recognized.

Other questions arise in the study of early Christianity:
* **The impossibly fast growth of Christianity in the first century** — We argue that a Christianity already existing during the time of Jesus is more plausible than the traditional model of a small sect that expanded only after his death, with founding texts written decades later.
* **The success of Christianity despite failed prophecies** — We argue that Jesus’s mission, as understood by his earliest followers, was in fact fulfilled and that the Kingdom did come.

Countless other questions arise from this analysis, and we invite readers to explore them throughout the book.
Not every question yields a certain answer, but the project aims to spark curiosity, deepen engagement with the Christian story, and inspire a renewed search for understanding through historical inquiry.
