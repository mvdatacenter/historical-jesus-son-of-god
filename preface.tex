\begin{itemize}
\item
  Historical Jesus as the Son of God: Glory to the Newborn King
\end{itemize}

\section{Preface}\label{par:preface}

Few figures in history have generated as many competing portraits as Jesus of Nazareth.
Few movements have been as dissected, debated, and reimagined as early Christianity.
Across centuries of inquiry, scholars and theologians have offered radically different answers to the question: Who was Jesus, and what was the origin of the movement that carried his name?

Some say he was the divine Son of God, performing miracles and rising from the dead.
Others see him as an apocalyptic prophet, a rabbi, a revolutionary zealot, a Cynic philosopher, a mystic, a Jewish messiah, a prophet of Allah, a purely mythical creation, or even an actual supernatural being.
Every one of these portraits points to real evidence, and every one has shaped entire schools of thought.

And yet — none has prevailed.
Each model explains something powerfully, but collapses when tested against the evidence as a whole.
This is why the “quest for the historical Jesus” has been renewed again and again for over two centuries: not because the question is trivial, but because no proposed solution has yet fit all the data.

The paradox is sharp: Jesus and early Christianity are among the best-documented figures and movements of antiquity, yet no reconstruction has managed to hold the evidence together without breaking at key points.

This book takes a different approach.
Its foundation is the use of artificial intelligence to synthesize an archive larger than any single human could absorb: two millennia of historical writing, theological speculation, textual criticism, archaeology, and cultural commentary.
Each generation of scholars left insights, but also blind spots.
AI allows us to read them all at once — to uncover patterns, to highlight contradictions, and to recover questions that have been hiding in plain sight.

With the AI-assisted synthesis we apply to it the scientific method.
Every hypothesis is tested against the entire body of data.
The moment a contradiction appears, the hypothesis must be revised or abandoned.
Only the model that survives these collisions — that accounts for more evidence and leaves fewer anomalies than all rivals — deserves to stand.

This is not a shortcut.
It is the opposite: the cumulative labor of countless scholars, disciplined by the logic of scientific testing, now brought into one field of vision.
Secondly, the AI can help us remove the biases that inevitably come from continued reliance on past scholarship and authority that often becomes so ingrained in anyone studying the subject in depth.
There have been countless brilliant points and insights that have been dismissed or ignored because they did not fit the prevailing narrative.
Devoted christians would not read a book openly criticizing religion.
Secular scholars would not listen to valid points made by religious scholars.
And scientific scholars would not consider points made by theories of Ancient Aliens.
And although the final analysis may not agree with a majority of viewpoints of many particular scholars, we find countless deeply insightful observations and fantastic arguments from all groups whether, secular, deeply religious, or even considered fringe theories.
We also look at many points, modern or not, that have often gained general acceptance but their consequences on the scholarship have not been fully explored.
The result is a sharper and more demanding inquiry into Jesus and the birth of Christianity than has ever before been possible.
And it leads us to a striking possibility: the answer may have been hiding in plain sight all along.

This is not a theological work, and it does not aim to change the reader’s religious beliefs.
Rather, it seeks to offer a new framework that both deeply committed Christians and those with no religious affiliation — whether casual readers or professional scholars — may find intellectually valuable.
Jesus and early Christianity can be studied historically with the same rigor applied to any other figure of antiquity.
While we do not seek to challenge anyone’s faith, we do seek to challenge assumptions widely held about the story of Jesus and his early followers.
Our aim is to invite readers to confront new and important questions about the historical Jesus and the origins of Christianity.

Beyond identifying questions, AI tools are also highly effective at spotting contradictions and inconsistencies in the traditional narratives.
In particular, we revisit the mainstream scholarship consensus on issues such as the dating and authorship of the Gospels and the original structure of the earliest Christian movement.
Examples of issues that seem to have a strong body of evidence behind them, but are nearly inexplicably dismissed in favor of alternatives with far poorer evidence, include:
* Johannine primacy and authorship - We will show mostly what we know as John's Gospel was likely written first based on a source who was an eyewitness woman, most likely Mary Magdalene.
* Late dating of Jesus’s birth - We show that Jesus birth narratives seem far more consistent and historically grounded than almost any historian or even theologian has recognized.

The other deeply intriguing questions arise in the study of early Christianity.
* The impossibly fast growth of Christianity in first century - We show Christianity that existed already during the time of Jesus is far more plausible than the traditional model of a small sect that exploded after Jesus's death and founding text being written many decades or centuries later.
* The success of Christianity despite failed prophecies - We show that the mission of Jesus as understood by his earliest followers was actually fulfilled and the Kingdom did come.

There are countless other fascinating questions that arise from this analysis, and we invite readers to explore them in depth throughout the book.
While not every question has a certain answer, this project ultimately aims to spark curiosity, deepen engagement with the Christian story, and inspire a renewed search for understanding through historical inquiry.

\section{The Historical Background}\label{par:background-historical}

On 31 Aug the year 326 BCE, Alexander the Great, King of the Macedon stood on the banks of the river Hydaspes in India and wept because there were no more worlds to conquer.
In the year 323 BCE, Alexander the Great died in Babylon and left his empire to the strongest among his men.
His empire was divided with the biggest share and the imperial title going to Seleucus Nicator.
Under the greek rule there came an era of enlightenment and prosperity in all the nations of the world.
In a short span of time countless colonies were founded and given the law, currency and culture of the Greeks.
Of all the cities under the sun, Ephesus, Antioch, Thessalonica, Laodiciea, Philippi, Corinth, Athens, Tarsus and Alexandria rose as the greatest seats of learning.

In 146 BCE, the Roman general Lucius Mummius destroyed Corinth, and Polybius lamented, ``The day will come when men will ask where once stood mighty Corinth.'' In the year 85 BCE, to the shock of the world, the Roman general Lucius Cornelius Sulla fought and destroyed the combined 350,000 strong army of the Greek would.
Athens, once the teacher of the world, lay in ruins.
In 31 BCE, at Actium, fortune truly turned away from the Greeks and embraced the Romans as Gaius Julius Caesar Octavianus defeated the combined forces of the Greek world and Mark Antony.
In the East the Greek world was attacked by the Parthians and Scythians.
Meanwhile, in the East, the unstoppable tide of Scythians pressed upon the remnants of Alexander's empire.
The last Greek king of Bactria, Strato II Soter, fell to king Rajuvula around 10 CE.
With that the fall of the entire Greek world was no more, well, not exactly\ldots{} When the general Sulla sued for peace he did not fully incorporate the Judea, a rebellious land, and permitted the Greek dynasties of the Hasmoneans and Herodians to continue to rule as client kingdoms of Gallilee, Samaria, Judea, Decapolis.
And so the imperial court officials of the Greeks, the Head of the Imperial Guard, the Keeper of Imperial Light and the Imperial Treasurer, came to Galilee from the East to seek the last rightful heir to the empire.

As we read form the works of the Jewish historian Flavius Josephus:
``About this time there lived Jesus, a wise man, if indeed one ought to call him a man.
For he was one who performed surprising deeds and was a teacher of such people as accept the truth gladly.
He won over many Jews and many of the Greeks.
He was the Christ.
When Pilate, upon hearing him accused by men of the highest standing amongst us, had condemned him to be crucified, those who had in the first place come to love him did not give up their affection for him.
On the third day he appeared to them restored to life, for the prophets of God had prophesied these and countless other marvelous things about him.
And the tribe of the Christians, so called after him, has still to this day not disappeared.
``
And so the Jesus Christ was called the Son of God, the King of Kings, the Lord of Lords, the Savior of the World, the Light of the World, the Prince of Peace, the Lamb of God, the Good Shepherd, the Way, the Truth, the Life, the Alpha, the Omega.
Jesus died for the sins of others so that those who believe in him may not perish but have eternal life.

There is a bird which is called the Phoenix.
This is the only one of its kind and lives five hundred years.
When the time of its dissolution draws near, it makes for itself a coffin of frankincense and myrrh and other spices, and when the time is fulfilled it enters it and dies.
But as its flesh decays, a worm is produced, which is nourished by the moisture of the dead creature and puts forth wings.
Then, when it has grown strong, it takes up that coffin and flies from the land of Arabia to Egypt, to the city of Heliopolis, and, in the daytime, in the sight of all, it places itself on the altar of the sun.

