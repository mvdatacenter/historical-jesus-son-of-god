„Poszukiwania historycznego Jezusa” to współczesny projekt badawczy, którego celem jest odtworzenie postaci Jezusa w ramach historii, a nie teologii.
Rozpoczął się w epoce Oświecenia, rozkwitł w XIX wieku w postaci wielu „Żywotów Jezusa”, został ograniczony przez wczesny dwudziestowieczny sceptycyzm, a od połowy XX wieku odradza się w kolejnych falach.
Od tego czasu niezliczone prace proponowały konkurujące hipotezy na temat tego, kim był Jezus i co oznaczało jego przesłanie.
Zanim przejdziemy do własnej analizy, warto zarysować najbardziej wpływowe z tych portretów, z których każdy przyciągnął poważną uwagę współczesnej nauki.

\section{Prorok apokaliptyczny.}\label{sec:apocalyptic}

W tym ujęciu Jezus wpisuje się w apokaliptyczne nurty późnego judaizmu epoki Drugiej Świątyni.
Jego głoszenie „królestwa Bożego” (βασιλεία τοῦ θεοῦ) rozumie się nie jako ponadczasową etykę, lecz jako ogłoszenie zbliżającego się w czasie działania Boga: „królestwo przybliżyło się” (ἤγγικεν; Mk 1,15).
Retoryka jest nagląca — czuwanie, rozdzielenie, żniwo, rozliczenie — a rama myślowa ta sama, którą znajdujemy w Księdze Daniela, 1 Henocha, 4 Ezdrasza i w Zwojach znad Morza Martwego: Bóg sądzi niegodziwych, usprawiedliwia sprawiedliwych, przywraca Izrael i porządkuje świat na nowo.

Zwolennicy tego ujęcia wskazują na grupę wypowiedzi, które naturalnie wpisują się w taki horyzont.
Jezus mówi o „Synu Człowieczym” (ὁ υἱὸς τοῦ ἀνθρώπου), który ma zostać objawiony, o znakach kosmicznych i o bliskości wydarzeń mierzonej „tym pokoleniem” (ἡ γενεὰ αὕτη).
„Mała Apokalipsa” (Mk 13 i paralelne miejsca synoptyczne) łączy w jednym łuku ucisk, zbezczeszczenie i przyjście Syna Człowieczego „na obłokach” (por. Dn 7,13).
Nawet odpowiedź Jezusa na pytanie Jana Chrzciciela — „niewidomi widzą, chromi chodzą, ubogim głosi się dobrą nowinę” — nawiązuje do eschatologicznej listy znaków (por. Iz 35; 61; 4Q521), wiążąc uzdrowienia z odnową czasów ostatecznych, a nie jedynie z samym współczuciem.
W takiej perspektywie prorockie gesty symboliczne — działanie w Świątyni, wybór Dwunastu, otwarte uczty — stają się czynami-znamionami zapowiadającymi bliską restaurację Izraela.

Niektórzy badacze ujmują ten sam materiał mniej w kategoriach kalendarza, a bardziej

\label{sec:mythical}

Stanowisko miticystów obejmuje zarówno całkowite zaprzeczanie istnieniu Jezusa, jak i twierdzenie, że nawet jeśli żył, to pod warstwą mitów nie da się już odzyskać żadnego wiarygodnego rdzenia historycznego.
Argument opiera się na dwóch filarach: późnym i anonimowym charakterze źródeł oraz gęstym splocie mitologicznych i literackich paralel w świecie starożytnym.

Sam Jezus niczego nie napisał.
Ewangelie, jakie znamy, po raz pierwszy pojawiają się tylko pośrednio, gdy Ireneusz pod koniec II wieku opisuje działalność Marcjona około 140 roku po Chr.
Choć często datuje się je na późny I wiek, istnieją mocne argumenty także za datami sięgającymi połowy II wieku.
Pisma z I i wczesnego II wieku nie zawierają jednoznacznych odniesień do znanych nam Ewangelii czy listów, a można by ich oczekiwać, gdyby teksty te już krążyły.
Nawet pod koniec I wieku większość uczniów Jezusa była zapewne już dawno martwa lub na tyle wiekowa, że trudno byłoby im tworzyć dzieła literackie.
Teksty są anonimowe, a przypisanie ich Mateuszowi, Markowi, Łukaszowi i Janowi nastąpiło dopiero znacznie później u ojców Kościoła, często na podstawie domysłów.
Ireneusz, który w innych miejscach formułuje odważne i kategoryczne twierdzenia, w kwestii autorstwa Ewangelii jest wyraźnie ostrożniejszy.
Wyjaśnia swoje rozumowanie, sprawiając wrażenie, jakby proponował uczoną rekonstrukcję, a nie dobrze zachowaną tradycję sięgającą apostołów.

Ewangelie nie przedstawiają niezależnych głosów naocznych świadków, lecz warstwową tradycję literacką: najpierw powstał Marek, Mateusz i Łukasz go przeredagowali, a Jan przepracował całość teologicznie.
Większość badaczy zgadza się, że mamy do czynienia z drugą lub trzecią redakcją ewangelii, a nasze teksty są kopiami kopii kopii, przy czym wiele pierwotnych treści prawdopodobnie przepadło.
Narracje są nasycone zapożyczeniami z Pisma.
Opis Męki przypomina świadome złożenie w całość motywów z Psalmu 22, Izajasza 53 i Księgi Daniela 7.
Mateusz jawnie nawiązuje do Wyjścia, Ozeasza i Micheasza, budując całą opowieść tak, by ukazać Jezusa jako nowoczesnego Mojżesza.

Paralele wykraczają jednak daleko poza samo Pismo.
Uzdrowienia, egzorcyzmy i biografia Jezusa mają setki uderzających podobieństw do życia Apolloniosa z Tiany.
Cud z winem w Kanie, Eucharystia i zmartwychwstanie przywołują kult Dionizosa.
Język „nowego narodzenia” i „nowego stworzenia” przypomina obrzędy Mitry, Izydy czy Eleusis.
Jego męka i śmierć odsyłają do motywów umierających i zmartwychwstających bóstw, takich jak Ozyrys, Attis czy Dionizos.
Opowiadanie o opętanym z Gerazy i Legionie wykazuje zdumiewające podobieństwo do historii Odyseusza i Polifema.
Modlitwa Pańska ma uderzająco wiele wspólnego ze słonecznymi hymnami do Atena.
Gołębica zstępująca przy chrzcie zachowuje się dokładnie tak, jak Zeus zstępujący na ziemię.

Nawet proces i śmierć wpisują się w znane schematy.
Szlachetnie cierpiący nauczyciel stający przed władzą państwową przywołuje proces Sokratesa.
Filozofowie skazywani za rzekome deprawowanie młodzieży lub kwestionowanie bogów tworzą jasne tło kulturowe.
W takiej lekturze ukrzyżowanie staje się śródziemnomorskim wariantem motywu sprawiedliwego mędrca niesłusznie straconego przez władzę, a mimo to ostatecznie usprawiedliwionego przez Boga.

Listy Pawła, czytane w ten sposób, nie opisują żadnego galilejskiego rabina.
Głoszą niebiańskiego Chrystusa objawionego w wizjach i przez Pismo.
Paweł nigdy nie przytacza przypowieści, nie wspomina Nazaretu, nie opowiada o cudach.
Jego Jezus jest kosmiczny, a nie biograficzny.

Najwcześniejsze zewnętrzne świadectwa są późne i pośrednie.
Józef Flawiusz bywa odrzucany jako tekst zinterpolowany.
U Tacyta dostrzega się relację o wierze chrześcijan, a nie niezależną wiedzę archiwalną.
„Chrestus” u Swetoniusza często uznawany jest za postać niezwiązaną z Jezusem.

W tej rekonstrukcji chrześcijaństwo nie wyrasta z pamięci o konkretnej osobie, lecz z mitu, alegorii i rytuału.
Paralele z kultami misteryjnymi, wiarą dionizyjską, Apolloniosem i procesami filozofów są tak gęste, że opowieść o Jezusie przestaje przypominać wyjątkową biografię, a zaczyna wyglądać jak kolaż znanych motywów kulturowych.
Niezależnie od tego, czy gdzieś u korzenia istniał realny człowiek, Jezus znany z tekstów jawi się jako mit od podstaw, konstrukt teologiczny, a nie nauczyciel historyczny.

Wielu krytyków twierdzi, że te paralele są wymuszone i nienaturalne, ale to, czy są one „prawdziwe”, ma niewielki wpływ na wiarygodność tego modelu jako całości.
Tak po prostu opowiadano historie w świecie Jezusa.
Nie znaczy to, że zasadniczy rdzeń opowieści jest fałszywy.
Inni krytycy podkreślają ekstremalną dokładność geograficznych i niezwiązanych z teologią szczegółów historycznych w tekstach.
Nawet jeśli uznamy poprawność wszystkich tych danych, one również niewiele wnoszą do zasadniczego pytania, czy Jezus był postacią mityczną.
Stan Lee pisał o Nowym Jorku i wydarzeniach tam rozgrywających się z niebywałą drobiazgowością, ale nie czyni to historii o Spider-Manie bardziej historycznymi.
Słabości tego modelu leżą gdzie indziej.

Najpoważniejszy problem tej teorii wiąże się z drobnymi szczegółami, które są istotne dla fabuły, a jednak wyraźnie nie do końca zrozumiane przez autorów tekstów.
W przeciwieństwie do realizmu tła, który można swobodnie wymyślać, takie fragmenty wyglądają jak autentyczne okruchy pamięci przepisane bez pełnego rozeznania.
Ewangelia Jana na przykład podaje, że Józef z Arymatei i Nikodem przynoszą mieszaninę mirry i aloesu, by namaścić ciało Jezusa.
Późny, całkowicie mityczny Jan raczej nie wprowadziłby środków leczniczych, gdyby Jezus miał być pospiesznie pochowany.
Albo Mateusz, który wspomina o Ojcu wynagradzającym „w ukryciu”.
Późny, mityczny Mateusz nie cytowałby podsłuchanej teologii egipskiej w żydowskim kontekście prorockim.
Albo późniejsze teksty, które nazywają magów Kacprem, Melchiorem i Baltazarem.
Późna, całkowicie mityczna narracja o narodzinach nie operowałaby tymi dworskimi tytułami, a jednocześnie nazywała ich magami.
Tło tych i wielu innych przykładów wyjaśnimy w kolejnych rozdziałach.

Dodatkową trudność stanowi fakt, jak trudno wyjaśnić liczbę autorów, którzy wydają się wybiórczo znać geografię i język wydarzeń, w których rzekomo uczestniczyli.
Znajdujemy zbyt wiele wzajemnych potwierdzeń w źródłach, które równocześnie mocno różnią się w kwestii elementów najpewniej zmyślonych.
Każda wersja opowieści przytacza inne cuda i inne słowa Jezusa na krzyżu, ale opis procesu jest uderzająco spójny.
Jesteśmy przekonani, że już w kilka lat po śmierci Jezusa istniały poważne spory co do jego tożsamości i przesłania.
Najwcześniejsi autorzy nie wahali się proponować odmiennych interpretacji, lecz byli bardzo powściągliwi, jeśli chodzi o wymyślanie samego szkieletu wydarzeń.
Trudno wyobrazić sobie mit, w którym część centralnych elementów byłaby tak bardzo rozbieżna, a wiele pozornie drugorzędnych szczegółów tak szeroko zgodnych.
Ten wzorzec równoczesnego potwierdzania i niezgody niezwykle trudno pogodzić z czystą fikcją.

Z takiej lektury wyłania się nie tyle fabrykacja, ile historia załamana przez pryzmat języka mitu — opowieść zakorzeniona w realnym doświadczeniu, ale wyrażona w idiomie epoki.
I po raz kolejny, choć elementy mityczne zapewne odpowiadają za znaczną część narracji, mało prawdopodobne, by stanowiły jej przeważającą całość.

\section{Uzdrowiciel, egzorcysta i prorok społeczny.}\label{sec:healer}

Jednym z najtrwalszych portretów Jezusa jest obraz uzdrowiciela i egzorcysty, którego czyny mocy gromadziły tłumy i nadawały jego misji natychmiastowy autorytet.
Ewangelie przedstawiają ten wymiar nie jako ozdobnik, lecz jako centrum: egzorcyzmy demonów, przywracanie wzroku i władzy w nogach, oczyszczanie trędowatych oraz działania, które przywracały wspólnotę osobom rytualnie nieczystym.
Tradycje te są liczne, wczesne i szeroko rozpowszechnione, a do tego współbrzmią z żydowskimi wspomnieniami świętych mężów, takich jak Choni Kreślarz Okręgów czy Chanina ben Dosa, przywoływanych jako orędownicy w modlitwie o deszcz czy uzdrowienie.
Harmonizują także ze śródziemnomorską tradycją cudotwórców i „boskich ludzi”, których autorytet opierał się na widocznej mocy.
Nic dziwnego, że powstała ogromna liczba prac rozwijających ten portret na różne sposoby: jedne podkreślają figurę żydowskiego świętego męża, inne skupiają się na reintegracji społecznej, jeszcze inne na gospodarce politycznej, religijnym doświadczeniu czy szerszych paralelach śródziemnomorskich.
Choć akcenty są różne, razem tworzą one obraz złożony: czyny mocy Jezusa były integralną częścią jego reputacji i przesłania.

W takim ujęciu uzdrowienia można czytać jako przypowieści w działaniu o królestwie Bożym.
Egzorcyzm oznacza pokonanie wrogich mocy.
Uzdrowienie staje się aktem odnowy Izraela.
Wspólny posiłek znosi granice czystości i statusu.
Odczytania społeczne podkreślają, że choroba oznaczała wykluczenie, a uzdrowienie — powrót do rodziny, wsi i życia w przymierzu.
Interpretacje polityczne wskazują, że kody czystości i systemy długów służyły jako narzędzia kontroli elit, tak że uzdrowienia i uczty Jezusa wcielały w życie alternatywny porządek społeczny.
Ujęcia doświadczeniowe akcentują jego poczucie synostwa i modlitwę jako źródło mocy.
Porównania z innymi tradycjami sytuują Jezusa wśród śródziemnomorskich uzdrowicieli i mędrców, przy czym jego językiem pozostaje Pismo Izraela, a horyzontem — odnowa przymierza.
Interpretacje liturgiczne pokazują, jak opowieści o uzdrowieniach stały się rytuałami pamięci, kształtując sposób nauczania, przepowiadania i modlitwy we wspólnotach.

Mocne strony tego portretu to szerokość i siła wyjaśniająca.
Pomaga on zrozumieć, dlaczego gromadziły się tłumy, dlaczego tworzyło się grono uczniów i dlaczego przeciwnicy reagowali tak ostro.
Pokazuje, jak mowa o „królestwie” przybierała konkretną formę w widzialnych działaniach, a nie tylko w słowach.
Wyjaśnia także trwałość pamięci o Jezusie: ludzie zapamiętują otrzymaną pomoc.

Równie istotne są jednak jego słabości.
Problemem nie jest to, że uzdrowienia są „niewiarygodne”, lecz że same w sobie nie wystarczają jako pełne wyjaśnienie.
W starożytności wielu ludzi zapamiętano jako uzdrowicieli, a niezliczeni kapłani przez wieki dokonywali egzorcyzmów.
Taka kategoria sama w sobie nie tłumaczy, dlaczego właśnie ten Galilejczyk stał się centrum ruchu, który przekształcił historię.

Oczywiste są też ograniczenia metodologiczne: historycy nie mogą weryfikować cudów tak, jak weryfikują monety czy inskrypcje, a kształt narracji zaciera granice między tym, co zaszło, a tym, jak to później opowiadano.
Tradycje o uzdrowieniach nakładają się z niemal każdym innym portretem: prorocy apokaliptyczni, nauczyciele mądrości, rewolucjoniści i prorocy odnowy włączają je jako potwierdzenie własnej ramy interpretacyjnej.
Wczesne listy, takie jak pisma Pawła — nasze pierwsze teksty chrześcijańskie — wspominają wizje i dary duchowe, ale nie przedstawiają Jezusa szczegółowo jako uzdrowiciela, co sugeruje, że opowieści o cudach zyskały na znaczeniu po jego śmierci.
Same historie różnią się między sobą — jedne uzdrowienia są natychmiastowe, inne stopniowe; jedne wyraźnie powiązane z wiarą, inne nie — co wskazuje zarówno na pamięć, jak i na kształtowanie przez wspólnotę.
I wreszcie najważniejsze: same uzdrowienia nie tłumaczą ukrzyżowania.
Rzym nie krzyżował ludzi za leczenie chorych.
Zagrożenie musiało tkwić w głoszeniu królestwa Bożego, w symbolicznym wyzwaniu rzuconym Świątyni, w roszczeniu dynastycznym albo w postrzeganym zagrożeniu politycznym.

W rezultacie powstaje portret, który zabezpiecza jeden nieodzowny nurt pamięci, lecz nie wystarcza jako samodzielne wyjaśnienie.
Jezus najprawdopodobniej rzeczywiście dokonywał uzdrowień i egzorcyzmów, a wspólnota pamiętała te czyny jako znaki mocy Bożej i odnowy społecznej.
Trzeba je jednak osadzić w szerszym horyzoncie — apokaliptycznej nagłości, roszczeń dynastycznych, wyzwania wobec władzy — aby zrozumieć, dlaczego został ukrzyżowany jako „Król Żydowski” i dlaczego jego uczniowie ogłaszali go jako Pana.

\section{Mędrzec i nauczyciel.}\label{sec:teacher}

Ten złożony portret gromadzi wątki, które widzą w Jezusie przede wszystkim nauczyciela.
Mówi on w przypowieściach i aforyzmach, które odwracają zwyczajową hierarchię honoru i wyznaczają nowe horyzonty moralne.
Debatuje nad halachą z wielką skrupulatnością, a zarazem kładzie nacisk na miłosierdzie i „ważniejsze sprawy Prawa”.
Wykonuje gesty symboliczne, które wskazują na odnowę przymierza, a nie tylko na prywatną pobożność.
W tej perspektywie jawi się jako nauczyciel mądrości w tradycji sapiencjalnej Izraela, rozmówca prawny bliski faryzeuszom, a miejscami wiejski mędrzec, którego styl może przypominać cyników przez ostrą mowę, dobrowolne ubóstwo i otwarty stół.

Atuty tego obrazu są znaczące.
Przypowieści stanowią najbardziej stabilny rdzeń zapamiętanej mowy i dobrze wpisują się w formy żydowskiej mądrości.
Zwięzłe aforyzmy łatwo krążą między wspólnotami i tłumaczą powstanie zbiorów wypowiedzi.
Spory prawne o szabat, czystość, rozwód i dziesięciny sytuują Jezusa wewnątrz żydowskiego Prawa, a nie poza nim.
Gesty symboliczne — zwłaszcza działanie w Świątyni i wybór Dwunastu — harmonizują z nadziejami na odnowę Izraela.
Porównania kulturowe pokazują, że jego sposób życia — wędrowne nauczanie, wspólne posiłki przekraczające podziały statusu, celna krytyka — znajduje realne analogie w świecie śródziemnomorskim, nie zacierając przy tym jego zakorzenienia w Piśmie.

Ten portret pomaga też zrozumieć, dlaczego wokół Jezusa gromadzili się uczniowie.
Nauczyciele przyciągają słuchaczy.
Zapadająca w pamięć mowa, wcielana w praktyki publiczne, tworzy wspólnoty, które ją zachowują i powtarzają.
Wyjaśnia także trwałość: mądrość lepiej znosi opóźnienia i rozczarowania niż precyzyjne harmonogramy wydarzeń.

Słabości tego ujęcia mają inny charakter.
„Nauczyciel” to kategoria zbyt pojemna, by tłumaczyć finał opowieści.
Rzym nie krzyżował mędrców za przypowieści.
Aby zrozumieć oskarżenie „Król Żydowski”, profil nauczyciela trzeba połączyć z roszczeniami, które brzmiały politycznie — z mową o królestwie, Świątyni, pokoleniu i władzy.
Paralela z cynikami może posunąć się za daleko, jeśli pomniejszy gęste zaangażowanie Jezusa w Torę i historię Izraela.
Akcent na sporach prawnych może spłaszczyć obraz, jeśli uczyni z niego po prostu łagodniejszego faryzeusza.
Motyw odnowy może zaś pozostać mglisty, jeśli nie doprecyzujemy, co dokładnie miało się zmienić w zakresie Świątyni, ziemi i królowania.

Istnieje też problem selektywności.
Z samych tylko wypowiedzi można złożyć niemal dowolny portret, jeśli dość ostro przycina się kontekst.
Przypowieści o miłosierdziu stoją obok ostrzeżeń o sądzie i obrazów królewskich.
Mądrość i eschatologia są w źródłach splecione, a nie starannie rozdzielone.

Najbardziej spójna lektura widzi nauczanie jako medium, a nie jako całą treść.
Przypowieści, aforyzmy i spory prawne są sposobem, w jaki Jezus prowadził większą sprawę: ogłaszał władzę Boga i odnowę Izraela.
W tym ujęciu portret mędrca i nauczyciela jest niezbędny, by zrozumieć, jak jego słowa zostały zapamiętane, przekazane i wcielone w życie.
Samodzielnie nie wyjaśnia jednak, dlaczego skazano go na śmierć przez ukrzyżowanie ani dlaczego jego uczniowie ogłaszali go Panem.
Podobnie jak inne portrety, wskazuje on realny środek ciężkości, który jednak wymaga połączenia z apokaliptycznym napięciem, symbolicznym królowaniem i publicznym wyzwaniem rzuconym władzy.

\section{Król dynastyczny lub pretendent do tronu.}

\label{sec:dynastic}

Kolejny portret widzi w Jezusie postać dynastyczną, pretendenta do tronu, którego domostwo zachowało ciągłość sukcesji po jego śmierci.
Do współczesnych prac rozwijających ten wątek należą James D.\ Tabor, \emph{The Jesus Dynasty}; Robert Eisenman, \emph{James the Brother of Jesus}; Hyam Maccoby, \emph{Revolution in Judea}; Barbara Thiering, \emph{Jesus the Man}; Simcha Jacobovici i Charles Pellegrino, \emph{The Jesus Family Tomb}; oraz Michael Baigent, Richard Leigh i Henry Lincoln, \emph{Holy Blood, Holy Grail}.

W całym tym dorobku powraca kilka linii argumentacji.
Pierwsza podkreśla napis na krzyżu „Król Żydowski” i odczytuje go wprost jako oskarżenie o roszczenia królewskie.
W tym ujęciu wjazd do Jerozolimy jest procesją królewską, działanie w Świątyni gestem ustrojowej suwerenności, a wybór Dwunastu podziałem administracji plemiennej pod odnowionym panowaniem.
Rodowody u Mateusza i Łukasza traktuje się jako zapisy dynastyczne mające potwierdzić legalne pochodzenie i prawo do władzy.

Druga linia akcentuje sukcesję rodzinną jako mechanizm ciągłości.
Bracia i krewni Jezusa, zapamiętani jako \emph{desposyni}, jawią się jako kontynuacja jego linii.
Jakub Sprawiedliwy przedstawiany jest jako bezpośredni następca w Jerozolimie, przewodniczący decyzjom halachicznym i kierujący wspólnotą.
Dynastia rozciąga się dalej przez krewnych, takich jak Symeon, syn Kleofasa, tworząc w pamięci sekwencję kolejnych pretendentów królewskich.

Jeszcze inny nurt podkreśla podwójny wzór urzędu mesjańskiego.
W tym modelu Jan Chrzciciel działa jako mesjasz kapłański, a Jezus jako mesjasz królewski.
Chrzest odczytuje się jako akt inwestytury, który łączy kapłaństwo i królowanie oraz ustanawia podwójną, wspólną legitymizację.

Kolejna linia argumentacji wiąże Jezusa bezpośrednio z dynastią hasmonejsko–herodiańską.
W tej rekonstrukcji pamięta się go nie tylko jako pretendenta z linii Dawida, lecz także jako wnuka lub potomka Heroda Wielkiego, wywodzonego od Antypatra lub innej gałęzi rodu.
Taki rodowód tłumaczyłby stałe łączenie jego narodzin z królewskim zagrożeniem, starania dworu Heroda, by go zgładzić, oraz polityczny charakter napisu na krzyżu.
Wyjaśniałby także ewangeliczne genealogie jako propagandę dynastyczną, która ujmuje jego linię w idiomie Izraela, a zarazem sytuując go w rywalizacji wobec innych dziedziców tronu.
W tej wersji portretu dynastycznego ukrzyżowanie Jezusa interpretuje się mniej jako los marginalnego proroka, a bardziej jako unieszkodliwienie groźnego pretendenta do władzy wewnątrz samego domu herodiańskiego.

Argumenty archeologiczne dodają jeszcze jeden wymiar.
Grób w Talpiot w Jerozolimie interpretuje się jako pochówek rodzinny z ossuariami opatrzonymi inskrypcjami o imionach odpowiadających postaciom ewangelicznym.
Statystyczna analiza tego skupiska imion ma dowodzić, że taki zestaw jest mało prawdopodobny jako czysty przypadek.
W ramach portretu dynastycznego miejsce to służy jako materialne potwierdzenie miejskiego domu o królewskich ambicjach.

Niektóre ujęcia rozwijają ten portret dalej w głąb historii, poprzez hipotezę linii krwi.
W tej lekturze „Święty Graal” rozumie się jako krew królewską (\emph{sang real}).
Genealogię prowadzi się przez wygnanie i pamięć, a przetrwanie rodu opowiada w tradycjach średniowiecznych i wczesnonowożytnych.

Przeciw portretowi dynastycznemu często podnosi się kilka zarzutów.
Rodowody Mateusza i Łukasza różnią się w szczegółach i wydają się sprzeczne z późniejszymi tradycjami o dziewiczym poczęciu.
Twierdzi się, że tytulatura królewska rzadko pojawia się w zachowanych odniesieniach, co skłania niektórych do wątpienia, czy Jezusa w ogóle pamiętano jako króla.
Grób w Talpiot budzi spór, ponieważ imiona, które zawiera, są z jednej strony powszechne, a z drugiej — skupione w sposób przypominający rodzinę z Ewangelii.
Te słabości zasługują na wnikliwą analizę, bo szczegółowe omówienie może istotnie wpłynąć na ocenę modeli dynastycznych.
Do każdej z nich wrócimy w kolejnych rozdziałach, gdzie sprawdzimy, czy są naprawdę zabójcze, czy da się je pogodzić w ramach spójnej rekonstrukcji dynastii.

O wiele trudniejszy jest problem natury społecznej.
Jeśli roszczenie Jezusa było po prostu roszczeniem żydowskiej dynastii, trudno zrozumieć, dlaczego po jego śmierci nie poszło za nim trwałe żydowskie poparcie, lecz niemal wyłącznie ruch pogański.
Najwcześniejsze świadectwa pokazują synagogi podzielone, ale w ciągu jednego pokolenia żydowska przynależność staje się marginalna, podczas gdy miasta greckie w całym imperium goszczą prężne zgromadzenia.
Jeżeli napis na krzyżu „Król Żydowski” dobrze oddawał jego roszczenie, dlaczego sami Żydzi nie zachowali jego królowania?
Dlaczego zamiast tego to poganie, którzy nie mieli żadnego udziału w linii Dawida, stali się jego głównymi wyznawcami?
Ta nierównowaga między żydowskim odrzuceniem a pogańską akceptacją pozostaje najtrudniejszym napięciem do pogodzenia w wąsko judejskim modelu dynastycznym.

\section{Postać nadprzyrodzona.}\label{sec:supernatural}

Większość współczesnych badań mieści się w ramach naturalistycznych, ale w niniejszym projekcie nie zakładamy z góry, że Bóg, Allah czy nawet nieludzkie inteligencje muszą być uznane za niemożliwe wyjaśnienia.
Portret „postaci nadprzyrodzonej” traktuje poważnie twierdzenie, że Jezus był kimś więcej niż tylko człowiekiem albo że rzeczywiście przemawiał i działał z mandatu boskiej lub nieludzkiej istoty.
Obejmuje to zarówno klasyczne wyznanie wiary w Boga wszechmogącego Stwórcę wcielonego, jak i rywalizujące starożytne wizje, w których Jezus zstępuje jako niebiański objawiciel, wysłannik wyższego Boga, prorok Allaha czy istota spoza porządku stworzenia, a także nowoczesne ujęcia przedstawiające go jako kogoś „z innego świata” w kategoriach technologicznych.
We wszystkich tych wariantach teza jest zasadniczo ta sama: Jezus działał z nieludzką sprawczością, a ludzie doświadczali go jako żywej mocy.

Pozytywny argument jest prosty.
Najwcześniejszy kult traktuje go jako \textgreek{κύριος} i Syna, a nie tylko nauczyciela.
Wspólnoty wzywają jego imienia w modlitwie, hymnach i podczas posiłków, tak jakby działała tam boska obecność.
Tradycje o cudach i egzorcyzmach są centralne, a nie ozdobne.
Rywalizujące ruchy, które negowały jego ciało lub oddzielały go od Boga Izraela, i tak zakładały nadprzyrodzonego Jezusa.
Trwałość i globalny zasięg pobożności wydają się niezwykłe, jeśli był on tylko wiejskim mędrcem.

Punktem nacisku jest kwestia ciągłości.
Jeśli istota o nieludzkiej sprawczości wkroczyła w historię, należałoby oczekiwać dalszej, publicznej interakcji ze światem.
Powinny pojawiać się znaki wyraźne, powtarzalne i dostępne dla postronnych obserwatorów, a nie tylko dla wewnętrznych.
Wcielony Bóg, niebiański wysłannik czy ponadludzki agent nie powinni znikać po krótkim galilejskim epizodzie, by potem pogrążyć się w milczeniu.
Można by oczekiwać trwałych śladów we wspólnotowym zapisie natury i życia społecznego.
Samo to oczekiwanie odzwierciedla oczywiście współczesne standardy naukowe, ale dobrze pokazuje lukę między starożytnym świadectwem a rodzajem ciągłych danych, jakich dziś się domagamy.

Nauka tylko zaostrza to wymaganie.
Współczesne badania szukają efektów, które da się z góry określić, zaobserwować w kontrolowanych warunkach i powtórzyć.
Szpitale, dane demograficzne i duże grupy pacjentów zapewniają ciągły pomiar wyzdrowień, śmiertelności i zjawisk nietypowych.
Astronomia, sejsmologia i badania klimatu śledzą niebo i ziemię z wysoką precyzją.
Na tym tle nie widać stabilnego, publicznego wzorca zjawisk, który wymagałby stałego odwołania do czynnika nadprzyrodzonego.
Relacje o cudownych wydarzeniach są sporadyczne, lokalne i niepowtarzalne.
Docierają do nas przez świadectwa, a nie przez bezpośredni pomiar przyrządami.
Nie składają się na sygnał, który wymusiłby korektę fizyki, biologii czy struktury przyczynowej, jaką dziś stosujemy.

Starożytne zapisy wykazują podobną selektywność.
Choć doniesienia o cudach i widzeniach są częste, cesarskie roczniki, archiwa świątynne i inskrypcje miejskie rzadko potwierdzają spektakularne znaki, jakich można by oczekiwać, gdyby interwencje były masowe.
Tam, gdzie pojawiają się cuda, idą raczej śladem pamięci wspólnotowej i polemiki niż długich szeregów danych typowych dla trwałych przyczyn.

Nie oznacza to, że wszystkie twierdzenia są fałszywe.
Brak powtarzalnego sygnału nie jest dowodem niemożliwości.
Świadectwa uzdrowień, uwolnień, nagłych przemian i wizji występują w wielu kulturach i epokach.
Niektóre wyzdrowienia pozostają medycznie niewyjaśnione.
Niektóre doświadczenia zmieniają życie i wspólnoty w sposób, który trudno opisać zwykłym ciągiem przyczyn.
Takie dane istnieją, ale nie tworzą jeszcze ciągłego śladu empirycznego spełniającego naukowe kryteria publicznej weryfikacji.

Wynik to wyważony osąd.
Portret nadprzyrodzony ma realny ciężar historyczny, ponieważ kult Jezusa jako kogoś więcej niż człowieka jest wczesny, intensywny i trwały.
Jego kluczową słabością pozostaje problem ciągłości: świat, który rzekomo pozostaje pod stałą nadprzyrodzoną ingerencją, nie wykazuje współmiernego, publicznego i powtarzalnego wzorca skutków.
W tej książce uznajemy świadectwa cudów i nie próbujemy ich falsyfikować.
Można natomiast stwierdzić, że nasza rekonstrukcja nie stoi w sprzeczności z portretem nadprzyrodzonym, a ocenę, czy przedstawione dowody lepiej układają się z tym obrazem czy bez niego, pozostawiamy czytelnikowi.
Zaznaczamy przy tym, że dostrzegamy wiele przesłanek wskazujących, iż liczne tradycje chrześcijańskie są bardziej wiarygodne, niż się zazwyczaj zakłada, i bardzo niewiele danych sugerujących, że powszechnie przyjmowane tradycje miałyby być fałszywe.

\medskip

Każdy z portretów Jezusa omówionych powyżej wspierają poważne argumenty.
Nie utrzymałyby się tak długo w dyskusji naukowej, gdyby nie odsłaniały realnych rysów w materiale źródłowym.

Historyczny osąd nie może się jednak opierać na samej liczbie argumentów.
Prawdopodobieństwo nie jest kwestią zliczania mocnych i słabych stron jak głosów.
Jedna wyraźna sprzeczność z dobrze ustalonymi danymi — chronologią, geografią, językiem, kontekstem politycznym — wystarczy, by teorię obalić.
Z kolei grupa drobnych napięć może osłabiać dane ujęcie, ale nie musi go niszczyć.
Pytanie nie brzmi, ile problemów ma dany model, lecz czy któryś z nich jest śmiertelny.

Na takim surowym gruncie każdy z dotychczasowych modeli okazuje się cząstkowy.
Wyjaśnia naglący ton, albo uzdrowienia, albo nauczanie mądrościowe, albo niebezpieczeństwo polityczne, ale zawsze kosztem pozostawienia jakiegoś stałego faktu w konflikcie.
W rezultacie żaden z omówionych dotąd modeli nie może być w pełni poprawny.
Droga naprzód nie polega na wyborze któregoś z nich, lecz na odzyskaniu tego, co pozostaje spójne, gdy usunie się sprzeczności, a elementy ocalałe połączy w jeden konsekwentny obraz.
To właśnie do tego tła i do takiej rekonstrukcji teraz się zwrócimy.
Tłem, które przywraca spójność, jest grecki świat instytucjonalny, kształtujący całe słownictwo polityczne, wszystkie struktury obywatelskie i wszelkie formy tekstowe, w jakich wczesny ruch chrześcijański wyrażał siebie.
Następny rozdział prowadzi do tego świata --- nie jako jednej z możliwych inspiracji, lecz jako ramy nadrzędnej, bez której nie da się czytać zachowanych źródeł w ich własnym kontekście historycznym.