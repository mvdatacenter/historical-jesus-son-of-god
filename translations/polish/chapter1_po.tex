„Poszukiwania historycznego Jezusa” to współczesny projekt badawczy, którego celem jest odtworzenie postaci Jezusa w ramach historii, a nie teologii.
Zainicjowano go w epoce Oświecenia, rozwinął się w XIX wieku wraz z wieloma „Żywotami Jezusa”, został ograniczony przez sceptycyzm pierwszej połowy XX wieku, a następnie odrodził się po połowie stulecia w kolejnych falach.
Od tego czasu niezliczone prace proponowały konkurencyjne hipotezy dotyczące tego, kim był Jezus i jakie znaczenie miał jego przekaz.
Zanim przejdziemy do naszej własnej analizy, warto nakreślić najbardziej wpływowe z tych ujęć, z których każde przyciągnęło poważną uwagę współczesnych badaczy.

\section{Prorok apokaliptyczny.}\label{sec:apocalyptic}

W tym ujęciu Jezus stoi w nurcie apokaliptycznym późnego judaizmu okresu Drugiej Świątyni.
Jego głoszenie „królestwa Bożego” (βασιλεία τοῦ θεοῦ) rozumie się nie jako ponadczasową etykę, lecz jako ogłoszenie wydarzenia bliskiego w czasie: „królestwo przybliżyło się” (ἤγγικεν; Mk 1,15).
Retoryka jest nagląca — czuwanie, rozłam, żniwo, rozliczenie — a rama interpretacyjna jest ta sama, co w Księdze Daniela, 1 Henocha, 4 Ezdrasza i Zwojach znad Morza Martwego: Bóg osądza niegodziwych, usprawiedliwia sprawiedliwych, odnawia Izrael i porządkuje świat na nowo.

Zwolennicy wskazują na grupę wypowiedzi, które naturalnie wpisują się w ten horyzont.
Jezus mówi o „Synu Człowieczym” (ὁ υἱὸς τοῦ ἀνθρώπου), który zostanie objawiony, o znakach kosmicznych oraz o bliskości wydarzeń wyznaczonej przez „to pokolenie” (ἡ γενεὰ αὕτη).
„Mała Apokalipsa” (Mk 13 i paralele synoptyczne) łączy w jednym ciągu ucisk, zbezczeszczenie oraz przyjście Syna Człowieczego „na obłokach” (por. Dn 7,13).
Nawet odpowiedź udzielona Janowi Chrzcicielowi — „niewidomi widzą, chromi chodzą, ubogim głosi się dobrą nowinę” — powtarza eschatologiczną listę znaków (por. Iz 35; 61; 4Q521), wiążąc uzdrowienia z odnową czasów ostatecznych, a nie jedynie ze współczuciem.
W tej interpretacji prorockie gesty-symboliczne — działanie w Świątyni, wybór Dwunastu, otwarte uczty — funkcjonują jako żywe przypowieści o nadchodzącej odnowie Izraela.

Niektórzy uczeni ujmują ten sam materiał mniej w kategoriach kalendarza, bardziej w kategoriach doświadczenia wizjonerskiego.
Tutaj autorytet Jezusa opiera się nie tylko na ogłaszaniu końca, lecz na bliskości z Bogiem w modlitwie, mowie objawieniowej i wizjach, takich jak chrzest i przemienienie.
Ten akcent „mistyczny” nie wypiera apokaliptyki, ale podkreśla jej doświadczalne jądro: Jezus mówił pilnie o panowaniu Boga, ponieważ wierzył, że już je zobaczył i zakosztował.

Siłą tego modelu jest dopasowanie historyczne.
Wyjaśnia on naglący charakter wezwania Jezusa, to, dlaczego jego ruch mógł być odczytywany jako destabilizujący zarówno przez elity kapłańskie, jak i przez Rzym, dlaczego język królewski brzmiał politycznie oraz dlaczego ukrzyżowanie — rzymska kara za pretensje do władzy — stało się zakończeniem jego działalności publicznej.
Wyjaśnia także, dlaczego najwcześniejsze głoszenie zachowuje presję eschatologiczną: listy Pawła, na przykład, rejestrują zarówno bliskość końca, jak i potrzebę porządkowania wspólnot w świetle tego oczekiwania.

Napięcia są jednak realne.
Oś czasu jest trudna: wypowiedzi o „niektórych stojących tutaj”, którzy zobaczą królestwo, albo o wydarzeniach mających nadejść przed upływem jednego pokolenia, nie pasują łatwo do opóźnienia.
Redakcja stanowi problem: Mk 13 wygląda na ukształtowany przez traumę po roku 70, co rodzi pytania, ile w nim Jezusa, a ile interpretacji kryzysowej.
Nie wszystkie zapamiętane wypowiedzi są napędzane kryzysem; przypowieści o miłosierdziu, nakazy przebaczenia i rady przeciw zamartwianiu się mogą brzmieć jak stabilna mądrość, a nie retoryka odliczania.
A jednak dla wielu historyków te trudności wzmacniają, a nie osłabiają model.
Niespełnione proroctwa to właśnie to, czego oczekuje się w realnych ruchach apokaliptycznych: ogłasza się śmiałe harmonogramy, spełnienie się opóźnia, a uczniowie dokonują korekt.
Nowy Testament nosi ten ślad wszędzie — wspólnoty Pawła zmagają się z opóźnieniem, Ewangelie różnie ujmują „to pokolenie” — co wskazuje nie na brak oczekiwania apokaliptycznego, lecz na żywy proces jego przesuwania w czasie.

Pomimo siły tego ujęcia portret proroka apokaliptycznego pozostawia decydujące paradoksy.
Jeśli Jezus był tylko jednym z wielu żydowskich kaznodziejów apokaliptycznych, jego przesłanie okazało się uderzająco nieskuteczne w jego własnej ojczyźnie: w ciągu jednego pokolenia większość Żydów je odrzuciła, a ruch zdobył jedynie marginalną pozycję w Judei.
Tymczasem to samo przesłanie okazało się zadziwiająco przekonujące dla Greków, rozprzestrzeniając się po miastach i kulturach w tempie, jakiego nie osiągnęła żadna inna sekta apokaliptyczna.
Równie uderzające jest to, że choć pamięć o Jezusie zachowana przez jego uczniów rzeczywiście zawiera ton apokaliptyczny, nigdy nie pojawia się on jako rdzeń jego przesłania.
Ewangelie i wczesne listy znacznie silniej podkreślają miłość, przebaczenie i życie wspólnotowe niż eschatologiczne harmonogramy.
Tekst często uznawany za najstarszy w Nowym Testamencie, List do Galatów, koncentruje się na usprawiedliwieniu przez wiarę, życiu w Duchu, jedności i miłości, a oczekiwanie apokaliptyczne jest jedynie tłem.
Podobnie najbardziej powszechnie potwierdzone nauki historycznego Jezusa — Modlitwa Pańska, Kazanie na Górze, przypowieści o miłosierdziu — również traktują czasy ostateczne jako tło, a nie główną treść.
I co najważniejsze, ukrzyżowanie nigdy nie było losem apokaliptycznego proroka za samo przepowiadanie.
Była to kara Rzymu za przestępstwa przeciw państwu — bunt, insurekcję lub roszczenia królewskie.
Gdyby Jezus jedynie ogłaszał koniec świata, nie zostałby stracony jako „Król Żydów”.
Model apokaliptyczny wyjaśnia pilność, konflikt i oczekiwanie, ale nie potrafi wyjaśnić ani porażki wśród Żydów, ani trwałej atrakcyjności wśród Greków, ani tego, dlaczego Rzym potraktował go jako zagrożenie polityczne, a nie nieszkodliwego wizjonera.
Ta nierozwiązana sprzeczność stanowi jego największe ograniczenie.
Apokaliptyczny Jezus może nie jest całą opowieścią, lecz pozostaje nieuniknionym horyzontem, na tle którego należy oceniać wszystkie inne ujęcia.

\section{Rewolucjonista lub zelota.}\label{sec:revolutionary}

Inny trwały portret przedstawia Jezusa jako buntownika politycznego przeciw Rzymowi.
Traktuje poważnie napis na krzyżu — „Król Żydów” — który najnaturalniej czyta się politycznie, i wskazuje, że ukrzyżowanie było rzymską karą za bunt.
W tym ujęciu Jezus nie był przede wszystkim mędrcem ani mistykiem, lecz rewolucyjnym pretendentem, którego ruch zagrażał stabilności Judei.

Najpełniejszą prezentację tej tezy przedstawił S. G. F. Brandon w *Jesus and the Zealots* (1967) oraz *The Fall of Jerusalem and the Christian Church* (1951).
Brandon twierdził, że Jezus stał w ciągłości z narodową tradycją zelocką, że jego działanie w Świątyni było aktem rewolucyjnym, a jego uczniowie byli nie tylko słuchaczami, lecz towarzyszami w oporze.
W przeciwieństwie do późniejszych propozycji utożsamiających Jezusa z konkretnymi przywódcami rebelii, wersja Brandona nie zależała od zrównania go z Judą lub Egipcjaninem, lecz widziała w nim osobną postać ukształtowaną przez te same prądy rewolucyjne.
To ujęcie jest przekonujące, ponieważ tłumaczy polityczne zabarwienie języka królewskiego, symboliczne wyzwanie rzucone Świątyni oraz decyzję Rzymu o jego ukrzyżowaniu.

Warianty tezy rewolucyjnej identyfikują Jezusa bezpośrednio z znanymi przywódcami powstań.
Jedna propozycja utożsamia go z „Egipcjaninem” wspomnianym przez Józefa Flawiusza, prorokiem, który w latach 50. I wieku poprowadził tysiące ludzi na Górę Oliwną i obiecywał obalić mury Jerozolimy.
Analogią kusi, ponieważ wyjaśnia, dlaczego Rzym zareagował tak surowo i dlaczego późniejsi chrześcijanie mogli pamiętać go jako tego, który zapowiadał upadek Świątyni.
Trudnością jest tu jednak chronologia, ponieważ Józef umieszcza Egipcjanina o dwie dekady za późno.

Inny wariant łączy Jezusa z Judą Galilejczykiem, który wszczął bunt przeciw Rzymowi podczas spisu Kwiryniusza w 6 roku i jest pamiętany jako założyciel ruchu zelotów.
Tutaj siłą argumentu jest geografia — Juda, podobnie jak Jezus, pochodził z Galilei — oraz tematyka, ponieważ obaj byli ogłaszani przywódcami kwestionującymi władzę Rzymu.
Słabością jest ponownie chronologia: bunt Judy poprzedza działalność Jezusa, a nie ma jasnych dowodów na to, że byli tą samą osobą.

Nawet przy tych problemach model rewolucyjny pozostaje przekonujący, ponieważ stawia na pierwszym planie polityczne zagrożenie, jakie Jezus stanowił.
Wyjaśnia, dlaczego Rzym go ukrzyżował, dlaczego napis brzmiał „Król Żydów” i dlaczego późniejsza pamięć nie mogła oddzielić go od oczekiwań narodowych i królewskich.
A jednak wszystkie warianty tej teorii cierpią na trudności chronologiczne.
Przesunięcie daty narodzin, działalności lub ukrzyżowania Jezusa o rok czy dwa jest możliwe.
Ale przesunięcie ich o dekady, jak to konieczne, aby utożsamić go z Egipcjaninem lub Judą Galilejczykiem, wykracza daleko poza historyczną wiarygodność.
I choć portret rewolucjonisty jest przekonujący, jasne jest również, że — podobnie jak model apokaliptyczny — nie może stanowić całej opowieści.

\section{Postać mityczna.}

\label{sec:mythical}

Stanowisko mitologiczne obejmuje spektrum od całkowitego zaprzeczenia istnieniu Jezusa po twierdzenie, że nawet jeśli żył, nic z jego postaci nie da się już wydobyć spod warstw mitu.
Argument opiera się na dwóch filarach: późnym powstaniu i anonimowości źródeł oraz gęstości mitologicznych i literackich paralel w świecie starożytnym.

Sam Jezus niczego nie napisał.
Ewangelie znane nam dzisiaj pojawiają się po raz pierwszy tylko pośrednio, gdy Ireneusz, piszący pod koniec II wieku, opisuje działalność Marcjona około 140 roku.
Choć często datuje się je na późny I wiek, istnieją mocne argumenty również za datowaniem na połowę II wieku.
Pisma z I wieku i wczesnego II wieku nie zawierają wyraźnych odniesień do znanych nam dziś ewangelii czy listów, a wprost sformułowana wzmianka byłaby spodziewana, gdyby już krążyły.
Nawet pod koniec I wieku większość uczniów Jezusa była zapewne dawno martwa albo przynajmniej zbyt stara, by tworzyć dzieła literackie.
Teksty są anonimowe, a ich przypisanie Mateuszowi, Markowi, Łukaszowi i Janowi nastąpiło dopiero dużo później za sprawą ojców Kościoła, często na podstawie domysłów.
Ireneusz, który w innych miejscach formułuje śmiałe i stanowcze twierdzenia, tutaj jest zauważalnie bardziej powściągliwy przy przypisywaniu autorstwa ewangelii.
Wyjaśnia swoje rozumowanie, sprawiając wrażenie wykształconej rekonstrukcji, a nie dobrze zachowanej tradycji przekazanej od apostołów.

Ewangelie nie przedstawiają niezależnych głosów naocznych świadków, lecz warstwową tradycję literacką: najpierw powstaje Marek, następnie Mateusz i Łukasz go przerabiają, a Jan kształtuje wszystko na nowo teologicznie.
Większość badaczy zgadza się, że mamy do czynienia z drugą albo trzecią edycją ewangelii, a nasze teksty są kopiami kopii kopii, z których wiele treści pierwotnych zapewne zaginęło.
Narracje są nasycone zapożyczeniami z Pisma.
Opis męki czyta się jak celową kompilację Psalmu 22, Iz 53 i Dn 7.
Mateusz jawnie nawiązuje do Księgi Wyjścia, Ozeasza i Micheasza i buduje całą opowieść tak, by ukazać Jezusa jako współczesnego Mojżesza.

Paralele wykraczają jednak daleko poza samo Pismo.
Uzdrowienia, egzorcyzmy i życiorys Jezusa mają setki uderzających podobieństw do życia Apoloniusza z Tyany.
Jego cud w Kanie, Eucharystia i zmartwychwstanie przywodzą na myśl kult Dionizosa.
Język „nowych narodzin” i „nowego stworzenia” odpowiada rytuałom Mitry, Izydy i Eleusis.
Jego męka i śmierć przywodzą na myśl motywy bogów umierających i zmartwychwstających, takich jak Ozyrys, Attis czy Dionizos.
Opowieści o opętanym z Gerazy i Legionie wykazują uderzające podobieństwo do historii Odyseusza i Polifema.
Modlitwa Pańska wykazuje uderzające podobieństwa do solarnych hymnów do boga Atona.
Gołębica zstępująca przy chrzcie jest dokładnie takim sposobem ukazywania się Zeusa, gdy schodzi na ziemię.

Nawet proces i śmierć wpisują się w znane schematy.
Szlachetnie cierpiący nauczyciel postawiony przed obliczem państwa przywodzi na myśl proces Sokratesa.
Filozofowie skazywani za deprawowanie młodzieży lub kwestionowanie bogów tworzą wyraźne tło kulturowe.
W tej interpretacji ukrzyżowanie staje się śródziemnomorską odmianą motywu sprawiedliwego mędrca niesłusznie skazanego przez władzę ludzką, lecz ostatecznie usprawiedliwionego przez boską.

Listy Pawła, czytane w tym kluczu, nie opisują galilejskiego rabina.
Głoszą niebiańskiego Chrystusa objawionego przez wizje i Pismo.
Paweł nigdy nie cytuje przypowieści, nie wspomina Nazaretu, nie opowiada o cudach.
Jego Jezus jest kosmiczny, a nie biograficzny.

Najwcześniejsi świadkowie zewnętrzni są późni i pośredni.
Józef Flawiusz bywa uznawany za tekst zniekształcony przez interpolacje.
Tacyt jest czytany jako ten, który jedynie relacjonuje wiarę chrześcijan, a nie niezależną wiedzę archiwalną.
„Chrestus” Swetoniusza bywa traktowany jako postać niezwiązana z Jezusem.

W takim ujęciu chrześcijaństwo nie zaczyna się od pamięci o człowieku, lecz od mitu, alegorii i rytuału.
Paralele do misteriów, wiary dionizyjskiej, Apoloniusza i procesów filozofów są tak gęste, że opowieść o Jezusie czyta się mniej jak wyjątkową biografię, a bardziej jak kolaż znanych motywów kulturowych.
Niezależnie od tego, czy u podstaw istniał kiedyś człowiek, Jezus tekstów jawi się jako mit od fundamentu, konstrukt teologiczny, a nie nauczyciel historyczny.

Choć wielu krytyków twierdzi, że te paralele są wymuszone i nieprawdziwe, sam fakt, czy są one realne, ma niewielkie znaczenie dla oceny historycznej wiarygodności tego modelu.
Tak po prostu opowiadano historie w kontekście Jezusa.
Nie znaczy to, że rdzeń opowieści jest fałszywy.
Inni krytycy zwracają uwagę na ekstremalną precyzję geograficznych i niezwiązanych ze sobą szczegółów historycznych w tekstach.
Nawet jeśli przyznamy słuszność wszystkim tym obserwacjom, one także mają znikomy wpływ na podstawowe pytanie, czy Jezus był postacią mityczną, czy nie.
Stan Lee pisał o Nowym Jorku i wydarzeniach tam się dziejących z drobiazgową dokładnością, a jednak nie czyni to w żadnym stopniu historii o Spider-Manie bardziej historycznie wiarygodnymi.
Słabości tego modelu leżą gdzie indziej.

Najpoważniejszy problem tej teorii wynika z drobnych szczegółów, istotnych dla opowieści, a jednocześnie wyraźnie niezrozumianych przez autorów tekstów.
W odróżnieniu od realizmu tła, który można dowolnie wymyślać, te fragmenty wyglądają jak autentyczne resztki wspomnień skopiowane bez pełnego zrozumienia.
Na przykład Ewangelia Jana podaje, że Józef z Arymatei i Nikodem przynoszą mieszaninę mirry i aloesu, by namaścić ciało Jezusa.
Późny, całkowicie mityczny Jan nie wprowadziłby środków medycznych w sytuacji pospiesznego pochówku Jezusa.
Albo Mateusz mówiący o Ojcu, który odpłaci ci w ukryciu.
Późny, całkowicie mityczny Mateusz nie cytowałby podsłuchanej teologii egipskiej w żydowskim, prorockim kontekście.
Albo późniejsze teksty nazywające magów Kasparem, Melchiorem i Baltazarem.
Późna, całkowicie mityczna opowieść o narodzeniu nie znałaby tych dworskich tytułów, a jednocześnie nazywała ich magami.
Tło tych i wielu innych przykładów wyjaśnimy w kolejnych rozdziałach.

Dodatkowo bardzo trudno wyjaśnić samą liczbę autorów, którzy wydają się wybiórczo znać geografię i język wydarzeń, w których rzekomo biorą udział.
Znajdujemy zbyt wiele wzajemnych potwierdzeń w źródłach, które jednocześnie wykazują ostre rozbieżności w kwestiach, które prawdopodobnie nigdy nie zaszły.
Każda wersja historii przedstawia inne cuda i inne słowa Jezusa na krzyżu, ale opis procesu jest uderzająco spójny.
Jesteśmy niemal pewni, że już w kilka lat po śmierci Jezusa istniały poważne spory co do jego tożsamości i przesłania.
Wczesnym autorom nie brakowało odwagi, by proponować odmienne interpretacje, ale wykazywali dużą powściągliwość wobec wymyślania samego szkieletu wydarzeń.
Trudno sobie wyobrazić mit, w którym niektóre elementy centralne byłyby tak głęboko rozbieżne, podczas gdy wiele pozornie drugorzędnych szczegółów byłoby tak szeroko zgodnych.
Taki wzór zgodności i rozbieżności bardzo trudno pogodzić z czystym wymysłem.

To, co się wyłania, nie jest fabrykacją, lecz historia załamana przez pryzmat języka mitycznego — relacja zakorzeniona w przeżytym doświadczeniu, lecz wyrażona idiomem epoki.
I znów, choć mit zapewne odpowiada za wiele elementów opowieści, mało prawdopodobne, by stanowił bardzo dużą część całej historii.

\section{Uzdrowiciel, egzorcysta i prorok społeczny.}\label{sec:healer}

Jednym z najbardziej trwałych portretów Jezusa jest obraz uzdrowiciela i egzorcysty, którego akty mocy przyciągały tłumy i nadawały jego misji natychmiastowy autorytet.
Ewangelie przedstawiają ten wymiar nie jako ozdobnik, lecz jako centralny: egzorcyzmy demonów, uzdrowienia niewidomych i chromych, oczyszczenia trędowatych i działania przywracające do wspólnoty rytualnie nieczystych.
Takie tradycje są wielokrotne, wczesne i szeroko rozpowszechnione, a ponadto współbrzmią z żydowskimi wspomnieniami świętych mężów, jak Choni Wyznaczający Kręgi czy Chanina ben Dosa, zapamiętanych z powodu modlitwy, sprowadzania deszczu czy uzdrowień.
Wpisują się także w śródziemnomorską tradycję cudotwórców i boskich mężów, których autorytet opierał się na widocznej mocy.
Nic dziwnego, że powstała ogromna liczba książek rozwijających ten obraz na różne sposoby: czasem akcentuje się żydowskiego świętego męża, czasem skupia na przywracaniu do wspólnoty, czasem na ekonomii politycznej, czasem na doświadczeniu religijnym, czasem na szerszych paralelach śródziemnomorskich.
Choć podejścia te różnią się akcentami, razem tworzą portret zbiorczy: czyny mocy Jezusa były integralne dla jego reputacji i przesłania.

W tej ramie uzdrowienia można czytać jako żywe przypowieści o panowaniu Boga.
Egzorcyzm sygnalizuje pokonanie sił wrogich.
Uzdrowienie urzeczywistnia odnowę Izraela.
Wspólny posiłek znosi granice czystości i statusu.
Lektury społeczne podkreślają, że choroba oznaczała wykluczenie, a uzdrowienie — ponowne włączenie do rodziny, wsi i życia przymierza.
Lektury polityczne wskazują, że przepisy czystości i systemy zadłużenia służyły jako narzędzia kontroli elit, tak że uzdrowienia i uczty Jezusa realizowały alternatywny porządek społeczny.
Lektury egzystencjalne akcentują jego poczucie synostwa i modlitwę jako źródło mocy.
Lektury porównawcze sytuują go wśród śródziemnomorskich uzdrowicieli i mędrców, ale zauważają, że jego językiem jest Pismo Izraela, a horyzontem — odnowa przymierza.
Lektury liturgiczne pokazują, jak opowieści o uzdrowieniach stawały się rytuałami pamięci, kształtując sposób nauczania, głoszenia i modlitwy wspólnot.

Siłą tego portretu zbiorczego jest jego rozległość i moc wyjaśniająca.
Tłumaczy, dlaczego gromadziły się tłumy, dlaczego powstawało uczniostwo i dlaczego przeciwnicy reagowali gwałtownie.
Pokazuje, jak mowa o „królestwie” przybierała ciało w widocznych działaniach, a nie tylko w słowach.
Wyjaśnia, dlaczego pamięć o Jezusie trwała: ludzie pamiętali otrzymaną pomoc.

Nie mniej ważne są jednak słabości.
Problemem nie jest to, że uzdrowienia są niewiarygodne, lecz to, że są niewystarczające jako pełne wyjaśnienie.
Wiele postaci w starożytności zapamiętano jako uzdrowicieli, a niezliczeni kapłani przez wieki odprawiali egzorcyzmy.
Ta kategoria sama w sobie nie może wytłumaczyć, dlaczego akurat ten Galilejczyk stał się centrum ruchu, który przekształcił dzieje.

Oczywiste są też ograniczenia metodologiczne: historycy nie mogą weryfikować cudów tak jak monet czy inskrypcji, a kształtowanie narracji zaciera granice między tym, co zaszło, a tym, jak opowiadano.
Tradycje te zazębiają się z niemal każdym innym portretem: prorok apokaliptyczny, nauczyciel mądrości, rewolucjonista czy prorok odnowy wszyscy włączają uzdrowienia jako dowód na rzecz własnych ram.
Wczesne pisma, takie jak listy Pawła — nasze pierwsze teksty chrześcijańskie — wspominają wizje i charyzmaty, ale nie opisują Jezusa szczegółowo jako uzdrowiciela, co sugeruje, że opowieści cudowne zyskały na znaczeniu po jego śmierci.
Same historie są zróżnicowane — jedne uzdrowienia są natychmiastowe, inne stopniowe; jedne zależą od wiary, inne nie — co wskazuje zarówno na kształtowanie przez wspólnotę, jak i na pamięć.
A co najważniejsze, samo uzdrawianie nie wyjaśnia ukrzyżowania.
Rzym nie skazywał na krzyż za leczenie chorych.
Zagrożenie musiało kryć się w jego głoszeniu panowania Boga, symbolicznym wyzwaniu rzuconym Świątyni, roszczeniu dynastycznym lub postrzeganym zagrożeniu politycznym.

W efekcie otrzymujemy portret, który zabezpiecza jeden nieodzowny nurt pamięci, lecz nie może stać samodzielnie.
Jezus najprawdopodobniej rzeczywiście dokonywał uzdrowień i egzorcyzmów, a pamiętano je jako znaki Bożej mocy i odnowy społecznej.
Trzeba je jednak osadzić w szerszym horyzoncie — apokaliptycznej pilności, roszczeń dynastycznych, wyzwania rzuconego władzy — jeśli mają wyjaśnić, dlaczego Jezusa ukrzyżowano jako „Króla Żydów” i dlaczego jego uczniowie wyznali go jako Pana.

\section{Mędrzec i nauczyciel.}\label{sec:teacher}

Ten portret zbiorczy zbiera wątki, które widzą w Jezusie przede wszystkim nauczyciela.
Mówi on w przypowieściach i aforyzmach, które odwracają zwyczajowe kody honoru i przerysowują horyzont moralny.
Debatuje nad halacha z wielką precyzją, a zarazem akcentuje miłosierdzie i „ważniejsze sprawy Prawa”.
Wykonuje gesty-symboliczne, które wskazują na odnowę przymierza, a nie na prywatną pobożność.
W tej ramie jest nauczycielem mądrości w tradycji sapiencjalnej Izraela, niemal faryzejskim dyskutantem prawa oraz — miejscami — wiejskim mędrcem, którego styl może przypominać cyników poprzez ostrą mowę, dobrowolne ubóstwo i otwarty stół.

Atuty tego ujęcia są znaczne.
Przypowieści stanowią najbardziej stabilny rdzeń zapamiętanej mowy i dobrze wpisują się w formy żydowskiej mądrości.
Krótkie aforyzmy łatwo krążą między wspólnotami i tłumaczą powstawanie zbiorów wypowiedzi.
Spory prawne o szabat, czystość, rozwód i dziesięciny sytuują Jezusa wewnątrz prawa żydowskiego, a nie na zewnątrz.
Gesty-symboliczne — zwłaszcza działanie w Świątyni i wybór Dwunastu — współbrzmią z nadziejami na odnowę Izraela.
Badania porównawcze pokazują, że jego styl życia — wędrowny, łamiący hierarchię przy posiłkach, ostro krytyczny — ma realne analogie śródziemnomorskie, nie zacierając przy tym jego zakorzenienia w Piśmie.

Ten portret wyjaśnia także, dlaczego Jezus przyciągał uczniów.
Nauczyciele gromadzą słuchaczy.
Zapamiętane słowa, wcielone w praktyki publiczne, rodzą wspólnoty, które je zachowują i powtarzają.
Wyjaśnia również trwałość: mądrość lepiej znosi opóźnienia i rozczarowania niż harmonogramy.

Jego słabości mają inny charakter.
„Nauczyciel” to pojęcie zbyt pojemne, by wyjaśnić finał historii.
Rzym nie krzyżował mędrców za przypowieści.
Aby zrozumieć zarzut „Król Żydów”, profil nauczyciela trzeba połączyć z roszczeniami, które brzmiały politycznie — królestwo, Świątynia, plemię, władza.
Analogią do cyników można łatwo nadużyć, jeśli pomniejsza się głębokie zaangażowanie Jezusa w Torę i historię Izraela.
Akcentowanie sporów o prawo może spłaszczać obraz, jeśli przemienia go w zwykłego faryzeusza o łagodniejszym tonie.
Motyw odnowy może zblednąć, jeśli nie doprecyzuje się, co miało się zmienić w Świątyni, na ziemi i w królowaniu.

Istnieje też problem selektywności.
Wypowiedzi można tak zestawiać, by uzyskać niemal każdy portret, jeśli odpowiednio okroi się kontekst.
Przypowieści o miłosierdziu stoją obok ostrzeżeń o sądzie i obrazów królewskich.
Mądrość i eschatologia są w źródłach splecione, a nie łatwo rozdzielne.

Najspójniejsze ujęcie traktuje nauczanie jako medium, a nie samą treść.
Przypowieści, aforyzmy i spory prawne są sposobem, w jaki Jezus prowadził większą sprawę o panowanie Boga i odnowę Izraela.
W tym ujęciu profil mędrca-nauczyciela jest nieodzowny, by zrozumieć, jak jego słowa zostały zapamiętane, przekazane i praktykowane.
Sam w sobie nie tłumaczy jednak, dlaczego został stracony przez Rzym ani dlaczego jego uczniowie wyznali go jako Pana.
Podobnie jak inne portrety, wskazuje na rzeczywisty środek ciężkości, który wciąż domaga się powiązania z presją apokaliptyczną, symbolicznym królowaniem i publicznym wyzwaniem rzuconym władzy.

\section{Król dynastyczny lub pretendent do tronu.}

\label{sec:dynastic}

Kolejny portret interpretuje Jezusa jako postać dynastyczną, pretendenta do tronu, którego dom zachował ciągłość sukcesji po jego śmierci.
Współczesne prace rozwijające ten wątek to między innymi James D.\ Tabor, \emph{The Jesus Dynasty}; Robert Eisenman, \emph{James the Brother of Jesus}; Hyam Maccoby, \emph{Revolution in Judea}; Barbara Thiering, \emph{Jesus the Man}; Simcha Jacobovici i Charles Pellegrino, \emph{The Jesus Family Tomb}; oraz Michael Baigent, Richard Leigh i Henry Lincoln, \emph{Holy Blood, Holy Grail}.

W tym nurcie pojawia się kilka linii argumentacji.
Jedna podkreśla wyrok na krzyżu „Król Żydów” i odczytuje go jako wprost sformułowane oskarżenie o królowanie.
W takim ujęciu wjazd do Jerozolimy jest procesją królewską, działanie w Świątyni gestem konstytucyjnej suwerenności, a wybór Dwunastu — redystrybucją administracji plemiennej pod odnowionym panowaniem.
Rodowody u Mateusza i Łukasza traktuje się jako rejestry dynastyczne, mające potwierdzić legalne pochodzenie i polityczne prawo do tronu.

Druga linia argumentu kładzie nacisk na sukcesję rodzinną jako mechanizm ciągłości.
Bracia i krewni Jezusa, zapamiętani jako \emph{desposyni}, jawią się jako przedłużenie jego linii.
Jakub Sprawiedliwy opisywany jest jako bezpośredni następca w Jerozolimie, przewodniczący decyzjom halachicznym i kierownictwu wspólnoty.
Dynastia rozciąga się dalej poprzez krewnych, takich jak Symeon, syn Kleofasa, tworząc zapamiętany ciąg pretendentów królewskich.

Inny nurt podkreśla podwójny wzór urzędu mesjańskiego.
W tym modelu Jan Chrzciciel pełni funkcję mesjasza kapłańskiego, a Jezus — mesjasza królewskiego.
Chrzest interpretowany jest jako inwestytura, która łączy kapłaństwo i królowanie oraz ustanawia wspólną ramę legitymizacji.

Jeszcze inna linia argumentacji łączy Jezusa bezpośrednio z dynastią hasmonejsko–herodiańską.
W tej rekonstrukcji pamięta się go nie tylko jako potomka Dawida, ale jako wnuka lub dalekiego potomka Heroda Wielkiego, poprzez Antypatra lub iną gałąź rodu.
Taka genealogia wyjaśniałaby konsekwentne wiązanie jego narodzin z królewskim zagrożeniem, próby dworu Heroda, by go usunąć, oraz polityczny charakter oskarżenia wypisanego na krzyżu.
Tłumaczyłaby także ewangeliczne rodowody jako propagandę dynastyczną, ułożoną tak, by ująć jego linię w idiomie Izraela, a zarazem ustawić go w pozycji rywala wobec innych spadkobierców tronu.
Wersja ta interpretuje jego ukrzyżowanie nie jako los marginalnego proroka, lecz jako unieszkodliwienie niebezpiecznego pretendenta do władzy wewnątrz samego domu herodiańskiego.

Argumenty archeologiczne dodają kolejny wymiar.
Grobowiec w Talpiot w Jerozolimie odczytuje się jako grobowiec rodzinny z ossuariami opatrzonymi inskrypcjami, których imiona odpowiadają postaciom z Ewangelii.
Statystyczna analiza tego skupiska imion służy tezie, że taka kombinacja jest mało prawdopodobna jako czysty przypadek.
W ramach portretu dynastycznego miejsce to funkcjonuje jako materialne potwierdzenie miejskiego domu o królewskich pretensjach.

Niektóre linie interpretacji rozciągają ten portret na późniejsze dzieje poprzez hipotezę o „linii krwi”.
W takim ujęciu „Święty Graal” rozumie się jako krew królewska (\emph{sang real}).
Sukcesję śledzi się przez wygnanie i pamięć, a przetrwanie linii opowiada się w tradycjach średniowiecznych i wczesnonowożytnych.

Przeciw portretowi dynastycznemu często podnosi się kilka słabości.
Rodowody Mateusza i Łukasza różnią się w szczegółach i wydają się sprzeczne z późniejszą tradycją o dziewiczym poczęciu.
Twierdzi się, że tytuły królewskie są rzadkie w zachowanych wzmian­kach, co skłania niektórych do wątpienia, czy Jezusa w ogóle pamiętano jako króla.
Grobowiec w Talpiot stał się przedmiotem sporu, ponieważ zawarte w nim imiona są zarazem pospolite i jednak zgrupowane w sposób, który przywodzi na myśl rodzinę ewangeliczną.
Te słabości zasługują na uważną analizę, ponieważ pogłębione omówienie może mieć znaczący wpływ na ogólną ocenę modeli dynastycznych.
Każdą z nich przeanalizujemy w kolejnych rozdziałach, sprawdzając, czy rzeczywiście są zabójcze dla tej teorii, czy też dają się pogodzić w spójnej rekonstrukcji dynastycznej.

Znacznie trudniejszy jest problem społeczny.
Jeśli roszczenie Jezusa było po prostu roszczeniem żydowskiej dynastii, trudno wyjaśnić, dlaczego po jego śmierci nie nastąpiła trwała wierność żydowska, lecz pojawił się ruch w przytłaczającej większości pogański.
Najwcześniejsze świadectwa pokazują synagogi rozdarte, ale w ciągu jednego pokolenia udział Żydów staje się marginalny, podczas gdy greckie miasta w całym imperium goszczą kwitnące zgromadzenia.
Jeśli napis na krzyżu „Król Żydów” oddawał istotę jego roszczenia, dlaczego sami Żydzi nie zachowali jego królowania?
Dlaczego zamiast tego to poganie, którzy nie mieli żadnego udziału w linii Dawida, stali się jego głównymi wyznawcami?
Ta dysproporcja między odrzuceniem przez Żydów a przyjęciem przez pogan pozostaje najtrudniejszym napięciem do pogodzenia w wąsko pojętym judejskim modelu dynastycznym.

\section{Postać nadprzyrodzona.}\label{sec:supernatural}

Większość współczesnych badań mieści się w ramach naturalistycznych, ale w tej pracy nie przyjmujemy z góry, że Boga, Allaha czy nawet inteligencje nieludzkie należy z góry wykluczyć jako niemożliwe wyjaśnienia.
Portret „postaci nadprzyrodzonej” traktuje poważnie twierdzenie, że Jezus był kimś więcej niż człowiekiem albo że rzeczywiście przemawiał z mandatu boskiej lub nieludzkiej istoty.
Obejmuje to zarówno ortodoksyjne wyznanie Boga wszechmogącego Stwórcy wcielonego, jak i dawne konkurencyjne wizje, w których zstępuje on jako niebiański objawiciel, wysłannik wyższego Boga, prorok Allaha czy istota spoza stworzonego porządku, a także nowoczesne przekształcenia widzące w nim byt „z innego świata” w kategoriach technologicznych.
We wszystkich tych wariantach teza rdzenna jest ta sama: Jezus działał z nieludzką sprawczością, a ludzie doświadczali go jako żywej mocy.

Pozytywny argument jest prosty.
Najwcześniejsza cześć traktuje go jako κύριος i Syna, a nie tylko nauczyciela.
Wspólnoty wzywają jego imienia w modlitwie, hymnie i posiłku, jakby działała tam obecność boska.
Tradycje cudów i egzorcyzmów są centralne, a nie dekoracyjne.
Ruchy rywalizujące, które zaprzeczały jego ciału albo oddzielały go od Boga Izraela, i tak zakładały nadprzyrodzonego Jezusa.
Trwałość i globalny zasięg jego kultu są niezwykłe, jeśli byłby tylko wiejskim mędrcem.

Decydującym punktem nacisku jest tu ciągłość.
Jeśli istota o nieludzkiej sprawczości weszła w historię, należałoby oczekiwać dalszej, publicznej interakcji ze światem.
Powinniśmy widzieć znaki wyraźne, powtarzalne i dostępne dla postronnych, nie tylko dla wtajemniczonych.
Wcielony Bóg, niebiański wysłannik czy ponadludzki agent nie zniknąłby w krótkim galilejskim epizodzie, by później zamilknąć.
Oczekiwalibyśmy trwałych śladów w wspólnym rejestrze natury i społeczeństwa.
Samo to oczekiwanie odzwierciedla współczesne kryteria naukowe, ale dobrze uwidacznia lukę między starożytnym świadectwem a typem nieprzerwanego materiału, jakiego dziś się żąda.

Nauka wyostrza to oczekiwanie.
Współczesne badania poszukują efektów dających się określić z góry, zaobserwować w kontrolowanych warunkach i powtórzyć.
Szpitale, dane demograficzne i wielkie kohorty dostarczają ciągłych pomiarów wyzdrowień, śmiertelności i wydarzeń anomalnych.
Astronomia, sejsmologia i szeregi klimatyczne śledzą niebo i ziemię z wysoką precyzją.
Na tym tle nie widać stabilnego, publicznego wzorca zdarzeń, który wymagałby stałego odwołania do czynnika nadprzyrodzonego.
Relacje o cudach są sporadyczne, lokalne i niepowtarzalne.
Przekazywane są przez świadectwo, a nie przez obserwację przyrządową.
Nie kumulują się w sygnał, który wymuszałby zmianę fizyki, biologii czy struktury przyczynowej, z jakiej korzystamy.

Źródła starożytne pokazują podobną wybiórczość.
Choć doniesienia o cudach i widzeniach są częste, cesarskie roczniki, archiwa świątynne i inskrypcje miejskie rzadko potwierdzają takie widowiskowe znaki, jakich można by oczekiwać przy szerokiej interwencji.
Tam, gdzie pojawiają się dziwy, podążają raczej śladami pamięci wspólnotowej i polemiki niż długich szeregów danych, które zwykle wskazują na przyczyny trwałe.

Nie obala to każdego roszczenia.
Brak powtarzalnego sygnału nie jest dowodem niemożliwości.
Świadectwa uzdrowień, uwolnień, nagłych przemian i wizji są rozpowszechnione w kulturach i epokach.
Niektóre wyzdrowienia pozostają medycznie niewyjaśnione.
Niektóre doświadczenia przeobrażają życie i wspólnoty w sposób, który zwykłej kauzalności trudno ująć.
Takie dane istnieją, lecz nie tworzą jak dotąd ciągłego śladu empirycznego, który spełniałby naukowe kryteria publicznej weryfikacji.

Wynik to osąd wyważony.
Portret nadprzyrodzony ma realny ciężar historyczny, ponieważ cześć dla Jezusa jako kogoś więcej niż człowieka jest wczesna, intensywna i trwała.
Jego zasadnicza słabość to problem ciągłości: świat rzekomo poddany stałej nadprzyrodzonej sprawczości nie wykazuje odpowiedniego, publicznego, powtarzalnego wzorca efektów.
W tej książce uznajemy świadectwa cudów i nie próbujemy ich falsyfikować.
Można natomiast powiedzieć tyle, że nasza rekonstrukcja nie popada w sprzeczność z portretem nadprzyrodzonym, a ocenę, czy przedstawione przez nas dane lepiej pasują do świata z interwencją nadprzyrodzoną, czy bez niej, pozostawiamy czytelnikowi.
Dodajmy natomiast, że znajdujemy wiele przesłanek sugerujących, iż liczne tradycje chrześcijańskie są bardziej wiarygodne, niż się zwykle przyjmuje, oraz bardzo niewiele przesłanek, że powszechnie przyjmowane tradycje są fałszywe.

\medskip

Każdy z dotychczas omówionych portretów Jezusa wspiera się na poważnych argumentach.
Nie utrzymałyby się w nauce, gdyby nie odsłaniały rzeczywistych cech materiału źródłowego.

Historyczny osąd nie może jednak opierać się wyłącznie na liczbie argumentów.
Prawdopodobieństwo nie polega na zliczaniu mocnych i słabych punktów jak głosów.
Jedna wyraźna sprzeczność z pewnymi danymi — chronologią, geografią, językiem, kontekstem politycznym — wystarcza, by złamać teorię.
Z kolei grupa drobniejszych napięć może osłabić tezę, ale jej nie obala.
Pytaniem nie jest to, ile problemów ma dany model, lecz czy któryś z nich jest śmiertelny.

Na tym surowym gruncie każdy ze standardowych modeli okazuje się częściowy.
Wyjaśnia pilność, albo uzdrowienia, albo nauczanie mądrościowe, albo zagrożenie polityczne, lecz zawsze za cenę pozostawienia innego, ustalonego faktu w konflikcie.
W efekcie żaden z dotąd omówionych modeli nie może być poprawny.
Droga naprzód nie polega na wyborze między nimi, lecz na odzyskaniu tego, co pozostaje spójne, gdy usunie się sprzeczności, a pozostałe elementy zintegruje w jedną, konsekwentną opowieść.
Wymaga to powrotu do tła, którym jest grecki świat instytucjonalny kształtujący całe słownictwo polityczne, każdą strukturę obywatelską i każdy typ tekstu, w których wczesny ruch chrześcijański wyraził siebie.
Następny rozdział zwróci się ku temu światu — nie jako opcjonalnemu wpływowi, lecz jako ramie nadrzędnej, bez której zachowane źródła nie mogą być odczytane w swoim własnym kontekście historycznym.