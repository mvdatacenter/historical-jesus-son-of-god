W poprzednich rozdziałach pokazaliśmy, że jeśli teksty o Jezusie są w większości poprawne, to Jezus musiał być postacią królewską o niezwykle głębokich związkach z greckimi i egipskimi tradycjami boskiego panowania.
W tym ujęciu Jezus jako prawowity Christos oraz jego towarzysze jawią się jako osoby o ogromnej władzy i wysokim wykształceniu.

Na tym etapie musimy jednak poważnie rozważyć możliwość, że Jezus — podobnie jak Amon, Ra, Zeus czy Dionizos — mógł być postacią mityczną.
Choć historycznie panował dość silny konsensus, że Jezus był realną postacią historyczną, coraz więcej badaczy zauważa, że argumenty za historycznością Jezusa, Pawła i apostołów są dużo słabsze przy wnikliwym zbadaniu.
Brak źródeł współczesnych, sprzeczności w źródłach oraz skrajne podobieństwo opowieści o Jezusie do wielu innych opowieści mitycznych epoki wskazują na możliwość, że Jezus był postacią mityczną.
Niewiarygodnie rozwinięte struktury kościelne obecne już w najwcześniejszych listach Pawła stawiają pod znakiem zapytania jego istnienie.

Można wysunąć bardzo mocny argument, że listy Pawła są w całości pismami pseudonimowymi, a ewangelie mogły powstać nawet tak późno jak około 150 roku.
I faktycznie, w ramach obecnych paradygmatów badań biblijnych, przy założeniu powstania chrześcijaństwa od zera jako żydowskiej sekty w roku 30, najbardziej prawdopodobne datowanie wypada dopiero na połowę II wieku.
Gdyby tak było, Jezus najprawdopodobniej byłby postacią mityczną, a ewangelie zostałyby napisane jako element greckiej propagandy imperialnej.

Jednak w ramach koncepcji Jezusa jako postaci królewskiej, wchodzącej w istniejące greckie i egipskie struktury królewskie oraz kościelne, datowanie ewangelii można ponownie przeanalizować.
Jeśli wszystkie dane rozważyć dokładniej, okazuje się, że materiał literacki powstał najprawdopodobniej dużo wcześniej, niż dotąd uważano — może nawet za życia samego Jezusa.

Co ważne, choć ewangelie traktuje się jako najwcześniejsze i najważniejsze źródła o Jezusie, trzeba pamiętać, że z pewnością miały one wiele wydań.
Dość silny konsensus zakłada, że znana nam dziś Ewangelia Jana jest jej trzecim wydaniem, bez końcowego rozdziału stwierdzającego, że umiłowany uczeń się starzeje.
Łukasz mógł mieć wcześniejszą wersję, przed opisem narodzin, którą być może znamy jako ewangelię Marcjona.
O Marku wiemy, że istniała wersja bez opisu zmartwychwstania.

Wraz z ponowną analizą datowania ewangelii możemy również zrewidować datowanie licznych wczesnych tekstów apokryficznych.
Historycznie odrzucono je wszystkie jako późne wytwory, ale wrzucenie ich do jednego worka było oczywistym błędem dawnych uczonych, który stworzył ogromne uprzedzenie.
Znaczna większość apokryfów jest bardzo późna i wyraźnie napisana jako literatura wtórna, pozbawiona wartości historycznej.
Jednak wciąż istnieje wiele tekstów, które są niewątpliwie bardzo wczesne, a czasem nawet wcześniejsze niż ewangelie.
Ewangelia Tomasza najprawdopodobniej poprzedza cztery ewangelie kanoniczne.
Protoewangelium Jakuba oraz Ewangelia Filipa stanowią ważną część wiary i teologii Kościoła katolickiego i prawosławnego.
Choć ojcowie Kościoła wymieniali listę pism, które ostatecznie uformowały dzisiejszą Biblię, przez piętnaście stuleci nikt nie przywiązywał do niej większej wagi i Protoewangelium Jakuba kopiowano równie często jak ewangelie.
Tekst ten rzeczywiście wypadł z łask dopiero po Soborze Trydenckim w połowie XVI wieku, kiedy Kościół katolicki formalnie skanonizował Biblię w obecnej formie.

Kiedy uświadomimy sobie, że istniała ogromna liczba tekstów pisanych natychmiast po wydarzeniach związanych z Jezusem, pisanych przez różnych autorów, często przez świadków naocznych, a ich kolejne wersje odzwierciedlały ważne wydarzenia — takie jak ukrzyżowanie Piotra czy zburzenie świątyni — możliwość, że ta historia jest mitem, spada praktycznie do zera.

\section{Ewangelia Jana}\label{sec:john-gospel}

\subsection{Datowanie Ewangelii Jana}\label{subsec:dating-of-the-gospel-of-john}

Jak wiadomo, Ewangelia Jana ma najwcześniejsze i najliczniejsze świadectwa tekstowe, a mimo to datuje się ją najpóźniej — wyłącznie z powodu uprzedzeń wynikających z błędnych założeń dotyczących „chrystologii”.
To jedno z tych dziewiętnastowiecznych założeń wczesnych biblistów, które całkowicie rozpada się pod najmniejszą analizą, a mimo to wciąż pozostaje dogmatem współczesnej biblistyki.

Jeśli chodzi o świadectwa, posiadamy papirus P52 z Ewangelii Jana, datowany na około 125 rok, a być może nawet na 100 rok.
Najstarsze rzekomo świadectwa Marka, Mateusza i Łukasza pochodzą dopiero od Ireneusza z około 180 roku, razem z kilkoma innymi ewangeliami, które Ireneusz uznał za heretyckie, a najstarszy rękopis Marka pochodzi z końca III wieku.
Najstarsza wzmianka o Marcjonie pochodzi od Tertuliana, ale sam Tertulian potwierdza działalność Marcjona około 140 roku.
Tak więc przy obecnej chronologii rękopisów i wzmianek kolejność świadectw jest następująca: najpierw Jan, następnie Marcjon, potem Maria Magdalena, Tomasz, Judasz, a dopiero później Marek, Mateusz i Łukasz.

Można to wyjaśnić przypadkiem historii — większość wczesnych źródeł zaginęła, a najstarszym zachowanym okazał się Jan, czysto przypadkowo.
Jednak musimy zauważyć, że Ewangelia Jana nie została znaleziona samotnie, lecz jako część dużej biblioteki zawierającej wiele tekstów, w tym inne ewangelie, a także różnego rodzaju dokumenty, również datowane dokumenty podatkowe.
Papirus oksyrynchijski P52 został jednoznacznie napisany stylem pisma pasującym do rękopisów datowanych na około 125 rok, ale inne ewangelie z Oksyrynchos są z pewnością starsze o 25–50 lat.

Choć datowanie może być dyskutowane, na podstawie geograficznej lokalizacji w Oksyrynchos w Egipcie mamy ewangelie, które najprawdopodobniej zostały przepisane w tym samym miejscu przez tę samą grupę osób, więc względne datowanie ewangelii w momencie ich dotarcia do Oksyrynchos jest w zasadzie pewne.

Oczywiście może być tak, że ewangelie synoptyczne trafiły do Oksyrynchos dużo później, czystym przypadkiem, jednak ten sam wzorzec powtarza się w drugim najstarszym zbiorze rękopisów — bibliotece z Nag Hammadi — gdzie Ewangelia Jana również jest najstarszym i najliczniejszym tekstem, a ewangelie synoptyczne nie występują w ogóle.
Rękopisy z Nag Hammadi są nieco wcześniejsze, choć ich pierwotne wersje rzekomo pochodzą z połowy II wieku.

Trzecim źródłem dość pewnie datowanym na II wiek, prawdopodobnie na jego początek, jest papirus Bodmera P66 zawierający Ewangelię Jana.
Podobnie jak w pozostałych przypadkach, Ewangelia Jana poprzedza ewangelie synoptyczne w kolekcji Bodmera co najmniej o 25–50 lat.

I to właściwie wszystkie odniesienia do ewangelii wcześniejsze niż około 140 rok.
Jedynymi możliwymi wcześniejszymi świadectwami są Papiasz, Polikarp i Klemens, którzy prawdopodobnie pisali przed 140 rokiem.
W przypadku Papiasza odnosi się on do ewangelii niepisanych po grecku, więc niemal na pewno nie chodzi o żadne ewangelie kanoniczne ani o szeroko znane ewangelie apokryficzne.
Klemens i Polikarp cytują słowa Jezusa znane z ewangelii, ale niezależnie od tego, czy Jezus rzeczywiście te słowa wypowiedział, nie wiadomo, czy cytowali ewangelie, czy ewangelie cytowały ich, czy — co najbardziej prawdopodobne — wszyscy korzystali z tych samych tekstów lub tradycji ustnych.
Kolejnym świadkiem jest Justyn Męczennik, i on również najprawdopodobniej cytuje Ewangelię Jana, co znamienne — nie słowa Jezusa zanotowane w Ewangelii Jana, lecz znany fragment prologu.

Późniejsze świadectwa od Tertuliana, Ireneusza i Orygenesa zawierają już jednoznaczne odniesienia do wszystkich ewangelii kanonicznych jako tekstów ugruntowanych i szeroko rozpowszechnionych, ale jesteśmy wtedy już w roku 180 i później.

Aby było zupełnie jasno: nie oznacza to, że Jan został napisany w 125 roku, ani że Marcjon został napisany w 140 roku.
Jan mógł zostać spisany zaraz po wydarzeniach, które opisuje, albo mógł powstać około 100 roku.

To natomiast wskazuje, że na podstawie dat rękopisów i wzmianek to właśnie Jan ma najwyższe prawdopodobieństwo bycia najwcześniejszą ewangelią — a nie Marek, Mateusz czy Łukasz.

Autorem Ewangelii Jana jest również autor trzech listów Jana, co jasno wynika ze stylu i słownictwa.
Choć Euzebiusz przypisuje Apokalipsę Janowi, ta sama analiza pokazuje, że autor Ewangelii Jana z całą pewnością nie jest autorem Apokalipsy.
A zatem Jan z Patmos — jedyny jawnie nazwany autor księgi Nowego Testamentu — nie jest tą samą osobą co autor Ewangelii Jana.
Dlaczego ma to znaczenie?
Ponieważ autor listów janowych z naciskiem podkreśla, że jest świadkiem naocznym, który słyszał, widział i dotykał Jezusa.

Jednym z częstych argumentów przeciw wczesnemu datowaniu Ewangelii Jana jest twierdzenie, że mowy Jezusa są zbyt literackie i zbyt długie jak na tradycję ustną.
To z pewnością prawda, ale powszechny wniosek, że wszystkie te mowy musiały więc zostać wymyślone i spisane dopiero sto lat później, wcale nie jest logiczny.
Dużo bardziej sensowne jest założenie, że mowy te istniały w formie pisemnej już wtedy, gdy Jezus je wygłaszał.
Widoczne w Ewangelii Jana odniesienia do logiów znanych z Ewangelii Tomasza mogą wskazywać nie na to, że Jan i Tomasz spierali się o teologię, lecz że obaj korzystali z tego samego źródła co Jezus.
Jak już podkreślono, Jezus musiał być niezwykle dobrze oczytany, więc można oczekiwać, że przynajmniej część jego nauk była spisywana — przez niego samego lub przez jego uczniów, takich jak Tomasz.

Istnieje wiele argumentów za datowaniem Ewangelii Jana po roku 60, opartych na ostatnim rozdziale, Jana 21.
Rozdział ten wygląda na napisany przez innego autora i jest wyraźnym dodatkiem do tekstu pierwotnego, który kończy się w rozdziale 20 jasnym zakończeniem:
„Teraz Jezus uczynił wiele innych znaków wobec uczniów, które nie są zapisane w tej księdze;
lecz te zapisano, abyście wierzyli, że Jezus jest Chrystusem, Synem Boga, i abyście wierząc mieli życie w imię Jego.”
To sprawia, że datowanie rozdziałów 1–20 wypada dość pewnie na okres wyraźnie wcześniejszy niż 60 rok.

Gdy odrzucimy argumenty oparte na rozdziale 21 oraz argumenty o „wysokiej chrystologii”, jedyne pozostałe argumenty dotyczą rzekomego rozłamu między chrześcijanami a Żydami.
Jak pokazujemy w tej książce, rozłam ten jest wielkim nieporozumieniem i nigdy nie nastąpił.
Wobec tego nie ma żadnych wartościowych argumentów przemawiających za datowaniem Ewangelii Jana później niż 60 rok.

\section{Tożsamość umiłowanego ucznia i głos naocznego świadka}\label{sec:identity-beloved-disciple}

\subsection{Ewangelia Jana jako świadectwo naoczne: Maria Magdalena, nie Jan syn Zebedeusza}\label{subsec:the-gospel-of-john-is-widely-accepted-to-be-one-gospel-that-indicates-it-was-written-by-an-eyewitness.}

Ewangelia Jana wyróżnia się spośród synoptyków swoim nieustannym twierdzeniem, że opiera się na bezpośrednim świadectwie naocznym: „ucznia, którego Jezus kochał”.
Tradycja od dawna utożsamiała tę postać z Janem, synem Zebedeusza, lecz wewnętrzna analiza tekstu wskazuje na coś zupełnie innego.
W przeciwieństwie do Marka, Łukasza i Mateusza — żaden z nich nie twierdzi, że jest autorem naocznym — narracja Jana jest przesycona szczegółami, które mógł zapamiętać jedynie ktoś obecny, emocjonalnie zaangażowany:
prywatne dialogi, sceny domowe, wewnętrzne reakcje najbliższych Jezusa.
A jednak tekst nigdzie nie nazywa umiłowanego ucznia Janem ani nie umieszcza go jednoznacznie w gronie Dwunastu.
Zamiast tego wzór obecności pod krzyżem, przy grobie i w scenach zmartwychwstania, a także wyjątkowy dostęp do kobiet z otoczenia Jezusa, wskazują na Marię Magdalenę jako najbardziej prawdopodobną autorkę świadectwa.
Jej centralna rola w narracjach paschalnych, jej wyjątkowe posłannictwo nadane przez Jezusa oraz jej konsekwentna obecność tam, gdzie mężczyźni–uczniowie uciekają lub znikają, czynią z niej najlepszą kandydatkę na umiłowanego ucznia.
Przypisanie autorstwa „Janowi” należy rozumieć jako późniejszy zwyczaj, być może odzwierciedlający redaktora lub imię skryby, lecz istota świadectwa jest jednoznacznie magdalenowa — to jej głos i jej pamięć uformowały rdzeń Czwartej Ewangelii.
To tłumaczy zarówno żywość narracji, jak i jej odejście od synoptyków, sytuując Marię Magdalenę jako głównego świadka stojącego za tekstem.

Argument ten ma dwa poziomy.
Po pierwsze, wewnętrzne świadectwo Ewangelii Jana wskazuje na autorkę kobietę:
rejestr zmysłowy, domowa topografia scen, powtarzające się role kobiet w kluczowych momentach narracji oraz słownictwo emocjonalne — wszystko to odpowiada wzorcowi literackiemu dobrze potwierdzonemu w źródłach greckich i egipskich.
Ten wniosek istnieje niezależnie od identyfikacji konkretnej osoby.
Dopiero po ustaleniu, że autorka najprawdopodobniej była kobietą, możemy przejść do pytania drugiego: która z kobiet z kręgu Jezusa była zdolna napisać taki tekst.
Maria Magdalena jest najsilniejszą kandydatką, choć nie jedyną, a argumenty na jej korzyść mają charakter wnioskowy, a nie strukturalny, oparte raczej na wczesnych tekstach chrześcijańskich spoza późniejszego kanonu niż na samych cechach literackich.

Archeologia wielokrotnie potwierdziła precyzję topografii Jerozolimy przedstawionej w Janie, wzmacniając argument o świadectwie naocznym.
Sadźawka Betesda, z jej pięcioma portykami (J 5,2), została odkryta w XIX wieku dokładnie tam, gdzie umieszcza ją Jan, ze strukturą pięciu kolumnad odpowiadającą opisowi.
Sadźawka Siloam (J 9,7) została odkopana w 2004 roku i potwierdzona jako ogromny rytualny zbiornik z okresu Drugiej Świątyni.
Jan wymienia dolinę Cedronu, Gabbata (Kamienny Dziedziniec), studnię Jakuba, Betanię za Jordanem i Efraim blisko pustyni — wszystkie te miejsca odpowiadają topografii Jerozolimy sprzed 70 roku z precyzją, jakiej nie osiąga żadna inna ewangelia.
Równie trafne są szczegóły galilejskie Jana:
relacja Kany do Kafarnaum, uwaga o „schodzeniu w dół” do nadjeziornych miejscowości oraz odległości między nimi — wszystko odpowiada rzeczywistym różnicom wysokości i geografii.
Spośród wszystkich czterech ewangelii to Jan zachowuje najdokładniejszą mikrotopografię Jerozolimy i Judei, dokładnie taką, jakiej należałoby oczekiwać od świadka, który znał to miasto dobrze przed jego zburzeniem w 70 roku.

\subsection{ὁ μαθητὴς ὃν ἠγάπα ὁ Ἰησοῦς}

\label{subsec:ux1f41-ux3bcux3b1ux3b8ux3b7ux3c4ux1f74ux3c2-ux1f43ux3bd-ux1f20ux3b3ux3acux3c0ux3b1-ux1f41-ux1f30ux3b7ux3c3ux3bfux1fe6ux3c2}

Częste odwołania do ucznia, którego Jezus miłował, mogą również wskazywać, że nie był on jednym z Dwunastu, ponieważ Ewangelia Jana w ogóle nie wymienia po imieniu grona apostołów.

Co znamienne, Ewangelia Jana rzeczywiście wymienia Dwunastu przy wielu okazjach, ale uczeń, którego Jezus miłował, nie wydaje się należeć do tego grona, choć logiczne byłoby zaznaczyć to wprost, gdyby tak było.
Uczeń, którego Jezus miłował, wydaje się należeć do kręgu innego niż grupa Piotra.
Co najważniejsze, wszystkie ewangelie zdają się mocno sugerować, że po aresztowaniu Jezusa wszyscy męscy uczniowie uciekli, a pozostały jedynie kobiety.
„Były tam również kobiety, które z daleka się przypatrywały\ldots{} Maria Magdalena, Maria, matka Jakuba\ldots{} oraz wiele innych kobiet\ldots{}”
Stoimy więc wobec albo bardzo poważnej sprzeczności między Janem a wszystkimi trzema synoptykami — tam, gdzie nie było żadnego powodu, by ją wprowadzać — albo, co znacznie bardziej prawdopodobne, umiłowany uczeń nie należał do Dwunastu i był kobietą, zachowaną anonimowo, lecz rozpoznawaną we wspólnocie jako autorytatywna świadek.

\subsection{Γύναι, ἰδοὺ ὁ υἱός σου}\label{subsec:ux3b3ux3cdux3b9-ux1f30ux3b4ux3bfux1f7a-ux1f41-ux3c5ux1f31ux3ccux3c2-ux3c3ux3bfux3c5}

Słowa Jezusa z krzyża — „Niewiasto, oto syn Twój” (J 19,26) — od dawna interpretowano jako skierowane do umiłowanego ucznia, a jednak niejednoznaczność tej frazy podsyca debatę.
Choć proponowano Joannę, znaczącą kobietę z dworu Heroda i kluczową patronkę Jezusa (Łk 8,3), jako możliwą kandydatkę na autora-świadka stojącego za Ewangelią Jana, całokształt dowodów nieustannie prowadzi z powrotem do Marii Magdaleny.
Pozycja Joanny i jej dostęp do elit czynią z niej atrakcyjną pośredniczkę lub redaktorkę tradycji, a jej powiązań (być może nawet z Junia, „znakomitą między apostołami”) nie można lekceważyć.
Mimo to narracyjne ognisko Ewangelii Jana — zwłaszcza pod krzyżem i przy grobie — pozostaje mocno skupione na Marii Magdalenie.
Jej obecność w każdym kluczowym momencie, jej emocjonalna bliskość Jezusa oraz wyjątkowa rola w opisie zmartwychwstania czynią z niej najlepsze wyjaśnienie perspektywy świadka, z której napisana jest Ewangelia.
Joanna pozostaje wiarygodną kandydatką jako kanał przekazu bądź późniejsza depozytariuszka tradycji, jednak wewnętrzne i zewnętrzne dane w przytłaczającej większości wskazują na Marię Magdalenę jako fundamentalną świadek, której świadectwo podtrzymuje cały tekst.

\subsection{Piotr kontra Maria Magdalena: wczesne świadectwo i umiłowany uczeń}\label{subsec:peter-vs-mary-magdalene-early-testimony}

Gdy wczesne teksty chrześcijańskie spoza późniejszego kanonu, takie jak Ewangelia Marii, poruszają kwestię tożsamości umiłowanego ucznia, centralnym konfliktem nie jest rywalizacja Jana z Marią, lecz Piotra z Marią Magdaleną.
Teksty te konsekwentnie ukazują Marię Magdalenę jako odbiorczynię szczególnego objawienia i uczennicę najbliższą Jezusowi, często w bezpośredniej opozycji do autorytetu Piotra.
Brak jakiejkolwiek tradycji, w której Jan, syn Zebedeusza, byłby przeciwstawiony Marii w walce o ten status, jest znamienny:
najwcześniejsze spory koncentrują się na tym, czy to Piotr, czy Maria jest prawowitą interpretatorką i przywódczynią.
Ten wzorzec dodatkowo wzmacnia wniosek, że to Maria Magdalena, a nie Jan, była zapamiętana w najwcześniejszych kręgach jako uprzywilejowany świadek i autorytatywny głos stojący za Ewangelią.

W Ewangelii Marii (koptyjski 17--18) Piotr skarży się:
„Czy naprawdę rozmawiał z kobietą na osobności, poza naszą wiedzą?
Czy mamy się obrócić i wszyscy jej słuchać?
Czy bardziej ją umiłował niż nas?”
Lewi (Mateusz) karci go:
„Piotrze, zawsze byłeś porywczy\ldots{} Jeśli Zbawiciel uznał ją za godną, kim ty jesteś, by ją odrzucać?
Z pewnością Zbawiciel zna ją bardzo dobrze.
Dlatego bardziej ją kochał niż nas.”
Ewangelia Filipa zachowuje inną wersję:
„Towarzyszką [Zbawiciela jest] Maria Magdalena.
[On kochał] ją bardziej niż [wszystkich] uczniów i często zwykł ją całować w [\ldots].”
W \textit{Pistis Sophia} Maria zadaje trzydzieści dziewięć z sześćdziesięciu czterech pytań; Jezus nazywa ją „błogosławioną”, „bardziej błogosławioną niż wszystkie kobiety na ziemi” — a Piotr skarży się, że nie przestaje mówić.

Są to trzy niezależne teksty — od różnych autorów, w różnych językach, prawdopodobnie z różnych wspólnot — a wszystkie zachowują tę samą formułę:
Jezus kochał Marię bardziej niż innych.
To stabilna linia w całej wczesnej literaturze.

Janowa formuła ὁ μαθητὴς ὃν ἠγάπα ὁ Ἰησοῦς („uczeń, którego Jezus kochał”) jest dziwna, jeśli chodzi po prostu o Jana, syna Zebedeusza — można by go było bez trudu nazwać po imieniu.
Natomiast motyw „umiłowanej bardziej niż pozostali” jest raz po raz przypisywany Marii Magdalenie w tekstach niekanonicznych.
Kanonizowany Jan i niekanoniczne Maria/Filip/\textit{Pistis Sophia} są więc dwiema różnymi próbami ujęcia tej samej, wspólnej pamięci:
że istniał uczeń, którego Jezus kochał „bardziej niż nas” i że w wielu kręgach tego ucznia znano jako Marię Magdalenę.

Jak zauważa Esther de Boer, w Ewangelii Marii, Ewangelii Tomasza i \textit{Pistis Sophia} to zawsze Piotr atakuje Marię, ponieważ jest kobietą i ponieważ otrzymała szczególne objawienie.
Nie istnieje żaden wczesny tekst, który zachowywałby rywalizację Jana, syna Zebedeusza, z Piotrem o autorytet.
Wszystkie wczesne tradycje sporu to Piotr kontra Maria — jedyna kandydatka pasująca do janowego wzorca.

\section{Struktury płciowe i technika literacka w Ewangelii Jana}\label{sec:gendered-structures-literary}

\subsection{Autorka kobieta najprawdopodobniej opisywałaby siebie przez pryzmat uczuć do innych — niewielu autorów mężczyzn pisze w ten sposób}\label{subsec:finally-a-woman-writer-would-have-been-very-likely-to-describe-themselves-based-on-the-feelings-towards-others-not-many-male-authors-write-like-this.}

Ona mówi, że Jezus kochał Łazarza — nie w sposób romantyczny, lecz tak, jak autorka opisałaby bliską przyjaźń.
Jezus płacze, co jest motywem, o którym raczej napisałaby kobieta, a nie „prawdziwy mężczyzna”.

\subsection{Parowanie płci w Ewangelii Jana: mierzalne dowody strukturalne}\label{subsec:jesus-interactions-towards-nicodemus-are-also-very-feminine-close-and-personal-jesus-is-talking-about-being-born-again.}

Ewangelia Jana systematycznie zestawia rozmówców męskich i żeńskich w parach, a w każdej z nich kobieta przewyższa mężczyznę pod względem zrozumienia, inicjatywy i jasności teologicznej.
Najbardziej widocznym przykładem jest dwuczęściowe zestawienie Nikodema w rozdziale 3 i Samarytanki w rozdziale 4.

Nikodem pojawia się jako żydowski przywódca, mężczyzna, który przychodzi do Jezusa nocą.
W całej rozmowie zabiera głos tylko trzy razy: w J 3,2, 3,4 i 3,9.
Po wersecie 9 całkowicie znika z dialogu, a Jezus kontynuuje monolog od J 3,10 dalej.
Scena z Nikodemem nie zawiera ani udanego zrozumienia, ani wyznania, ani aktu świadectwa, ani żadnego ruchu narracyjnego wywołanego przez samego Nikodema.

Dla kontrastu, Samarytanka w rozdziale 4 otrzymuje najdłuższą, najspójniejszą, indywidualną rozmowę jeden na jeden, jaką Jezus prowadzi z kimkolwiek w ewangeliach.
Bezpośredni dialog rozciąga się przez około dwadzieścia kolejnych wersetów (J 4,7--26), z licznymi wymianami zdań po obu stronach.
Kobieta wchodzi w wielowarstwową dyskusję teologiczną, zadaje dociekliwe pytania, stopniowo rozpoznaje tożsamość Jezusa i w końcu staje się pierwszą publiczną ewangelizatorką tej Ewangelii.
Ewangelista stwierdza wprost, że „wielu Samarytan uwierzyło z powodu słowa kobiety, które świadczyła” (J 4,39).

Kontrasty te dają się obiektywnie zmierzyć:

\begin{itemize}
\item Nikodem: 3 wypowiedzi (3,2.4.9); dialog załamuje się w niezrozumieniu i wycofaniu.
\item Samarytanka: ok. 20 wersetów nieprzerwanego dialogu (4,7--26); udane rozpoznanie i publiczne głoszenie (4,28--29.39).
\item Rezultat: mężczyzna–elitarny przywódca nie pojmuje nawet podstawowych obrazów; społecznie marginalizowana kobieta uchwytuje prawdę chrystologiczną i rozpowszechnia ją dalej.
\end{itemize}

To nie jest kwestia interpretacji, lecz architektury strukturalnej.
Jan przeznacza ogromną przestrzeń narracyjną, pełen rozwój teologiczny i sukces ewangelizacyjny dla kobiety, podczas gdy odpowiadająca jej postać męska otrzymuje minimum dialogu i schodzi ze sceny w zamęcie.
Takie odwrócenie zwykłego wzorca literatury starożytnego basenu Morza Śródziemnego — gdzie kobiety zazwyczaj zajmują role pomocnicze lub negatywne — jest wyjątkowo rzadkie.
Znakomicie wpisuje się ono w hipotezę, że Jan został napisany przez kobietę, której doświadczenie życiowe i priorytety ukształtowały narrację.

\subsection{Samarytanka (J 4,1--42) --- intymne spotkanie przy studni}\label{subsec:the-samaritan-woman-john-41-42-an-intimate-encounter-at-the-well}

Jezus spotyka ją samą przy studni — miejscu, które w Starym Testamencie często symbolizuje relacje małżeńskie lub przymierne (np. Izaak i Rebeka, Jakub i Rachela).
Rozmowa ma głęboko osobisty charakter, bo Jezus ujawnia, że zna jej przeszłość (pięciu mężów), a mimo to jej nie potępia.
Stopniowo prowadzi ją do rozpoznania go jako Mesjasza, a ona staje się kluczową postacią w szerzeniu jego przesłania.
Emocjonalna przemiana — od sceptycyzmu do radości — czyni tę scenę jednym z najbardziej przemieniających, osobistych spotkań w całych ewangeliach.

\subsection{Szczegółowy opis męki uwiarygodnia twierdzenie, że ewangelia została napisana przez naocznego świadka}\label{subsec:the-detailed-description-of-the-passion-seems-to-add-credibility-to-the-claim-that-the-gospel-was-written-by-an-eyewitness-or-at-least-someone-portraying-themselves-as-an-eyewitness.}

Ewangelie zdają się wskazywać, że apostołowie uciekli — i mieli ku temu dobry powód, bo zapewne słusznie bali się oskarżenia o wspieranie Jezusa w rzekomej zbrodni przeciwko państwu rzymskiemu.
Dużo bardziej prawdopodobne jest natomiast, że kobiety mogły pozostać i nie były ścigane przez Rzymian, pochodzących z jeszcze bardziej patriarchalnego społeczeństwa, które w ogóle nie brałoby pod uwagę, że kobieta może stanowić poważne zagrożenie dla państwa.

\subsection{Maria Magdalena (J 20,11--18) --- chwila czystego oddania}\label{subsec:mary-magdalene-john-2011-18-a-moment-of-pure-devotion}

Maria płacze samotnie przy pustym grobie, okazując głęboki żal.
Jezus się pojawia, lecz ona nie rozpoznaje go od razu, biorąc go za ogrodnika.
Scena staje się głęboko osobista w momencie, gdy Jezus po prostu wypowiada jej imię: „Mario”, a ona natychmiast go rozpoznaje.
Nazywa go ῥαββωνί (Nauczycielu), co ujawnia bardzo osobistą więź.
Jezus powierza jej następnie pierwsze ogłoszenie zmartwychwstania, czyniąc ją pierwszą świadkinią Wielkanocy.

Jeśli jest jedna rzecz, którą powszechnie mówi się o dziewczynach, to to, że mają słabość do nauczycieli.
Nauczyciel czyni ją wyjątkową, powierzając jej pierwsze ogłoszenie zmartwychwstania.

\subsection{Motyw oblubieńca}\label{subsec:the-bridegroom-motif}

Jan Chrzciciel nazywa Jezusa oblubieńcem (J 3,29).
Pierwszy cud dokonuje się na weselu (J 2), co jest nietypowym otwarciem narracji ewangelicznej.
Idea Jezusa–oblubieńca ma silną wymowę symboliczną w tradycji żydowskiej i wczesnochrześcijańskiej, czasem łącząc się z obrazem boskiego małżeństwa (Bóg i Izrael, Chrystus i Kościół).
Niektóre tradycje gnostyckie i ezoteryczne później podkreślały więź Marii Magdaleny z Jezusem, widząc w niej część tego oblubieńczego obrazu.

We wszystkich głównych tradycjach literackich mężczyźni oczywiście tworzą sceny z udziałem wesel, ale nie budują wokół nich całej narracji.
Od Homera po Szekspira i kino współczesne, autorzy mężczyźni używają wesel jako komicznych finałów, chwytów fabularnych lub rytuałów społecznych, nigdy jako konstrukcyjnego kręgosłupa opowieści.
Narracje nasycone obrazami ślubnymi, słownictwem oblubieńczym, symboliką małżeńską i intymnością relacyjną powstają w przeważającej mierze w kręgu autorek.
Ewangelia Jana należy do tego wzorca:
jest narracją opakowaną w motyw weselny, zakodowaną w języku małżeństwa, a jej architektura tematyczna bardziej pasuje do kobiecej wrażliwości literackiej niż do jakiejkolwiek znanej tradycji męskiej.

\subsection{Pamięć węchowa jako odcisk palca autorki: J 12,3 i zapach pomieszczenia}\label{subsec:the-anointing-of-jesus-by-a-woman-with-expensive-perfume-john-123-mark-143-9}

J 12,3 zawiera szczegół zmysłowy, którego brakuje we wszystkich trzech paralelnych opowiadaniach synoptycznych.
W Mk 14,3--9, Mt 26,6--13 i Łk 7,36--50 kobieta namaszcza Jezusa wonnym olejkiem, lecz żadne z tych opowiadań nie opisuje zapachu pomieszczenia.
Tylko Jan pisze:

\begin{quote}
„A dom napełnił się wonią olejku.”
\end{quote}

Proste sprawdzenie w konkordancji potwierdza, że jest to jedyny opis narracyjny w całym Nowym Testamencie, który mówi, jak pachniało wnętrze pomieszczenia.
Pozostałe ewangelie nie zawierają opisów zapachu wnętrz.
Dzieje Apostolskie również ich nie mają.
Paweł używa języka „wonności” jedynie metaforycznie (np. 2 Kor 2,14--16; Ef 5,2; Flp 4,18), nigdy jako konkretnej relacji zmysłowej.
Apokalipsa posługuje się obrazem kadzidła symbolicznie (np. Ap 8,3--4), a nie jako opisem atmosfery fizycznej przestrzeni.

Jan jako jedyny dostarcza sensorycznego, domowego, wypełniającego pokój detalu węchowego.

Szczegół ten jest zestawiony z innym wyraźnym momentem węchowym w ewangeliach:
ostrzeżeniem Marty przy grobie Łazarza: „Panie, już cuchnie” (J 11,39).
Są to dwa najdłuższe odniesienia do węchu w ewangeliach kanonicznych i oba pojawiają się w scenach napędzanych głosami lub działaniami kobiet (Maria z Betanii w J 12; Marta w J 11).

Tego typu szczegół odpowiada znanym wzorcom z papirusów dokumentarnych:
listy kobiet z grecko–rzymskiego Egiptu znacznie częściej niż pisma mężczyzn koncentrują się na przestrzeni domowej, zarządzaniu gospodarstwem, jedzeniu, ubraniach i ciałach członków rodziny — w sposób rzadko spotykany w pismach obywatelskich czy filozoficznych autorstwa mężczyzn.
Węchowa specyfika J 12 i J 11 wpisuje się w ten wzorzec domowej, ucieleśnionej uwagi.

Te dwa sensoryczne momenty są więc czymś więcej niż ozdobą nastroju.
To jedyne dwa rozbudowane opisy zapachu w ewangeliach, oba powiązane z kobietami, oba związane z kontekstem domowym lub cielesnym i oba umieszczone w kluczowych punktach przejścia narracyjnego (przygotowanie do pogrzebu w J 12; przygotowanie do zmartwychwstania w J 11).
To materialnie weryfikowalny ślad głosu narracyjnego, swobodnie poruszającego się w przestrzeni domowej i w szczególe fizycznym — cech, które naturalnie kojarzą się z kobiecą ręką autorską.

\subsection{Głos wnętrza domowego: paralele klasyczne}

\label{subsec:domestic-interior-voice}

W literaturze greckiej i rzymskiej pisanej przez mężczyzn przestrzeń domowa ma charakter czysto funkcjonalny, nie jest opisywana.
Przestrzenie domowe rzadko pojawiają się w narracji, chyba że w celach moralizatorskich lub satyrycznych.
Opis zmysłowy jest skąpy.

Jan opisuje zapach perfum wypełniających dom (12,3), troskę o odór rozkładu (11,39), drobiazgowe sceny domowe w Betanii i Kanie, intymne wnętrza, takie jak spoczynek w sali na górze, oraz gesty relacyjne z fizycznym dotykiem i łzami.

To jest znak rozpoznawczy kobiecego autorstwa w starożytności.
Fragmenty Safony podkreślają przestrzeń domową i doświadczenie zmysłowe.
Poetyka Korynny koncentruje się na przestrzeni zajmowanej przez kobiety.
„Kobiety na święcie Adonisa” Teokryta (Idylla 15) przedstawiają wnętrze domu z zapachem i fakturą, widziane z kobiecego punktu widzenia.
Papirusy epistolarne z epoki hellenistycznej z Oksyrynchos pokazują, że listy kobiet konsekwentnie skupiają się na gospodarstwie domowym, jedzeniu, ubraniach i ciałach członków rodziny.

Wzorzec zmysłowej narracji Jana odpowiada gatunkom pisanym przez kobiety, a nie męskiej biografii.

\subsection{Fokalizacja płciowa i sceny rozpoznania}\label{subsec:gendered-focalization}

W greckiej tradycji retorycznej autorzy mężczyźni eksponują działanie, honor publiczny, przemówienia, genealogie oraz wydarzenia militarne lub sądowe.
Autorki albo narracje prowadzone kobiecym głosem eksponują stany emocjonalne wnętrza, prywatne cierpienie, dynamikę relacji i osobiste objawienie.

Ewangelia Jana jest wyjątkowa wśród starożytnych biografii naciskiem na łzy (płacząca Maria Magdalena, płaczący Jezus), skupieniem na szeptanych rozmowach (Nikodem nocą), wydłużonymi łukami emocjonalnymi (namaszczenie przez Marię z Betanii i reakcja Jezusa), rozpoznaniem poprzez jedno osobiste słowo („Mario!”) oraz bliskością wobec śmierci i żałoby (Łazarz).

Scena rozpoznania w rozdziale 20 jest szczególnie uderzająca.
Grecka teoria narracji (Konstan, Burrus) zauważa, że sceny rozpoznania oparte na wywołaniu imienia są ponadproporcjonalnie częste w narracjach autorstwa kobiet lub prowadzonych kobiecym głosem.
Jan 20, z sekwencją „Mario!” — „Rabbuni!”, jest najbardziej intymną sceną \textit{anagnorisis} w całym Nowym Testamencie.

Narrator Jana zachowuje się jak ktoś piszący od środka doświadczenia emocjonalnego, a nie z zewnątrz.
To kobieco kodowany podpis narracyjny.

\section{Maria Magdalena i janowa teologia rozpoznania}\label{sec:johannine-theology-recognition}

\subsection{Żydowska tradycja mądrościowa i Prolog}\label{subsec:jewish-wisdom-tradition}

Literatura żydowska zna kobiece narracje objawienia:
Córka Syjonu lamentuje w Lamentacjach, Mądrość/Sophia mówi w pierwszej osobie w Prz 8 i w Księdze Mądrości, Debora przemawia w Sdz 5, Miriam pojawia się w tradycji wyjścia.
Kobiecy głos objawienia i kobiecy autorytet prorocki nie były obce myśli żydowskiej.

Prolog Jana wpisuje się w tę tradycję.
Obraz Logosu nawiązuje do żydowskiej Mądrości/Sophii, a nie do męskiej biografii rzymskiej.
Logos Filona łączy żydowską Mądrość, filozofię grecką i świątynną Szekinah.
Gdyby autorką była kobieta wykształcona w judaizmie aleksandryjskim, miałaby idealną pozycję, by połączyć kobiece tradycje Mądrości z hellenistycznym językiem filozoficznym.
Prolog Jana dokładnie odpowiada temu wzorcowi.

\subsection{Księgi pamięci kobiet}\label{subsec:womens-memory-books}

Archiwa papirusowe z rzymskiego Egiptu pokazują, że kobiety prowadziły księgi pamięci, zachowywały wspomnienia narodzin, śmierci i chorób, zapisywały doświadczenia emocjonalne i duchowe oraz utrwalały wypowiedzi mężów lub synów.

To do tego gatunku upodabnia się Jan:
do narracji pamięciowej, w której szczegół domowy, faktura zmysłowa i refleksja teologiczna splatają się ze wspomnieniem.
Ewangelia Jana czyta się jak kobieca księga wspomnień, a nie jak męski \textit{bios}.

\subsection{Składnia kobieca i słownictwo emocjonalne}\label{subsec:feminine-syntax}

Teksty z głosem kobiecym w starożytności, zwłaszcza listy papirusowe z rzymskiego Egiptu, chętnie korzystają z czasowników percepcji i emocji: widzieć, wiedzieć, wierzyć, lękać się, płakać, kochać.
Ich język jest wewnętrzny, relacyjny i zakorzeniony w doświadczeniu życia codziennego.

Ewangelia Jana dzieli ten profil.
Narrator używa czasowników percepcji częściej niż jakikolwiek inny autor Nowego Testamentu.
Ewangelia sięga po pełne słownictwo emocji — Jezus wzruszony, poruszony, płaczący — i obraca się wokół czasowników miłości (\textgreek{ἀγαπᾶν}, \textgreek{φιλεῖν}).

Ten wzorzec językowy przypomina głos kobiecych listów domowych znacznie bardziej niż podniosłą prozę męskich autorów elitarnych.
Składnia Jana i jego słownik emocji są zgodne z narracją pisaną kobiecą ręką w starożytności.

\subsection{Pieśń nad Pieśniami i spotkanie w ogrodzie}\label{subsec:song-of-songs}

Ogród w J 20, gdzie Maria spotyka zmartwychwstałego Jezusa, przywołuje Pieśń nad Pieśniami.
Maria szuka Umiłowanego, płacząc i pytając ogrodnika, gdzie on jest.
Rozpoznanie dokonuje się w chwili, gdy Umiłowany woła ją po imieniu.
Ojcowie Kościoła i komentatorzy średniowieczni regularnie interpretują Oblubienicę z Pieśni nad Pieśniami jako Marię Magdalenę lub Kościół.
Współcześni badacze, tacy jak André Feuillet, jasno zestawiają J 20 z Pieśnią nad Pieśniami.

W Pieśni nad Pieśniami Umiłowana jest kobietą; to jej tęsknota, poszukiwanie i głos tworzą narracyjne ramy.
Janowa fraza „uczeń, którego Jezus miłował” znajduje się wewnątrz tego biblijnego wzorca, gdzie Bóg/Chrystus jest oblubieńcem, a Umiłowana — partnerką w rodzaju żeńskim.
Kanoniczne tło dla słowa „umiłowany” stanowi Pieśń nad Pieśniami i w tym wzorcu postać umiłowana jest zawsze kobieca.

\subsection{Łazarz i krąg betanijski}\label{subsec:lazarus-bethany}

Niewielka, ale poważna grupa badaczy uważa, że umiłowany uczeń to Łazarz.
J 11,3 mówi: „Panie, oto choruje ten, którego kochasz.”
J 11,5: „A Jezus miłował Martę i jej siostrę, i Łazarza.”
J 11,36: „Patrzcie, jak go miłował.”
Ten sam czasownik pojawia się w J 13, 19, 20 i 21 w odniesieniu do ucznia, którego Jezus miłował.

Jeśli Łazarz jest umiłowanym uczniem, czyj punkt widzenia tłumaczy betanijskie sceny, szczegół domowy oraz prywatną wiedzę o jego chorobie, śmierci i wskrzeszeniu?
Najbardziej oczywistą kandydatką jest Maria z Betanii, kobieta najbliższa Łazarzowi.
Nawet jeśli recenzenci wzdragają się przed Marią jako umiłowanym uczniem, poważni badacze już odrywają tę postać od Jana, syna Zebedeusza, i zakorzeniają motyw „umiłowanego” w kręgu Betanii, gdzie w centrum stoi Maria.

\subsection{Krzyż i formuła redakcyjna}\label{subsec:cross-editorial-formula}

J 19,25 wymienia cztery kobiety stojące pod krzyżem, w tym Marię Magdalenę.
W wersetach 26--27 Jezus mówi do swojej matki i do ucznia, którego miłował, stojącego tam.
Tekst nigdzie wprost nie stwierdza, że umiłowany uczeń jest mężczyzną ani że nie jest jedną z wymienionych przed chwilą kobiet.
Gramatycznie Jezus mógł powierzać swoją matkę innej kobiecie stojącej obok.

J 21,24 odróżnia „tego, który daje świadectwo” (umiłowanego ucznia) od „my”, którzy owo świadectwo potwierdzają.
Jeśli późniejsi przywódcy-mężczyźni czuli się niekomfortowo z kobietą jako głównym autorytetem, można by się spodziewać następującego zabiegu:
jej rola jako umiłowanej świadek zostaje zachowana, jej imię znika, a męski liczebnik mnogi „my” służy do uwierzytelnienia jej świadectwa.
J 21,24 robi dokładnie to.

\section{Kobiety jako pierwsi świadkowie i protagonistki narracji}\label{sec:women-primary-witnesses}

\subsection{Maria Magdalena pierwsza na listach}\label{subsec:mary-first-in-lists}

W listach kobiet w synoptykach Maria Magdalena niemal zawsze pojawia się na pierwszym miejscu, tak jak Piotr w listach Dwunastu.
Badaczki takie jak Carla Ricci zwracają uwagę na tę strukturalną paralelę:
pozycja Marii Magdaleny wśród uczennic odpowiada pozycji Piotra wśród apostołów.

Jan nadaje Umiłowanemu Uczniowi rolę równoległą do Piotra, lecz bardziej wierną, bardziej wnikliwą, bliższą Jezusowi.
Teksty poza kanonem mówią, że Maria była kochana bardziej niż inni, a Piotr żywił o to urazę.
Najprostsza synteza: janowa tradycja o umiłowanym uczniu jest zmaskulinizowaną, zanonimizowaną formą pierwotnej pamięci, która wskazywała Marię Magdalenę jako uczennicę, którą Jezus kochał bardziej niż innych.

\subsection{Kobiety w kluczowych momentach zwrotnych narracji: policzalne dowody strukturalne}\label{subsec:the-gospel-of-john-passes-the-bechdel-test.}

Ścisły, współczesny test Bechdel (dwie nazwane kobiety rozmawiające ze sobą o czymś innym niż mężczyzna) nie da się bezpośrednio zastosować do starożytnej narracji teologicznej.
Ewangelia Jana nie zachowuje na kartach tekstu rozbudowanych dialogów między kobietami.
Jeśli jednak dostosujemy test w sposób historycznie adekwatny — pytając, czy kobiety mówią własnym głosem o Bogu, prawdzie i świecie, a nie jedynie wzmacniają wypowiedzi mężczyzn — Jan zdecydowanie ten test przechodzi.

Dowody strukturalne są policzalne.
Na wszystkich głównych zwrotnych momentach narracji — miejscach, w których fabuła zmienia kierunek — stoją kobiety:

\begin{enumerate}
\item \textbf{Kana (J~2:1--11):} pierwszy znak Jezusa i początek jego publicznego objawienia dokonują się z inicjatywy jego matki.

\item \textbf{Samarytanka (J~4:4--42):} pierwszy rozbudowany dialog teologiczny i pierwsza masowa wiara poza Judeą rodzą się z rozmowy i świadectwa kobiety.

\item \textbf{Marta (J~11:17--27):} pierwsze wyraźne wyznanie chrystologiczne („Wierzę, że Ty jesteś Chrystusem, Synem Bożym") wypowiada kobieta.

\item \textbf{Maria z Betanii (J~12:1--8):} przejście do Męki wyznacza namaszczenie dokonane przez kobietę, interpretowane przez Jezusa jako przygotowanie do pogrzebu.

\item \textbf{Maria Magdalena (J~20:11--18):} pierwsze spotkanie zmartwychwstałego i pierwszy nakaz apostolski („idź do moich braci") powierzono kobiecie.
\end{enumerate}

Są to pięć kluczowych punktów zwrotnych Ewangelii — i wszystkie pięć zapośredniczone jest przez inicjatywę kobiet lub ich głos.
Dla kontrastu, żaden z męskich uczniów — ani Piotr, ani Andrzej, ani Filip, ani Tomasz — nie zajmuje w narracji roli porównywalnego węzła zwrotnego.

Aby uniknąć zarzutu wybiórczości, należy odnotować również męskocentryczne momenty kulminacyjne: powołanie pierwszych uczniów (J~1:35--51), mowę o Chlebie Życia (J~6) czy wyznanie Tomasza (J~20:28).
Niewątpliwie są one istotne.
Ale funkcjonują one wewnątrz faz narracyjnych, a nie jako punkty przejścia między fazami.
Gdy historia zmienia kierunek — od życia prywatnego do publicznej działalności, od Judei do Samarii, od życia do śmierci, od śmierci do zmartwychwstania — narracyjnym agentem na każdym progu jest kobieta.

Ten rozkład ról jest niezwykle nietypowy w kontekście starożytnej narracji śródziemnomorskiej, w której kobiety zwykle pojawiają się w rolach pomocniczych, dekoracyjnych lub ostrzegawczych.
Jan przeciwnie — czyni z kobiet rozmówczynie teologiczne, inicjatorki wydarzeń i główne świadkinie objawienia.
Wzorzec ten jest mierzalny, konsekwentny i zamierzony.
Ściśle odpowiada hipotezie, że Ewangelia została skomponowana przez kobietę, której perspektywa, pamięć i priorytety ukształtowały samą architekturę narracji.

\subsection{Kobiety towarzyszące Jezusowi i uczniom są również powiązane z rodziną królewską Heroda.}\label{subsec:note-the-women-traveling-with-jesus-and-the-disciples-are-also-linked-to-the-royal-family-of-herod.}

„Byli z nim Dwunastu, a także niektóre kobiety, które zostały uzdrowione z złych duchów i chorób: Maria (zwana Magdaleną), z której wyszło siedem demonów; Joanna, żona Chuzy, zarządcy domu Heroda; Zuzanna; i wiele innych.
Kobiety te wspierały ich ze swojego majątku." Maria Magdalena, Joanna i Zuzanna są wymienione z imienia jako finansowe patronki działalności.
Joanna była związana z dworem Heroda, co oznacza, że miała dostęp do bogactwa i wpływów.
Sformułowanie „ze swojego majątku" sugeruje, że osobiście finansowały działalność Jezusa.
Zauważmy, że podobnie jak Sepphoris, Migdala znajduje się w tej samej jednostce osadniczej co Nazaret.
Jeśli Maria Magdalena również zajmowała wysoką pozycję na dworze Heroda, wówczas naturalne byłoby, że dysponowała umiejętnościami pisarskimi — choć zapewne nie na poziomie najwybitniejszych klasyków.
To wyjaśniałoby głęboką znajomość królewskiego kultu greckiego przy jednoczesnym zachowaniu pewnego braku literackiego wyrafinowania.

\subsection{Męskie zaimki nie wykluczają kobiecego autorstwa}\label{subsec:the-authorship-of-mary-is-typically-dismissed-as-a-possibility-because-of-the-use-of-male-pronouns-in-the-gospel.}

Standardowy zarzut głosi, że umiłowany uczeń musiał być mężczyzną, ponieważ Ewangelia używa męskich form gramatycznych: ὁ μαθητής („uczeń”) oraz męskiego zaimka względnego ὃν („którego”).
Zarzut ten działa jednak tylko u osób myślących po angielsku.
Angielski nie ma rodzaju gramatycznego.
Anglojęzyczni odbiorcy podświadomie zakładają więc, że oznaczanie rodzaju w innych językach musi odzwierciedlać płeć biologiczną.

Użytkownicy języków indoeuropejskich z rodzajem gramatycznym natychmiast wiedzą, że to nieprawda.

Po polsku na przykład: Ofiara była mężczyzną oraz Ewangelista był kobietą.
Oba zdania są całkowicie naturalne.
Ofiara jest rzeczownikiem gramatycznie żeńskim, ewangelista — gramatycznie męskim.
Formy czasownika była i był podążają za rodzajem gramatycznym rzeczownika, a nie za płcią osoby.
Każdy użytkownik czeskiego, niemieckiego, francuskiego, hiszpańskiego, włoskiego, litewskiego czy greckiego rozumie to intuicyjnie:
gramatyka idzie za rzeczownikiem, nie za ciałem.

Zjawisko to jest wszechobecne:
niemieckie die Person war ein Mann — żeńska Person, męski referent.
Francuskie la victime était un homme — żeńska victime, męski referent.
Hiszpańskie la persona era un hombre — żeńska persona, męski referent.
Arabski żeński al-daḥiyya („ofiara”) używany wobec mężczyzn:
kānat al-daḥiyya rajulan („ofiara była mężczyzną”).

Anglojęzyczni odbiorcy mają z tym trudność, ponieważ ich język w ogóle nie koduje rodzaju gramatycznego, dlatego fałszywie przenoszą założenia o płci biologicznej na gramatykę grecką.

Greka koine — wspólna greka, która po podbojach Aleksandra stała się lingua franca wschodniego basenu Morza Śródziemnego — dostarcza idealnych paraleli.
W papirusach i inskrypcjach: μαῖα („położna”, forma wyłącznie żeńska) używana jest wobec mężczyzn (P.Oxy.\ 119, 744); ἡ γραμματεύς („sekretarz”, z rodzajnikiem żeńskim) używana jest wobec mężczyzny–urzędnika (P.Tebt.\ 51; P.Mich.\ 462).
W samym Nowym Testamencie Febe w Rz 16,1 jest διάκονος, rzeczownikiem gramatycznie męskim, choć jest kobietą — imiesłów οὖσαν zaznacza płeć żeńską, rzeczownik pozostaje męski.
Odwrotnie, rzeczowniki gramatycznie żeńskie, takie jak ψυχή („dusza, osoba”), często odnoszą się do mężczyzn.

Najsilniejszą paralelę nowotestamentową stanowi 1 P 3,20, gdzie explicite mowa o Noem i jego rodzinie:

\begin{quote}
ὅτε ἀπεξεδέχετο ἡ τοῦ Θεοῦ μακροθυμία ἐν ἡμέραις Νῶε, κατασκευαζομένης κιβωτοῦ, εἰς ἣν \ldots{} ὀκτὼ ψυχαὶ διεσώθησαν δι' ὕδατος.

„kiedy za dni Noego cierpliwość Boga oczekiwała, a budowano arkę, do której \ldots{} niewiele, to jest osiem dusz, zostało uratowanych przez wodę.”
\end{quote}

Struktura jest jasna.
Νῶε (Noe) jest wymieniony po imieniu.
Wyrażenie ὀκτὼ ψυχαί („osiem dusz”) odnosi się bezpośrednio do Noego i jego domu — Noego plus siedem innych osób.
ψυχή jest rzeczownikiem gramatycznie żeńskim, dlatego zaimek względny ἣν również musi być w rodzaju żeńskim.
Noe jest w tym zdaniu „gramatycznie żeński” nie dlatego, że ma płeć żeńską, ale dlatego, że ψυχή jest rzeczownikiem żeńskim.

To jednoznaczny wewnętrzny dowód nowotestamentowy, że rodzaj gramatyczny w grece podąża za rzeczownikiem, a nie za płcią referenta.

Jest to dokładnie ta sama zasada co po polsku:
Ofiara była mężczyzną (gramatyka żeńska, referent męski) i Ewangelista był kobietą (gramatyka męska, referent żeński).

Kiedy to zrozumiemy, zarzut znika.
Jeśli greka może odnosić się do Noego za pomocą żeńskiej morfologii, ponieważ ψυχή jest żeńska, to może także odnosić się do umiłowanego ucznia w formie męskiej, ponieważ μαθητής jest rzeczownikiem rodzaju męskiego.
Rodzaj gramatyczny nie jest tożsamy z płcią biologiczną.
Męska gramatyka u Jana nie może być użyta jako argument przeciwko kobiecemu autorstwu.
Odzwierciedla jedynie odmianę μαθητής, a nie płeć autora czy ucznia.

\section{Transmisja, zatarcie i synteza}\label{sec:transmission-erasure-synthesis}

\subsection{Tekstowe dowody na zatarcie kobiet}\label{subsec:textual-evidence-of-womens-erasure}

Hipoteza, że tożsamość kobiety–autorki została zaciemniona, nie jest czystą spekulacją.
Dysponujemy twardymi dowodami rękopiśmiennymi, że skrybowie celowo modyfikowali kobiecą widoczność we wczesnych tekstach chrześcijańskich.

Najbardziej oczywistym przypadkiem jest Junia z Rz 16,7.
Paweł pozdrawia Ἰουνίαν (Junię) jako „wybitną wśród apostołów”.
Imię jest gramatycznie i historycznie żeńskie; nie istnieje poświadczona męska forma „Junias”.
A jednak przez wieki skrybowie i tłumacze czynili z niej mężczyznę, bo zakładali, że kobieta–apostoł jest niemożliwa.
Współczesne badania krytyczne Eldona Eppa, Michaela Birda i innych jednoznacznie wykazały, że maskulinizacja była ideologicznym zniekształceniem.
To udokumentowany przykład sytuacji, w której apostolski status kobiety został zaciemniony przez późniejszą transmisję.

Ten sam mechanizm widać wewnątrz samej Ewangelii Jana.
Artykuł Elizabeth Schrader Polczer „Was Martha of Bethany Added to the Fourth Gospel?” (JBL 2017) pokazuje znaczną niestabilność tekstową w fragmentach o Marii i Marcie w J 11--12.
Dane są uderzające:
wczesne rękopisy, w tym P66 i P75, pokazują, że imiona „Maria” i „Marta” zamieniają się miejscami; „Maria” pojawia się tam, gdzie późniejsze rękopisy mają „Martę”, albo obie zostają całkowicie zamienione.
Niektóre warianty sugerują, że pierwotnie występowała tylko jedna siostra, o imieniu Maria, bez Marty.
Co najważniejsze, kulminacyjne wyznanie w J 11,27 — „Tak, Panie, ja uwierzyłam, że Ty jesteś Chrystus, Syn Boży, który miał przyjść na świat” — mogło pierwotnie zostać włożone w usta Marii, a nie Marty.

Wniosek Schrader Polczer jest prosty:
dowody tekstowe są zgodne z hipotezą, że pierwotnie w J 11 istniała jedna silna postać wyznająca wiarę, o imieniu Maria, która później została podzielona na dwie i częściowo „złagodzona”, prawdopodobnie po to, by zmniejszyć znaczenie tej Marii.

To nie teoria.
To realne warianty lekturowe, podmiany imion i niestabilności rękopiśmienne, świadczące o celowym, skrybskim przekształcaniu postaci kobiecych w Janie.
Jeżeli tradycja janowa pierwotnie zachowywała jeszcze mocniejsze kobiece świadectwo chrystologiczne, które skrybowie później próbowali osłabić, to dokładnie taki proces mógł także przekształcić „Maria, umiłowany uczeń” w anonimowego „ucznia, którego Jezus miłował”.

\subsection{Maria Magdalena jako czwarty filar: synteza dowodów}\label{subsec:mary-magdalene-as-the-fourth-pillar-synthesis}

Gdy połączymy ze sobą wątki świadectwa rękopiśmiennego, ogniskowania narracji, wczesnochrześcijańskich sporów i analizy językowej, staje się jasne, że Maria Magdalena nie była postacią marginalną, lecz stała obok Piotra, Jakuba i Pawła jako filar pierwotnego Kościoła.
Jej świadectwo naoczne, zachowane w Ewangelii Jana, ukształtowało zrozumienie tożsamości Jezusa i jego zmartwychwstania we wspólnocie.
Rozpoznanie Marii Magdaleny jako umiłowanego ucznia przywraca jej należne miejsce „apostołki apostołów”, status od dawna zaciemniony przez późniejsze przypisania i redakcję.
Obok Piotra, Jakuba i Pawła Maria Magdalena jawi się jako czwarty wielki filar początków chrześcijaństwa.

Nie jest to rewizjonistyczna spekulacja.
Tradycje prawosławne i bizantyjskie zachowują dokładnie taką pamięć.
W Kościołach greckich i słowiańskich Maria Magdalena jest konsekwentnie tytułowana ἰσαπόστολος — „równą apostołom” — status ten dzieli jedynie garstka świętych.
Pamięta się o niej jako o „Niosącej mirra i uczennicy Chrystusa”, „tej, która poszła za Nim aż na Golgotę” i „niosącej wieść o Jego zmartwychwstaniu uczniom”.
Coroczne hymny liturgiczne opisują ją jako „umiłowaną przez Mistrza, stojącą z Nim pod krzyżem”.

Istnieje także długa tradycja wschodnia, wedle której Maria Magdalena udała się z apostołem Janem do Efezu i tam zmarła, a jej relikwie przeniesiono później do Konstantynopola.
Jest to dokładnie ten sam obszar geograficzny, w którym osadzana jest Czwarta Ewangelia.
Pamięć o tym, że środowisko janowe w Azji Mniejszej zostało ukształtowane przez obecność Marii, przetrwała w źródłach greckich i bizantyjskich — poza łacińskim Zachodem, gdzie później zredukowano ją do roli skruszonej nierządnicy.

Nie istnieje żaden równorzędny, wczesny przekaz przedstawiający Jana, syna Zebedeusza, jako „umiłowanego ucznia”.
Najwcześniejsza żywa pamięć o „umiłowanym uczniu” wiąże się z Marią, a nie z Janem Zebedeuszem.
Nawet jeśli szczegóły tej tradycji mają charakter legendarny, konsekwentna pamięć Wschodu o Marii jako „równej apostołom”, podróżującej z Janem i umierającej w Efezie, wygląda jak późniejsze folklorystyczne skrystalizowanie wcześniejszej rzeczywistości historycznej:
że świadectwo Marii Magdaleny było fundamentem wspólnoty janowej.

\subsection{Macierz dowodów autorstwa: Maria Magdalena a Joanna}\label{subsec:authorship-evidence-matrix}

\begin{center}
\begin{tabular}{|l|c|c|}
\hline
\textbf{Authorship Evidence} & \textbf{Mary Magdalene} & \textbf{Joanna} \\
\hline
Present at Cross and Tomb & \checkmark & \checkmark (at tomb) \\
Named as Eyewitness in John & \checkmark &  \\
Central in Resurrection Narrative & \checkmark &  \\
Possible Link to Elite Circles & \checkmark & \checkmark \\
Early Church Rivalries (Peter vs.) & \checkmark &  \\
Tradition as “Apostle to Apostles” & \checkmark &  \\
Possible Redactional Role &  & \checkmark \\
Plausible Conduit for Traditions &  & \checkmark \\
\hline
\end{tabular}
\end{center}

\subsection{Umiłowany uczeń odgrywał ważną rolę we wczesnej wspólnocie chrześcijańskiej.}\label{subsec:the-beloved-disciple-is-thought-to-have-played-an-important-role-in-the-early-christian-community.}

Niektórzy badacze twierdzą, że Jan jako męski apostoł miałby większy wpływ we wczesnym Kościele, a zatem wydaje się mało prawdopodobne, by kobieta, taka jak Maria Magdalena, mogła być tą, która pisała lub pełniła rolę figury centralnej.
Obraz Marii Magdaleny we wczesnym Kościele jest jednak znacznie bardziej złożony, niż się to często przedstawia.
Choć została z czasem zepchnięta na margines, pierwotnie zajmowała ważne miejsce we wczesnych tradycjach chrześcijańskich.
Ewangelia Marii, tekst niekanoniczny, sugeruje, że Maria Magdalena pełniła wpływową rolę, być może nawet przywódczą w gronie uczniów.
Ponadto wczesne wspólnoty chrześcijańskie nie były tak sztywne pod względem ról płciowych, jak się je często rysuje, a mamy dowody, że kobiety zajmowały znaczące pozycje we wczesnym Kościele (np. Pryscylla, Febe, Tekla).

\subsection{Ewangelia Jana podkreśla, że Jezus był jednorodzonym Synem Bożym.}\label{subsec:the-gospel-of-john-emphasizes-the-jesus-was-the-only-begotten-son-of-god.}

Wyjątkowa relacja z Bogiem:
Jezus ukazany jest jako Ten, który ma z Bogiem Ojcem relację wyjątkową i intymną, niepodobną do żadnej innej.
W J 1,14.18 oraz 3,16 Jezus nazwany jest „jednorodzonym Synem” (greckie: μονογενής, monogenes).
Termin ten podkreśla wyjątkowość synostwa Jezusa — żaden inny człowiek nie uczestniczy w tej samej, boskiej relacji synowskiej.

\subsection{Interesująca perspektywa łącząca gnostycką Ewangelię Tomasza z Janem:}\label{subsec:here-there-is-a-very-interesting-perspective-linking-the-gnostic-gospel-of-thomas-with-john}

„Jezus powiedział: \texttt{Jeśli\ powiedzą\ wam,\ Skąd\ przyszliście?} powiedzcie im: \texttt{Przyszliśmy\ z\ światła,\ z\ miejsca,\ gdzie\ światło\ powstało\ samo\ z\ siebie,\ ustanowiło\ siebie\ i\ objawiło\ się\ poprzez\ swój\ obraz.}
\texttt{Jeśli\ zapytają\ was,\ Kim\ jesteście?} powiedzcie: \texttt{Jesteśmy\ jego\ dziećmi\ i\ jesteśmy\ wybranymi\ żyjącego\ Ojca.}”
W kilku miejscach Tomasz wydaje się sugerować, że „my” wszyscy jesteśmy dziećmi Boga.
Wyjątkowość synostwa jest w Ewangelii Jana podkreślana tak mocno, że niemal staje się osią całej księgi.
Co ciekawe, równocześnie Jan kreuje Tomasza na nonkonformistę i niedowiarka.
Rozdział 11.
Wszyscy uczniowie boją się iść do Jerozolimy, ale Tomasz mówi:
„Chodźmy i my, abyśmy razem z nim pomarli.”
Rozdział 14.
Jezus mówi, że odchodzi do Ojca.
Tomasz wątpi.
Wtedy Jezus stwierdza:
„Ja jestem drogą, prawdą i życiem.
Nikt nie przychodzi do Ojca inaczej jak tylko przeze Mnie.”
Do Ojca można przyjść wyłącznie przez jednorodzonego Syna.
Rozdział 20.
Tomasz rozpoznaje:
„Pan mój i Bóg mój.”
To pokazuje, że mimo iż Tomasz był bratem lub „bliźniakiem” Jezusa i sam mógł rościć sobie znaczące prawa, Jan próbuje podkreślić, że mógł początkowo kwestionować Jezusa, lecz ostatecznie uznał Go za jedynego prawowitego dziedzica.

14 J 12,31 (NIV): „Teraz odbywa się sąd nad tym światem, teraz władca (archon) tego świata zostanie wyrzucony.”
W tym wersecie „archon” odnosi się do Rzymu lub „księcia tego świata”.
Określenie to oznacza władzę i panowanie nad światem, ale użyte jest w sensie negatywnym na oznaczenie władcy zła.

\subsection{Zniszczenie Świątyni jest zwykle używane jako wyznacznik datowania Ewangelii.}\label{subsec:temple-destruction-is-typically-used-as-a-marker-for-the-dating-of-the-gospels.}

Argument ten stosuje się zazwyczaj do Ewangelii synoptycznych, lecz w przypadku Jana wymaga nieco odmiennego podejścia.
Warto zauważyć, że w przypadku Jana — „Zburzcie tę świątynię, a Ja w trzy dni ją wzniosę" — w przeciwieństwie do mowy na Górze Oliwnej, identyfikacja i proroctwo o zniszczeniu konkretnej świątyni w Jerozolimie w 70 roku są znacznie bardziej dyskusyjne.
Jeśli nie przywiązujemy nadmiernej wagi do mojżeszowych przekonań Ewangelii Jana, to świątynia mogła być jedną z wielu wielu innych zniszczonych świątyń, niekoniecznie główną świątynią w Jerozolimie.
Zauważmy, że świątynia w Jerozolimie została już wcześniej zburzona przez Babilończyków, a potem odbudowana, i była już oblegana przez Rzymian w 63 roku przed Chr.
Nie był to wcale tak daleki wniosek, że mogła zostać zniszczona ponownie.
Zauważmy też, że Sepphoris zostało również zburzone przez armię rzymską za życia Jezusa, a Jezus był silnie zaangażowany w odbudowę Sepphoris.

\subsection{Kobiety okresu greckiego były bardzo upodmiotowione, w przeciwieństwie do kultury rzymskiej i żydowskiej.}\label{subsec:the-females-of-the-greek-period-were-very-empowered-but-not-in-roman-and-jewish-culture.}

To prawda --- w dynastii Ptolemeuszy i w królestwie Hasmoneuszy było wiele władczyń.
Ptolemeusze mieli kilka Kleopatr, Berenik i Arsinoe rządzących samodzielnie lub jako współwładczynie.
Nawet w okresie hasmonejskim Salome Aleksandra (76--67 przed Chr.) rządziła jako królowa i była jedną z najpotężniejszych postaci swoich czasów.
To zdecydowanie osłabia argument, że kobieta nie mogła napisać Ewangelii Jana tylko z powodu płci.
Jeśli elitarne kobiety w tych kręgach miały władzę, bogactwo i wpływy, to kobieta–autorka Ewangelii (jak Maria Magdalena lub inna kobieta związana z dworem) wcale nie jest niemożliwa.
W J 11,27 Marta mówi: „ναί, κύριε· ἐγὼ πεπίστευκα ὅτι σὺ εἶ ὁ Χριστὸς ὁ υἱὸς τοῦ θεοῦ ὁ εἰς τὸν κόσμον ἐρχόμενος.”
Używa „Χριστός” (Christos), a nie „Μεσσίας” (Mesjasz).
To istotne, ponieważ Jan używa „Μεσσίας” w innych miejscach (J 1,41 oraz J 4,25), lecz tutaj Marta posługuje się terminologią grecką.
Zgadza się to z ogólnym, grecko–imperialnym schematem Jana, a nie z czysto żydowskim oczekiwaniem mesjańskim.

Jan nigdy nie używa Μεσσίας (Mesjasz) samodzielnie --- natychmiast tłumaczy ten tytuł jako Χριστός, co oznacza, że jego odbiorcy nie rozpoznawaliby naturalnie słowa „Mesjasz” jako tytułu niosącego znaczenie.
J 1,41 --- „Ten spotkał najpierw swego brata Szymona i powiedział do niego: \texttt{Znaleźliśmy\ Mesjasza} (to znaczy Chrystusa).” (εὑρήκαμεν τὸν Μεσσίαν, ὅ ἐστιν μεθερμηνευόμενον Χριστός).
J 4,25 --- „Rzekła do niego kobieta: \texttt{Wiem,\ że\ przyjdzie\ Mesjasz} (to znaczy Chrystus).” (Οἶδα ὅτι Μεσσίας ἔρχεται, ὁ λεγόμενος Χριστός).
Za każdym razem pojawia się to w mowie niezależnej, co oznacza, że są to postacie żydowskie (Andrzej i Samarytanka) używające żydowskiego terminu, a nie narrator czy autor.
To jest sytuacja, w której Grecy próbują przekonać Żydów, by przyjęli Jezusa także jako ich prawowitego władcę, żeby dodatkowo wzmocnić jego roszczenie i zgromadzić szersze poparcie.

Wyznanie Natanaela: „Rabbuni, Ty jesteś Synem Bożym!
Ty jesteś Królem Izraela!”
Oznacza to, że Natanael rozpoznaje, iż Jezus, uznany za Christos (grecki tytuł królewski), jest nie tylko władcą świata greckiego (Christos), lecz także królem Żydów w bardziej lokalnym sensie.
Tytuł „Król Izraela” potwierdza żydowskie pojęcie królewskości, rozumiane jako władza nad Izraelem, odrębna wobec szerszego, greckiego kontekstu królowania.
Wyznanie Andrzeja wobec Szymona: „Znaleźliśmy Mesjasza.”
Andrzej wskazuje, że Jezus, określany jako Christos (grecki tytuł królewski), nie jest tylko królem greckim (Christos), lecz także Mesjaszem, tytułem, który szczególnie rezonuje z żydowskimi oczekiwaniami króla.
Mówi Szymonowi, że Jezus wypełnia obie role --- greckiego Christos i żydowskiego Mesjasza --- i że zarówno Żydzi, jak i Grecy powinni rozpoznać w Nim władcę obu sfer.

\subsection{W tamtych czasach królów i cesarzy nazywano boskimi.}\label{subsec:in-that-time-kings-and-emperors-were-often-called-divine-not-necessarily-because-people-believed-they-were-literal-gods-but-because-divinity-was-part-of-royal-language-and-legitimacy.}

W tamtym czasie królów i cesarzy często nazywano boskimi, niekoniecznie dlatego, że ludzie wierzyli, iż są dosłownie bogami, lecz dlatego, że boskość była częścią języka królewskiego i legitymizacji władzy.

Dla Marii Magdaleny nazwanie Jezusa swoim „Bogiem” nie musiało oznaczać, że postrzegała Go jako jakiegoś abstrakcyjnego bóstwa, ale raczej że był Jej prawdziwym, namaszczonym władcą --- Jej Christos --- tak jak cesarzy nazywano boskimi.
Widziała w Nim prawowitego króla, tego, który ma panować, a osobista pobożność wobec Niego naturalnie wyrażałaby się w najwyższych możliwych tytułach.
Dlatego gdy nazywa Go „Panem” albo gdy Tomasz mówi „Pan mój i Bóg mój” w Ewangelii Jana, może to odzwierciedlać ten sam rodzaj królewskiego języka używanego wobec cesarzy --- głęboki szacunek, lojalność i uznanie najwyższego autorytetu --- bardziej niż w pełni rozwiniętą teologiczną doktrynę boskości.

\subsection{Patrzenie na jego śmierć musiało być nie do zniesienia.}\label{subsec:watching-him-die-would-have-been-unbearable.}

Patrzenie na jego śmierć musiało być nie do zniesienia.

Złożyła w Nim całą swoją wiarę, miłość i oddanie --- nie tylko jako władcy, ale jako centrum swojego świata.
Kiedy Rzymianie go ukrzyżowali, nie był to tylko koniec życia jednego człowieka; zawaliło się wszystko, w co wierzyła.
Żałoba potrafi potężnie oddziaływać na umysł.
Wobec takiej straty mogła doświadczać wizji, snów lub przytłaczającego poczucia, że on nie mógł naprawdę odejść.
Myśl, że zmartwychwstał, nie musiała początkowo przybrać postaci teologicznego twierdzenia --- mogła zacząć się jako rozpaczliwa, emocjonalna odpowiedź na nie do zniesienia ból utraty.
W tej chwili nie była po prostu świadkiem; była tą, która nie potrafiła go puścić.

\subsection{Opowieść o zmartwychwstaniu w Ewangelii Jana to głęboko osobiste doświadczenie.}\label{subsec:the-resurrection-story-in-john-isnt-just-a-theological-claimits-a-deeply-personal-emotional-experience-that-reflects-the-profound-impact-jesus-had-on-her-life.}

Opowieść o zmartwychwstaniu w Ewangelii Jana nie jest tylko twierdzeniem teologicznym --- to głęboko osobiste, emocjonalne doświadczenie, odzwierciedlające ogromny wpływ, jaki Jezus wywarł na jej życie.
To tłumaczy, dlaczego janowy opis zmartwychwstania jest tak intymny i osobisty, koncentrując się na żalu, zagubieniu i radości Marii.
To wyjaśnia, dlaczego tylko Maria widzi Jezusa jako pierwsza i dlaczego początkowo go nie rozpoznaje --- jej spojrzenie przesłaniają łzy i rozpacz.

\subsection{Opis męskiej nagości}\label{subsec:description-of-male-nudity}

Pojawia się niezwykły szczegół dotyczący Szymona Piotra, który jest nagi w obecności ucznia, którego Jezus miłował.
Ten szczegół wydaje się nie mieć oczywistego miejsca w narracji i nie jest jasne, dlaczego został włączony.

Choć zaproponowano bardziej wyrafinowane interpretacje tego fragmentu, najbardziej bezpośrednie wyjaśnienie samo się narzuca.
Kobieta, emocjonalnie bliska Piotrowi lub Jezusowi, pamiętająca to dziwne, nieco intymne zdarzenie:
„Był praktycznie nagi, zobaczył Jezusa i w popłochu próbował się zakryć, zanim wskoczył do wody.”
Piotr jako „γυμνός” (nagi lub nie w pełni ubrany) to rodzaj szczegółu, który kobieta prędzej zapamiętałaby wyraziście i wplotła w swoją relację, zwłaszcza jeśli wspominała historie z młodości.
Brzmi to mniej jak teologia, bardziej jak pamięć.

I wreszcie, zmartwychwstanie Jezusa opisane jest przez wzmiankę o całunie pozostawionym w grobie, z domyślną sugestią, że Jezus był nagi, kiedy spotkał Marię Magdalenę.
Przedstawienie Jezusa jako nagiego, spotykającego kobietę, byłoby bardzo adekwatnym zakończeniem dla kobiecej autorki o wrażliwości romansowej, ale miałoby niewielki sens w jakimkolwiek innym scenariuszu.

\subsection{Argument z milczenia cielesnego: brak opisów fizycznych}\label{subsec:the-lack-of-description-of-physical-appearance-of-jesus-in-john-seems-to-contradict-the-idea-that-the-author-was-an-eyewitness-a-woman-and-a-lover-of-jesus.}

Jan nie zawiera żadnych opisów fizycznego wyglądu kogokolwiek.
Nie ma opisu Jezusa.
Nie ma umiłowanego ucznia.
Nie ma Piotra.
Nie ma Marii Magdaleny.
Nie ma Nikodema.
Nie ma Samarytanki.
Nie ma Łazarza.
Nie ma Piłata.
Nie ma Judasza.
Nie ma Jana Chrzciciela.

W całej Ewangelii nie pojawia się ani jeden portret, ani jeden przymiotnik opisujący urodę, ani jedna cecha cielesna.
Ani „przystojny”.
Ani „młody”.
Ani „silny”.
Ani „wysoki”.
Ani „promienny”.
Ani „jasny”.
Ani „kształtny”.
Nic.

To milczenie jest nie do pogodzenia z autorstwem męskim motywowanym homoerotycznie.
We wszystkich starożytnych tekstach autorstwa mężczyzn lub zakodowanych męskim pragnieniem homoerotycznym znajdujemy: język piękna (κάλλος), centralny w gejowskiej literaturze greckiej (Platon, Ksenofont, Anakreont, Teognis); język spojrzenia („spojrzałem”, „pragnąłem”, „podziwiałem jego postać”); język młodości („młodzieniec”, „chłopiec”, „piękny młodzieniec”); idealizowane ciała; erotyzowany dotyk.
Jan nie ma nic z tego.

Umiłowany uczeń spoczywający „na piersi” Jezusa nie jest erotyzowany, nie jest opisany fizycznie, nie ma żadnych cielesnych szczegółów.
Scena zakodowana jest teologicznie, a nie fizycznie.
Nagość (γυμνός) Piotra w J 21 jest komiczna, wstydliwa, pospieszna --- bez szczegółu fizycznego, bez estetycznego zachwytu, bez erotycznego podtekstu.
Czyta się to jak wspomnienie kobiety o mężczyźnie, który się skompromitował, a nie jak narracja gejowska o spotkaniu z nagim męskim ciałem.

To milczenie jest natomiast dokładnie tym, czego należałoby się spodziewać po autorstwie kobiecym.
Autorki i kobiece strażniczki pamięci w starożytności często unikają opisywania męskich ciał --- nie dlatego, że nie potrafią, lecz z powodu konwencji skromności, pamięci relacyjnej raczej niż wizualnej, skupienia na emocjach, a nie na formie, oraz uwagi skierowanej ku sferze domowej, a nie erotycznej.
Opisują szczegół zmysłowy, domowy, emocjonalny: zapach perfum, zapach rozkładu, łzy, domy, ogrody, prywatne pokoje.
Dokładnie to robi Jan.

Rozstrzygającą wskazówką jest sam Jezus.
Gdyby jakikolwiek starożytny autor --- zwłaszcza emocjonalnie związany z innym mężczyzną --- opisywał fizycznie Jezusa, którego kochał, prędzej czy później powiedziałby coś o jego wyglądzie.
Żadnego wzrostu.
Żadnej twarzy.
Żadnej brody.
Żadnych oczu.
Żadnych włosów.
Żadnej cery.
Żadnej urody.
Żadnej młodości.
Żadnej siły.
Żadnych blizn.
Nic.

Męski autor homoseksualny --- zwłaszcza rzekomo zakochany w Jezusie --- nigdy nie napisałby biografii ukochanego bez żadnego opisu jego wyglądu.
Kobieta wspominająca człowieka, którego kochała, może natomiast celowo unikać opisywania go w kategoriach wizualnych, a zarazem opowiadać o rzeczywistości emocjonalnej i relacyjnej, używając jego głosu, obecności i bliskości, ale nie jego ciała.
Dokładnie taki wzorzec widzimy w kobiecej literaturze żałobnej, lamentacjach wdów, osobistych listach kobiet w starożytności i w kobieco autorowanej hagiografii.
Kobiety konsekwentnie opisują to, co ktoś robił lub znaczył, a nie to, jak wyglądał.

Częściowym wyjaśnieniem jest fakt, że Filon z Aleksandrii, od którego autorka lub autor Jana wydaje się zapożyczać wiele idei, również unika opisywania wyglądu fizycznego.
To jednak tylko wzmacnia argument za autorstwem kobiecym:
męski autor o skłonności homoerotycznej przejąłby jednocześnie teologię filonową i zachował męski, estetyzujący „spojrzeniowy” język.
Jan zachowuje to pierwsze i całkowicie pozbawiony jest tego drugiego.

Ewangelia Jana czyta się jak kobieca pamięć o Jezusie, a nie jak męska, erotyczna admiracja.

\subsection{Jan pomija ustanowienie Eucharystii.}\label{subsec:john-leaves-out-the-institution-of-the-eucharist.}

Jan pomija ustanowienie Eucharystii.

Ostatnia Wieczerza jest kluczowym momentem w ewangeliach synoptycznych, kiedy Jezus łamie chleb i podaje kielich uczniom, ustanawiając Eucharystię.
U Jana ten moment zostaje zastąpiony sceną umycia nóg (J 13,1–17), która podkreśla służbę i pokorę, a nie sakramentalny akt komunii.

W kontekście Jezusa jako obserwatora tradycji religijnych świata greckiego Eucharystia jawi się jako ryt już zastany.
W przeciwieństwie do uczt greckich żydowski posiłek paschalny koncentrowałby się na baranku, a obrzędy chleba i wina nie byłyby centrum całego posiłku.
Co szczególnie istotne, szeroko rozpowszechnione w tym czasie uczty dionizyjskie były oparte właśnie na rytuałach chleba i wina.
Podczas gdy picie metaforycznej krwi i spożywanie metaforycznego ciała Dionizosa było praktyką szeroko znaną, picie krwi i spożywanie ciała w żydowskiej uczcie byłoby uznane za rażące bluźnierstwo.

Nawet jeśli odłożymy na bok ewangelie synoptyczne i List do Koryntian, Didache, bardzo wczesny tekst chrześcijański, również wspomina Eucharystię jako tradycję już dobrze ugruntowaną.

Jeśli Eucharystia była solidnie ugruntowaną tradycją i już w końcu I wieku stała się centrum wiary chrześcijańskiej, to jak można wyjaśnić fakt, że Jan miałby ją „przeoczyć”?

Warto też zauważyć, że Jan opisuje, jak Jezus mówi o ciele i krwi w kontekście Logosu, a więc w zmodernizowanej formie uczty grecko–rzymskiej.
Jest to dokładnie ujęcie, jakie zaakceptowałby Filon, i bardzo prawdopodobne, że autorka lub autor Jana starał się przekazać tę samą myśl.

Warto odnotować, że Justyn Męczennik i Tertulian oskarżają wyznawców Mitry o kopiowanie Eucharystii od chrześcijan.
Jest to jednak silna przesłanka, że podobny rytuał był już rozpowszechniony w świecie greckim przed Jezusem.

\subsection{Dokładność topograficzna Jana potwierdza wczesne świadectwo naocznego świadka.}\label{subsec:johns-topographical-accuracy-confirms-an-early-eyewitness.}

Dokładność topograficzna Jana potwierdza wczesne świadectwo naocznego świadka.

Ewangelia Jana opisuje konkretne miejsca w Jerozolimie, takie jak sadzawka Betesda z pięcioma krużgankami, kamienny dziedziniec Gabbata czy dolina Cedronu.
Szczegóły te zostały potwierdzone dopiero przez nowoczesną archeologię i nie byłyby dostępne pisarzowi z II wieku, opierającemu się wyłącznie na wyobraźni.
To mocna wskazówka, że Jan odzwierciedla bezpośrednią pamięć lub świadectwo kogoś obecnego w mieście przed jego zniszczeniem w 70 roku po Chr.

\subsection{Semickie idiomy zachowane u Jana wskazują na wczesne pochodzenie.}\label{subsec:semitic-idioms-preserved-in-john-indicate-early-origin.}

Semickie idiomy zachowane u Jana wskazują na wczesne pochodzenie.

Greka Jana zawiera niepodważalne semickie zwroty, a nawet nieprzetłumaczone arameizmy, takie jak ῥαββουνί (J 20,16), Μεσσίας (Mesjasz, natychmiast objaśniony jako Christos w J 1,41 i 4,25) oraz Γολγοθᾶ (Golgota, J 19,17).
Nie są to późniejsze formuły kościelne, lecz surowe transliteracje mowy, zachowane nawet tam, gdzie autorka lub autor natychmiast tłumaczy je dla greckich odbiorców.
Inne zwroty, takie jak „Zaprawdę, zaprawdę, powiadam wam” (ἀμὴν ἀμὴν λέγω ὑμῖν), powielają semicki styl podwójnego potwierdzenia, który w grece brzmi niezgrabnie, ale w aramejskim czy hebrajskim jest całkowicie naturalny.
Wyrażenie „woda żywa” (ὕδωρ ζῶν, J 4,10) odzwierciedla semickie idiomy (\texthebrew{יִם חַיִּים}, \textit{majim chajim}) z Biblii Hebrajskiej.
Gęstość tych semityzmów jest zbyt duża, by wyjaśnić ją jako późne „stylizowanie” przez greckiego autora kilka pokoleń później;
wskazuje raczej na kogoś, kto myślał po aramejsku lub hebrajsku, a dopiero wtórnie wyrażał się po grecku.

Ma to daleko idące konsekwencje dla kwestii autorstwa.
Jeśli umiłowany uczeń nie był jednym z Dwunastu, ale kobietą z kręgu Jezusa, to zachowanie tak bezpośrednich idiomów sugeruje, że do tekstu dotarł jej własny głos lub pamięć.
Niektóre wczesne tradycje łączą Marię, Matkę Jezusa, z refleksjami i powiedzeniami cenionymi w środowisku janowym.
Jeśli ona lub inna Palestynka ukształtowała materiał źródłowy, wyjaśniałoby to, dlaczego tylko Jan zachowuje tak uporczywie semickie wzorce mowy, podczas gdy synoptycy wygładzają je w bardziej płynną grekę.
Być może w Janie słyszymy więc nie „stylizowaną, późną teologię”, lecz echo prawdziwych głosów najwcześniejszych towarzyszy Jezusa --- być może nawet samej Świętej Marii.

\subsection{Wczesna liturgia odzwierciedla wpływ tradycji janowej.}\label{subsec:early-liturgy-reflects-johannine-influence.}

Wczesna liturgia odzwierciedla wpływ tradycji janowej.

Didache, datowana na późny I lub bardzo wczesny II wiek, zawiera modlitwy eucharystyczne, które brzmią bardziej w tonie janowym --- z motywami życia, jedności i światła --- niż w stylu synoptycznych opisów Ostatniej Wieczerzy.
Pokazuje to, że teologia janowa już bardzo wcześnie kształtowała praktyki liturgiczne, podważając twierdzenie, że Jan jest późną innowacją.

\subsection{Brak wyraźnego proroctwa o 70 roku po Chr. u Jana.}

\label{subsec:absence-of-explicit-70ad-prophecy-in-john.}

Gdyby Ewangelia Jana powstała po upadku Jerozolimy, należałoby się spodziewać mocno wyeksponowanego motywu spełnionej zapowiedzi prorockiej.
Tymczasem janowe odniesienie do świątyni („zburzcie tę świątynię, a Ja w trzech dniach ją wznoszę”) ujęte jest w kategoriach symbolicznych, chrystologicznych, a nie jako bezpośrednia aluzja do roku 70 po Chr.
Milczenie na temat oblężenia rzymskiego jest mocnym argumentem za wcześniejszym horyzontem powstania dzieła.

\subsection{Wrodzy świadkowie wcześnie zaakceptowali autorytet Jana.}\label{subsec:hostile-witnesses-accepted-johns-authority-early.}

Już w połowie II wieku krytycy tacy jak Celsus czy polemiści żydowscy podejmują wątki janowe, nie twierdząc jednak, że Ewangelia jest świeżą, dopiero co wymyśloną historią.
Gdyby Jan pojawił się późno, jego nowość stałaby się celem ataków.
Tymczasem jego autorytet wydaje się już utrwalony, co jest zgodne z obrazem tekstu krążącego w obiegu od kilku wcześniejszych dziesięcioleci.

\subsection{Polikarp i Ireneusz zachowują ciągłość.}\label{subsec:polycarp-and-irenaeus-preserve-continuity.}

Polikarp, działający około 90 roku po Chr., znał środowisko Jana, a Ireneusz wprost poświadcza janowe autorstwo.
Tak bliskie pokrewieństwo pokoleń sprawia, że mało wiarygodna staje się hipoteza późnego spreparowania Ewangelii, która mimo to miałaby zostać powszechnie przyjęta jako apostolska.
Sama linia recepcji wspiera zatem tezę o wczesnym powstaniu.

\subsection{Paweł już miał wysoką chrystologię.}\label{subsec:paul-already-had-high-christology.}

Paweł, piszący w latach 50. po Chr., opisuje Jezusa w kategoriach wywyższonych: jako sprawcę stworzenia w 1 Liście do Koryntian 8 oraz jako tego, któremu nadano imię ponad wszelkie imię w Liście do Filipian 2.
Jan nie wymyśla nowej doktryny, lecz narracyjnie oprawia to, co Paweł już głosił.
Dlatego dawne założenie, że „wysoka chrystologia oznacza późną datę”, jest nie do utrzymania.

\subsection{Filon z Aleksandrii już miał wysoką chrystologię.}\label{subsec:philo-of-alexandria-already-had-high-christology.}

Powszechnie uznaje się, choć wciąż konsekwentnie bagatelizuje, że Czwarta Ewangelia i pisma Filona z Aleksandrii oddychają tym samym teologicznym powietrzem.
Filon, aktywny w pierwszej połowie I wieku, usystematyzował doktrynę \emph{Logosu}, która w dużej mierze antycypuje to, co współczesne badania nazywają chrystologią janową.
W jego dziełach Logos nie jest drobną metaforą, lecz centralnym pośrednikiem boskiego panowania, opisanym w całej gamie ról:
jako boski plan i narzędzie stworzenia (\emph{De Opificio Mundi}~20, 146),
„pierworodny syn” i „archanioł” Boga (\emph{De Confusione Linguarum}~146--147),
kosmiczny arcykapłan wstawiający się za stworzeniem (\emph{De Specialibus Legibus}~1.81; 4.123--125),
„obraz Boga” i „więź” spinająca kosmos (\emph{De Somniis}~1.215--239; \emph{De Cherubim}~35--36),
oraz przewodnik i nauczyciel dusz pielgrzymujących (\emph{De Migratione Abrahami}~6).
Konsekwentnie przypisuje Logosowi synostwo, mediację, oświecanie, funkcję kapłańską i zamieszkującą obecność --- dokładnie te kategorie, którymi Jan posługuje się w Prologu i w całej Ewangelii.

Paralele są konkretne i niepodważalne.
J 1,3 ogłasza, że „wszystko przez Nie się stało”, co bezpośrednio odzwierciedla naukę Filona o Logosie jako narzędziu i planie stworzenia.
Janowy opis Logosu jako „Jednorodzonego Syna” (1,14.18) koresponduje z filonowym określeniem Logosu jako „pierworodnego syna” Boga.
Janowe stwierdzenie, że „Boga nikt nigdy nie widział, lecz Jednorodzony Syn Go objaśnił” (1,18), powtarza przekonanie Filona, iż Logos jest pomostem między niewidzialnym Bogiem a widzialnym światem.
Dualizm światła i ciemności w J 1,4--5 odpowiada platońskiej terminologii Filona, przeciwstawiającej światło umysłowe materialnej niewiedzy.
Gdy Jan mówi, że Logos „rozbił namiot” (zamieszkał) pośród nas (1,14), powtarza filonową ideę Logosu jako obecności Boga zamieszkującej kosmos jak świątynię.
Nawet Prologowe zestawienie „łaska i prawda” (1,14.17) przypomina częste łączenie przez Filona Logosu z Bożym miłosierdziem i prawdą.
Późniejsze wypowiedzi janowe („Ja jestem drogą, prawdą i życiem”, 14,6) współbrzmią z obrazem Logosu jako przewodnika dusz.
Krótko mówiąc, niemal każdy kluczowy motyw Prologu janowego ma swój filonowy odpowiednik.

Współcześni badacze długo zmagali się z tym, jak opisać tę relację.
C.~H.~Dodd już dawno twierdził, że Ewangelia Jana szeroko czerpie z doktryny Logosu judaizmu hellenistycznego, którego najjaśniejszym świadkiem jest Filon.
Harold Attridge trafnie określił Jana i Filona jako „dwie wariacje na temat jednego Logosu”: dzielą ten sam schemat pojęciowy, przy czym wyjątkową nutą Jana jest wcielenie.
Peder Borgen najpełniej zbadał paralele, pokazując, że Jan i Filon w podobny sposób odczytują tradycje Wyjścia --- jak manna i Namiot Spotkania --- przez pryzmat Logosu, co czyni ich zbieżność zbyt specyficzną, by była przypadkiem.
Raymond E.~Brown, choć ostrożniejszy, przyznawał, że Logos Filona stanowi jedno z tła, obok żydowskich tradycji \emph{Memra} i Mądrości.
David Runia udokumentował, jak późniejsi chrześcijanie swobodnie sięgali po Filona, jednocześnie podkreślając, że bezpośredniej zależności Jana od Filona nie da się udowodnić ponad wszelką wątpliwość.
Daniel Boyarin zaproponował ujęcie zorientowane na judaizm: Prolog Jana da się zrozumieć jako midrasz do „Memry” w ramach binitarnego monoteizmu judaizmu drugiej świątyni, z Filonem lub bez niego.
Właśnie to potwierdza wagę Filona, ponieważ jego pisma pokazują, że taka „wysoka chrystologia” już istniała w judaizmie aleksandryjskim.
Marian Hillar doprowadził tę linię do końca: według niego filonowa teologia Logosu jest sama w sobie chrystologiczna, antycypując kategorie, które Jan później zastosuje do Jezusa.

Suma tych obserwacji jest uderzająca.
To, co Jan ogłasza w Prologu --- że boski Logos, od dawna wyznawany jako Syn, Światłość, Pośrednik i Arcykapłan, teraz przyjął ciało --- nie jest nagłą innowacją.
Jest krystalizacją teologii judaizmu aleksandryjskiego, już wcześniej wyartykułowanej przez Filona.
Wysoka chrystologia nie czekała na zmartwychwstanie; była żywa w greckim judaizmie jeszcze przed ziemskim życiem Jezusa.

\subsection{Gatunek Ewangelii Jana odpowiada wczesnej biografii grecko-rzymskiej.}\label{subsec:johns-genre-fits-early-greco-roman-biography.}

Gatunek Ewangelii Jana odpowiada wczesnej biografii grecko–rzymskiej (\emph{bios}): starannie dobrane znaki publiczne, rozbudowane mowy, kulminacyjny proces i śmierć oraz koncentracja na legitymizacji władzy.
Wpisuje się to w pierwszowieczne wzorce biografii królewskiej.
Dla kontrastu, apokryfy II wieku przyjmują styl ezoteryczny lub czysto „sentencjonalny”, który wygląda na późniejszy.
Jan należy więc do wcześniejszego nurtu biograficznego.

\subsection{Podsumowanie i wnioski.}\label{subsec:summary-and-consequence.}

Zebrane razem --- P52 i P66, dokładność archeologiczna, semickie idiomy, echa liturgiczne, milczenie o roku 70, wrogo i życzliwie nastawieni odbiorcy, ciągłość przez Polikarpa, zgodność z Pawłem, gatunek \emph{bios}, synoptyczna faktura narracji i zależność apokryfów --- wszystko to zbiega się w jednym wniosku.
Ewangelia Jana nie jest najpóźniejsza, lecz jedna z najwcześniejszych.
Synoptycy również należą raczej do pierwszego pokolenia niż do okresu po roku 70.
Ewangelie funkcjonują jako królewskie biografie pisane szybko i równie szybko rozpowszechniane, nie jako teologiczne fikcje powstające wieki później.
Co najważniejsze, janowy świadek naoczny najprawdopodobniej jest tożsamy z Marią Magdaleną --- czwartym filarem obok Piotra, Jakuba i Pawła --- przy czym Joanna pozostaje wiarygodnym kandydatem na pośredniczkę czy przekazicielkę tradycji, ale nie na głównego świadka.

\subsubsection{Przesłanie}\label{subsubsec:the-message}

Na koniec tradycyjne założenie, że Jan był mężczyzną.
Musimy pamiętać, że większość badaczy zgadza się, iż we wczesnych wspólnotach chrześcijańskich kobiety miały bardzo silny wpływ.
Postacie takie jak Maria Magdalena, Tekla czy Pryscylla stały na czele głoszenia i patronatu.
Kiedy ewangelie trafiły w kanały literatury rzymskiej, teksty przepisano pod oczekiwania patriarchalnego audytorium.
Kobiece przywództwo zniknęło z pamięci, Maria Magdalena została zakodowana jako prostytutka, a głosy popawłowe posłużyły do uciszenia kobiet.
W tym kontekście ewangelia spisana przez Marię Magdalenę lub Joannę w przewidywalny sposób krążyłaby pod męskim tytułem „Jan”.
Brak imiennie wskazanego „Umiłowanego Ucznia” w Ewangeliach synoptycznych nie stanowi więc problemu, lecz podpowiedź, że umiłowany świadek należy do grupy kobiet, które one jednak wymieniają z imienia.

To, co Czwarta Ewangelia nazywa swoimi „znakami” i „chwałą”, pełni funkcję królewskiej legitymizacji oraz wejścia w \emph{nauczanie tajemne}.
Janowe słownictwo światła i ciemności, tego co „z góry” i „z dołu”, widzenia i poznania, trwania i ponownych narodzin czyta się jak katechezę dla wewnętrznego kręgu.
Jezus rozmawia nocą na osobności z Nikodemem, przy studni z Samarytanką, a Mowy Pożegnalne zarezerwowane są dla grona najbliższych, którym obiecuje Parakleta, który „wszystkiego ich nauczy”.
To dokładnie wzorzec \emph{mysteriów}: publiczne głoszenie dla wszystkich i głębsze pouczenie dla wtajemniczonych.

Egipt jest miejscem, gdzie to głębsze nauczanie przetrwało najbardziej widocznie.
Zbiory z Oksyrynchos zachowują fragmenty janowe i teksty pokrewne obok papirusów dokumentujących wczesny horyzont czasowy.
Biblioteka z Nag Hammadi koło Chenoboskion przechowała \emph{Ewangelię Tomasza}, \emph{Ewangelię Filipa}, \emph{Apokryf Jana} i pokrewne traktaty, których słownictwo nakłada się na janowe motywy światła, prawdy, poznania i narodzin na nowo.
Maria Magdalena pojawia się w \emph{Ewangelii Marii} jako autoryzowana interpretatorka, przewyższająca Piotra --- dokładnie tak, jak sugeruje to rywalizacja wokół Umiłowanego Ucznia w Janie.
Te egipskie świadectwa później wrzucono do jednego worka „herezji” pod etykietą „gnostycyzmu” nie dlatego, że były późnymi falsyfikatami, lecz dlatego, że przechowały nurty \emph{nauczania ezoterycznego}, biegnące równolegle do coraz bardziej rzymskiego, klerykalizującego się kanonu.

Aleksandryjska kultura intelektualna wyjaśnia Logos Jana i jego filozoficzny rejestr.
Metoda alegoryczna Filona i jego język Logosu dostarczają miejskiego idiomu, w którym świadectwo Marii mogło zostać opracowane dla czytelnika greckiego.
Ta sama miejska praktyka transmisji szkolnej tłumaczy, dlaczego teksty janowe czytają się jak podręcznik zaawansowanego nauczania, a nie prosta kronika.

Stąd przesłanie Jana jest podwójne: królewskie i ezoteryczne.
To historia miłości Marii do swojego króla i katechizm dworu szkolącego wysłanników.
To, co późniejsi biskupi zdyskredytowali jako „gnostycyzm”, lepiej rozumieć jako \emph{ezoteryczny poziom} tego samego królewskiego Ewangelium, które na poziomie jawnym było głoszone na ulicach.
Czytane w ten sposób, przypisanie Ewangelii „Janowi” nie miało zanegować kobiety stojącej za tekstem, lecz osłonić jej głos, tak aby jej \emph{gnosis} mogła nadal żywić Kościoły od Egiptu po skraj greckiego świata.

\subsubsection{Cuda w Ewangelii Jana}\label{par:miracles-in-john}

W Ewangelii Jana jest siedem cudów, nazywanych często „znakami”, ponieważ wskazują na tożsamość Jezusa i Jego misję.
Warto podkreślić, że nie są to przypadkowe popisy magiczne, lecz wszystkie przekazują królewską władzę Jezusa.
Możliwe też, że cuda nie są czystą metaforą.
Jezus jako władca wysoce wykształcony mógł być rzeczywiście biegłym uzdrowicielem, albo przynajmniej tak był postrzegany, przy wzmocnieniu przez efekt potwierdzenia.
Nietrudno uwierzyć, że jeśli Jezus odwiedził chorego i ten wyzdrowiał choćby z samego efektu placebo, to osoba pisząca jego królewską biografię bardzo chętnie by o tym opowiedziała.
Podobnie, jeśli Jezus dysponował zasobami, by nakarmić głodnych, a okazał przy tym królewską hojność, autor biografii chętnie wyeksponowałby takie wydarzenie.

Siedem znaków w Ewangelii Jana to:

\paragraph{1. Woda w wino (J 2,1-11)} Przemiana wody w wino oznacza nadejście nowego przymierza, podobnie jak królewskie uczty potwierdzały legitymizację królów.

\paragraph{2. Uzdrowienie syna urzędnika (J 4,46-54)} Uzdrowienie syna urzędnika pokazuje władzę nad życiem i chorobą, które w świecie starożytnym pozostawały w gestii władzy bosko–królewskiej.

\paragraph{3. Uzdrowienie paralityka (J 5,1-9)} Uzdrowienie przy sadzawce Betesda ukazuje władzę przywracania tego, co złamane, królewskie prawo naprawy tego, co społeczeństwo porzuciło.

\paragraph{4. Nakarmienie 5 000 (J 6,1-14)} Nakarmienie pięciu tysięcy odbija obraz władcy troszczącego się o swój lud, bliski idei boskiego królowania w tradycji hellenistycznej i żydowskiej.

\paragraph{5. Chodzenie po wodzie (J 6,16-21)} Chodzenie po wodzie przywołuje panowanie nad żywiołami, znak imperialnej i boskiej władzy zarówno w myśli greckiej, jak i żydowskiej.

\paragraph{6. Uzdrowienie niewidomego od urodzenia (J 9,1-12)} Przywrócenie wzroku temu, który był niewidomy od urodzenia, ukazuje władzę nad losem wrodzonym, zdolność odwrócenia tego, co uchodziło za nieodwracalne od chwili narodzin.

\paragraph{7. Wskrzeszenie Łazarza (J 11,1-44)} Uzdrowienie niewidomego i wskrzeszenie Łazarza podkreślają władzę nad samym losem --- kwestionując nawet najbardziej niezmienne warunki ludzkiej egzystencji.

\subsection{Brak cudu przy śmierci Jezusa.}\label{subsec:the-non-miracle-at-the-death-of-jesus.}

Zwraca też uwagę fakt, że Ewangelia Jana jako jedyna nie wspomina cudów towarzyszących śmierci Jezusa.
Jako Ewangelia napisana przez świadka naocznego, bardzo prawdopodobne, że autorka lub autor byli obecni przy ukrzyżowaniu Jezusa i mogliby widzieć rzekome cuda z tym związane.

\subsection{Zakończenie}\label{subsec:conclusion}

Brak wzmianki o cudach przy śmierci Jezusa w Ewangelii Jana jest mocną wskazówką, że autorka lub autor faktycznie próbowali zapisać rzetelną relację z wydarzeń, a nie tylko stworzyć traktat teologiczny.
Bo jeśli wprowadza się cud „na potrzeby teologii”, to byłoby to najoczywistsze miejsce, aby to zrobić.
Zatem choć siedem cudów nie musi być w pełni historycznie dokładnych, nie ma przymusowego powodu, by uznawać je za coś więcej niż lekko wyolbrzymione wersje prawdziwych wydarzeń --- momentów, w których Jezus okazał swoje talenty przywódcze, znajomość medycyny i hojność --- po prostu ozdobione tak, jak oczekiwano od królewskiej biografii, zwłaszcza jeśli autorka była w Jezusie zakochana.

Warto naprawdę rozważyć, że wydarzenia siedmiu znaków zostały opisane przez Umiłowanego Ucznia tak, jak Jezus je przed nią lub przed nim przedstawiał.
Albowiem, powiedzmy szczerze: niech pierwszy rzuci kamieniem ten, kto nigdy nie wyolbrzymiał swoich męskich dokonań przed osobą, w której był zakochany.

\subsection{Argument z Ewangelii Jana}

\label{subsec:the-gospel-of-john-point}

Kiedy rozważamy historycznego Jezusa i historycznego Jana, wciąż powraca jedna myśl.

Niemal każda dyskusja o Ewangelii Jana dotyczy jej teologii.
Spieramy się o doktrynę, słownictwo i historyczną wiarygodność, ale być może w ten sposób umyka nam to, co w tekście jest najważniejsze.

Utraciliśmy autorkę lub autora.
Utraciliśmy wspólnotę, która jako pierwsza niosła tę Ewangelię.
Utraciliśmy Jezusa, którego oni znali, oraz wiarę pierwszego Kościoła.
Przetrwały kopie kopii kopii, kształtowane przez stulecia zmieniającej się teologii i kultur.

A jednak być może najlepszym świadectwem rzeczywistego autorstwa Ewangelii Jana nie jest żaden pojedynczy twardy dowód, lecz samo ogólne przesłanie Ewangelii.
Co, jeśli przesłanie miłości jest w niej tak silne i tak głębokie, że stało się korzeniem chrześcijaństwa?
Być może, kiedy miliardy chrześcijan czytają Ewangelię Jana, nie zdając sobie z tego sprawy, podświadomie odczuwają miłość, jaką autorka lub autor żywił wobec Jezusa.
Być może starożytny list o utraconej miłości stał się najsilniejszym i najbardziej trwałym przesłaniem o miłości w dziejach ludzkości.
Być może prawdą jest, że miłość jest jedyną rzeczą, którą potrafimy postrzegać jako przekraczającą wymiary czasu i przestrzeni.

\section{Ewangelie synoptyczne}\label{sec:synoptic-gospels}

\subsection{Herod umiera w 10 roku po Chr., a nie w 4 roku przed Chr., a Ewangelie Mateusza i Łukasza nie są zgodne co do daty narodzenia Jezusa.}\label{subsec:herod-dies-in-10ad-not-4bc-and-the-gospels-of-matthew-and-luke-are-not-in-agreement-on-the-date-of-jesuss-birth.}

Istnieje pewna trudność z datowaniem narodzin Jezusa na 7 rok przed Chr., co jest powszechną datą używaną we współczesnej biblistyce.

Mamy pozorną sprzeczność między Ewangeliami Mateusza i Łukasza: Ewangelia Mateusza stwierdza, że Jezus urodził się za panowania Heroda Wielkiego, który zmarł w 4 roku przed Chr., natomiast Ewangelia Łukasza umieszcza narodziny Jezusa w czasie rzymskiego spisu ludności przeprowadzonego za Kwiryniusza, namiestnika Syrii w latach 6–12 po Chr.
Rzymskie spisy rejestrowały ludność i majątek na potrzeby podatków; były w Judei głęboko znienawidzone, ponieważ oznaczały bezpośrednią kontrolę rzymską i wywołały przynajmniej jedno zbrojne powstanie.

Nawet jeśli datujemy Ewangelie bardzo późno, na koniec II wieku, trudno wyjaśnić, jak autorzy mogli mieć tak dobrą wiedzę o wielu szczegółach tła historycznego, a zarazem tak bardzo się mylić w tak kluczowym punkcie historii.
Poza samymi narodzinami, Łukasz musiałby się mylić co do wieku Jezusa w chwili chrztu, co również jest elementem absolutnie centralnym.
Dodając do tego, w ramach niniejszej książki pojawia się jeszcze dodatkowa rozbieżność: pielgrzymka Kacpra, Melchiora i Baltazara w 7 roku przed Chr. nie miałaby sensu, ponieważ tron grecki był wciąż zajmowany przez Strato III, a to królestwo wcale nie stało jeszcze na krawędzi upadku.
Uczeni głównego nurtu odrzucają sprzeczności między Mateuszem i Łukaszem jako dowód zmyślenia, ale korekta zaćmienia na rok 12 po Chr. pokazuje, że obie tradycje da się zharmonizować --- co oznacza, że obie mogą zachowywać realną pamięć, a nie jedynie fikcję.

Znacznie przecenia się to, co naprawdę wiemy historycznie o tej dacie.
Wiemy, że Kwiryniusz był namiestnikiem Syrii od 6 do 12 roku po Chr.
Warto mocno podkreślić, że fakt, iż Kwiryniusz był namiestnikiem Syrii, nie oznacza, że Judea została włączona do prowincji rzymskiej następnego dnia po jego nominacji.
Nie sprzeciwia się żadnym znanym świadectwom historycznym założenie, że Judea została formalnie zaanektowana dopiero tuż przed opuszczeniem przez niego urzędu w 12 roku po Chr.
Wiemy, że spis miał miejsce w czasie jego namiestnictwa, ale nie wiemy, kiedy dokładnie.
Józef Flawiusz nie twierdzi, że spis odbył się po śmierci Heroda Wielkiego.
Wiemy, że Józef umieszcza śmierć Heroda Wielkiego przy zaćmieniu Księżyca tuż przed Paschą.
Zwraca przy tym wyraźnie uwagę, że Herod zmarł po długiej chorobie, co dobrze współgra z możliwością, że przez kilka ostatnich lat władza była faktycznie przesunięta na jego syna Archelaosa.

I tu pojawia się już poważny błąd biblistyki głównego nurtu.
Fragment Józefa, na którym opiera się datowanie śmierci Heroda Wielkiego, stwierdza, że „zaćmienie Księżyca, które zaszło przed jego śmiercią, nastąpiło dziesiątego dnia miesiąca Lous, czyli Nisan”.
Tymczasem mało kto przyznaje, że w latach 7 przed Chr. do 12 po Chr. miało miejsce 51 zaćmień Księżyca.
Spośród nich trzy zaszły nocą w miesiącu Nisan i były na tyle wyraźne, że ludzie mogli je obserwować.
Konsensus uczonych głosi, że Józef odnosi się tu do zaćmienia z 4 roku przed Chr., jednak przy bliższym przyjrzeniu się okazuje się, że w 4 roku przed Chr. zaćmienie nastąpiło w Adar, cały miesiąc wcześniej.

Jedyną możliwą datą zaćmienia Księżyca, które zaszło blisko 10 Nisan między 16:00 a 4:00 czasu UTC, a więc mogło być widoczne dla mieszkańców Jerozolimy, jest: 0012 Apr 24--22:38:24 10410 -24585 41 N -a 1.4693 0.1335 -0.8101 96.4 - - 11S 64E

Warto zauważyć, że zaćmienia Księżyca zawsze mają miejsce w pełnię, a 15 Nisan to pełnia miesiąca Nisan.
Ponieważ jednak kalendarze księżycowe nie są zsynchronizowane z kalendarzem słonecznym, kalendarz żydowski w latach przestępnych wprowadza dodatkowy miesiąc Adar tuż przed Nisan.
Tak się składa, że w roku 12 po Chr. dodatkowy Adar przesunął 15 Nisan o kilka dni później, co tłumaczy, dlaczego Józef mógł powiedzieć, iż zaćmienie nastąpiło 10 Nisan.
We wszystkich pozostałych możliwych terminach zaćmienie następowało w odległości jednego dnia od 15 Nisan.

To przesunięcie porządkuje wiele problemów związanych z datowaniem narodzin Jezusa i spisu ludności.
I wówczas Ewangelie pozostają w pełnej zgodzie ze sobą.
Kwiryniusz mógł przeprowadzić swój spis już w 6 roku po Chr., Jezus mógł urodzić się również w 6 roku po Chr.
Byłoby dość czasu, aby w pierwszych dwóch latach życia Jezusa spędzić pewien okres w Jerozolimie, nawiedzić świątynię i dokonać obrzezania.
Mędrcy mogliby przybyć z wizytą w 8 roku po Chr., gdy Jezus był jeszcze bardzo małym dzieckiem, dokładnie w momencie upadku królestwa grecko–indyjskiego.
Herod miałby czas, by w 8 roku po Chr. wydać rozkaz rzezi niewiniątek, a Jezus mógłby uciec do Egiptu i pozostać tam przez kolejne cztery lata.
Następnie rodzina mogłaby osiedlić się w Nazarecie, gdy Herod Antypas objął tron w Judei.

Przy takim nowym datowaniu pojawiają się oczywiście pewne pozorne sprzeczności ze znanymi zapisami.
Najbardziej rzuca się w oczy tradycyjne datowanie panowania Heroda Archelaosa, syna Heroda Wielkiego, który miał rządzić Judeą od 4 roku przed Chr. do 6 roku po Chr.
Warto jednak zauważyć, że Herod Wielki był tetrarchą Judei, a nie królem, a tytuł ten obejmował Judeę właściwą, Galileę, Pereę, Idumeę, Samarię oraz inne obszary przylegające do Nabatei i Dekapolu.

Archelaos otrzymał wówczas tytuł etnarchy, czyli „króla Żydów”, a nie władcy całego terytorium.
Józef Flawiusz bardzo wyraźnie stwierdza również, że Archelaos sprawował rządy jeszcze za życia Heroda Wielkiego.

\begin{center}
\begin{tabular}{@{}lll@{}}
\toprule
\textbf{Dziedzic} & \textbf{Terytorium} & \textbf{Tytuł} \\
\midrule
Archelaos & Judea, Samaria, Idumea & Etnarcha (nie król) \\
Herod Antypas & Galilea, Perea & Tetrarcha \\
Filip & Gaulanitis, Batanea, Trachonitis, Iturea & Tetrarcha \\
\bottomrule
\end{tabular}
\end{center}

Ważne jest także podkreślenie, że długość urzędowania rzymskich namiestników Judei jest powiązana z czasem panowania Heroda Archelaosa, a znane daty nie są potwierdzone niezależnie innymi źródłami.
Jeśli więc panowanie Heroda Archelaosa zakończyło się w 12 roku po Chr., nadal jest wystarczająco dużo miejsca, aby w wiarygodny sposób umieścić w tej osi czasu Koponiusza, Ambibulusa, Annusza Rufusa i Waleriusza Gratusa, poprzedników Poncjusza Piłata.

\subsection{Głównym argumentem za datowaniem Ewangelii synoptycznych po roku 70 po Chr. jest zburzenie świątyni przez Rzymian.}\label{subsec:the-main-argument-for-the-dating-of-the-synoptic-gospels-to-after-year-70ad-is-the-destruction-of-the-temple-by-the-romans.}

\subsection{Większość uczonych uznaje, że mowa na Górze Oliwnej w Ewangeliach synoptycznych jest proroctwem zburzenia świątyni.}\label{subsec:most-scholars-argue-the-olivet-discourse-in-the-synoptic-gospels-is-a-prophecy-of-the-destruction-of-the-temple.}

\subsection{Jednocześnie analiza tekstualna Marka wskazuje, że te wersety zostały dodane później.}\label{subsec:at-the-same-time-the-textual-analysis-of-mark-shows-that-these-verses-were-added-later.}

który jest uderzająco bardziej formalny i techniczny niż otaczający go język Marka.
W pozostałych częściach Ewangelii Marek ma tendencję do stosowania bardziej bezpośrednich i przystępnych sformułowań nauczania i proroctw Jezusa, często opartych na jasnych, sugestywnych obrazach.
„Ohyda spustoszenia” jest bardzo specyficznym, technicznym terminem, znacznie bliższym językowi proroctw, liturgii czy późniejszych rozważań teologicznych.
To sugeruje, że może chodzić o dodatek redakcyjny, wprowadzony po to, by odzwierciedlić późniejsze troski teologiczne (jak zburzenie świątyni w roku 70 po Chr.) lub by dopasować tekst do znanego schematu starotestamentowego.

\subsection{\texorpdfstring{``Lecz w owe dni, po tym ucisku\ldots{}'' (Mk 13,24)}{4.
``But in those days, after that tribulation\ldots'' (Mark 13:24)}}\label{subsec:but-in-those-days-after-that-tribulation-mark-1324}

Zwrot „po tym ucisku” wprowadza bardziej formalną, uporządkowaną frazę apokaliptyczną, która kontrastuje z bardziej bezpośrednim i natychmiastowym tonem wcześniejszych partii Ewangelii Marka.
Ewangelia Marka zazwyczaj posługuje się prostym, rozmownym językiem, natomiast ta formuła nosi znamiona stylizowanego tonu apokaliptycznego, charakterystycznego dla późniejszych pism prorockich.
Otwiera ona wizję kosmicznego wstrząsu, który jest typowym motywem późniejszej literatury apokaliptycznej (w tym Apokalipsy oraz niektórych żydowskich pism apokaliptycznych), a znacznie rzadziej pojawia się w wcześniejszych partiach Ewangelii.

\subsection{``Gdy więc ujrzycie ohydę spustoszenia, stojącą tam, gdzie nie powinna'' (Mk 13,14)}\label{subsec:when-you-see-the-desolating-sacrilege-set-up-where-it-ought-not-to-be-mark-1314}

Kwestia języka: ten fragment zawiera złożone sformułowania, takie jak „ohyda spustoszenia” i zwrot „stojąca tam, gdzie nie powinna”.
Podczas gdy „ohyda spustoszenia” w starszych proroctwach jest dobrze ugruntowanym terminem, ta wariacja wydaje się bardziej zawiła i odległa od typowego słownictwa nauczania Marka.
Wprowadza dodatkową warstwę złożoności, bliższą późniejszym autorom apokaliptycznym, którzy używali takiego języka, by wskazać rytualne zbezczeszczenie w sposób dość odmienny od prostego, bezpośredniego stylu Marka.

\subsection{``Niebo i ziemia przeminą, ale słowa moje nie przeminą'' (Mk 13,31)}\label{subsec:heaven-and-earth-will-pass-away-but-my-words-will-not-pass-away-mark-1331}

Kwestia języka: motyw wiecznotrwałości słów Jezusa jest znacznie częstszy w literaturze janowej (J 1,1; 6,68; 14,23).
Choć Marek przedstawia Jezusa jako autorytatywnego nauczyciela, to stwierdzenie akcentuje rozwinięty w późniejszej teologii motyw trwałości Jego nauki, który nieco odbiega od bardziej „ludzkiego” profilu Jezusa w Marku.
Zdanie to może także przywoływać tematy spotykane w późniejszych pismach chrześcijańskich, które koncentrują się na wiecznej ważności nauczania Jezusa w sposób trochę odmienny od ziemskiej misji akcentowanej przez Marka.

\subsection{``Lecz o owym dniu lub godzinie nikt nie wie, ani aniołowie w niebie, ani Syn, tylko Ojciec'' (Mk 13,32)}\label{subsec:but-concerning-that-day-or-that-hour-no-one-knows-not-even-the-angels-in-heaven-nor-the-son-but-only-the-father-mark-1332}

Kwestia języka: podkreślenie niewiedzy Syna co do nadejścia Królestwa kontrastuje z wysoką chrystologią, która rozwija się w późniejszych tekstach, takich jak Ewangelia Jana, gdzie Jezus ma pełną wiedzę i autorytet.
Możliwe, że odzwierciedla to markowy nacisk na człowieczeństwo Jezusa, jednak sposób sformułowania wypowiedzi wydaje się lekko dysonujący wobec reszty Ewangelii, gdzie Jezus prezentowany jest jako mający wyraźną władzę nad czasem i znajomość Bożych planów.
Akcent na niewiedzę wybrzmiewa wyraźnie inaczej niż pełen pewności ton późniejszych chrześcijańskich refleksji o Jezusie.

\subsection{``Kto czyta, niech rozumie'' (Mk 13,14)}\label{subsec:let-the-reader-understand-mark-1314}

Już sam ten zwrot jest wskazówką, że mamy do czynienia z wtrętem.
To dokładnie taki rodzaj zdania, jakiego późniejszy skryba mógłby użyć, aby doprecyzować sens lub dostarczyć dodatkowego kontekstu dla fragmentu, który mógł być dla czytelnika niejasny.

\subsection{Pojawia się więc pytanie: jeśli Mk 13 i Mk 16 zostały dodane do znanej nam Ewangelii, to kiedy dokonano tych redakcji?}\label{subsec:now-the-question-is-if-mark-13-and-mark-16-were-added-to-the-gospel-known-to-us-then-when-would-the-edits-have-been-made}

Najpewniej dość szybko po samym wydarzeniu, kiedy było ono jeszcze świeże w pamięci ludzi.
Gdyby poprawki wprowadzono dopiero dziesięciolecia później, w czasie gdy tak wiele kluczowych wydarzeń dla polityki światowej oraz historii żydowskiej i chrześcijańskiej nakładało się na siebie, to inne fakty byłyby w umysłach redaktorów bardziej palące.
Już na tej podstawie można przyjąć górną granicę datowania pierwotnego rękopisu Marka na okres sprzed 70 roku po Chr.

\subsection{Pozostałe argumenty za dolną granicą datowania Ewangelii koncentrują się na rzekomej potrzebie „rozwinięcia” teologii wczesnego Kościoła.}\label{subsec:the-other-arguments-for-the-lower-bound-of-the-dating-of-the-gospels-focus-on-the-need-to-develop-the-theology-of-the-early-church.}

W perspektywie Jezusa jako dziedzica imperium greckiego Ewangelia Marka nie wymagałaby żadnego specjalnego „rozwoju” teologii, ponieważ jest przede wszystkim relacją historyczną o ostatecznym upadku imperium greckiego.
Elementem teologicznym jest w niej zmartwychwstanie Jezusa, którego w oryginalnej wersji Ewangelii Marka nie było.
W związku z tym nie ma większego powodu, by oczekiwać znacznej przerwy czasowej między wydarzeniami historycznymi a powstaniem Ewangelii Marka.

\subsection{Ewangelia Marka zaczyna się od chrztu Jezusa przez Jana Chrzciciela.}

\label{subsec:the-gospel-of-mark-starts-with-the-baptism-of-jesus-by-john-the-baptist.}

Chrzest Jezusa jest w Ewangeliach wydarzeniem kluczowym, ponieważ wyznacza początek Jego panowania jako prawowitego króla imperium greckiego.
Dworscy skrybowie bardzo często rozpoczynali opowieść od początku panowania władcy, i chrzest Jezusa jest najpewniej wydarzeniem, które uznano za początek Jego rządów.
Tradycyjna biblistyka nie potrafi dobrze wyjaśnić, dlaczego historia Jezusa miałaby zaczynać się od chrztu w Jordanie udzielonego przez Jana Chrzciciela, ale w perspektywie Jezusa jako prawowitego dziedzica imperium greckiego ma to całkowicie oczywisty sens.
Scena chrztu jest też przedstawiona jako wydarzenie bardzo publiczne, co odpowiadałoby doniosłemu aktowi, o którym ludzie musieli wiedzieć.
We wszystkich czterech Ewangeliach podczas chrztu Jezusa rozlega się głos z nieba: „To jest mój Syn umiłowany, w którym mam upodobanie” (Mt 3,17; Mk 1,11; Łk 3,22; J 1,34).
Ma to fundamentalne znaczenie, bo w starożytnych tradycjach żydowskich i hellenistycznych ogłoszenie kogoś „synem Bożym” bardzo często miało konotacje królewskie, oznaczało bowiem boską legitymizację lub wyznaczenie do królowania.
Tytuł „Syn Boży” był używany dla królów w pismach hebrajskich (np. 2 Sm 7,14), a w tradycji hellenistycznej dla cesarzy i władców.

Chrzest Jezusa podąża za ustalonym wzorcem królewskiej koronacji w starożytnym świecie śródziemnomorskim.
Psalm 2, królewski psalm intronizacyjny Izraela, dostarcza zasadniczej matrycy: „Tyś moim Synem, Jam Ciebie dziś zrodził” (Ps 2,7).
To formuła wypowiadana przez Boga do nowo ustanowionego króla — nie stwierdzenie biologicznego synostwa, lecz deklaracja politycznej legitymizacji i boskiego ustanowienia.
Gdy więc przy chrzcie Jezusa głos z nieba oznajmia „To jest mój Syn”, przywołuje dokładnie ten sam królewski język.

Tradycyjna interpretacja posługi Jana Chrzciciela opiera się na frazie „chrzest nawrócenia na odpuszczenie grzechów” (μετάνοια εἰς ἄφεσιν ἁμαρτιῶν), występującej w Mk 1,4 i Łk 3,3.
Takie tłumaczenie zaciemnia jednak zakres znaczeniowy greckich terminów i przenosi na realia I wieku judejskiego późniejszą chrześcijańską teologię moralną.
Greckie słowo μετάνοια nie oznacza przede wszystkim „odczuwać żal z powodu win moralnych”, lecz „zmianę myślenia” albo „ponowne rozważenie”, często używane w kontekście politycznym lub strategicznym na opis zmiany stanowiska czy przynależności.
Podobnie ἄφεσις oznacza zasadniczo „uwolnienie” lub „wypuszczenie”, a w Septuagincie jest terminem na jubileuszowe uwolnienie z długów, niewoli i zobowiązań ziemskich (hebr. \texthebrew{דְּרוֹר} \textit{deror}, \texthebrew{שְׁמִטָּה} \textit{szemita}).
Słowo ἁμαρτία, zazwyczaj oddawane jako „grzech”, wywodzi się z czasownika oznaczającego „chybiać celu” albo „błądzić” — \textit{hamartia} u Arystotelesa w tragedii to raczej tragiczny błąd lub pomyłka niż wrodzone moralne zepsucie.

Odczytana w tym leksykalnym kontekście fraza μετάνοια εἰς ἄφεσιν ἁμαρτιῶν może znaczyć „zmianę przynależności dla uwolnienia od dotychczasowych zobowiązań i błędów” — język znacznie lepiej pasujący do politycznego lub rytualnego przejścia niż do prywatnej skruchy moralnej.
Ten odczyt wzmacnia dodatkowo kontekst geograficzny i kulturowy: Jan działał w regionie Zajordania, konkretnie w Perei (por. J 1,28: „Betania po tamtej stronie Jordanu”), terytorium herodiańskim na wschód od Jordanu, silnie przesiąkniętym grecką administracją i kulturą.
Ta „pograniczna”, zhellenizowana strefa była klasycznym miejscem formowania się alternatywnych pretensji i ruchów działających poza bezpośrednią kontrolą jerozolimskiego kapłaństwa.

Sam tekst wspiera takie odczytanie.
W J 3,25 spór, który powstaje na temat chrztu Jana, opisany jest jako περὶ καθαρισμοῦ — „o oczyszczenie”.
Greckie καθαρισμός oznacza obmycie, oczyszczenie rytualne, a nie μετάνοια (nawrócenie) czy ἄφεσις (odpuszczenie).
To język używany na opis rytuału przygotowania przed objęciem świętego urzędu, a nie wyznania win moralnych.
Ewangelia Jana klasyfikuje chrzest jako ryt konsekracji, nie ryt pokuty.

Co istotne, cała Ewangelia Jana całkowicie unika słownictwa związanego z pokutą.
Słowa μετάνοια i μετανοέω („nawracać się”) nie pojawiają się w niej ani razu.
Zamiast tego czwarta Ewangelia konsekwentnie ujmuje ruch Jana i Jezusa w kategoriach oczyszczenia (καθαρισμός), nowego narodzenia (γεννηθῆναι ἄνωθεν), obmycia i wiary.
Nie jest to przypadkowy wybór stylu, lecz przejaw zasadniczo odmiennego rozumienia skutku chrztu Jana.
Tam, gdzie Ewangelie synoptyczne używają języka nawrócenia zaczerpniętego z profetycznych wezwań do odnowy Przymierza, Jan posługuje się słownictwem rytualnej przemiany i zmiany tożsamości — dokładnie tego można by oczekiwać, gdyby chrzest pełnił funkcję inwestytury królewskiej, a nie moralnej reformy.

Kiedy tylko ujmiemy chrzest w tej kategorii rytualnej, struktura koronacyjna staje się widoczna.
Narracje o chrzcie zawierają stałą trójczłonową sekwencję obecną w starożytnych ceremoniach koronacyjnych: rytualne obmycie, boskie namaszczenie lub udzielenie mocy oraz publiczne ogłoszenie synostwa królewskiego.
Ten wzorzec występuje w tradycjach koronacyjnych Egiptu, Bliskiego Wschodu i Izraela, a nie w obrzędach pokutnych czy w scenach powołania prorockiego.
Relacje ewangeliczne przedstawiają wszystkie trzy elementy w zachowanej kolejności: Jezus poddaje się zanurzeniu (obmycie), Duch zstępuje na Niego (boskie namaszczenie), a głos z nieba ogłasza „To jest mój Syn” (proklamacja królewska).
Rytuały koronacyjne w Egipcie ptolemejskim podążały tym samym schematem — król był oczyszczany, otrzymywał boską inwestyturę od bóstw i był ogłaszany prawowitym władcą.
Ten sam wzór pojawia się w asyryjskich inskrypcjach królewskich, gdzie król przechodzi rytualne przygotowanie, otrzymuje łaskę bogów i zostaje ogłoszony ich wybrańcem.
Namaszczenie Dawida w 1 Sm 16 przebiega identycznie: rytualny akt Samuela, Duch Pański spoczywający na Dawidzie i uznanie go za wybranego króla Boga.

Rytualne oczyszczenie przed objęciem królewskiego czy kapłańskiego urzędu było standardową praktyką w całym starożytnym świecie śródziemnomorskim i na Bliskim Wschodzie.
Kontekst geograficzny jest tu kluczowy: Jan Chrzciciel działał w rejonie Zajordania („Betania po tamtej stronie Jordanu”, J 1,28), będącym pod wpływem greckiej władzy politycznej i kultury, a nie w samej Judei.
W takim hellenistycznym otoczeniu greckie obrzędy oczyszczenia (καθαρισμοί) stosowano przy przejściach stanowych, przy powołaniu prorockim czy przed objęciem funkcji królewskich — nie w znaczeniu późnochrześcijańskiej „pokuty moralnej”.
W tekstach ugaryckich z epoki brązu królowie dokonują obmyć rytualnych — lustracji — w wyznaczonych momentach w ramach swoich kultowych i politycznych zadań, oczyszczając się przed składaniem ofiar i wykonywaniem funkcji królewskich.
Hasmonejscy władcy Judei, łączący rolę króla i arcykapłana, byli zobowiązani do zachowywania przepisów czystości kapłańskiej, obejmujących obowiązkowe obmycia, określone szaty i konsekrację przed wejściem do przestrzeni świętej czy podjęciem czynności kultowych.
W tradycji greckiej i rzymskiej podobne obrzędy oczyszczenia (κάθαρσις po grecku, \textit{lustratio} po łacinie) stosowano wobec miast, armii, urzędników i funkcjonariuszy kultu, używając wody, dymu ofiarnego i ofiar, aby usunąć zanieczyszczenie (\textit{miasma}) przed podjęciem funkcji publicznych.
Pojęcie oczyszczenia rytualnego jako przygotowania do nowej roli politycznej lub sakralnej nie było marginalne, lecz stanowiło strukturę nośną starożytnych koncepcji legitymizacji władzy.

Kluczowe jest to, że obrzędy te nie dotyczyły przede wszystkim win moralnych, lecz zmiany statusu: wyznaczały przejście z jednego stanu do drugiego, uwalniały jednostkę od wcześniejszych zobowiązań i przygotowywały ją do nowych obowiązków.
Dlatego właśnie ἄφεσις („uwolnienie”) jest tu terminem adekwatnym — nie jako przebaczenie prywatnych grzechów, lecz formalne zwolnienie z dotychczasowych więzi, aby można było objąć nowy urząd lub tożsamość.

Interpretację koronacyjną dodatkowo wzmacnia wewnętrzna logika narracji, która kłóci się z interpretacją pokutną.
W Mt 3,14 Jan sprzeciwia się chrztowi Jezusa, mówiąc: „To ja potrzebuję chrztu od Ciebie, a Ty przychodzisz do mnie?”.
Taki protest nie ma sensu, gdyby chodziło o ryt pokuty za grzechy moralne — Jan po prostu udzieliłby chrztu, jak wszystkim innym.
Jeśli jednak rytuał wynosi ochrzczonego na wyższy status od celebransa, nadając mu władzę królewską, wówczas opór Jana jest całkowicie zrozumiały: czuje się niegodny udzielić obrzędu, który stawia Jezusa ponad nim samym.
Dokładnie tak zachowałby się kapłan koronacyjny poproszony o namaszczenie kogoś, kto ma stać się jego zwierzchnikiem.
Model pokutny zmusza do mnożenia późniejszych wyjaśnień teologicznych dla tego protestu; model koronacyjny wyjaśnia go naturalnie w ramach samej opowieści.

Wczesne warianty tekstowe potwierdzają tę interpretację koronacyjną, zachowując bardziej dosłowną formułę z Psalmu 2.
Ewangelia Hebrajczyków i niektóre rękopisy Łukasza przekazują przy chrzcie słowa „Dziś Cię zrodziłem”, cytując Ps 2,7 dosłownie zamiast łagodniejszego „To jest mój Syn umiłowany”.
To dokładne brzmienie formuły intronizacyjnej Dawidowego króla — deklaracji ogłaszanej w chwili objęcia tronu.
Obecność takiego wariantu pokazuje, że wczesne wspólnoty chrześcijańskie rozumiały chrzest jako koronację Jezusa, moment formalnego ustanowienia Go Mesjaszem–Królem.
Późniejsze rękopisy złagodziły tę formułę, gdy polityczne konsekwencje stały się niewygodne, podobnie jak zmieniano odniesienia do zburzenia świątyni czy niektóre relacje o zmartwychwstaniu, aby podkreślić wymiar religijny, a nie dynastyczny.

Wstęp Marka najlepiej czytać jako narrację koronacyjną, a nie „ogólny” wstęp do działalności Jezusa.
Kiedy Marek pisze: „Początek Ewangelii o Jezusie Chrystusie, Synu Bożym”, nie daje neutralnej przedmowy, lecz świadomie sięga po precyzyjny język \textit{euangelion} z dekretów koronacyjnych, to samo słowo, którego używa inskrypcja z Priene w ogłoszeniu panowania Augusta.
Gdy cytuje Iz 40,3 — „Głos wołającego na pustyni: Przygotujcie drogę Panu, prostujcie ścieżki dla Niego” — nie tylko przywołuje proroctwo, lecz ogłasza królewski wjazd Jezusa.
Marek następnie opisuje gromadzenie się tłumów: „Schodziła się do niego cała kraina judzka i wszyscy mieszkańcy Jerozolimy”, co czyta się jak zgromadzenie świadków intronizacji.
Jan kontynuuje formalnym ogłoszeniem następcy: „Idzie za mną mocniejszy ode mnie, któremu nie jestem godzien, by się schylić i rozwiązać rzemyk u Jego sandałów”.
Potem na scenę wchodzi Jezus, przyjmuje chrzest i otwierają się niebiosa.
Zstąpienie Ducha „jak gołębica” i niebiańska deklaracja „Tyś jest mój Syn umiłowany, w Tobie mam upodobanie” pełnią funkcję aklamacji boskiego synostwa.
Na koniec Jezus udaje się na czterdzieści dni na pustynię.
Ten okres postu i próby odpowiada rytom oczyszczenia i testu znanym z tradycji ptolemejskich i bliskowschodnich, które przygotowywały władców do objęcia królewskich funkcji.

Cała ta sekwencja bywa w nowoczesnych wydaniach dzielona na osobne perykopy, co zaciera jej jedność.

Kiedy Marek pisze, że Duch zstąpił „jak gołębica”, nie jest to przypadkowy obraz.
Zeus był powszechnie znany z tego, że przekazywał swoje boskie orędzia za pośrednictwem białego gołębia.
Hezjod opisuje gołębie z Dodony przy jednym z najstarszych i najsłynniejszych wyroczni Zeusa: kapłanki nazywano „gołębiami”, a samo proroctwo mówiono, że płynie z dębu Zeusa.
W takim kontekście gołębica była rozpoznawalnym emblematem zstąpienia i komunikacji Zeusa, w bardzo podobny sposób, w jaki Duch Święty zaczął później być przedstawiany w tradycji chrześcijańskiej.

Wydarzenie to bywa dzielone na osobne, zatytułowane fragmenty, co wprowadza czytelnika w błąd, sugerując, że chodzi o kilka epizodów, a nie o ciągłą narrację jednego wydarzenia.
Czytane w całości, jako opis ceremonii koronacyjnej, jest jednak absolutnie przejrzyste.
Każda koronacja w Egipcie ptolemejskim była połączona z rytuałem oczyszczenia, postu i próby na pustyni.
Motyw Zeusa zstępującego jako biały gołąb, by przemówić do ludu, jest także wyjątkowo częsty w hellenistycznej ikonografii królewskiej.

\subsection{11.3 Hymny liturgiczne jako królewska proklamacja}

\label{subsec:liturgical-hymns-as-royal-proclamation}

Kolejnym silnym wskazaniem, że przed spisaniem Ewangelii nie była potrzebna długa faza dojrzewania teologii, jest obecność kompletnych materiałów hymnicznych już w ich obrębie.
Te hymny nie są przypadkowymi modlitwami, lecz mają strukturę w tym samym idiomie, co greckie hymny królewskie i cesarskie świata hellenistycznego i rzymskiego.
Czytają się jak pieśni koronacyjne przeniesione bezpośrednio z Augusta lub Ptolemeuszy na Jezusa.

\paragraph{Magnificat (Łk 1:46--55).}
Pieśń Marii zwykle traktuje się jako echo psalmu żydowskiego, ale jej forma literacka jest hymnem intronizacyjnym.
Otwarcie „Wielbi dusza moja Pana i raduje się duch mój w Bogu, Zbawcy moim” używa czasownika \emph{megalyno} („wielbić, wywyższać”) i tytułu \emph{soter} („zbawca”), czyli dwóch kluczowych terminów hellenistycznych hymnów władczych.
Dekrety ptolemejskie i inskrypcje augustiańskie wychwalały monarchę jako „Zbawcę świata” i rozpoczynały się formułami uwielbienia niemal identycznymi w tonie.
Środkowa część opisuje wielką odwróconą hierarchię: „Strącił władców z tronów, a wywyższył pokornych, głodnych nasycił dobrami, a bogatych z niczym odprawił”.
To nie jest jedynie retoryka prorocka Starego Testamentu; to podstawowy schemat hymnów na objęcie władzy, w których nowy król upokarza tyranów, wywyższa ubogich i rozdziela ludowi zboże i bogactwo.
Inskrypcje z czasów Augusta i Nerona wychwalają cesarza dokładnie za to, że „rzuca w pył pyszałków, podnosi pokornych i nasyca głodnych chlebem”.
Końcowe wersy („Ujął się za swoim sługą, Izraelem, pomny na miłosierdzie swoje”) odpowiadają zakończeniom hellenistycznych encomiów, w których dobrodziejstwo władcy zakorzenione jest w tradycji przodków i boskim mandacie.
W sumie Magnificat jest hymnem królewskim: ogłasza, że dziecko zrodzone z Marii jest prawdziwym królem–dobroczyńcą, który realizuje odwrócenia wychwalane w propagandzie cesarskiej.

\paragraph{Hymn o Chrystusie (Flp 2:6--11).}
Jeszcze bardziej uderzający jest przedpawłowy hymn cytowany w Liście do Filipian:
„On, istniejąc w postaci Bożej,
nie skorzystał ze sposobności, aby być na równi z Bogiem,
lecz ogołocił samego siebie,
przyjąwszy postać sługi,
stawszy się posłusznym aż do śmierci — i to śmierci krzyżowej.
Dlatego też Bóg wywyższył Go nad wszystko
i darował Mu imię ponad wszelkie imię,
aby na imię Jezusa
zgięło się wszelkie kolano
na niebie, na ziemi i pod ziemią
i aby wszelki język wyznał,
że Jezus Chrystus jest Panem”.
Struktura jest jednoznaczna: zstąpienie w uniżeniu, weryfikacja, wywyższenie i powszechna aklamacja.
To dokładnie ta sama gramatyka, co w greckich hymnách cesarskich i formułach aklamacyjnych.
W prowincjach poddani mieli obowiązek zginać kolana i wyznawać panowanie Augusta czy Tyberiusza; w inskrypcjach wotywnych zachowały się formuły nakazujące „wszystkim narodom i wszystkim językom” ogłaszać tytuły cesarza.
Nadanie „imienia ponad wszelkie imię” odpowiada ceremonialnemu nadaniu władcy boskich epitetów — \emph{Soter}, \emph{Euergetes}, \emph{Kyrios}.
Hymn kończy się powszechnym chórem poddania, czyli dokładnie w scenerii liturgii intronizacyjnej.
Tyle że tutaj honory nie zostają przeniesione na Cezara, lecz na Chrystusa ukrzyżowanego i zmartwychwstałego.
To bezpośrednie przejęcie form imperialnych na rzecz konkurencyjnego pretendenta.

\paragraph{Wniosek.}
Te dwa hymny same w sobie wystarczą, by obalić rozpowszechniony pogląd, że teologia potrzebowała dekad „fermentacji” po ukrzyżowaniu.
Wręcz przeciwnie, Ewangelie już zawierają cesarskie formy liturgiczne w pełni przepisane na Jezusa.
Nie jest to powolny rozwój doktrynalny, lecz natychmiastowa proklamacja polityczna w idiomie greckiej hymnografii dworskiej.
Magnificat intronizuje nienarodzone jeszcze dziecko jako króla–dobroczyńcę; hymn z Listu do Filipian wywyższa ukrzyżowanego jako Pana całego kosmosu.
Oba świadectwa pokazują, że od samego początku przesłanie chrześcijańskie było formułowane w ustalonym języku dworskim imperium, ogłaszając Jezusa prawowitym władcą.

\subsection{Warianty opowieści o chrzcie w Ewangelii Hebrajczyków}\label{subsec:gospel-of-the-hebrews-variants-of-the-baptism-story}

W niektórych fragmentach Jezus przedstawiany jest jako ten, który chrzci innych, a nie tylko jako przyjmujący chrzest.

Odwraca to standardową narrację i stawia Jezusa w pozycji inicjatora, tego, który rozkazuje i umacnia.
Taki akt odzwierciedla przywilej cesarski — król nie tylko sam zostaje zainicjowany, lecz jest tym, który odtąd sprawuje obrzędy swojego panowania.
Podkreśla to Jezusa jako króla nowego porządku, który już sprawuje władzę królewską.

Ewangelia Hebrajczyków w sposób wyjątkowy nazywa Ducha Świętego „Matką” Jezusa, która chwyta Go za włosy i przenosi na Górę Tabor.
Jest to echo starożytnych scen inwestytury królewskiej, w których bóstwo (lub żeńskie uosobienie Mądrości) objawia lub błogosławi nowego władcę.
Góra Tabor jest symbolem — „wysokim miejscem”, często używanym w tradycjach żydowskich i hellenistycznych jako sceneria objawień boskich i imperialnych.
Motyw uniesienia i poniesienia przez Ducha przywołuje obrazy intronizacji, w których nowy król jest symbolicznie podnoszony.

\subsection{Psalmy królewskie:}\label{par:royal-psalms}

Psalmy zawierają liczne odniesienia do króla jako pomazańca Boga (np. Ps 2,7; Ps 45,7), a scena chrztu wyraźnie nawiązuje do tych motywów królewskich.
Gdy Ewangelie identyfikują Jezusa jako Syna Bożego i namaszczają Go Duchem, mogą w ten sposób odwoływać się do tych tradycji królewskich, sugerując wydarzenie o charakterze koronacyjnym, a nie jedynie ryt religijnego obmycia.

\subsection{Gołębica zstępująca na Jezusa przy chrzcie jako symbol boskiej aprobaty i umocnienia.}\label{subsec:the-dove-descending-on-jesus-at-his-baptism-is-a-symbol-of-divine-approval-and-empowerment.}

Gołębica jako symbol boski: w kulturach greckich i hellenistycznych gołębica była często kojarzona z różnymi przejawami boskości, szczególnie z bogami takimi jak Zeus czy Apollo.
Gołębica zstępująca z nieba stanowiła powszechny symbol boskiej aprobaty lub udzielenia mocy, jak w niektórych mitach, gdzie Zeus przybiera postać gołębicy.
W hellenistycznej ideologii królewskiej motyw boskich znaków, takich jak pojawienie się gołębicy w momencie koronacji króla lub wodza, wcale nie był rzadki.
Gołębica i „namaszczenie” władców: greccy królowie i rzymscy cesarze byli często postrzegani jako wybrani przez bogów, a znaki boskiej aprobaty — takie jak gołębica — miały podkreślać tę ideę.
Użycie motywu gołębicy przy chrzcie Jezusa można więc odczytać jako znak, że Jezus zostaje „namaszczony” na króla, zgodnie z tym, jak władcy hellenistyczni byli symbolicznie namaszczani przez bóstwa.

\subsection{Ewangelia Mateusza zawiera szczegółowy opis narodzin Jezusa z wizytą Mędrców.}\label{subsec:the-gospel-of-matthew-includes-the-details-of-the-birth-of-jesus-with-the-visit-of-the-magi.}

Wizyta Mędrców jest powszechnie przyjmowana jako alegoria, ale — jak wspomniano w poprzednim rozdziale — tytuły Mędrców są niezwykle trudne do sfabrykowania nawet dla najbardziej wykształconych uczonych w imperium rzymskim, którzy nie byliby lojalistami dawnego imperium greckiego.
Rzeź niewiniątek dokonana przez Heroda jest również powszechnie uznawana za alegorię, lecz — jak pokazano w \emph{The Jesus Dynasty} — „rzeź niewiniątek” najprawdopodobniej odnosi się do Heroda zabijającego wielu własnych synów i członków ich rodzin.
Jeśli Józef był sam z rodu hasmonejskiego, to narodziny w mieście dynastii hasmonejskiej, Betlejem, są znacznie bardziej prawdopodobne jako wydarzenie historyczne niż czysta alegoria.
Podobnie ucieczka do Egiptu jest najpewniej zdarzeniem realnym, ponieważ szlachetna rodzina bez trudu dysponowałaby środkami, by uchodzić do Egiptu i zostać tam przyjętą na kilka lat.

\subsection{Wreszcie Ewangelia Mateusza zawiera wielki nakaz misyjny, który najprawdopodobniej jest redakcyjnym rozwinięciem pierwotnej Ewangelii Marka.}\label{subsec:finally-the-gospel-of-matthew-includes-the-great-commission-which-is-likely-an-edit-to-the-original-gospel-of-mark.}

πορευθέντες οὖν μαθητεύσατε πάντα τὰ ἔθνη — Czyńcie uczniami wszystkie narody — roznoście naukę Jezusa do wszystkich ludów świata.
Tutaj słowo „narody” jest tłumaczeniem greckiego terminu „ethne”, który można oddać także jako „poganie” lub „nie-Żydzi”.
„Chrzcząc je w imię Ojca i Syna, i Ducha Świętego” — włączajcie ich do wspólnoty wiary przez chrzest.
„Uczcie je zachowywać wszystko, co wam przykazałem” — kontynuujcie nauczanie i prowadzenie nowych wierzących, aby żyli według nauki Jezusa.
Jezus obiecuje swoją obecność: „A oto Ja jestem z wami przez wszystkie dni, aż do skończenia świata”.
Wielki nakaz misyjny jest w istocie tym, co robili wysłannicy imperium greckiego: roznosili dekrety króla po całym jego władztwie.

\subsection{Termin ἔθνη jest zazwyczaj tłumaczony jako „narody”, ale może też oznaczać „poganie” lub „nie-Żydzi”.}\label{subsec:the-term-ux1f14ux3b8ux3bdux3b7-is-typically-translated-as-nations-but-can-also-be-translated-as-gentiles-or-non-jews.}

W kontekście imperium greckiego termin ten odnosiłby się do wszystkich Greków rozproszonych po imperium, do wszystkich ludów, królestw i plemion wchodzących w skład świata greckiego.

\subsection{Kuszenie Jezusa na pustyni po chrzcie.}\label{subsec:the-temptation-of-jesus-in-the-wilderness-following-the-baptism.}

Zaraz po chrzcie — swoim namaszczeniu jako Syna i Chrystusa — Jezus zostaje wyprowadzony na pustynię na czterdzieści dni postu (Mk~1,12--13; Mt~4,1--11; Łk~4,1--13).
Tam staje wobec potrójnej królewskiej próby: \emph{chleba} (zaopatrzenie materialne), \emph{widowiska} (cud dla udowodnienia boskiej przychylności) i \emph{królestw} (panowanie nad światem).
Jan zachowuje tę samą logikę, rozciągając ją na całe publiczne działanie Jezusa zamiast skupiać w jednej scenie: tłumy domagają się chleba (J~6,26), krewni naciskają na publiczne znaki (J~7,3--4), a lud usiłuje obwołać go królem siłą (J~6,15).
Nie jest to przedstawione jako prorok odhaczający kolejne „spełnione wersety” ani jako opowieść zapożyczona z jakiejś obcej szkoły.
Czyta się to jako próbę na progu koronacji zaczerpniętą z repertuaru starego jak samo królowanie.

\paragraph*{Tło egipskie (objęcie tronu przez faraona).}
W egipskiej teologii państwowej wyniesienie króla wymagało rytualnego zwycięstwa nad \emph{isfet} — chaosem, fałszem, pustynną „czerwoną ziemią” — aby ustanowić \emph{maat} — porządek, prawdę, sprawiedliwość.
Cykl koronacyjny łączył oczyszczenie (wstrzemięźliwość, obmycia, przywdzianie szat) z liturgicznym „miażdżeniem wrogów”, boskimi przemówieniami i nadaniem insygniów.
Przykłady tej biegunowości i wzorca obrzędu wyniesienia są widoczne już w Tekstach Piramid i Tekstach Sarkofagów oraz w programach świątyń w Karnaku, Luksorze i Medinet Habu, kulminujących w cyklu z Edfu znanym jako „koronacja sokoła”.
Geografia ma tu znaczenie: w wyobraźni egipskiej pustynia należy do sfery chaosu.
Wzmianka Marka, że Jezus był „ze zwierzętami dzikimi”, osadza jego próbę dokładnie w tej strefie nieładu.
Forma jest inna (publiczny ryt vs.\ samotny post), ale funkcja ta sama (próba w domenie chaosu bezpośrednio przed objęciem rządów).

\paragraph*{Tło izraelskie (Mojżesz, Izrael, Dawid).}
Mojżesz spędza \emph{czterdzieści dni} na Synaju „bez chleba i wody”, zanim otrzyma Prawo (Wj~34,28; Pwt~9,9).
O Izraelu mówi się, że był \emph{czterdzieści lat} „doświadczany” na pustyni głodem, „wystawianiem Boga na próbę” i wiernością wyłącznie jemu (Pwt~8,2--3; 6,13; 6,16) — to właśnie te wersety Jezus cytuje w odpowiedzi Szatanowi.
Próba Dawida przy obejmowaniu władzy ma charakter etyczny, a nie ascetyczny: dwukrotnie otrzymuje możliwość skrócenia drogi do tronu przez zabicie Saula, ale odmawia sięgnięcia po władzę (1 Sm~24; 26).
Nie są to te same opowieści co Ewangelie, ale uczą tej samej zasady królowania: prawowita władza przychodzi przez posłuszeństwo i próbę, a nie przez zawłaszczenie.

\paragraph*{Porównania indoiranijskie i grecko-filozoficzne.}
Najbardziej \emph{precyzyjną} paralelą narracyjną jest próba Siddharthy Gautamy pod drzewem Bodhi: Mara oferuje (1) pożądliwość zmysłową, (2) przerażające wyzwanie domagające się cudownego dowodu ocalenia oraz (3) panowanie nad światem, jak relacjonują \emph{Lalitavistara} (rozdz.~24) i \emph{Nidanakatha}.
Strukturalne dopasowanie do ewangelicznej triady \emph{chleb–widowisko–królestwa} jest uderzające.
Konfrontacja Zaratusztry z Angra Mainju ma podobnie trójczłonowy przebieg: (1) bogactwo i majątek, (2) kobiety i rozkosz oraz (3) królowanie i władza; prorok odrzuca każdą z tych ofert, pozostając wierny \emph{asza} (prawda/porządek), jak przekazują \emph{Vendidad}~19 i \emph{Denkard}~7.
Grecka pedagogika moralna proponuje heraklejską matrycę, szczególnie w przypowieści Prodikosa zachowanej u Ksenofonta (\emph{Mem}.~2,1), gdzie królewski bohater musi wybrać trudną drogę cnoty zamiast łatwej drogi występku.
Formy są różne, ale logika progu pozostaje ta sama: nadrzędna pozycja wymaga zdecydowanego odrzucenia pożądania, popisu i dominacji.

\paragraph*{Praktyka obejmowania tronu w świecie hellenistycznym i perskim.}
Również historyczni królowie przechodzili próby tuż \emph{przed} intronizacją.
Cyrus zostaje rozpoznany po udowodnionym szlachectwie pośród skromnego wychowania (Herodot~1,114--122; Ksenofont, \emph{Cyropaideia}).
Dariusz~I przedstawia swój pierwszy rok w inskrypcji behistuńskiej jako pokonanie konkurentów do korony.
Aleksander szuka boskiego potwierdzenia w Delfach i Siwa przed podbojem świata.
Są to próby o charakterze militarnym, politycznym lub wyrocznym, a nie kuszenia, ale moment jest ten sam: test bezpośrednio przed objęciem władzy.

\paragraph*{Co jest wyjątkowe w ewangelicznej triadzie.}
(1) \textbf{Czas:} scena stoi \emph{natychmiast} po namaszczeniu i tuż przed publicznymi roszczeniami królewskimi.
(2) \textbf{Miejsce:} \emph{pustynia}, klasyczna strefa chaosu zarówno dla Egiptu, jak i Izraela.
(3) \textbf{Forma:} \emph{potrójna} oferta, która czysto nakłada się na starożytny katalog królewskich pokus (\emph{chleb/bogactwo}, \emph{widowisko/dowód lub przyjemność}, \emph{królestwo/władza}).
(4) \textbf{Rozstrzygnięcie:} inaczej niż Budda (wyrzeka się królowania dla oświecenia) czy budowniczowie imperiów (zwyciężają siłą), Jezus odrzuca wszystkie skróty i wybiera drogę ofiary jako ścieżkę do królowania.

\paragraph*{Wniosek.}
Ta scena nie jest mozaikową składanką ani importem z buddyzmu.
Gdyby zamiarem było skopiowanie Mojżesza, scenerią byłby Synaj z czterdziestoma dniami przed obliczem Boga, a nie sataniczna konfrontacja na pustyni z trójstopniową eskalacją.
Gdyby celem było skopiowanie Buddy, pojawiłyby się córki Mary, demoniczne armie i oświecenie, ale ich nie ma, a buddyzm pozostaje całkowicie nieobecny w tkance narracyjnej Ewangelii.
Tekst posługuje się natomiast toposem królewskiej próby, który przesyca świat starożytny.

Fakty diagnostyczne:
(1) \textbf{Czas:} scena ma miejsce natychmiast po namaszczeniu i bezpośrednio przed publicznym wystąpieniem jako król, czyli dokładnie na progu objęcia władzy.
(2) \textbf{Miejsce:} pustynia jest strefą chaosu (egipskie \emph{isfet}, izraelskie miejsce próby), co podkreśla wzmianka „był ze zwierzętami dzikimi” (Mk~1,13).
(3) \textbf{Forma:} próba ma postać zwartej triady — chleb (zaopatrzenie materialne), widowisko (publiczny dowód), królestwa (suwerenność) — a nie ogólnego cierpienia.
(4) \textbf{Gramatyka biblijna:} każda odpowiedź Jezusa cytuje teksty o próbie na pustyni z Księgi Powtórzonego Prawa (Pwt~8,3; 6,13; 6,16), wiążąc jego próbę z czterdziestoletnią próbą Izraela, ale bez powtarzania sceny Synaju.
(5) \textbf{Egipska logika koronacji:} faraonowie oczyszczają się przez wstrzemięźliwość, stają wobec pustynnego chaosu (\emph{isfet}) w rytuale „miażdżenia wrogów” i otrzymują insygnia od bogów; król zdaje egzamin na granicy chaosu, zanim zostanie intronizowany.
(6) \textbf{Triady irańskie i indyjskie:} Zaratusztra odrzuca bogactwo, pożądliwość i władzę, zanim rozpocznie swoją misję; Siddhartha odrzuca pożądanie, przerażające widowisko i panowanie, zanim dozna przebudzenia; kategorie pokus odpowiadają ewangelicznej triadzie, nawet jeśli rozwiązania są różne.
(7) \textbf{Hebrajska etyka królewska:} Dawid odrzuca skrót do tronu Saula; królowanie przychodzi przez wierność i próbę, a nie przez chwytanie.
(8) \textbf{Hellenistyczny zwyczaj progu:} Cyrus, Dariusz i Aleksander stają wobec kryzysów lub prób wyroczni właśnie w momencie przejścia do władzy; próba mieści się tuż przed koroną.

Wniosek:
Mozaikowa kopia zachowałaby teofanię na górze Synaj i pominęłaby stopniowane propozycje Szatana.
Bezpośrednia kopia buddyjska jest mało prawdopodobna, bo choć filozofia Jezusa i Buddy ma pewne powinowactwa, przy rzeczywistym wpływie można by się spodziewać przynajmniej śladowej obecności motywów buddyjskich gdzieś w najwcześniejszej literaturze chrześcijańskiej.
Zastany pakiet — \emph{pustynna} sceneria tuż po namaszczeniu, odpowiedzi z Pwt, \emph{trójczłonowe} królewskie oferty, odrzucenie skrótów i rozpoczęcie publicznego królowania — dokładnie odpowiada starożytnemu wzorcowi \emph{królewskiej próby}.

A zatem:
ewangeliści piszą jak historycy dworscy, a nie jak kompilatorzy tekstów–dowodów czy kolekcjonerzy egzotycznych motywów.
Egipt dostarcza logiki objęcia tronu (chaos ujarzmiony przed koronacją).
Izrael dostarcza schematu czterdziestu dni/czterdziestu lat próby i słów odpowiedzi.
Iran i Indie dostarczają trójdzielnej gramatyki pożądliwości, widowiska i suwerenności.
Grecja dostarcza nawyku próby na progu władzy.
Ewangelie krystalizują te nici w jednej scenie objęcia władzy: namaszczony król wchodzi w pustynny chaos, odrzuca chleb, widowisko i królestwa, i wybiera krzyż jako jedyną prawowitą drogę do panowania.

\subsection{Być może najbardziej uderzającym elementem Ewangelii Mateusza jest odwoływanie się do pism żydowskich i wypełniania proroctw na niemal każdym kroku.}\label{subsec:possibly-the-most-prominent-element-of-the-gospel-of-matthew-is-the-reference-to-the-jewish-scriptures-and-the-fulfillment-of-the-prophecies-at-seemingly-every-turn.}

Jako całość Ewangelia ta z bardzo dużym prawdopodobieństwem wydaje się dziełem wybitnego żydowskiego kapłana, który był zarazem lojalistą imperium greckiego.
Istnieje teoria poparta znaczącymi przesłankami tekstowymi, że Ewangelia Mateusza została napisana przez kapłana żydowskiego Mattatiasza ben Teofilosa.
Teoria ta nie jest powszechnie przyjmowana, ponieważ zakłada datowanie Ewangelii Mateusza na okres przed 70 rokiem, jednak w ramach założeń omawianych w tej książce argument ten traci moc, a w świetle innych obserwacji poczynionych w tym ujęciu teoria ta staje się bardziej prawdopodobna niż tradycyjne datowanie.

\subsection{Ewangelia Łukasza zawiera także szereg redakcyjnych dodatków, z których najbardziej rzuca się w oczy narracja o narodzinach Jezusa.}\label{subsec:the-gospel-of-luke-also-includes-a-number-of-edits-most-notably-the-birth-narrative-of-jesus.}

Opowiadanie o narodzinach jest pełne wewnętrznych sprzeczności i stoi w napięciu wobec Ewangelii Mateusza, której relacja zawiera kilka niezwykle intrygujących faktów potwierdzających resztę przedstawionej tu teorii.
Genealogia Jezusa w Ewangelii Łukasza ma, co znamienne, jeszcze więcej autentycznych postaci z dynastii hasmonejskiej aż do ojca Marii, Helego, co jest skrótem od Eliakim, tego samego imienia, które tradycja Protoewangelii Jakuba podaje jako Joachima, ojca Marii.
Jeśli przyjmiemy, że Ewangelia Łukasza powstała jako korespondencja z autorem Ewangelii Mateusza, wówczas jej datowanie również nie byłoby odsunięte w czasie.

\subsection{Ewangelia Marcjona zasługuje w tym kontekście na ponowne rozpatrzenie.}\label{subsec:the-gospel-of-marcion-deserves-another-look-in-this-context.}

Ewangelia wiązana z Marcjonem (ok.~140 r.) zachowała się wyłącznie poprzez wrogie świadectwa, takie jak Tertulian (\emph{Adversus Marcionem}, ok.~207 r.), Epifaniusz (\emph{Panarion}, ok.~374 r.) oraz \emph{Dialog Adamancjusza} (koniec III / początek IV w.).
Mimo polemicznej oprawy źródła te zachowują wystarczająco danych, by odtworzyć kluczowe cechy Ewangelii Marcjona.
Otwierała się ona na Łk~3,1 („W piętnastym roku panowania Tyberiusza”), nie zawierała narracji o dzieciństwie, pomijała wszelką wyraźną zapowiedź zburzenia świątyni w 70 roku i konsekwentnie prezentowała prostsze brzmienia niż kanoniczny Łukasz.

Cechy te od dawna wywołują spór.
Ojcowie Kościoła utrzymywali, że Marcjon okaleczył Łukasza, wycinając to, co nie pasowało do jego teologii.
Jednak wielu nowożytnych krytyków tekstu — od Harnacka po nowsze badania — wysunęło alternatywę, że Ewangelia Marcjona odzwierciedla wcześniejszą redakcję Łukasza.
Brak opowieści o dzieciństwie i późniejszych rozwinięć łatwiej wyjaśnić jako cechę wcześniejszej wersji niż jako skutek masowych skreśleń.
Prostszy język i zbieżność z wariantami „zachodniego” typu tekstowego wzmacniają wrażenie pierwotności.

Z punktu widzenia datowania jest to istotne.
Jeśli Łukasz miałby powstać około 80–90 roku, Ewangelia Marcjona musiałaby zostać radykalnie „okaleczona” dopiero około 140 roku.
Jeżeli jednak Ewangelia Łukasza krążyła w wielu formach, tekst Marcjona byłby po prostu tą wersją, która funkcjonowała w jego wspólnotach w Azji Mniejszej, podczas gdy dłuższy Łukasz kanoniczny reprezentuje późniejszą, rozszerzoną redakcję.

Kluczowy punkt polega na tym, że teza o pierwszeństwie Marcjona nie jest marginalną spekulacją, lecz poglądem poważnie rozważanym przez wybitnych krytyków tekstu.
Zamiast widzieć w Marcjonie innowatora, który „stworzył” Ewangelię, o wiele rozsądniej jest rozumieć go jako transmitującego wcześniejszą formę Łukasza.
Powszechne nieporozumienie polega na twierdzeniu, że „Marcjon napisał Ewangelię około 140 roku”, podczas gdy w rzeczywistości posługiwał się wersją już istniejącą.
Dla celów wczesnego datowania kluczowa lekcja jest taka, że liczne redakcje Łukasza żyły w I i wczesnym II wieku, a świadectwo Marcjona stanowi jedno z najczystszych okien na tę różnorodność.
W ramach koncepcji wczesnego Łukasza brak motywu zburzenia świątyni w 70 roku staje się całkowicie zrozumiały: świątynia wciąż stała, a autor pisał przed jej upadkiem, nie po nim.
Ten sam punkt widzenia wyjaśnia, dlaczego marcjońska kolekcja listów Pawła nie zawierała Listów pasterskich.
Te pisma — 1 i 2 Tymoteusz oraz Tytus — wszystkie otwierają się wyraźnymi roszczeniami do autorytetu Pawła, a jednak współczesna biblistyka jest niemal jednomyślna, że są to utwory pseudonimowe z wczesnego II wieku.
Ich późniejsze dodanie miało oswoić Pawła, czyniąc z niego strażnika porządku instytucjonalnego, podczas gdy korpus Marcjona zachował autentycznego, radykalnego Pawła.
Wzmacnia to wniosek, że biblioteka Marcjona odzwierciedla wcześniejszy, bardziej pierwotny etap zarówno Łukasza, jak i Pawła, zanim późniejsze rozszerzenia i interpolacje przekształciły tradycję.
Sam wstęp Ewangelii Łukasza — „Wielu już starało się ułożyć opowiadanie przede mną” — wskazuje wprost na taką sytuację.
Ewangelia Marcjona może być właśnie takim wcześniejszym „opowiadaniem”, podczas gdy nasz kanoniczny Łukasz jest wersją zredagowaną i rozszerzoną.
W tym sensie postać, którą nazywamy Łukaszem, może nie być odpowiedzialna za dłuższą redakcję zachowaną w kanonie, lecz raczej za krótszy oryginał, który przekazał Marcjon.
Jest również całkowicie możliwe, że sam Łukasz skomponował zarówno pierwszą wersję przed 70 rokiem, jak i drugą, rozszerzoną redakcję po upadku Jerozolimy.
W takim ujęciu Marcjon miałby dostęp do wcześniejszego szkicu, podczas gdy tekst kanoniczny odzwierciedla późniejszą wersję, ukształtowaną przez traumę zburzenia Świątyni.
Wiemy też, że autor Łukasza z pewnością napisał ciąg dalszy, Dzieje Apostolskie, więc bardzo rozsądne jest założenie, że mógł skomponować zarówno pierwszą, jak i drugą wersję swej Ewangelii.
W takim scenariuszu pierwotna wersja mogła nie zawierać narracji o narodzinach po prostu dlatego, że autor nie miał jeszcze do niej źródła, a materiał ten został dodany dopiero później w rozszerzonej redakcji.

\subsection{Ewangelia Łukasza opisuje, jak Jezus jako dziecko wchodził w dialog z przywódcami świątynnymi i żydowską elitą.}\label{subsec:the-gospel-of-luke-describes-how-jesus-engaged-with-temple-leaders-and-the-jewish-elite-when-he-was-a-child.}

Czytamy, że Józef i Maria osiedlili się w Galilei, ale co roku podróżowali do Jerozolimy na święto Paschy.
Jest to w pełni zgodne z tezą, że Maria należała do dynastii hasmonejskiej, ponieważ Hasmoneusze byli znani z silnego związku ze świątynią jerozolimską.
Ta teoria w pełni wyjaśnia, w jaki sposób Jezus mógł realistycznie w tak młodym wieku wchodzić w dialog z przywódcami świątynnymi i żydowską elitą.
Zgodnie z opinią o nim jako o mędrcu, jest bardzo prawdopodobne, że Jezus był dziecięcym geniuszem, który już jako chłopiec był chętny i zdolny rozmawiać z przywódcami świątyni o Pismach i Prawie.
Jeśli Jezus był dziecięcym geniuszem, bardzo możliwe, że studiował pisma żydowskie i greckie oraz filozofię zarówno w Egipcie, jak i w Galilei.
Biblioteki Galilei w Seforis byłyby wręcz idealnie zaopatrzone w dzieła filozofów greckich i pisma żydowskie, w pełni wystarczające, by wyjaśnić nauczanie Jezusa.

\subsection{Łukasz posługuje się greckim stylem historiograficznym, a ewangelia jest napisana jako dzieło historyczne, nie jako czysty tekst religijny.}\label{subsec:luke-is-using-greek-historical-writing-style-and-the-gospel-is-written-not-a-religious-text.}

Formuła wstępna starożytnych historyków greckich: wstęp Łukasza w Łk~1,1–4 naśladuje dobrze utrwaloną konwencję greckiego pisarstwa historycznego.
Typowa formuła u takich historyków jak Herodot czy Tukidydes rozpoczyna się od wyraźnej deklaracji zamiaru przedstawienia relacji faktograficznej oraz powołania się na wiarygodne źródła.
Historyk zazwyczaj zapowiada, że przekaże narrację opartą na starannych badaniach i rozmowach ze świadkami.
Podobnie Łukasz wprowadza swoją Ewangelię, stwierdzając, że wielu już podjęło się sporządzenia opowiadań, a on sam zamierza napisać „dokładny (uporządkowany) wykład” na podstawie tego, co dokładnie zbadał (Łk~1,3).
Ten wstęp nie tylko odwołuje się do autorytetu świadectwa naocznych świadków, ale też ustanawia wiarygodność autora — cechę niezwykle cenioną w greckiej historiografii.

Posługując się tą formułą historyka, Łukasz dystansuje swoje dzieło od opowieści mitycznych czy legendarnych.
Podobnie jak Herodot odróżnia się od bajarzy i poetów, dążąc do wierności faktom.
Jego wyraźne roszczenie do przeprowadzenia rzeczowego dochodzenia jest dokładnie techniką greckiej historiografii.

Pisze, że wielu już starało się ułożyć opowiadanie o wydarzeniach, które dokonały się pośród nas, tak jak przekazali je ci, którzy od początku byli naocznymi świadkami i sługami słowa.
(zauważmy: on sam przedstawia się jako sługa słowa)

Nacisk na dokładność historyczną: Łukasz konsekwentnie osadza swoją narrację w świecie realnych, znanych postaci politycznych i konkretnych odniesień chronologicznych.
Wspomina na przykład panowanie Cezara Augusta (Łk~2,1), namiestnictwo Kwiryniusza (Łk~2,2) oraz rządy Poncjusza Piłata (Łk~23,1–25).
Jest to uderzająco podobne do metody Tukidydesa, który zakotwicza swoją relację w precyzyjnie oznaczonych latach i działaniach politycznych (np. w odniesieniu do wojny peloponeskiej).
Akty publiczne i dekrety: rzymski spis ludności (Łk~2,1–3) jest kluczowym wydarzeniem wiążącym opowieść z szerszym światem polityki imperialnej.
To cecha charakterystyczna greckich historyków, którzy czynią z oficjalnych dekretów czy kampanii wojskowych punkty zwrotne narracji.
Podobnie Dzieje Apostolskie posługują się odniesieniami do procesów cesarskich i rzymskich procedur prawnych (np. proces Pawła przed Gallionem, Dz~18,12–17), aby legitymizować chrześcijaństwo i wyjaśnić jego rozprzestrzenianie się w imperium.
Cesarze rzymscy i przywódcy lokalni: wzmianka o Herodzie Wielkim w narracji osadza wydarzenia w konkretnych realiach politycznych.
Podobnie jak Polibiusz używa działań lokalnych przywódców jako okna na funkcjonowanie całego imperium, tak Łukasz czyni z Heroda figurę regionalną, której panowanie jest istotne nie tylko samo w sobie, ale jako element większej układanki politycznej.

Precyzyjne użycie nazw geograficznych, takich jak Judea, Galilea i Kafarnaum, wzmacnia tę ramę historyczną, zapewniając kontekst dla działań Jezusa i pierwotnego Kościoła w sposób bardzo podobny do metody dawnych historyków, którzy zakotwiczali swoje opowieści w dobrze znanych realiach.

Spójna narracja: Łukasz i Dzieje Apostolskie tworzą razem jeden, ciągły przekaz.
Sama struktura jest wysoce metodyczna i przypomina sposób, w jaki historycy tacy jak Polibiusz dzielili swoje dzieła na księgi lub tomy, podążając za chronologicznym rozwojem kluczowych wydarzeń.
Dzieło Łukasza jest podzielone między historię życia i działalności Jezusa (Ewangelia Łukasza) a wczesne dzieje Kościoła chrześcijańskiego (Dzieje Apostolskie).
Przejście od życia Jezusa do misji apostołów odpowiada temu, jak starożytni historycy najpierw opisywali życie władcy, a następnie przechodzili do dziejów jego imperium.

Ruchy polityczne i religijne: tak jak Polibiusz śledzi wzlot i upadek Rzymu, a Swetoniusz przedstawia życie cesarzy, tak Łukasz koncentruje się na postaciach politycznych i religijnych oraz ich ruchach w obrębie imperium.
Jego opis podróży misyjnych Pawła (w Dziejach) odpowiada strukturze kronik królewskich, ukazując wyzwania i procesy sądowe, z jakimi mierzy się wybitna postać polityczna (Paweł) podczas swoich wędrówek po imperium.

Mowa jako narzędzie polityczne: Łukasz często wykorzystuje przemówienia Jezusa i Pawła, aby wyrazić kluczowe idee teologiczne i polityczne.
Przemówienia te przypominają formalną retorykę obecną w historiografii greckiej, gdzie królowie lub przywódcy polityczni wygłaszają mowy, by wyjaśniać swoje cele i potwierdzać swoją legitymację.
W Dziejach, na przykład, mowy Pawła przed królami (Agryppa, Dz~26) czy Sanhedrynem (Dz~23) są osadzone w kontekście historycznym, w którym postać ta potwierdza swoją tożsamość polityczną i religijną.
Strategiczne wykorzystanie procesów i mów obrończych: podobnie jak Plutarch czy Swetoniusz przytaczają kluczowe przemówienia z procesów cesarzy lub ważnych polityków, tak Łukasz wykorzystuje procesy i obrony (takie jak Pawła przed Festusem i Agryppą), aby podkreślić legitymizację przesłania chrześcijańskiego.
Procesy te są strategicznie rozmieszczone w narracji, aby ukazać nie tylko polityczne zmagania, lecz także teologiczną obronę — wskazując, że nowy ruch w pewnych aspektach zgodny jest z polityką imperialną, a w innych stanowi dla niej wyzwanie.

Boska przychylność wobec Jezusa i Pawła: sposób przedstawiania boskiej misji Jezusa i apostolskiego autorytetu Pawła odpowiada temu, jak historycy greccy często ukazywali królów lub cesarzy jako obdarzonych boską łaską.
Konsekwentne ukazywanie cudów Jezusa oraz wizji i boskich interwencji w życiu Pawła przedstawia wczesny ruch chrześcijański jako przedsięwzięcie z ustanowienia Bożego — podobnie jak królowie i władcy w historiografii greckiej byli ukazywani jako wybrańcy bogów.

\subsection{Łukasz umieszcza rodowód Jezusa przy chrzcie, nie przy narodzinach.}\label{subsec:luke-places-the-lineage-of-jesus-at-the-baptism-not-at-the-birth.}

Uczeni twierdzą, że umieszczenie genealogii Jezusa po narracji o narodzinach jest dowodem na to, że opowieść o narodzeniu została dodana jako późniejsza interpolacja.
Nie jest to jednak argument przekonujący, ponieważ gdyby skryba zadał sobie trud dodania obszernego fragmentu i tak starannego wkomponowania go w narrację i główną myśl autora, trudno sobie wyobrazić, by „zapomniał" przenieść rodowód na początek tekstu.

Odpowiedź jest oczywista: rodowód nie był przeznaczony na początek, lecz na moment chrztu Jezusa, kiedy ogłaszano Go prawowitym królem.

\subsection{Późniejsze ewangelie i teksty apokryficzne}\label{subsec:later-gospels-and-apocryphal-texts}

Warto podkreślić, że aby ocenić znaczenie datowania tekstów — a w istocie jakichkolwiek twierdzeń w dziele historycznym — należy brać pod uwagę całe dostępne świadectwo, a nie tylko to, które wspiera daną tezę.
Chociaż niektóre pisma, takie jak Ewangelia Hebrajczyków, Ewangelia Marcjona, Protoewangelium Jakuba czy Ewangelia Tomasza, są bardzo silnie powiązane z niezależnymi tradycjami źródłowymi, istnieje duża grupa tekstów, które — mimo że rozbudowują narrację — wydają się w całości oparte na wcześniejszych pismach i nie wnoszą żadnych nowych źródeł.
Przykładem jest bardzo rozpowszechniona Ewangelia Nikodema, czyli Akta Piłata, która — choć wprowadza wiele scen z udziałem ludzi — nie wydaje się wnosić żadnych świeżych ani sprzecznych informacji, których nie można by uznać za naturalne upiększenie lub interpolację wcześniejszych tekstów.
Wszystko wskazuje na to, że została ona skomponowana wyłącznie na podstawie znanych już pism, a nie w oparciu o zaginione źródło lub bezpośrednie świadectwo pierwszych czy drugich świadków.

\subsection{Dowody synoptyczne na wczesne datowanie.}\label{subsec:synoptic-evidence-for-early-dating.}

Greka Marka jest szorstka i nieoszlifowana, co byłoby nietypowe, gdyby tekst powstał późno, w środowisku ukształtowanego Kościoła.
Mateusz wykazuje znajomość podatku świątynnego i praktyk, co najlepiej wyjaśnić, zakładając, że świątynia wciąż stała.
Łukasz posługuje się wyrafinowanymi konwencjami greckiej historiografii, podobnymi do Herodota czy Polibiusza, co wskazuje na warsztat historyka współczesnego opisywanym wydarzeniom, a nie późniejszego teologa.
Wszystkie te cechy przemawiają za datowaniem do pokolenia świadków.

Jeżeli Ewangelie są dokumentami pierwszego pokolenia, zakotwiczonymi w żywej pamięci uczestników oraz ujętymi w ramy greckich konwencji literackich, to ruch, który opisują, nie jest późną konstrukcją teologiczną, lecz zorganizowaną rzeczywistością polityczną istniejącą już za życia Jezusa.
Uświadomienie sobie tego przesuwa pytanie z „czy wczesny ruch chrześcijański powstawał stopniowo" na „jak taka struktura mogła już istnieć w miastach i sieciach, w których powstały same Ewangelie".
