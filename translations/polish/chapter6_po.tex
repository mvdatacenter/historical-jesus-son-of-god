To jest wielka tajemnica wiary:
Chrystus umarł, Chrystus zmartwychwstał, Chrystus powróci.

To jest sedno wiary chrześcijańskiej.
Chrześcijanie głosili miłość i przebaczenie oraz obiecywali życie wieczne wszystkim, którzy wierzą w Chrystusa.
Przyciągało to wielu nowych wyznawców.
A gdy ich liczba rosła, tajemnica stawała się coraz jaśniejsza.
Chrystus powróci.
Nie tylko jako duchowy zbawca, lecz jako nowy cesarz świata greckiego.
I tak Grecy stopniowo rozbudowywali wspólnotę chrześcijańską, przenikając Imperium Rzymskie, niższe warstwy społeczne, ośrodki miejskie i elity rządzące.
Aż w końcu samo Imperium Rzymskie zostało schrystianizowane, a dawne imperium greckie Aleksandra Wielkiego zostało przywrócone jako królestwo Boże na ziemi.

Powszechne, niepodważalne, a zarazem całkowicie błędne założenie współczesnej nauki głosi, że cała wczesna literatura chrześcijańska była pisana po grecku, ponieważ grecki był lingua franca Imperium Rzymskiego.
Gdyby tak było, należałoby oczekiwać, że Juliusz Cezar, Lud i Senat Rzymu, Wergiliusz czy Seneka wszyscy biegle posługiwaliby się greką i obficie w niej pisali oraz mówili.
W rzeczywistości to grecki był językiem administracyjnym niemal całego dawnego imperium greckiego, które później stało się znane jako wschodnia część Imperium Rzymskiego.
Znamienne, że niemal wszyscy pisarze chrześcijańscy, w tym Klemens Rzymski, Ignacy z Antiochii, Polikarp, Justyn Męczennik, Ireneusz, Orygenes, pisali po grecku.
Pierwszym niegreckim ojcem Kościoła jest Tertulian, dopiero pod koniec II wieku.
W tym rozdziale chcemy podkreślić, że chrześcijaństwo było religią wyłącznie grecką w swoim literackim i intelektualnym fundamencie, choć pojawiały się liczne wzmianki o Jezusie w źródłach niegreckich, a zwłaszcza teksty religijne pisane w koptyjskim nie były uważane za tę samą religię co chrześcijaństwo.
W tym rozdziale analizujemy pisma ojców Kościoła przez pryzmat odrodzenia Cesarstwa Bizantyjskiego.

\section{Feniks: symbolika imperialna w najwcześniejszej literaturze chrześcijańskiej}\label{sec:phoenix-symbolism}

Jedno z najwcześniejszych pism chrześcijańskich spoza kanonu Nowego Testamentu, \emph{1 Klemens} (ok.~96 r. n.e.), skierowane z Rzymu do Koryntu, zawiera rozbudowany fragment o feniksie:

\begin{quote}
Istnieje ptak zwany Feniksem.
Jest jedyny w swoim rodzaju i żyje pięćset lat.
Kiedy zbliża się czas jego rozpadu, przygotowuje dla siebie trumnę z kadzidła, mirry i innych wonności, a gdy czas jest wypełniony, wchodzi do niej i umiera.
Ale gdy jego ciało się rozkłada, powstaje robak, który odżywia się wilgocią martwego stworzenia i wyrasta mu skrzydła.
Następnie, gdy już się wzmocni, bierze tę trumnę i leci z ziemi Arabii do Egiptu, do miasta Heliopolis, i za dnia, na oczach wszystkich, kładzie ją na ołtarzu słońca.
(\emph{1 Klemens} 25:1--5)
\end{quote}

Współczesna nauka niemal zawsze interpretuje ten fragment jako alegorię zmartwychwstania Chrystusa oraz powszechnego zmartwychwstania umarłych.
Takie odczytanie wyrywa jednak tekst z politycznego kontekstu, w którym feniks funkcjonował już jako ikonografia imperialna.
Feniks nie był chrześcijańskim wynalazkiem; w I wieku był już ustalonym symbolem \emph{imperium renovatum} — „odnowionego imperium”.

Monety rzymskie od okresu julijsko-klaudyjskiego regularnie zestawiały feniksa z legendami takimi jak AETERNITAS AVG, SAECULUM NOVUM czy FEL TEMP REPARATIO.
Za Hadriana feniks pojawia się z promienistą aureolą, stojąc na gałązce laurowej, jako rewers złotych i srebrnych emisji reklamujących „wieczność cesarza” (RIC II Hadrian 246--247).
Późniejsze, IV-wieczne monety Konstansa i innych władców konstantyńskich ukazują feniksa na skalistym kopcu z legendą FEL TEMP REPARATIO — „pomyślne odnowienie czasów” — wyraźnie oznaczającą odnowę losów imperium po kryzysie.
W tym rzymskim użyciu feniks należy do cesarza: siada na jego globie, stoi obok personifikowanej Aeternitas albo wieńczy odbudowę jego wieku.

Wersja historii w \emph{1 Klemensie} wygląda podobnie na powierzchni, lecz jej geografia jest zupełnie inna.
Feniks nie leci do Rzymu, na Kapitol ani na jakikolwiek ołtarz imperialny w Italii.
Niesie trumnę z kośćmi przodka z „krainy Arabii do Egiptu, do miasta zwanego Heliopolis” i tam, w pełnym świetle dnia, składa ją na „ołtarzu słońca”, po czym wraca do dawnego miejsca zamieszkania.
Ta trasa — Arabia $\rightarrow$ Heliopolis $\rightarrow$ ołtarz słoneczny — nie jest przypadkową scenerią mityczną.
Dokładnie odpowiada starszym tradycjom egipskim i hellenistycznym, które umieszczają kult feniksa w Heliopolis i łączą go z Bennu, słonecznym ptakiem Ra, „panem jubileuszy”, czczonym w wielkiej świątyni słońca w On.
Powrót feniksa do Heliopolis jest powrotem do centrum kultowego grecko–egipskiego Wschodu, dawnego serca państwa Ptolemeuszów.

Kontrast z monetami imperialnymi jest wyraźny.
Na rzymskich monetach feniks jest związany z osobą Augusta i z „odnowionym wiekiem” Rzymu.
W \emph{1 Klemensie} ten sam ptak dokonuje odnowy, która całkowicie omija Rzym.
Jego ofiara dokonuje się na ołtarzu słonecznym w Egipcie, symbolicznym centrum greckiego Wschodu, a nie w mieście cezarów.
W tym świetle fragment jest co najmniej zawłaszczeniem, a być może cichą inwersją rzymskiej imperialnej ideologii feniksa.
Znak wieczności cesarskiej zostaje opowiedziany na nowo, ze wschodnią trasą i przeniesioną ofiarą.

Klemens Aleksandryjski, piszący około 190 r. n.e., zna ten sam zespół idei i czyni imperialne zabarwienie całkowicie wyraźnym.
W swoich \emph{Stromata} opisuje feniksa jako \emph{porphyrous} (πορφυροῦς), „purpurowego”, używając tego samego przymiotnika, który stosowano do płaszczy cesarskich i królewskich tkanin.
W grece zarówno πορφυροῦς, jak i pokrewne terminy, jak φοῖνιξ, oznaczają głęboką czerwień–purpurę związaną ze statusem wysokiego rodu i rangą królewską.
Literatura męczeńska przejmuje dokładnie ten język w wyrażeniach typu „purpura krwi” (το πορφυρουν αιμα), opisując krew męczennika jako rodzaj królewskiego barwnika.
Feniks, purpurowy od królewskiego upierzenia, i męczennik, odziany w purpurę krwi, zostają wciągnięci w to samo semantyczne pole cierpienia, suwerenności i odnowy kosmicznej.

Na początku III wieku feniks staje się standardowym chrześcijańskim emblematem zmartwychwstania w literaturze i sztuce, lecz jego imperialne podteksty trwają nadal.
Tertulian w \emph{O zmartwychwstaniu ciała} 13 powtarza historię feniksa jako naturalny dowód, że ciała mogą umrzeć i powstać na nowo, explicite traktując ją jako znak „zmartwychwstania naszych ciał”.
W tym samym czasie lub nieco później łaciński poemat \emph{De ave phoenix}, powszechnie przypisywany Laktancjuszowi, rozwija motyw tysiącletniego życia feniksa, jego ognistej śmierci i odrodzenia w rajskim ogrodzie Wschodu; choć nie jest jawnie chrześcijański, został szybko odczytany jako alegoria zmartwychwstania Chrystusa i rajskiej ojczyzny odkupionych.
Obrazy feniksa — zwykle ptaka siedzącego na gałęzi palmy — pojawiają się w III-wiecznej sztuce funeralnej, m.in. w katakumbach Pryscylii (kompleks kaplicy greckiej), jako wizualny skrót idei życia wiecznego.
Średniowieczne opisy mozaiki apsydy starej bazyliki św. Piotra (poświadczone w tradycji \emph{Liber Pontificalis} oraz w notatkach Giacomo Grimaldiego przed renesansową przebudową) wspominają ptaka ponad krzyżem i drzewem życia, powszechnie interpretowanego przez historyków sztuki jako feniksa, który przenosi imperialny język „wieczności” w klucz chrystologiczny.

Późnoantyczna ideologia imperialna nie pozostała w miejscu, gdy to wszystko się działo.
Od IV wieku sami cesarze chrześcijańscy bili monety z feniksem, z legendami jednoznacznie odnowicielskimi, takimi jak FEL TEMP REPARATIO, głoszącymi „pomyślne czasy przywrócone”, łącząc tym samym dawną pogańską symbolikę wieczności z chrześcijańską retoryką opatrznościowej odnowy.
W tym sensie dyskursy feniksa krzyżują się: cesarze przywłaszczają ptaka do własnej chrześcijańskiej autoprezentacji, a biskupi i teologowie coraz częściej włączają go do narracji o zmartwychwstaniu, męczeństwie i przyszłej chwale Kościoła.
W okresie bizantyjskim feniks przetrwa głównie jako obraz literacki i liturgiczny, a nie jako formalny emblemat państwowy, lecz jego funkcja pozostaje taka sama: oznacza wspólnotę lub organizm polityczny, który przeszedł przez śmierć i stanął na nowo.

W tym szerszym kontekście alegoria feniksa w \emph{1 Klemensie} nie jest odosobnioną ciekawostką.
To pierwszy punkt długiej chrześcijańskiej tradycji, w której feniks niezmiennie symbolizuje zmartwychwstanie i życie wieczne — w kazaniach, w poezji łacińskiej, w sztuce katakumbowej, w mozaikach apsyd.
Tym, co wyróżnia \emph{1 Klemensa}, jest to, że droga feniksa wiedzie nie do Rzymu, lecz do Heliopolis, i że jego ofiara składana jest na ołtarzu słońca w Egipcie, a nie na rzymskim ołtarzu imperialnym.
Ten wybór geografii i miejsca kultu był oczywisty dla każdego, kto znał heliopolitańskie tradycje feniksa.
Koduje on — w pozornie niewinnym przykładzie zaczerpniętym z natury — opowieść o odnowie kosmicznej i politycznej zakorzenionej we Wschodzie greckim.
Ten sam ptak, który na rzymskich monetach gwarantuje „wieczność cesarza”, tutaj gwarantuje odnowienie innego porządku, powracając do własnego sanktuarium i ołtarza na Wschodzie, a nie do miasta cezarów.
Nie jest to wyłącznie opowieść o indywidualnym zmartwychwstaniu; to mit odnowy imperialnej, opowiedziany na nowo z perspektywy wspólnoty chrześcijańskiej oczekującej nadejścia nowego wieku i ponownego wyniesienia Wschodu.

\section{Aleksandria w Egipcie doznała ogromnych trudów i prześladowań pod rządami Rzymu.}\label{sec:alexandria-egypt-suffered-enormous-hardship-and-persecution-during-the-roman-empire.}

Dwustuletnia walka Rzymu o podbicie całego świata greckiego zakończyła się nałożeniem ogromnych ciężarów podatkowych na nowo zdobyte terytoria.
Niemal cały dochód podatkowy Imperium Rzymskiego w czasach Jezusa i krótko po nich pochodził z nowo podbitych ziem Greków, a niemal połowa z samego Egiptu.
Często błędnie zakłada się, że Egipt był po prostu tak bogaty i rozwinięty, iż mógł odpowiadać za tak wielką część wpływów podatkowych.
Egipt rzeczywiście był najbogatszy, lecz w żadnym razie nie aż o tak wielki margines.
Podatek był po prostu tak wysoki, aby przetransferować bogactwo Greków do Rzymu i zniszczyć serce świata greckiego wraz z jego możliwościami oporu w przyszłości.
Biorąc pod uwagę trudności, jakich doświadczyli Grecy, nie może budzić wątpliwości, że dążyli do przywrócenia własnego królestwa Bożego, a nie do życia pod panowaniem bestii Rzymu.
Egipt był traktowany jako własność cesarza i nie podlegał zwykłemu systemowi rządów senacko–cesarskich.
Stłumienie greckiej autonomii przez rzymskie podatki stworzyło nie tylko głęboki sprzeciw, lecz także teologiczno–polityczną nadzieję: że imperium Greków, obecnie zmiażdżone rzymskim trybutem, powstanie na nowo jako prawdziwe królestwo Boże.
Dlatego chrześcijaństwo narodziło się całkowicie w formach literackich greckich — każda epistoła, każda ewangelia, każdy traktat apologetyczny — ponieważ należało do intelektualnych i politycznych spadkobierców Aleksandra, a nie spadkobierców Mojzesza.
Zniszczenie Jerozolimy przez Rzym (70 r. n.e.) pozostawiło judaizm rozproszony, lecz greckie miasta Wschodu — Aleksandrię, Antiochię, Efez, Smyrnę — stały się centrami teologii chrześcijańskiej.

\section{Wschodnie Morze Srodziemne: synteza trzystu lat zjednoczonej cywilizacji}\label{sec:eastern-mediterranean-synthesis}

Przed Aleksandrem wschodnie Morze Srodziemne było mozaiką odrębnych ludów z odrębnymi bogami.
Egipcjanie służyli Ozyrysowi, Izydzie i gęstemu panteonowi świątynnemu zakorzenionemu w królewskości faraonów.
Tradycje izraelskie i judejskie mówiły o Elu, Elohim, Jahwe, Baalu i Aszerze, splatając je w nakładających się na siebie nurtach kultu i polemiki.
Grecy czcili Zeusa, Atene, Apollina, Dionizosa i grono olimpijskie w kulcie miejskim.
Miasta fenickie wielbiły Melkarta, Astarte i lokalnych Baalów wzdłuż wybrzeża Lewantu.
Ośrodki syryjskie i mezopotamskie łączyły kult Hadada, Atargatis, Isztar i lokalnych bogów burzy.
Każda grupa miała własny język, własne kapłaństwo, własny kalendarz kultowy i własne instytucje polityczne.

Podboje Aleksandra nie sprowadzały się do prostego narzucenia tym ludom „greckości”.
Wymusiły trwałe współistnienie i zapoczątkowały świadomą politykę fuzji.
Nowi władcy macedońscy promowali jeden wspólny język, grekę, jako medium administracji, filozofii i handlu.
Zakładali miasta mieszane, w których greckie rady i zgromadzenia rządziły ludnością pozostającą etnicznie i religijnie zróżnicowaną.
Ptolemeusze w Egipcie przyjęli tytulaturę faraonów i przedstawiali się jako następcy starożytnych królów egipskich.
Seleukidzi w Syrii i Mezopotamii wydawali dekrety łączące greckie formy prawne z lokalnymi przywilejami kultowymi.
Rządy hellenistyczne tworzyły w ten sposób wspólną cywilizacyjną skorupę, nie wymazując przy tym lokalnej treści.

Od początku wymiana była dwukierunkowa.
Grecy, którzy osiedlili się w Egipcie, zetknęli się ze starszą kulturą kapłańską i czytali ją jako źródło mądrości, a nie barbarzyństwo.
Utożsamili Ozyrysa z Dionizosem i interpretowali mity egipskie w kategoriach własnej filozofii.
Egipskie nauki o sądzie pośmiertnym, zmartwychwstaniu i ważeniu serca weszły do dyskursu greckiego jako czcigodne obrazy ładu moralnego.
Grecy, którzy spotkali Żydów w Aleksandrii i diasporze, napotkali tradycję pism, która głosiła jednego Boga Stwórcę wszystkich narodów.
Usłyszeli historie o Noem, Adamie i Ewie, Abrahamie, Józefie, Mojzeszu i Dawidzie jako uniwersalną historię moralną, a nie tylko folklor jednej grupy plemiennej.

Greckie tłumaczenie Pism hebrajskich w III i II wieku przed Chr. stanowi punkt zwrotny w tym procesie.
Septuaginta została przygotowana przez Żydów, po grecku, na potrzeby środowiska greckojezycznego, w którym Żydzi i nie-Żydzi posługiwali się tym samym językiem.
Gdy Tora, a potem prorocy, psalmy i księgi mądrości pojawiły się w koiné, mogły krążyć wszędzie tam, gdzie przemieszczały się wykształcone elity.
Te teksty można było przepisywać w tych samych skryptoriach, które przechowywały Homera i Platona.
Można było przywoływać je w argumentacji filozoficznej i chwalić za starożytność oraz rygor moralny.
Jednocześnie Żydzi w Aleksandrii i innych poleis przyjmowali grecką retorykę, słownictwo filozoficzne i struktury obywatelskie.
Synagogi funkcjonowały z urzędnikami i zgromadzeniami na wzór grecki, nawet jeśli zachowywały odrębną liturgię i Prawo.
Filon z Aleksandrii stoi na końcu tej drogi, przedstawiając Mojzesza jako archetypicznego filozofa w całkowicie greckiej gramatyce pojęciowej.

Wkład Egiptu w rodzącą się syntezę nie ograniczał się do rytuału i mitu.
Oferował w pełni rozwiniętą teologię królewskości, w której władca ucieleśniał wyższą rzeczywistość boską.
Oferował uporządkowaną eschatologię, w której życie jednostki było ważone po śmierci według standardu sprawiedliwości.
Takie obrazy czyniły czymś naturalnym myślenie o historii, prawie i władzy pod spojrzeniem jednego ładu moralnego.

Miasta fenickie dały infrastrukturę, która spinała ten świat.
Ich porty i szlaki żeglugowe łączyły Egipt, Syrię, Cypr i Morze Egejskie na długo przed Aleksandrem.
Ich wcześniejszy alfabet spółgłoskowy, na wiele stuleci przed epoką hellenistyczną, wpłynął na rozwój alfabetu greckiego.
W I wieku naszej ery codziennym językiem pisanym w tych portach była greka, ale morska sieć przenosząca teksty greckie nadal spoczywała na fenickich fundamentach.

W późnym okresie hellenistycznym i wczesnorzymskim wschodnie Morze Srodziemne dzieliło ten sam zasadniczy zestaw pojęć.
Greka stała się wspólnym językiem edukacji od Aleksandrii po Antiochię, od Efezu po Korynt.
Elity miejskie w całej tej strefie uczyły się tego samego kanonu poetów, historyków i filozofów.
Szkoły filozoficzne coraz częściej mówiły o jednej najwyższej zasadzie albo jednym bogu stojącym ponad wielością kultów.
Stoicki logos, platońskie Jedno i Bóg Izraela mogły być słyszane jako różne akcenty tej samej tezy o źródle kosmosu.
Nawet tam, gdzie dawni bogowie miejscy trwali w praktykach rytualnych, inskrypcje i hymny zaczęły przywoływać „Boga Najwyższego” ponad nimi.

Materiał epigraficzny czyni tę konwergencję widoczną.
W Azji Mniejszej, na Lewancie i na Cyprze od epoki hellenistycznej po czasy cesarstwa rzymskiego pojawiają się dedykacje dla Theos Hypsistos, „Boga Najwyższego”.
Te inskrypcje są pisane po grecku i często pochodzą z nieżydowskich środowisk.
Zazwyczaj są anikoniczne, używając ołtarzy i formuł zamiast pełnych posągów kultowych.
Opisują jednego najwyższego boga językiem, który może zarazem przywoływać filozoficzny monoteizm i żydowskie frazy biblijne.
Zacierają granice między Zeusem, filozoficzną pierwszą zasadą, a Bogiem Izraela, nie utożsamiając ich jednak wprost.

Ta wspólna gramatyka cywilizacyjna okazała się trwalsza niż jakakolwiek pojedyncza dynastia czy sanktuarium.
Egipt przechodził spod władzy rodzimych królów pod Persów, Macedończyków i Rzymian, a jego mądrość kapłańska przetrwała każdy kolejny reżim.
Ateny zostały pokonane przez Sparte w 404 r. przed Chr. i podporządkowane Macedonii w 338 r. przed Chr., a jednak grecka paideia wkrótce kształciła dzieci ich zwycięzców.
Jerozolima została zniszczona przez Babilon w 586 r. przed Chr., a jednak na wygnaniu Żydzi utrwalili Pismo i nauczyli się być ludem bez rodzimej monarchii.
Korynt został złupiony przez Rzym w 146 r. przed Chr., po czym odrodził się jako kolonia rzymska, która szybko odzyskała znaczenie greckojezycznego ośrodka handlowego.
Aleksandria straciła ostatniego władcę z dynastii Ptolemeuszy w 30 r. przed Chr., ale aleksandryjska nauka i teologia wkrótce zaczęły kształtować elity cesarskie w całym basenie Morza Srodziemnego.

W czasach Jezusa wschodnie Morze Srodziemne nie było już patchworkiem odizolowanych kultów etnicznych.
Było jedną oikoumene, w której miasta dzieliły język, kanon edukacyjny i nakładające się na siebie założenia teologiczne.
Lokalni bogowie i tradycje trwali, lecz interpretowano ich teraz w ramach wspólnego schematu zakładającego istnienie jednego Boga najwyższego ponad wszystkimi.
W tej celowo zespolonej i już zjednoczonej cywilizacji pojawiają się najwcześniejsze ekklesie chrześcijańskie i wysuwają swoje roszczenia wobec tytułu prawdziwego władcy świata.

\section{Co dokładnie wydarzyło się około roku 70 n.e.?}\label{sec:what-exactly-happen-near-year-70-ad}

Rok 70 n.e. jest absolutnie kluczowy w badaniach nad wczesnym chrześcijaństwem.
To rok kotwiczny, według którego datuje się ewangelie, oraz moment, w którym chrześcijaństwo rzekomo przeszło od nieudanego ruchu apokaliptycznego do religii duchowej.
Musimy jednak zapytać, dlaczego właśnie ten rok uważa się za tak rozstrzygający.
To był rok, w którym świątynia w Jerozolimie została zniszczona przez Rzymian.
A jednak wszyscy pisarze chrześcijańscy tego okresu, włącznie z Pawłem, żyli daleko od Jerozolimy i świątyni i nie utrzymywali realnej więzi z jej kapłańską religią.
Dlatego powszechne założenie, że zniszczenie świątyni wymusiło całkowitą zmianę teologiczną w chrześcijaństwie, ma bardzo słabe podstawy.
Nie sposób uznać za wiarygodne, że jedno wydarzenie w mieście, w którym niemal żaden z wczesnych autorów nie mieszkał, spowodowało rzekome całkowite przepisanie wszystkich ewangelii, listów i tradycji albo spalenie ich pierwotnych wersji.

Trzeba dostrzec, że zniszczenie świątyni w Jerozolimie było tylko jednym z wielu gwałtownych wydarzeń, które naznaczyły tę samą dekade.
Koniec lat 60. i początek lat 70. n.e. był czasem ogromnej niestabilności w greckim Wschodzie Imperium Rzymskiego.
W Galilei Józef Flawiusz walczył z Rzymianami jako dowódca armii północnej w \textit{pierwszej wojnie żydowsko–rzymskiej} (66--73 n.e.).
W tym samym czasie w Aleksandrii wybuchła zaciekła \textbf{wojna grecko–żydowska} w 66 r. n.e., gdy grecka i żydowska dzielnica miasta pogrążyły się we wzajemnej rzezi; legiony wysłane przez Wespazjana, by przywrócić porządek, ostatecznie zrównały z ziemią znaczną część miasta (Józef Flawiusz, \textit{Wojna} 2.497--507; Filon, \textit{In Flaccum}).
Podobne \textbf{zamieszki międzyetniczne} wybuchły w Antiochii i Damaszku, gdzie greccy i żydowscy mieszkańcy wymordowali się nawzajem w tysiącach (Józef Flawiusz, \textit{Wojna} 7.43--45).
Tymczasem w Efezie, Sardes i w całej Azji Mniejszej dochodziło do \textbf{powstań antyrzymskich} przeciw podatkom i przymusowej rekrutacji, odnotowanych przez Tacyta (\textit{Dzieje} 2.81) i Kasjusza Diona.
W Cyrenajce i w Egipcie \textbf{wojna Kitos} (rozpoczynająca się około 70 r. n.e. i rozpalająca się na nowo za Trajana) kontynuowała tę samą falę niepokojów.
To praktycznie każde główne miasto greckie w Imperium Rzymskim, wszystkie zbuntowane przeciw rzymskim rządom, wszystkie w tym samym czasie.
Nawet w samym Rzymie kilku senatorów i ekwitów oskarżono o zmowę z buntownikami ze Wschodu i stracono za zdradę.
Nie była to odosobniona wojna żydowska — był to ogólny bunt grecki i orientalny przeciw rzymskiej władzy, w którym lokalne konflikty etniczne i powstania antyimperialne stopiły się w jeden wielki rozpad ładu obywatelskiego.

Sama rzymska naczelna komenda widziała w tych buntach jeden system.
Zachowana tradycja takytejska u Sulpicjusza Sewera opisuje radę wojenną Tytusa, debatującą nad tym, czy oszczędzić świątynie, czy ją zniszczyć.
Tytus ostatecznie rozkazuje jej zburzenie „aby religia Żydów i chrześcijan została tym pełniej zdeptana”, ponieważ oba ruchy „wyrastają z tego samego pnia, a jeśli korzeń zostanie zniszczony, gałąź uschnie” (Sulpicjusz Sewer, \textit{Chronica} 2.30.6--7).
To nie teologia.
To rzymski dokument wojenny, który identyfikuje Żydów i ludzi, którzy później otrzymają nazwę „chrześcijanie”, jako jeden blok polityczno–religijny, którego siła opierała się na tym samym centrum w Jerozolimie.

Właśnie na ten okres historycy przypadają najlepsze szacunki dat egzekucji Pawła (ok.~64--67 n.e.), Piotra (ok.~64--67 n.e.) i Jakuba (ok.~62 n.e.).
Wszyscy trzej zginęli w Rzymie, sercu imperium, daleko od Jerozolimy.
Wszyscy zostali straceni za bunt, nie za herezję religijną.

I tu wyłania się punkt krytyczny.
Józef Flawiusz, Paweł, Piotr i Jakub nie byli przeciwnikami na rozproszonym krajobrazie religijnym.
Działali w tym samym świecie politycznym: ruchu odnowy imperialnej, który skupiał greckie, żydowskie i wschodnie elity w jednym antyrzymskim horyzoncie.
Nie dzielili tych samych rytuałów ani szkół, ale służyli tej samej sprawie w różnych rejestrach.
Ich różne „religie” były gałęziami tej samej sprawy królewskiej, dostosowanymi do różnych odbiorców.
Józef Flawiusz w \textit{Testimonium Flavianum} opisuje Jezusa jako przywódcę, który przyciągnął wielu zwolenników.
W odczytaniu królewskim to stwierdzenie nie jest wcale paradoksalne.
Józef pozostał wierny religii Mojzesza, ale uznał także Jezusa za \textit{Christos} — królewskiego przywódcę Galilei — którego ruch, choć pokonany, stanowił część szerszego buntu, który wstrząsnął całym światem greckim około roku 70 n.e.

\subsection{Korelacja geograficzna: sieć ekklesii jako infrastruktura polityczna}

Związek między ustaleniami geograficznymi z Rozdziału 5 a powstaniami około roku 70 n.e. jest nie do uchwycenia w inny sposób niż jako ścisła zależność.
Rozdział 5 pokazał, że wczesne chrześcijańskie \emph{ekklesiai} pojawiły się we \emph{wszystkich} głównych miastach greckich — Aleksandrii, Antiochii, Efezie, Sardes, Smyrnie, Koryncie, Filipii, Tesalonice — i w \emph{zeru} głównych miastach łacińskojezycznych.
Niniejszy rozdział pokazuje, że w późnych latach 60. i wczesnych 70. n.e. \emph{dokładnie te same miasta} — Aleksandria, Antiochia, Efez, Sardes oraz miasta całej Azji Mniejszej — stanęły w ogniu skoordynowanych powstań antyrzymskich.

Nie może to być zbieg okoliczności.

Problem nie sprowadza się do prostej korelacji.
Miasta, które wybuchły buntem w latach 60. — Aleksandria, Antiochia, Efez, Sardes — to te same miasta, które tworzyły hellenistyczną monoteistyczną sieć miejską w greckim Wschodzie.
W każdym węźle pojawia się ta sama wspólna ideologia: jeden Bóg, prawo przodków, odrzucenie kultu cezara i oczekiwanie władcy uprawomocnionego przez Boga.
W każdym węźle widzimy tę samą wspólną formę instytucjonalną: greckie zgromadzenia obywatelskie, rady i stowarzyszenia dobrowolne — \emph{ekklesiai} — a nie wiejskie synagogi czy plemienne milicje.
W całym kryzysie przewijają się te same kategorie osób: wykształceni po grecku monoteiści, kapłani, nauczyciele filozofii, elity miejskie, a nie odizolowani judejscy zeloci.
I spina je ta sama chronologia: Aleksandria 38, Antiochia lata 40., Azja Mniejsza w latach 50., Judea 66--73.
To fronty tego samego świata politycznego, a nie odosobnione przypadki.

Sieć \emph{ekklesii} nie była stowarzyszeniem religijnym, które przypadkowo pokryło się z granicami politycznymi; stanowiła część tej samej strukturalnej infrastruktury, która wydała z siebie bunt.
Gdy nanosimy na mapę miasta wymienione w Dziejach Apostolskich i Listach oraz miasta, które zbuntowały się w latach 66--73 n.e., pokrycie jest niemal doskonałe.
Zgromadzenia, do których Paweł kierował swoje listy w Koryncie, Efezie, Filipii i Tesalonice, nie były biernymi wspólnotami religijnymi; były węzłami sieci restauracyjnej, która po uruchomieniu wyzwoliła skoordynowane powstanie w greckojezycznym Wschodzie.

Statystyczna nieprawdopodobność tego wzorca jest trudna do przecenienia.
Gdyby wczesne wspólnoty chrześcijańskie były po prostu kościołami misyjnymi szerzącymi się dzięki ustnej ewangelizacji, należałoby oczekiwać losowego rozmieszczenia geograficznego, stopniowej dyfuzji ponad granicami językowymi i politycznymi oraz braku korelacji z wydarzeniami politycznymi.
Tymczasem obserwujemy: (1) wyłączną koncentrację na dawnych terytoriach imperium greckiego, (2) równoczesne pojawienie się we wszystkich głównych miastach greckich do lat 60. n.e., (3) skoordynowane powstania dokładnie w tych miastach w latach 66--73 n.e. oraz (4) natychmiastowy rozpad tej sieci po stłumieniu buntów, a następnie (5) jej odtworzenie dopiero wtedy, gdy patronat cesarski przesunął się ku Wschodowi za Konstantyna.

Nie jest to wzorzec ruchu religijnego.
Jest to wzorzec programu restauracji politycznej wykorzystującego zgromadzenia obywatelskie (\emph{ekklesiai}) jako mechanizm organizacyjny.

Nałożenie mapy powstań i mapy ekklesii nie jest przypadkiem ani prostą korelacją.
Odbija ono pojedynczą infrastrukturę hellenistyczno–monoteistyczną, rozciągniętą od Aleksandrii przez Antiochie i Azje Mniejszą aż po Jerozolime.
Zgromadzenia, na które natrafiał Paweł, nie były zgromadzeniami religijnymi unoszącymi się ponad polityką.
Były miejskimi, greckimi instytucjami obywatelskimi, przez które działał każdy poważny ruch na Wschodzie — filozoficzny, polityczny czy powstańczy.
Gdy nadeszła dekada ognia, to właśnie te miasta budziły strach Rzymu, to one wybuchły i to je Tytus wyraźnie utożsamił z tym samym ideologicznym korzeniem, który zamierzał odciąć.
\emph{Ekklesiai} nie były widownią buntu; należały do tego samego świata, który go zrodził.
Gdy Rzym niszczył świątynie „aby religia Żydów i chrześcijan została obalona”, nie rozróżniał wyznań; miażdżył polityczną architekturę całego grecko–monoteistycznego Wschodu.

Ostateczne zwycięstwo przyszło trzy stulecia później, gdy wschodnie cesarstwo rzymskie (Bizancjum) zostało ustanowione jako chrześcijańskie imperium.

\section{Literatura martyrologiczna: uniwersalna literatura ludów okupowanych}\label{sec:martyrdom-literature-occupied-peoples}

Współczesna nauka traktuje ogromną obfitość literatury martyrologicznej we wczesnym chrześcijaństwie jako zjawisko religijne — dowód wiary, pobożności i gotowości umierania za doktrynę.
Takie ujęcie ignoruje jednak najbardziej oczywisty kontekst porównawczy: literatura męczeńska jest standardową produkcją literacką \emph{wszystkich} ludów okupowanych, które stawiają opór dominacji imperialnej.

Wszędzie tam, gdzie przyglądamy się krajom okupowanym o silnej tożsamości narodowej — Polsce w okresie zaborów, Wietnamowi pod rządami francuskimi, Algierii pod okupacją francuską, Irlandii pod dominacją angielską — znajdujemy te same cztery elementy: obfitą literature martyrologiczną, program wyzwolenia narodowego poprzez moralną wyższość nad okupantem, kult przywódców duchowych i bohaterów narodowych zarówno z czasów niedawnych, jak i odległych oraz zintensyfikowaną organizacje religijną skupioną wokół zgromadzeń wspólnotowych.

Męczennicy są czczeni nie za osobistą świętość, lecz jako symbole zbiorowego oporu.
Ich śmierć dowodzi, że naród nadal żyje i zostanie odnowiony.
Samo męczeństwo jednak nie wystarcza: lud okupowany musi także \emph{stać się lepszy} niż jego ciemiężcy — bardziej cnotliwy, bardziej zdyscyplinowany, bardziej godny suwerenności.
Musi odwoływać się do dawnych przywódców — starożytnych królów, proroków, wojowników i mędrców — jako dowodu dawnej chwały i wzorców dla przyszłej odnowy.
Musi też regularnie gromadzić się w kościołach, świątyniach, domach ludowych czy konspiracyjnych zgromadzeniach, aby zachować tożsamość zbiorową, gdy bezpośrednia działalność polityczna jest tłumiona.
Taki jest standardowy program ruchów wyzwoleńczych: moralna i obywatelska przemiana, inspirowana przez bohaterskich przodków, podtrzymywana przez wspólnotowe zgromadzenia religijne, jako przygotowanie do restauracji politycznej.

Polska, rozebrana i okupowana w latach 1795–1918, daje najczystszy przykład.
Polski romantyzm — zwłaszcza twórczość Adama Mickiewicza, Juliusza Słowackiego i Zygmunta Krasińskiego — stworzył ogromny korpus narracji martyrologicznych, który stał się fundamentem polskiej tożsamości narodowej.
Koncepcja Polski jako „Chrystusa narodów” (\emph{Chrystus narodów}) — narodu, który znosi ukrzyżowanie pod obcym panowaniem, ale zmartwychwstanie — przenikała poezję, dramat i prozę w całym okresie zaborów.
Nie była to mistyka religijna; była to literatura polityczna zaszyfrowana w obrazach chrześcijańskich.
Męczennicy tacy jak Tadeusz Kościuszko, straceni lub wygnani za przewodzenie oporowi, stali się świeckimi świętymi, których cierpienie dowodziło dalszego istnienia Polski jako narodu.
Gdy w 1918 r. niepodległość została przywrócona, polityczna funkcja tego gatunku w znacznej mierze wygasła.

Literatura wietnamskiego i algierskiego ruchu oporu w podobny sposób ujmowała ich zmagania jako moralne oczyszczenie przygotowujące drogę do niepodległości.
Schemat jest uniwersalny: ludy okupowane tworzą narracje martyrologiczne i głoszą samodoskonalenie nie jako cele same w sobie, lecz jako drogę do restauracji narodowej.

Wczesnochrześcijańska literatura martyrologiczna dokładnie wpisuje się w ten wzorzec.
\emph{Akta męczenników}, \emph{Męczeństwo Polikarpa}, \emph{Męka Perpetuy i Felicyty} i niezliczone inne teksty nie były przede wszystkim dokumentami teologicznymi; były literaturą polityczną tworzoną przez lud okupowany, by podtrzymać nadzieję na restaurację.
Męczennicy umierali nie za abstrakcyjną doktrynę, lecz za odmowę uznania rzymskiej władzy imperialnej za ostateczną.
Skazywano ich za \emph{maiestas} — zdradę majestatu cesarza — pod tym samym zarzutem, pod którym zginęli Paweł, Piotr i Jakub.

Kluczowy szczegół polega na tym, że literatura martyrologiczna rodziła się niemal wyłącznie we greckojezycznym Wschodzie, na tych samych terytoriach dawnego imperium greckiego, które dokumentuje Rozdział 5.
Łacińskie martyrologia pojawiają się znacznie później i mają wyraźnie pochodny charakter.
Pierwotny korpus pism o męczennikach pochodzi ze Smyrny, Antiochii, Aleksandrii, Azji Mniejszej — z miast, które zbuntowały się w latach 66--73 n.e. i które później utworzyły rdzeń cesarstwa bizantyjskiego.

To nie jest przypadek.
Literatura martyrologiczna rozkwita tam, gdzie restauracja polityczna pozostaje żywą nadzieją.
Gdy wschodnie cesarstwo rzymskie zostało skutecznie schrystianizowane za Konstantyna, gatunek uległ przekształceniu: męczennicy przestali być buntownikami przeciw Rzymowi, a stali się obrońcami ortodoksji przeciw herezji.
Polityczna funkcja męczeństwa — podtrzymywanie nadziei na wyzwolenie pod okupacją — zniknęła, gdy wyzwolenie zostało osiągnięte.

Ogromna liczba chrześcijańskich tekstów martyrologicznych w pierwszych trzech stuleciach nie jest więc dowodem niezwykłej pobożności, lecz trwałego oporu politycznego.
To literatura ludu, który wierzył, że jego królestwo zostanie przywrócone, który w każdym męczenniku widział dowód, że Bóg go nie opuścił, i który ostatecznie doprowadził do powstania chrześcijańskiego cesarstwa bizantyjskiego.
Męczennicy nie ginęli za samo niebo; ginęli za restaurację królestwa Bożego w świecie greckojezycznym.

\section{Symbol Chi-Rho i wschodnia restauracja imperialna Konstantyna}\label{sec:chi-rho-constantinan-restoration}

Utworzenie wschodniego cesarstwa rzymskiego jako królestwa Bożego rządzonego przez Chrystusa, przez chrześcijańską armię chrześcijańskiego cesarza Konstantyna, było wydarzeniem o ogromnym znaczeniu w dziejach świata.
W tym momencie symbolem żołnierzy nie był krzyż.
Symbolem żołnierzy był znak Chi-Rho, będący symbolem Christosa.
Zwróćmy uwagę, że tarcze oznaczały wyłącznie Christosa, a nie bezpośrednio Jezusa, co mogło stanowić oznaczenie: „jesteśmy żołnierzami Christosa", prawowitego króla królestwa Bożego, oraz znak bitew o restaurację królestwa Christosa.
Gdyby wojnę pojmowano przede wszystkim jako religijną, zapewne widzielibyśmy dużo silniejszy nacisk na krzyż i masowe konwersje na chrześcijaństwo, czego jednak nie obserwujemy w przekazach historycznych dotyczących właśnie tej wojny.
Widzimy natomiast silne podkreślenie zmian politycznych i przywrócenia rządów na Wschodzie.
Użycie przez Konstantyna znaku Chi-Rho na monetach, tarczach i sztandarach nie było osobistą dewocją wobec Jezusa z Nazaretu, lecz stworzeniem chrystycznego monogramu imperialnego — znaku, że samo imperium należy teraz do Christosa, namaszczonego władcy.
Euzebiusz (\emph{Życie Konstantyna}, Księga 1) podkreśla, że imperium zostało odnowione pod rządami Chrystusa, jasno pokazując, iż chrześcijaństwo funkcjonowało jako religia legitymizująca nowy porządek cesarski.

Chronologia jest kluczowa: chrystianizacja Rzymu zbiegła się dokładnie z podziałem imperium na część zachodnią i wschodnią.
Restauracja wschodniego cesarstwa rzymskiego była jawnym celem tego ruchu, co potwierdzają pisma Euzebiusza.
Nazywa on Konstantynopol „Nowym Rzymem” zgodnym z wolą Bożą, przedstawiając podział imperium i chrystianizację Wschodu jako jedno, zintegrowane wydarzenie — wypełnienie teologii restauracji, która napędzała wczesne pisma chrześcijańskie przez trzy stulecia.

\section{Apokalipsa: bestia Rzymu i królestwo przywrócone}\label{sec:revelation-political-restoration}

\emph{Uwaga: Rozdział~\ref{subsec:revelation-as-imperial-restoration-by-revolt-from-rome} traktuje Apokalipsę jako dokument operacyjny — siedem miast jako obwód inspekcyjny, pieczęcie i trąby jako plan kampanii, znamię jako administrację kontrpaństwową.
Niniejsza sekcja traktuje Apokalipsę jako teologię restauracji: pokazuje, jak jej wizja upadku Babilonu i zstąpienia Nowego Jeruzalem wpisuje się w szerszy łuk greckich nadziei imperialnych.}

Apokalipsa (ok.~95 n.e.) łączy obie funkcje: jest zarówno operacyjnym schematem oporu, jak i teologiczną wizją restauracji.

\section{Orygenes z Aleksandrii (ok.~185--254 n.e.): uniwersalne królestwo przez Logos}\label{sec:origen-universal-kingdom}

Orygenes, pisząc z Aleksandrii — intelektualnego centrum greckiego Wschodu — formułuje wizję kosmicznej restauracji poprzez Logos, która łączy filozofię grecką z chrześcijańską eschatologią.
W \emph{Contra Celsum} wprost broni chrześcijaństwa przed oskarżeniami o zdradę stanu, argumentując, że chrześcijanie nie dążą do obalenia Rzymu siłą, lecz do jego przemiany przez prawdę Bożą.
Ale przemiany ku jakiemu celowi?

Orygenes pisze w \emph{Contra Celsum} 8.68:

\begin{quote}
Mówimy, że ostatecznie nadejdzie czas, gdy Słowo zatriumfuje nad całym światem istot rozumnych i doprowadzi każdą duszę do swojej doskonałości, kiedy to przekaże królestwo Bogu Ojcu, gdy wszystkie rozumne byty zostaną ukształtowane przez Słowo Boże w jedną doskonałość.
\end{quote}

Nie jest to ucieczka do innego świata.
Orygenes widzi czas, gdy „Słowo zatriumfuje nad całym światem istot rozumnych” — nad \emph{oikoumene}, światem cywilizowanym.
„Królestwo”, które Chrystus „przekaże Bogu Ojcu”, jest rzeczywistością polityczną: powszechnym królestwem zjednoczonym pod władzą Boża.

Koncepcja \emph{apokatastasis} (ἀποκατάστασις) — przywrócenia wszystkiego — jest często uduchawiana przez współczesnych interpretatorów, ale język Orygenesa jest jawnie polityczny.
Opisuje on ostateczne poddanie wszystkich \emph{archontes} (ἄρχοντες, władcy) i \emph{exousiai} (ἐξουσίαι, zwierzchności) Chrystusowi (\emph{De Principiis} 3.6.6), używając słownictwa administracji cesarskiej.
To teologia restauracji: rekonstrukcja politycznego kosmosu uporządkowanego przez Boga, skoncentrowanego we greckojezycznym Wschodzie, gdzie pisał Orygenes.

\section{Ireneusz z Lyonu (ok.~130--202 n.e.): opatrzność i nadchodzące królestwo}\label{sec:irenaeus-providence-kingdom}

Ireneusz, biskup Lyonu (greckojezycznego miasta w Galii), przedstawia millenarystyczną eschatologię w \emph{Przeciw herezjom}, w której oczekuje dosłownego królestwa ziemskiego rządzonego przez Chrystusa.
Pisząc około 180 r. n.e., Ireneusz interpretuje wizję czterech królestw z Księgi Daniela (Dn 2,31–45) jako proroctwo następstwa imperiów — babilońskiego, medyjskiego, perskiego i wreszcie rzymskiego — przy czym królestwo Boże jest piątym, ostatecznym królestwem, które zmiażdży i zastąpi wszystkie władze ziemskie.

W \emph{Przeciw herezjom} 5.26.1 Ireneusz pisze:

\begin{quote}
Na końcu kamień odcięty bez udziału rąk [z Daniela] uderzy w posag po nogach z żelaza i gliny, rozbije je na kawałki, a sam wypełni całą ziemię.
Odnosi się to do królestwa naszego Pana... które rozbije i położy kres wszystkim królestwom, a samo trwać będzie na wieki.
\end{quote}

Nie jest to alegoria duchowa — to jawne proroctwo polityczne.
„Kamień”, który uderza w nogi z żelaza i gliny (Rzym), to królestwo Chrystusa, które fizycznie „wypełni całą ziemię" i „rozbije... wszystkie królestwa”.
Ireneusz rozumie to jako ustanowienie tysiącletniego królestwa na ziemi, w którym zmartwychwstali święci będą królować z Chrystusem (5.32.1--5.35.2), w bezpośredniej paraleli do Ap 20,4–6.

Kluczowy jest wymiar geograficzny: królestwo ma zostać przywrócone w Jerozolimie i ziemiach ościennych, ale Jerozolima pełni tu funkcję symbolicznej stolicy odnowionej \emph{oikoumene} — świata cywilizowanego znajdującego się teraz pod władzą Boża, a nie pod rzymskim imperium.

\section{Tertulian (ok.~155--240 n.e.)}\label{sec:tertullian-c.-155240-ad---in-his-writings-such-as-apology-and-on-the-resurrection-of-the-flesh-tertullian-often-implies-the-eventual-triumph-of-christianity-within-the-roman-empire-framing-it-as-part-of-a-divine-plan.-while-he-doesnt-directly-speak-of-the-restoration-of-the-empire-there-is-a-sense-of-christianity-fulfilling-the-destiny-of-the-roman-state.}

W swoich pismach, takich jak \emph{Apologetyk} i \emph{O zmartwychwstaniu ciała}, Tertulian często sugeruje ostateczne zwycięstwo chrześcijaństwa wewnątrz Imperium Rzymskiego, ujmując je jako element planu Bożego.
Choć nie mówi wprost o „restauracji” cesarstwa, pojawia się u niego wizja chrześcijaństwa wypełniającego przeznaczenie państwa rzymskiego.

\section{Euzebiusz z Cezarei (ok.~260--340 n.e.)}\label{sec:eusebius-of-caesarea-c.-260340-ad}

W swoich dziełach \emph{Historia kościelna} i \emph{Życie Konstantyna} Euzebiusz explicite przedstawia wzrost znaczenia Konstantyna i ustanowienie chrześcijaństwa jako wypełnienie planu Bożego wobec Imperium Rzymskiego.
Postrzega rządy Konstantyna jako „restaurację" cesarstwa, uzgadniającą jego strukturę z wolą Bożą.
Wyraża to przekonanie, że chrześcijaństwo nie tylko odnowi cesarstwo, lecz doprowadzi je do jego prawdziwego, chrześcijańskiego celu.

\section{Atanazy z Aleksandrii (ok.~296--373 n.e.)}\label{sec:athanasius-of-alexandria-c.-296373-ad}

W swoich pismach, zwłaszcza w polemice z arianizmem i dziełach teologicznych, Atanazy mówi o kosmicznym zwycięstwie Chrystusa nad złem, co ma doniosłe konsekwencje dla restauracji imperium.
Często ujmuje cesarza chrześcijańskiego jako prawowitego władce działającego pod przewodnictwem Bożym, co można odczytać jako powiązanie restauracji cesarstwa ze zwycięstwem Chrystusa.

\section{Wiktoryn z Petawium (ok.~250--303 n.e.)}\label{sec:victorinus-of-pettau-c.-250303-ad}

W swoim \emph{Komentarzu do Apokalipsy} Wiktoryn rysuje zależności między Imperium Rzymskim a ostatecznym triumfem chrześcijaństwa.
Jak wielu jego współczesnych, uważa, że cesarstwo jest częścią planu Bożego i że jego ostateczne przeobrażenie w imperium chrześcijańskie doprowadzi do wypełnienia proroctw.
Można to rozumieć jako formę „restauracji" poprzez chrystianizację cesarstwa.

\section{Bunt nie jest militarny — Rzym wybiera duchowe nawrócenie na chrześcijaństwo, a wschodnie cesarstwo rzymskie zostaje przywrócone pokojowo, podczas gdy pisarze chrześcijańscy zaczynają wychwalać Rzym.}\label{sec:the-revolt-is-not-militaristic-rome-chose-to-spiritually-convert-to-christianity-and-the-eastern-roman-empire-was-restored-peacefully-while-the-christian-writers-started-to-praise-rome.}

Laktancjusz, piszący krótko przed zwycięstwem Konstantyna, ogłasza w \emph{Bożych naukach}, że przeznaczeniem Rzymu jest stać się chrześcijańskim i że samo cesarstwo jest narzędziem Boga do zjednoczenia świata.
Augustyn z Hippony (354--430 n.e.) rozszerza tę wizję w \emph{Państwie Bożym}, gdzie upadek Imperium Rzymskiego ujmuje w perspektywie chrześcijańskiej.
Twierdzi, że schyłek cesarstwa nie oznacza porażki Opatrzności.
Choć skupia się na duchowym wymiarze imperium, uznaje jego rolę w przygotowaniu świata na królestwo chrześcijańskie, sugerując „restaurację" Imperium Rzymskiego jako przyszłego bytu chrześcijańskiego.

\section{Pasterz Hermasa: najpopularniejszy tekst chrześcijański i nostalgia za grecką chwałą}\label{sec:the-shepherd-of-hermas-c.-100-160-ad}

\emph{Pasterz Hermasa} (ok.~100--160 n.e.) nie był jednym z wielu wczesnych tekstów chrześcijańskich — był \emph{najpowszechniej czytanym dziełem chrześcijańskim} w II i III wieku, bardziej popularnym niż większość tekstów, które później uznano za kanoniczne.
Świadectwa rękopiśmienne, cytaty u Ojców Kościoła i wczesne katalogi chrześcijańskich bibliotek dowodzą, że \emph{Hermas} krążył szerzej niż Ewangelie Marka czy Jana w wielu regionach, był włączany do niektórych wczesnych zbiorów Nowego Testamentu (Kodeks Synajski) i był traktowany jako Pismo przez Ireneusza, Tertuliana, Orygenesa i Klemensa Aleksandryjskiego.

Ta niezwykła popularność domaga się wyjaśnienia.
Współczesna nauka traktuje \emph{Hermasa} jako nieco bezbarwną alegorię moralną o pokucie i dyscyplinie kościelnej, co wydaje się mało nośne jako podstawa masowej recepcji.
Odczytanie polityczne ujawnia jednak, dlaczego tekst tak silnie rezonował z greckojezycznymi odbiorcami pod rządami rzymskimi: \emph{Hermas} jest rozbudowaną alegorią restauracji imperialnej, kodującą grecką nostalgię za utraconą suwerennością w obrazach starzejącej się kobiety odzyskującej młodość, zrujnowanej wieży odbudowywanej do dawnej chwały oraz ludu oczyszczonego i przywróconego do należnego mu miejsca.

\subsection{Wizja przemienionej starej kobiety: restauracja greckiej witalności}

Centralna wizja \emph{Hermasa} (Widzenia 1--4) przedstawia starą kobietę, która stopniowo młodnieje.
W pierwszym widzeniu ukazuje się jako osoba w podeszłym wieku, siedząca na krześle i trzymająca księge.
W kolejnych widzeniach staje się coraz młodsza, jej twarz nabiera wigoru, aż w czwartym widzeniu pojawia się jako jaśniejąca oblubienica.
Hermas pyta Pasterza, kim jest ta kobieta, i otrzymuje odpowiedź: „Kościołem”.

Tekst mówi, że kobieta jest „Kościołem”, ale dokładnie tak przemawia alegoria polityczna w starożytności.
Platon ukrywa plany ustrojowe pod językiem duszy.
Ezop ukrywa sztukę rządzenia pod bajkami o zwierzętach z morałem.
4 Księga Ezdrasza ukrywa antyrzymskie proroctwo w wizjach wyjaśnianych jako wezwania do pobożności.
Sybilliny ukrywają ideologię powstańczą pod płaszczem kosmicznych przepowiedni.
Apokalipsa ukrywa krytykę imperialną w głosie Jezusa.
Religijny komentarz nie jest znaczeniem; jest kamuflażem, który pozwala znaczeniu przetrwać pod rządami Rzymu.

Ale \emph{którego} Kościoła i \emph{jakiego} rodzaju przemiana?

Współcześni interpretatorzy uduchawiają ten obraz jako moralną odnowę Kościoła przez pokutę.
Tymczasem obraz jest jawnie polityczny: stara kobieta, która staje się młoda, nie oznacza poprawy moralnej, lecz \emph{przywrócenie młodzieńczej siły} — dokładnie takiego języka używano w greckim i rzymskim dyskursie politycznym na określenie odnowy imperium.
Łacińskie \emph{renovatio imperii} i greckie \emph{ananeosis tes arkhes} (ἀνανέωσις τῆς ἀρχῆς) opisują imperium przywrócone do dawnej potęgi po okresie upadku.

Stara kobieta reprezentuje greckojęzyczny świat pod rządami Rzymu: zestarzały, osłabiony, podbity.
Jej przemiana w młodą oblubienicę oznacza restaurację tego świata do dawnej chwały — ponowne ustanowienie suwerenności greckiej, odnowę hellenistycznej potęgi imperialnej.
Nie jest to abstrakcyjna eklezjologia; to alegoria polityczna, typowa dla literatury krążącej pod okupacją, gdy mowa wprost była niebezpieczna, a literatura symboliczna mogła kodować opór.

Stopniowe odmładzanie kobiety odzwierciedla strukturę ruchów oporu: najpierw rozpoznanie upadku (stara kobieta, podbita i ujarzmiona), potem nadzieja na restaurację (jej stopniowe odmładzanie), wreszcie wizja pełnej odnowy (jaśniejąca oblubienica).
To dokładny schemat narracyjny literatury ludów okupowanych: utrata, opór i obiecana restauracja.

\subsection{Wizja wieży: odbudowa greckiej oikoumene}

Najbardziej rozbudowaną alegorią w \emph{Hermasie} jest wizja wznoszonej wieży (Widzenie 3, Podobieństwo 9).
Wieża wznoszona jest z kamieni, z których część zostaje odrzucona, a część wymaga obróbki i oczyszczenia, zanim będzie nadawać się do wbudowania w konstrukcję.
Wieża reprezentuje „Kościół”, ale obraz architektoniczny ma jednoznacznie grecki charakter: wieża (\emph{pyrgos}, πύργος) zbudowana z starannie dopasowanych kamieni (\emph{lithoi}, λίθοι) przywołuje monumentalną architekturę greckiego miasta.

Miasta greckie na obszarze dawnego państwa Ptolemeuszów i Seleukidów — Aleksandria, Antiochia, Efez, Pergamon — wyróżniały się monumentalnymi wieżami, bramami i kolumnowymi ulicami.
Pod rządami Rzymu wiele z tych struktur popadło w ruinę, a ich utrzymanie zaniedbano, gdy trybut płynął do Rzymu, a nie na lokalne inwestycje miejskie.
Wizja wieży odbudowywanej kamień po kamieniu, z rygorystyczną selekcją i oczyszczaniem materiału, jest przejrzystą alegorią odbudowy greckojezycznego świata: każdy kamień to miasto lub region, każdy akt oczyszczenia to przygotowanie do restauracji politycznej.

Co istotne, kamienie odrzucane lub wyrzucane to te, których nie można wbudować w strukturę — ci, którzy odmawiają uczestnictwa w programie restauracji.
Nie chodzi tu o indywidualne zbawienie; chodzi o zbiorowy udział w odnowionym ciele politycznym.
Język oczyszczenia i pokuty (\emph{metanoia}) w \emph{Hermasie} ma charakter obywatelski, nie tylko moralny: wezwanie jest wezwaniem do stania się kamieniem godnym miejsca w odbudowanej wieży, do przygotowania się na restaurację królestwa.

\subsection{Dlaczego Hermas był najpopularniejszym tekstem chrześcijańskim?}

Niezwykła popularność \emph{Hermasa} we wschodnim świecie greckim staje się zrozumiała, gdy czytamy go jako alegorię restauracji.
Był popularny właśnie dlatego, że w sugestywnych i poruszających obrazach wyrażał centralną nadzieję polityczną Greków pod okupacją: że ich świat — postarzały i pokonany — zostanie przywrócony do młodzieńczej witalności, że ich miasta — zrujnowane i zaniedbane pod rządami Rzymu — zostaną odbudowane w chwale, a oni sami zostaną oczyszczeni i uczynieni godnymi tej restauracji.

Tekst nie wzywa do natychmiastowego buntu zbrojnego.
Głosi \emph{metanoia} — pokutę, przemianę, cnoty obywatelskie — jako drogę do restauracji.
To standardowy program ruchów oporu pod okupacją: odnowa moralna i obywatelska jako przygotowanie do wyzwolenia politycznego.
Wizja jest gradualistyczna, nie rewolucyjna, ale cel ostateczny jest jasny: wieża zostanie ukończona, kobieta w pełni odnowiona, a wierni będą panować w odnowionym królestwie.

Dlatego \emph{Hermas} krążył szerzej niż większość tekstów kanonicznych w II i III wieku.
Nie był traktatem teologicznym; był popularną literaturą restauracji imperium greckiego, zapisaną w formie alegorycznej, by ominąć cenzurę rzymską, ale przejrzystą dla zamierzonych odbiorców.
Grecy, którzy przepisywali, rozpowszechniali i czcili \emph{Hermasa}, doskonale rozumieli, co obiecuje: restaurację ich królestwa, odbudowę ich miast i potwierdzenie, że Bóg nie porzucił ich na zawsze w stanie poddaństwa.

Gdy Konstantyn w IV wieku ustanowił chrześcijańskie cesarstwo bizantyjskie, popularność \emph{Hermasa} zaczęła słabnąć.
Wizja została spełniona: wieża została odbudowana jako Konstantynopol, stara kobieta odnowiona jako młode imperium chrześcijańskie, a kamienie zostały dopasowane, gdy miasta greckiego Wschodu włączono na powrót w chrześcijański porządek imperialny.
Tekst spełnił swoją rolę, a jego funkcja polityczna — podtrzymywanie nadziei w czasie okupacji — przestała być konieczna.

\section{Klemens Aleksandryjski i \emph{Zachęta dla Greków} (ok.~190 n.e.)}\label{sec:clement-of-alexandrias-exhortation-to-the-greeks-c.-190-ad}

W swoich pismach Klemens łączy chrześcijańską eschatologię z filozofią grecką, głosząc powrót „Logosu" i ostateczną restaurację ludzkości do harmonii z Bogiem.
Choć nie jest apokaliptyczny w sensie wizji na wzór Apokalipsy, jego wizja przyszłości współbrzmi z ideą kosmicznej odnowy.
Apokaliptyczne motywy u Klemensa obejmują ostateczne odnowienie świata przez Logos, idee, która wiąże się z przywróceniem ładu Bożego w sposób zbliżony do eschatologicznych wizji znanych z Apokalipsy.

\section{\emph{O upadłych} Cypriana z Kartaginy (ok.~250 n.e.)}\label{sec:cyprian-of-carthages-the-lapsed-c.-250-ad}

Cyprian pisze o prześladowaniach chrześcijan i bliskim powrocie Chrystusa.
Jego dzieła zapowiadają sąd ostateczny, zwycięstwo sprawiedliwych i ustanowienie królestwa Bożego.
Podobnie jak inni wczesnochrześcijańscy autorzy apokaliptyczni, Cyprian wierzył, że Kościół zostanie odnowiony i zatriumfuje nad swoimi prześladowcami, odzwierciedlając szerszą nadzieję apokaliptyczną obecną w Apokalipsie.
