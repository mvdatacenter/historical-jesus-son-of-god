Niewiele postaci w dziejach doczekało się tak wielu konkurujących ze sobą wizerunków jak Jezus z Nazaretu.
Równie mało ruchów religijnych zostało poddanych tak szczegółowej analizie, tylu dyskusjom i tak wielu reinterpretacjom jak wczesne chrześcijaństwo.
Przez stulecia badań uczeni i teologowie udzielali skrajnie odmiennych odpowiedzi na to samo pytanie: Kim był Jezus i skąd wziął się ruch, który działał pod jego imieniem?

Jedni widzą w nim proroka apokaliptycznego, inni nauczyciela moralności, rewolucyjnego zelotę, filozofa cynickiego, mistyka, żydowskiego mesjasza, proroka Allaha, całkowicie mityczną postać albo wręcz realną istotę nadprzyrodzoną.
Każdy z tych obrazów opiera się na pewnych świadectwach i każdy stworzył własną tradycję interpretacyjną.

A jednak żaden z nich nie okazał się ostateczny.
Każda rekonstrukcja uchwyca jakiś istotny element, ale każda pęka pod naporem pełnego materiału dowodowego.
Nawet ujęcie, które przez ostatnie dekady często uchodziło za akademicki punkt wyjścia — prorok apokaliptyczny — jest stale podważane, a przeciwstawiają mu się mocne argumenty wysuwane przez licznych historyków, teologów i filozofów.
„Poszukiwania historycznego Jezusa” wcale się nie zakończyły; przeciwnie, muszą być podejmowane raz po raz od ponad dwóch stuleci — nie dlatego, że pytanie jest błahe, lecz dlatego, że żadna dotychczasowa odpowiedź nie potrafi objąć wszystkich danych bez wewnętrznych sprzeczności.

Paradoks jest uderzający: Jezus i początki chrześcijaństwa należą do najlepiej udokumentowanych zjawisk starożytności, a mimo to nie powstała rekonstrukcja, która utrzymałaby wszystkie świadectwa w spójnym porządku.
Niniejsza książka przyjmuje inne podejście.
Jej podstawą jest wykorzystanie sztucznej inteligencji do syntezy archiwum większego, niż mógłby ogarnąć ktokolwiek: dwóch tysięcy lat badań historycznych, spekulacji teologicznych, krytyki tekstów, znalezisk archeologicznych i komentarzy kulturowych.
Każde pokolenie uczonych zostawiło nam wglądy, ale także ślepe punkty.
SI pozwala zobaczyć je wszystkie naraz — dostrzec wzorce, uwidocznić sprzeczności i wydobyć pytania, które dotąd leżały na wyciągnięcie ręki, lecz pozostawały niezauważone.

Dzięki syntezie wspomaganej SI możemy stale rewidować ogromną liczbę szczegółowych hipotez na temat Jezusa i wczesnego chrześcijaństwa.
Każdą hipotezę można szybko skonfrontować z całym dostępnym zasobem danych.
A zgodnie z prawami logiki, jeśli pojawi się choć jedna sprzeczność, hipoteza musi upaść: sprzeczności między źródłami mogą istnieć, lecz najmniejsza niespójność wewnętrzna modelu jest dla niego śmiertelna.
Przetrwać może tylko taki model, który potrafi objąć więcej faktów i nie zawiera sprzeczności.
Nawet najlepsze systemy SI potrafią produkować istotne halucynacje, ale w tym projekcie służą wyłącznie do czytania tekstów i identyfikowania wzorców, a nie do podejmowania decyzji czy formułowania wniosków.

Nie jest to droga na skróty.
Wręcz przeciwnie — to skumulowana praca niezliczonych badaczy, poddana dyscyplinie logiki i nieustannej konfrontacji, zebrana w jednym polu widzenia.
SI pozwala przeczytać cały dostępny korpus i zachować tylko to, co przetrwa ścisłą kontrolę.
W praktyce o tym, co jest czytane, decydują czas, instytucjonalne uwarunkowania i bariery językowe.
Wiele wartościowych prac historycznych nienależących do obszaru teologii lub pochodzących z innych kultur i języków pozostaje nietkniętych.
My czytamy wszystko, zachowujemy to, co wytrzymuje krytykę, a resztę odrzucamy.
Pobożni chrześcijanie częściej omijają książki krytyczne wobec religii.
Badacze świeccy często pomijają argumenty sformułowane teologicznie.
Myślący naukowo z kolei omijają wszystko, co uznane zostaje za „niszowe”.
Tymczasem prace oznaczone jako „niszowe” nierzadko opierają się na słabych założeniach, lecz zawierają zaskakująco trafne spostrzeżenia albo stawiają pytania, których główny nurt nawet nie podejmuje.
Wina przez skojarzenie — jeden z najcięższych błędów logicznych — jest w tej dziedzinie nagminna.
Pozornie oczywiste argumenty odrzuca się wyłącznie dlatego, że pojawiają się obok błędów innego rodzaju albo w tekstach później uznanych za „heretyckie”.
We wszystkich obozach — pobożnych, świeckich i niekonwencjonalnych — znajdujemy wyjątkowo trafne obserwacje i mocne argumenty.
Zderzamy się także z poglądami powszechnie akceptowanymi, których konsekwencje nie zostały dotąd w pełni zbadane.
Ostatecznie powstaje obraz znacznie ostrzejszy i bardziej wymagający niż w dotychczasowych badaniach nad Jezusem i początkiem chrześcijaństwa.
A wszystko to prowadzi do zaskakującej możliwości: odpowiedź mogła od początku leżeć na widoku.

Nie jest to książka teologiczna i nie ma ambicji zmieniać czyichkolwiek przekonań religijnych.
Jej celem jest raczej zaproponowanie nowego sposobu myślenia, który zarówno głęboko wierzący chrześcijanie, jak i osoby niereligijne — czy to zwykli czytelnicy, czy zawodowi badacze — mogą uznać za intelektualnie wartościowy.
Jezusa i wczesne chrześcijaństwo można badać historycznie z taką samą rygorystycznością jak każdą inną postać starożytności.
Nie próbujemy podważać wiary, ale chcemy zakwestionować powszechnie przyjmowane założenia dotyczące Jezusa i jego pierwszych zwolenników.
Naszym celem jest zachęcić czytelników do zmierzenia się z nowymi, ważnymi pytaniami o historycznego Jezusa i początki chrześcijaństwa.

Poza samą identyfikacją pytań narzędzia SI znakomicie wykrywają także sprzeczności i nieciągłości w tradycyjnych narracjach.
Niektóre elementy dominującego konsensusu są tak głęboko zakorzenione, że powtarza się je jak pewniki, mimo że przytłaczające świadectwa wskazują gdzie indziej.
W szczególności ponownie analizujemy kwestie datowania i autorstwa Ewangelii.
Przykłady:
\begin{itemize}
\item \textbf{Prymat i autorstwo Jana} --- gdzie sprawdzimy, czy Ewangelia Jana, od dawna postrzegana jako najpóźniejsza, może w rzeczywistości zawierać najwcześniejsze świadectwo — przekazane przez kobietę naocznego świadka.
\item \textbf{Datowanie narodzin Jezusa} --- gdzie pokażemy, że opisy narodzenia mają solidne podstawy historyczne, a mimo to zostały niemal powszechnie odrzucone jako niehistoryczne na podstawie błędnego XIX-wiecznego datowania zaćmienia księżyca, które wielu uczonych wielokrotnie prostowało.
\end{itemize}

W badaniu wczesnego chrześcijaństwa pojawiają się także inne pytania:
\begin{itemize}
\item \textbf{Zadziwiająco szybki i niezwykle selektywny rozwój chrześcijaństwa w I wieku} --- Twierdzimy, że chrześcijaństwo jako ruch istniejący już za życia Jezusa jest znacznie bardziej wiarygodne niż tradycyjny model małej sekty, która rozwinęła się dopiero po jego śmierci, a swoje podstawowe teksty spisała dopiero dziesiątki lat później.
\item \textbf{Sukces chrześcijaństwa mimo niespełnionych proroctw} --- Utrzymujemy, że misja Jezusa nie była chybioną zapowiedzią, lecz została spełniona: Królestwo rzeczywiście nadeszło w wyrazistej formie historycznej, choć nie w takiej, jakiej wielu oczekiwałoby dziś.
\end{itemize}

Z tej analizy wynikają dalsze pytania i zapraszamy czytelników do ich śledzenia w całej książce.
Nie każde pytanie prowadzi do pewnej odpowiedzi, ale projekt ma pobudzać ciekawość, pogłębiać kontakt z historią chrześcijaństwa i inspirować do ponownego poszukiwania zrozumienia poprzez badania historyczne.