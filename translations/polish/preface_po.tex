Niewiele postaci w historii wywołało tak wiele konkurujących portretów jak Jezus z Nazaretu.
Niewiele ruchów było tak szczegółowo analizowanych, dyskutowanych i reinterpretowanych jak wczesne chrześcijaństwo.
Przez stulecia badań uczeni i teologowie oferowali radykalnie odmienne odpowiedzi na to samo pytanie: kim był Jezus i jakie było pochodzenie ruchu, który nosił jego imię?

Jedni widzą w nim proroka apokaliptycznego, nauczyciela moralnego, rewolucyjnego zelotę, filozofa cynickiego, mistyka, żydowskiego mesjasza, proroka Allaha, czysto mityczne stworzenie, a nawet realną nadnaturalną istotę.
Każdy z tych portretów wskazuje na rzeczywiste świadectwa i każdy ukształtował całe szkoły myślenia.

A jednak żaden nie okazał się ostateczny.
Każda rekonstrukcja uchwytuje coś prawdziwego i doniosłego, lecz każda pozostawia sprzeczności nierozwiązane, gdy zostanie zestawiona z całokształtem danych.
Nawet model często traktowany w ostatnich dekadach jako domyślny wśród uczonych — prorok apokaliptyczny — spotyka się z nieustanną krytyką, z potężnymi kontrargumentami wysuwanymi przez szerokie grono historyków, teologów i filozofów.
Daleko od rozstrzygnięcia, „poszukiwania historycznego Jezusa” musiały być podejmowane raz po raz przez ponad dwa stulecia: nie dlatego, że pytanie jest błahe, lecz dlatego, że żadnemu pojedynczemu rozwiązaniu nie udało się jeszcze objąć wszystkich danych.

Paradoks jest wyraźny: Jezus i wczesne chrześcijaństwo należą do najlepiej udokumentowanych postaci i ruchów starożytności, a jednak żadna rekonstrukcja nie zdołała utrzymać całości materiału dowodowego bez załamania się w kluczowych punktach.
Ta książka podejmuje inne podejście.
Jej fundamentem jest użycie sztucznej inteligencji do syntezy archiwum większego, niż mógłby przyswoić jakikolwiek pojedynczy człowiek: dwóch tysiącleci pism historycznych, spekulacji teologicznych, krytyki tekstualnej, archeologii i komentarzy kulturowych.
Każde pokolenie uczonych pozostawiało wglądy, ale także ślepe punkty.
AI pozwala nam czytać je wszystkie naraz — odkrywać wzorce, podkreślać sprzeczności i wydobywać pytania, które ukrywały się na widoku.

Dzięki syntezie wspomaganej przez AI możemy nieustannie rewidować ogromne liczby szczegółowych hipotez dotyczących Jezusa i wczesnego chrześcijaństwa.
Każdą hipotezę można szybko przetestować na tle całego korpusu danych.
Jak mówią prawa logiki, w chwili pojawienia się sprzeczności hipoteza musi zostać porzucona: sprzeczności w źródłach są możliwe, ale nawet najmniejsza sprzeczność w samym modelu jest zabójcza.
Tylko model, który przetrwa te zderzenia — uwzględniając więcej dowodów i nie pozostawiając żadnych sprzeczności — zasługuje na to, by pozostać.
Choć nawet najlepsze systemy AI są podatne na poważne halucynacje, w tym projekcie służą wyłącznie jako narzędzia pomagające nam czytać i identyfikować wzorce w ogromnej literaturze, a nie podejmować decyzje czy formułować wnioski.

To nie jest droga na skróty.
Jest wręcz przeciwnie: skumulowana praca niezliczonych uczonych, zdyscyplinowana prawami logiki i nieustannym krzyżowym przesłuchaniem, zebrana w jedno pole widzenia.
AI pomaga nam przeczytać całe archiwum i zachować tylko to, co przetrwa krzyżowe sprawdzenie.
W praktyce czas, motywacje i bariery językowe filtrują to, co zostaje przeczytane.
Nawet wartościowe niereligijne dzieła historyczne lub dorobek naukowy z innych krajów i kultur często pozostają nietknięte.
Czytamy wszystko, zachowujemy to, co okazuje się solidne, a resztę odrzucamy.
Pobożni chrześcijanie mają tendencję omijać książki krytyczne wobec religii.
Świeccy uczeni mają tendencję omijać argumenty ujęte teologicznie.
Myślący naukowo mają tendencję omijać wszystko, co oznaczone jako „fringe”.
Jednocześnie dzieła oznaczone jako „fringe” często mają słabe centralne założenia, a jednak nierzadko zawierają głębokie spostrzeżenia lub pytania, które główny nurt przeoczył.
Wina przez skojarzenie — śmiertelny błąd logiczny — jest w tej dziedzinie powszechna.
Pozornie oczywiste argumenty są często odrzucane wyłącznie dlatego, że pojawiają się w pracach zawierających niezwiązane błędy lub w tekstach później oznaczonych jako „heretyckie”.
We wszystkich obozach — pobożnych, świeckich lub niekonwencjonalnych — znajdujemy głęboko wnikliwe obserwacje i potężne argumenty.
Konfrontujemy także przyjmowane punkty, których konsekwencje nie zostały w pełni zbadane.
Rezultatem jest ostrzejsze, bardziej wymagające dochodzenie dotyczące Jezusa i narodzin chrześcijaństwa niż było to wcześniej możliwe.
I prowadzi to do uderzającej możliwości: odpowiedź mogła przez cały czas ukrywać się na widoku.

To nie jest dzieło teologiczne i nie ma na celu zmieniać religijnych przekonań czytelnika.
Raczej stara się zaproponować nowy schemat interpretacyjny, który zarówno głęboko oddani chrześcijanie, jak i osoby bez religijnej afiliacji — czy to zwykli czytelnicy, czy zawodowi uczeni — mogą uznać za intelektualnie wartościowy.
Jezus i wczesne chrześcijaństwo mogą być badane historycznie z taką samą rygorystycznością, jak każda inna postać starożytności.
Nie dążymy do kwestionowania wiary, ale dążymy do kwestionowania założeń powszechnie przyjmowanych na temat historii Jezusa i jego pierwszych naśladowców.
Naszym celem jest zaproszenie czytelników do zmierzenia się z nowymi i ważnymi pytaniami dotyczącymi historycznego Jezusa i początków chrześcijaństwa.

Poza identyfikowaniem pytań narzędzia AI są również wyjątkowo skuteczne w wykrywaniu sprzeczności i niespójności w tradycyjnych narracjach.
Niektóre elementy głównego konsensusu są tak głęboko zakorzenione, że powtarza się je tak, jakby były poza wszelką dyskusją, choć przytłaczające dowody wskazują w innym kierunku.
W szczególności ponownie analizujemy kwestie takie jak datowanie i autorstwo Ewangelii.
Przykłady obejmują:
* **Prymat i autorstwo Janowe** — gdzie przetestujemy, czy Ewangelia Jana, długo traktowana jako najpóźniejsza, może w rzeczywistości zachowywać najwcześniejsze świadectwo, poprzez kobietę–naocznego świadka.
* **Datowanie narodzin Jezusa** — gdzie pokażemy, że narracje o narodzinach wydają się historycznie ugruntowane, a jednak zostały powszechnie odrzucone jako niehistoryczne na podstawie argumentów opartych wyłącznie na błędzie datowania zaćmienia Księżyca z XIX wieku, który wielu uczonych wielokrotnie wykazało jako błędny.

Inne pytania pojawiają się w badaniu wczesnego chrześcijaństwa:
* **Niemożliwie szybki i niemożliwie selektywny wzrost chrześcijaństwa w pierwszym wieku** — Twierdzimy, że chrześcijaństwo jako ruch istniejący już za życia Jezusa jest znacznie bardziej prawdopodobne niż tradycyjny model małej sekty, która rozwinęła się dopiero po jego śmierci, z tekstami założycielskimi spisanymi dekady później.
* **Sukces chrześcijaństwa mimo rzekomych nieudanych proroctw** — Twierdzimy, że misja Jezusa nie była błędną przepowiednią, lecz została w rzeczywistości wypełniona: Królestwo rzeczywiście nadeszło w wyrazistej formie historycznej, choć nie w sposób, jakiego większość dziś by się spodziewała.

Niezliczone inne pytania wynikają z tej analizy i zapraszamy czytelników do ich zgłębiania w całej książce.
Nie każde pytanie przynosi pewną odpowiedź, ale projekt ma na celu wzbudzić ciekawość, pogłębić zaangażowanie w historię chrześcijaństwa i zainspirować do odnowionych poszukiwań zrozumienia poprzez badania historyczne.
