Niewiele postaci w historii wywołało tak wiele konkurujących portretów jak Jezus z Nazaretu.
Niewiele ruchów zostało tak głęboko przeanalizowanych, przedyskutowanych i przekształconych jak wczesne chrześcijaństwo.
Przez stulecia dociekań uczeni i teologowie proponowali radykalnie odmienne odpowiedzi na to samo pytanie: Kim był Jezus i jakie było pochodzenie ruchu, który nosił jego imię?

Jedni widzą w nim proroka apokaliptycznego, inni nauczyciela moralności, rewolucyjnego zelotę, filozofa cynickiego, mistyka, żydowskiego mesjasza, proroka Allaha, czysto mityczne stworzenie lub nawet rzeczywistą istotę nadprzyrodzoną.
Każdy z tych portretów wskazuje na realne świadectwa i każdy ukształtował całe szkoły myślenia.

A jednak żaden z nich nie okazał się rozstrzygający.
Każda rekonstrukcja uchwyca coś rzeczywistego i istotnego, lecz każda pozostawia nierozwiązane sprzeczności, gdy podda się ją próbie wobec całości materiału.
Nawet model często traktowany w ostatnich dekadach jako domyślny w badaniach — prorok apokaliptyczny — podlega nieustannej krytyce, a stanowcze kontrargumenty wysuwają liczni historycy, teologowie i filozofowie.
Daleko od rozstrzygnięcia, „poszukiwania historycznego Jezusa” musiały być ponawiane raz po raz przez ponad dwa stulecia: nie dlatego, że pytanie jest błahe, lecz dlatego, że żadna propozycja nie zdołała dotąd objąć całych danych bez pęknięć w kluczowych punktach.

Paradoks jest ostry: Jezus i wczesne chrześcijaństwo należą do najlepiej udokumentowanych postaci i ruchów starożytności, a jednak żadna rekonstrukcja nie potrafiła scalić świadectw bez załamań.
Niniejsza książka przyjmuje inne podejście.
Jej fundamentem jest użycie sztucznej inteligencji do syntezy archiwum większego, niż mógłby ogarnąć jakikolwiek pojedynczy człowiek: dwóch tysięcy lat pism historycznych, spekulacji teologicznych, krytyki tekstów, archeologii i komentarza kulturowego.
Każde pokolenie uczonych pozostawiło spostrzeżenia, ale także ślepe punkty.
SI pozwala przeczytać je wszystkie naraz — odkryć wzorce, uwypuklić sprzeczności i odzyskać pytania ukryte na widoku.

Dzięki syntezie wspomaganej SI możemy stale korygować ogromną liczbę szczegółowych hipotez o Jezusie i wczesnym chrześcijaństwie.
Każda hipoteza może być szybko sprawdzona wobec całego korpusu danych.
Jak mówią prawa logiki, w chwili pojawienia się sprzeczności hipoteza musi zostać odrzucona: sprzeczności w źródłach są możliwe, lecz nawet najmniejsza sprzeczność wewnątrz modelu jest fatalna.
Tylko model, który przetrwa te zderzenia — uwzględniając więcej świadectw i nie pozostawiając sprzeczności — zasługuje na uznanie.
Choć nawet najlepsze systemy SI są podatne na poważne halucynacje, w tym projekcie służą wyłącznie jako narzędzia pomagające czytać i rozpoznawać wzorce w ogromnym piśmiennictwie, a nie jako podmioty podejmujące decyzje lub formułujące wnioski.

To nie jest droga na skróty.
To jej przeciwieństwo: skumulowany wysiłek niezliczonych uczonych, zdyscyplinowany prawami logiki i nieustanną krzyżową weryfikacją, zebrany w jednym polu widzenia.
SI pomaga nam przeczytać całe archiwum i zachować tylko to, co przetrwa krzyżową kontrolę.
W praktyce czas, uwarunkowania instytucjonalne i bariery językowe filtrują to, co jest czytane.
Nawet wartościowe niereligijne dzieła historyczne czy badania z innych krajów i kultur często pozostają nietknięte.
Czytamy wszystko, zachowujemy to, co okazuje się solidne, a resztę odrzucamy.
Pobożni chrześcijanie mają skłonność omijać książki krytyczne wobec religii.
Uczeni świeccy często omijają argumenty sformułowane teologicznie.
Myślący naukowo omijają wszystko, co nosi etykietę „niszowe”.
Jednocześnie prace określane jako „niszowe” często mają słabe założenia centralne, lecz nierzadko zawierają głębokie intuicje lub pytania przeoczone przez główny nurt.
Wina przez skojarzenie — fatalny błąd logiczny — jest w tej dziedzinie powszechna.
Na pozór oczywiste argumenty bywają odrzucane wyłącznie dlatego, że pojawiają się w pracach zawierających niezwiązane błędy albo w tekstach później uznanych za „heretyckie”.
We wszystkich obozach — pobożnych, świeckich i niekonwencjonalnych — znajdujemy przenikliwe obserwacje i mocne argumenty.
Napotykamy też szeroko akceptowane twierdzenia, których konsekwencje nie zostały w pełni zbadane.
Rezultatem jest ostrzejsze, bardziej wymagające badanie Jezusa i narodzin chrześcijaństwa niż kiedykolwiek wcześniej.
A prowadzi ono do uderzającej możliwości: odpowiedź mogła cały czas znajdować się na widoku.

Nie jest to dzieło teologiczne i nie ma na celu zmiany przekonań religijnych czytelnika.
Raczej oferuje nowy schemat, który zarówno głęboko wierzący chrześcijanie, jak i osoby bez religijnej afiliacji — niezależnie od tego, czy są to czytelnicy okazjonalni, czy zawodowi uczeni — mogą uznać za intelektualnie wartościowy.
Jezusa i wczesne chrześcijaństwo można badać historycznie z taką samą rygorystycznością jak każdą inną postać starożytności.
Nie dążymy do podważenia wiary, lecz do zakwestionowania założeń powszechnie przyjmowanych w opowieści o Jezusie i jego pierwszych uczniach.
Naszym celem jest zachęcenie czytelników do zmierzenia się z nowymi i ważnymi pytaniami dotyczącymi historycznego Jezusa i początków chrześcijaństwa.

Poza identyfikacją pytań narzędzia SI są również niezwykle skuteczne w wykrywaniu sprzeczności i niespójności w tradycyjnych narracjach.
Niektóre elementy dominującego konsensusu są tak głęboko zakorzenione, że powtarza się je jakby były poza dyskusją, choć przytłaczające świadectwa wskazują w innym kierunku.
W szczególności ponownie badamy takie kwestie jak datowanie i autorstwo Ewangelii.
Przykłady obejmują:
\begin{itemize}
\item \textbf{Prymat i autorstwo Jana} --- gdzie sprawdzimy, czy Ewangelia Jana, od dawna uważana za najpóźniejszą, może w rzeczywistości zachowywać najwcześniejsze świadectwo — poprzez kobietę naoczną.
\item \textbf{Datowanie narodzin Jezusa} --- gdzie pokażemy, że narracje o narodzeniu są historycznie zakorzenione, lecz powszechnie odrzucane jako niehistoryczne na podstawie argumentów opartych wyłącznie na dziewiętnastowiecznej pomyłce w datowaniu zaćmienia księżyca, którą wielu uczonych wielokrotnie wykazywało jako błędną.
\end{itemize}

Inne pytania pojawiają się w badaniu wczesnego chrześcijaństwa:
\begin{itemize}
\item \textbf{Niemożliwie szybki i niemożliwie selektywny wzrost chrześcijaństwa w pierwszym wieku} --- Twierdzimy, że chrześcijaństwo jako ruch istniejący już za życia Jezusa jest znacznie bardziej prawdopodobne niż tradycyjny model małej sekty, która rozwinęła się dopiero po jego śmierci, z tekstami założycielskimi pisanymi dekady później.
\item \textbf{Sukces chrześcijaństwa mimo niespełnionych proroctw} --- Twierdzimy, że misja Jezusa nie była błędnym proroctwem, lecz została faktycznie wypełniona: Królestwo rzeczywiście nadeszło w wyrazistej historycznej formie, choć nie w taki sposób, jak wielu oczekiwałoby dziś.
\end{itemize}

Niezliczone inne pytania wynikają z tej analizy i zapraszamy czytelników do ich badania w całej książce.
Nie każde pytanie przynosi pewną odpowiedź, ale projekt ma pobudzać ciekawość, pogłębiać zaangażowanie w historię chrześcijaństwa i inspirować do ponownego poszukiwania zrozumienia poprzez badania historyczne.